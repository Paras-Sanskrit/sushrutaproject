% !TeX encoding = UTF-8
% !TeX root = incremental_SS_Translation.tex

\nocite{*} % include everything from the bib file in the bibliography

\printbiblist[%
    heading=biblistintoc,
    title=Editions and Abbreviations,
    notkeyword=botanical
    %
    ]
    {shorthand} % the Abbreviations (shorthands)

\indexprologue{\emph{\footnotesize Numbers after the final 
        colon refer to pages in this book.}} 

\printindex[manuscripts]

\printbibliography[notkeyword=edition,
    notkeyword=shorthand,
    notkeyword=botanical,
    heading=bibintoc]

\newpage

\begin{footnotesize}
    
% Skt-Eng and Eng-Skt of \se{}{} words
%\printindex[lexical]  
   
\printshorthands[heading=biblistintoc,
title=Materia Medica Reference Works,
keyword=botanical]

\renewcommand{\glossaryname}{Materia Medica}
% If you want to print all the glossary entries, 
% use the selection=all option in \GlsXtrLoadResources (see 
% xelatex-glossaries.sty.
%
\printunsrtglossaries % plant names.  iterate over all defined entries 
%% the setup for the glossaries package is in xelatex-indexing etc.
%%

\chaptermark{Glossary}\sectionmark{Glossary}
\printindex[lexical]  
\clearpage
%
\end{footnotesize}

\thispagestyle{empty}

