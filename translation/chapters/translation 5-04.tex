% !TeX root = incremental_SS_Translation.tex
\chapter{Kalpasthāna 4: Snakes and Invenomation}

\section{Introduction} 

The fourth chapter of the Kalpasthāna of the \emph{Suśrutasaṃhitā}
addresses the topic of snake bites and snake venom. Unusually for the
Nepalese version of the \SS, the discussion is framed as a question from
Suśruta to the wise Dhanvantari.  Suśruta's questions are about the
number of snakes, how they are classified, the symptoms of their bites
and the pulses or stages of toxic shock experienced by a victim of
snakebite and related topics.  The taxonomy of snakes is presented in 
tabular form in Figures~\ref{snakes1} and
\ref{snakes2}.\footnote{On the idea of notational variants in scientific
    translation, see \cites{elsh-2008}{saru-2016}[81--83]{wuja-2021}.}

\section{Literature} 

A brief survey of this chapter's contents and a detailed assessment of
the existing research on it to 2002 was provided by
Meulenbeld.\footcite[IA, 292--294]{meul-hist} There also exists an
substantial herpetological literature from colonial India as well as more
recent studies of snakes in the context of cultural and religious life.

The ophiological literature of the colonial period begins with
\citet{fayr-1874}, whose work included striking colour paintings of
snakes.\footnote{The first edition of Fayrer's work was published two
    years earlier, in 1872.} \citeauthor{fayr-1874} provided a biological
    taxonomy of snakes as well as chapters on mortality statistics during the
    nineteenth century, treatment and effects of poison, and experimental
    data. \citet{ewar-1878} included descriptions of appearance and behaviour
    of poisonous snakes and sometimes their local names; he also
    distinguished his publication by fine colour
    illustrations.\footnote{Calling his work a supplement to
        \citet{fayr-1874}, but also being cited by Fayrer, \cite{ewar-1878}
        evidently also collected local knowledge from his “snake-man” (p.\,22)}.
        \citet[75--124]{wall-1913} provided a useful analysis of the medical
        effects of snake envenomation in India arranged by the varied
        symptomology of different snakes.  He also discussed the difference
        between the symptoms of toxicity and fright (69--75) and also the
        difficulties arising out of uncertainty aabout the effects of snake-bite
        (124--126). \citet{wall-1921} provided a wealth of detail of the snakes
        of Sri Lanka, including line drawings.
        
\citet{doni-2015} provided a good survey of snakes as protagonists in
religious literature from the \emph{Atharvaveda} through the epics,
\emph{Purāṇas} and Buddhist literature. \citet[31--33 \emph{et
    passim}]{slou-2016} discussed the \SS's \emph{Kalpasthāna} as a precursor
and influence on later Tantric traditions of snake-bite interpretation
and therapy.  \citet{seme-1979} traced semiotics of the term \emph{nāga}
through Vedic, Pali and Sanskrit literature.
    
     %
    %Translations of this chapter
    % since 2000 have appeared by
    %\textcites[131--139]{wuja-2003}[3,
    % 1--15]{shar-1999}{srik-2002}.\footnote{For a
    %    bibliography of translations to 2002, including Latin (1847),
    % English
    % (1877),
    %Gujarati (1963)
    %    and Japanese (1971), see \cite[IB, 314--315]{meul-hist}.}
    
A discussion of this chapter specifically in the light of the Nepalese
manuscripts was published by Harimoto.\footcite[101--104]{hari-2011} After a
close comparative reading of lists of poisonous snakes, Harimoto concluded
that, “the Nepalese version is internally consistent while the [vulgate]
editions are not.”  Harimoto showed how the vulgate editions had been
adjusted textually to smooth over inconsistencies, and gave insights into
these editorial processes.\footnote{The two editions that Harimoto noted,
    \cite{vulgate} and \cite{bhat-1889}, present identical texts.}


\section{Translation}

\begin{translation}
    \item[1] Now we shall explain the \se{kalpa}{procedure} about what should be 
    known concerning the venom in those who have been bitten by
snakes.\footnote{The \emph{Sarvāṅgasundarī}, commenting on
    \Ah{1.16.17}{246}, glossed \dev{kalpa} as \dev{prayoga}.}
    
    \item[3] Suśruta, grasping his feet, questions the wise Dhanvantari, the 
    expert in all the sciences.
    
    \item[4]
    
    “My Lord, please speak about the number of snakes, and their divisions,
the symptoms of someone who has been bitten, and the knowledge
about the \se{vega}{successive shocks} of poisoning”.\footnote{The
    expression “successive shocks” translates \dev{vega}, which is other
    contexts may mean “(natural) urge.”  Here, it is rather the discrete
    stages or phases of physiological reaction to envenomation.  Cf.\ the
    symptoms of cobra poisoning described by \citet[80]{wall-1913}.}
        
    \item[5]
    
    On hearing his query, that distinguished physician spoke.
    
    “The venerable snakes such as Vāsukī and Takṣaka are uncountable. 
    
\item[6--9ab]

“They are snake-lords who support the earth, as bright as the ritual fire,
ceaselessly roaring, raining and scorching. They hold up the earth, with its
oceans, mountains and continents. If they are angered, they can destroy the
whole world with a breath and a look.  Honour to them. They have no role
here in medicine.

“The ones that I shall enumerate in due order are those mundane
ones with poison in their fangs who bite humans.\footnote{The next few
    verses are discussed in detail by \citet[101--104]{hari-2011}, who shows
    that in the taxonomy of snakes, the Nepalese version of the \SS\ has greater
    internal coherence than the vulgate recension.}


\end{translation}

    \begin{figure}
        \centering
        \Tree [.Snakes{ (80)}  
        [.Darvīkara {26 kinds} ]
        [.Maṇḍalin  {22 kinds} ]  
        [.Rājimant  {10 kinds} ]   
        [.Nirviṣa     {12 kinds} ]  
    [.Vaikarañja [.{3 kinds} {7 kinds} ] ]  ]
         \caption{The taxonomy of snakes in the vulgate, \Su{5.4.9--13ab}{571}.}
         \label{snakes1}
\end{figure}
\begin{figure}
\centering          
            \Tree [.Snakes{ (80)}  
            [.Darvīkara {26 kinds} ]
            [.Maṇḍalin  {26 kinds} ]  
            [.Rājimant  {13 kinds} ]   
            [.Nirviṣa     {12 kinds} ]  
            [.Vaikarañja {3 kinds} ]  ]
        \caption{The taxonomy of snakes in the Nepalese version.}
        \label{snakes2}
        \end{figure}
    
    \begin{translation}
        \item[9cd--10]    
        
        “There are eighty kinds of snakes and they are divided in five ways:
        Darvīkaras, Maṇḍalins, Rājimats, and Nirviṣas.  And Vaikarañjas that are
        traditionally of three kinds.\footnote{\citet{hari-2011} translated these
            names as “hooded,” “spotted,” “striped,” “harmless,” and “hybrid.” Figure 
            \ref{snakes1} shows the taxonomy described in the vulgate text; Figure 
            \ref{snakes2} shows the different and more logical division of the Nepalese 
            version of the \SS.}
            
    \item [11] 
    
    “Of those, there are twenty and six hooded snakes, and the same number
of Maṇḍalins are known.\q{Or “There are 20 phaṇins and 6 maṇḍalins.  The
    same number are known. There are 13 Rājīmants.”  Or even, “there are 20
    Phaṇins and six of them are Maṇḍalins.” Are phaṇins really the same as
    darvīkaras?}  There are thirteen Rājīmants.\footnote{The phrasing of
    this śloka is awkward.}
    
    \item [12]
    
    “There are said to be twelve Niriviṣas and, according to tradition, three 
    Vaikarañjas.
    
    \item [13--14ef]
    
“If they are trodden on, ill-natured or provoked or even just looking for
food, those very angry snakes will bite.  And that is said to happen in
three ways: \se{sarpita}{serpented}, \se{darita}{torn} and thirdly
\se{nirviṣa}{without venom}.  Some experts on this want to add “hurt by the
snake's body”.\footnote{This might refer to constriction.  The phrase reads
    like a commentarial addition rather than the main text of the \SS.}

\item[15--16]

“The physician can recognize the following as “\se{sarpita}{ophidian}”:
Where a rearing snake  makes one, two or more puncture-marks of its teeth,
when they are deep and without much blood,\footnote{\label{pada-snakes} The
    word \dev{udvṛtta} “aroused” was glossed by Ḍalhaṇa at \Su{5.4.15}{571} as
    \dev{unmoṭya}, a word not found as such in standard dictionaries
    \citep{moni-sans,apte-prac,mayr-kurz,josi-maha}. Semantic considerations
    suggest that the word is not related to $\surd$\emph{muṭ} “break” or
    \emph{mūta/mūṭa} “woven basket.” Perhaps it is related to the Tamil
    \texttamil{மோடி} (\emph{mōṭi},) whose meanings include “arrogance, grandeur,
    display” \citep[\#5133]{burr-1984} or to faintly-documented forms like
    \emph{moṭyate} “is twisted” \citep[\#10186]{CDIAL}. Ḍalhaṇa's \dev{unmoṭya}
    may thus mean “twisting up” or “making an arrogant display.” \par Note that
    \dev{pada} “puncture-mark” (more literally, “footprint”) is being used in
    the same sense as in \Su{1.13.19}{57} when describing the marks on the body
    where a knife scarifies the skin before leeching. See footnote
    \ref{pada-leeches}.} accompanied by a \se{cuñcumālaka}{little ring of
        spots},\footnote{The usual dictionary lexeme is \dev{cañcu}\,, not 
        \dev{cuñcu}
        as in the Nepalese witnesses.  We translate “spots” following Ḍalhaṇa and
        Gayadāsa on \Su{5.4.15}{571}, where they described a group of spots or
        swellings at the site of the bite. On the history of the word \dev{mālaka},
        see \cite{kief-1996}.} lead to degeneration, and are close together and
        swollen.

\item [17]  Where there are streaks with blood, whether it be blue or white, the 
physican should recognize that to be “\se{darita}{torn},” having a small amount of 
venom. 

\item [18]

% got to here
    
    % 
    %https://global.oup.com/us/companion.websites/9780190200886/student/chapter10/gline/quotation/
\end{translation}