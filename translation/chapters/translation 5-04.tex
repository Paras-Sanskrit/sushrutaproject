% !TeX root = ../incremental_SS_Translation.tex
\chapter{Kalpasthāna 4: Snakes and Invenomation}

\section{Introduction} 

The fourth chapter of the Kalpasthāna of the \emph{Suśrutasaṃhitā}
addresses the topic of snake bites and snake venom. Unusually for the
Nepalese version of the \SS, the discussion is framed as a question from
Suśruta to the wise Dhanvantari.  Suśruta's questions are about the
number of snakes, how they are classified, the symptoms of their bites
and the pulses or stages of toxic shock experienced by a victim of
snakebite and related topics.  The taxonomy of snakes is presented in 
tabular form in Figures~\ref{snakes1} and
\ref{snakes2}.\footnote{On the idea of notational variants in scientific
    translation, see \cites{elsh-2008}{saru-2016}[81--83]{wuja-2021}.}

\section{Literature} 

A brief survey of this chapter's contents and a detailed assessment of
the existing research on it to 2002 was provided by
Meulenbeld.\footnote{\volcite{IA}[292--294]{meul-hist}. In addition to the
    translations mentioned by \tvolcite{IB}[314--315]{meul-hist}, a translation
    of this chapter was included in \volcite{3}[35--45]{shar-1999}.}  There
    also exists an substantial herpetological literature from colonial India
    as well as more recent studies of snakes in the context of cultural and
    religious life.

%Translations of this chapter
% since 2000 have appeared by
%\textcites[131--139]{wuja-2003}[3,
% 1--15]{shar-1999}{srik-2002}.\footnote{For a
%    bibliography of translations to 2002, including Latin (1847),
% English
% (1877),
%Gujarati (1963)
%    and Japanese (1971), see \cite[IB, 314--315]{meul-hist}.}


The ophiological literature of the colonial period began in the late
nineteenth century with the work of Fayrer, whose publication included
striking colour paintings of snakes.\footnote{\cite{fayr-1874}, first
    published in 1872.} \citeauthor{fayr-1874} provided a biological taxonomy
    of snakes as well as chapters on mortality statistics during the
    nineteenth century, treatment and effects of poison, and experimental
    data. \citet{ewar-1878} included descriptions of appearance and behaviour
    of poisonous snakes and sometimes their local names and reproducing
    Fayrer's illustrations.\footnote{Calling his work a supplement to
        \citet{fayr-1874}, but also being cited by Fayrer, \cite{ewar-1878}
        evidently also collected local indigenous knowledge from his “snake-man”
        (p.\,22).} \citet[75--124]{wall-1913} provided a useful analysis of the
        medical effects of snake envenomation in India arranged by the varied
        symptomology of different snakes.  He also discussed the difference
        between the symptoms of toxicity and fright (69--75) and also the
        difficulties arising out of uncertainty about the effects of snake-bite
        (124--126).  The \SS\ too recognized the emotional and somatic effects of
        fright (see note \ref{fright} below). \citet{wall-1921} provided a wealth
        of detail of the snakes of Sri Lanka, including line drawings.
        
\citet{doni-2015} provided a good survey of snakes as protagonists in
religious literature from the \emph{Atharvaveda} through the epics,
\emph{Purāṇas} and Buddhist literature. \citet{seme-1979} traced
semiotics of the term \emph{nāga} through Vedic, Pali and Sanskrit
literature.  \citet[31--33 \emph{et passim}]{slou-2016} discussed the
\SS's \emph{Kalpasthāna} as a precursor and influence on later Tantric
traditions of snake-bite interpretation and therapy.  In particular, the
Tantric \emph{Kriyākālaguṇottara} text that Slouber presented
divided snakes into two basic categories, divine and mundane, as the \SS\
does.\footcite[144--145]{slou-2016}  But unlike the \SS, in the
\emph{Kriyākālaguṇottara} the chief taxonomic principle for both groups
is the four \emph{varṇa}s.  
    
A discussion of this chapter specifically in the light of the Nepalese
manuscripts was published by Harimoto.\footcite[101--104]{hari-2011} After a
close comparative reading of lists of poisonous snakes, Harimoto concluded
that, “the Nepalese version is internally consistent while the [vulgate]
editions are not.”  Harimoto showed how the vulgate editions had been
adjusted textually to smooth over inconsistencies, and gave insights into
these editorial processes.\footnote{The two editions that Harimoto noted,
    \cite{vulgate} and \cite{bhat-1889}, present identical texts.}


\section{Translation}

\begin{translation}
    \item[1] Now we shall explain the \se{kalpa}{procedure} about what should be 
    known concerning the venom in those who have been bitten by
snakes.\footnote{The \emph{Sarvāṅgasundarī}, commenting on
    \Ah{1.16.17}{246}, glossed \dev{kalpa} as \dev{prayoga}.}
    
    \item[3] Suśruta, grasping his feet, questions the wise Dhanvantari, the 
    expert in all the sciences.
    
    \item[4]
    
    “My Lord, please speak about the number of snakes, and their divisions,
the symptoms of someone who has been bitten, and the knowledge
about the \se{vega}{successive shocks} of poisoning”.\footnote{The
    expression “successive shocks” translates \dev{vega}, which is other
    contexts may mean “(natural) urge.”  Here, it is rather the discrete
    stages or phases of physiological reaction to envenomation.  Cf.\ the
    symptoms of cobra poisoning described by \citet[80]{wall-1913}.}
        
    \item[5]
    
    On hearing his query, that distinguished physician spoke.
    
    “The venerable snakes such as Vāsukī and Takṣaka are uncountable. 
    
\item[6--9ab]

“They are snake-lords who support the earth, as bright as the ritual fire,
ceaselessly roaring, raining and scorching. They hold up the earth, with its
oceans, mountains and continents. If they are angered, they can destroy the
whole world with a breath and a look.  Honour to them. They have no role
here in medicine.

“The ones that I shall enumerate in due order are those mundane
ones with poison in their fangs who bite humans.\footnote{The next few
    verses are discussed in detail by \citet[101--104]{hari-2011}, who shows
    that in the taxonomy of snakes, the Nepalese version of the \SS\ has greater
    internal coherence than the vulgate recension.}


\end{translation}

    \begin{figure}
        % The \Tree command calls on the qtree package.
        \centering
        \Tree [.Snakes{ (80)}  
        [.Darvīkara {26 kinds} ]
        [.Maṇḍalin  {22 kinds} ]  
        [.Rājimant  {10 kinds} ]   
        [.Nirviṣa     {12 kinds} ]  
    [.Vaikarañja [.{3 kinds} {7 kinds} ] ]  ]
         \caption{The taxonomy of snakes in the vulgate, \Su{5.4.9--13ab}{571}.}
         \label{snakes1}
\end{figure}
\begin{figure}
\centering          
            \Tree [.Snakes{ (80)}  
            [.Darvīkara {26 kinds} ]
            [.Maṇḍalin  {26 kinds} ]  
            [.Rājimant  {13 kinds} ]   
            [.Nirviṣa     {12 kinds} ]  
            [.Vaikarañja {3 kinds} ]  ]
        \caption{The taxonomy of snakes in the Nepalese version.}
        \label{snakes2}
        \end{figure}
    
    \begin{translation}
        \item[9cd--10]    
        
        “There are eighty kinds of snakes and they are divided in five ways:
        Darvīkaras, Maṇḍalins, Rājīmats, and Nirviṣas.  And Vaikarañjas that are
        traditionally of three kinds.\footnote{\citet{hari-2011} translated these
            names as “hooded,” “spotted,” “striped,” “harmless,” and “hybrid.” Figure 
            \ref{snakes1} shows the taxonomy described in the vulgate text; Figure 
            \ref{snakes2} shows the different and more logical division of the Nepalese 
            version of the \SS.}
            
    \item [11] 
    
    “Of those, there are twenty and six hooded snakes, and the same number
of Maṇḍalins are known.\q{Or “There are 20 phaṇins and 6 maṇḍalins.  The
    same number are known. There are 13 Rājīmats.”  Or even, “there are 20
    Phaṇins and six of them are Maṇḍalins.” Are phaṇins really the same as
    darvīkaras?}  There are thirteen Rājīmats.\footnote{The phrasing of
    this śloka is awkward.}
    
    \item [12]
    
    “There are said to be twelve Niriviṣas and, according to tradition, three 
    Vaikarañjas.
    
    \item [13--14ef]
    
“If they are trodden on, ill-natured or provoked or even just looking for
food, those very angry snakes will bite.  And that is said to happen in
three ways: \se{sarpita}{serpented}, \se{darita}{torn} and thirdly
\se{nirviṣa}{without venom}.  Some experts on this want to add “hurt by the
snake's body”.\footnote{This might refer to constriction.  The phrase reads
    like a commentarial addition rather than the main text of the \SS.}

\item[15--16]

“The physician can recognize the following as “\se{sarpita}{ophidian}”:
Where a rearing snake  makes one, two or more puncture-marks of its teeth,
when they are deep and without much blood,\footnote{\label{pada-snakes} The
    word \dev{udvṛtta} “aroused” was glossed by Ḍalhaṇa at \Su{5.4.15}{571} as
    \dev{unmoṭya}, a word not found as such in standard dictionaries
    \citep{moni-sans,apte-prac,mayr-kurz,josi-maha}. Semantic considerations
    suggest that the word is not related to $\surd$\emph{muṭ} “break” or
    \emph{mūta/mūṭa} “woven basket.” Perhaps it is related to the Tamil
    \texttamil{மோடி} (\emph{mōṭi},) whose meanings include “arrogance, grandeur,
    display” \citep[\#5133]{burr-1984} or to faintly-documented forms like
    \emph{moṭyate} “is twisted” \citep[\#10186]{CDIAL}. Ḍalhaṇa's \dev{unmoṭya}
    may thus mean “twisting up” or “making an arrogant display.” \par Note that
    \dev{pada} “puncture-mark” (more literally, “footprint”) is being used in
    the same sense as in \Su{1.13.19}{57} when describing the marks on the body
    where a knife scarifies the skin before leeching. See footnote
    \ref{pada-leeches}.} accompanied by a \se{cuñcumālaka}{little ring of
        spots},\footnote{The usual dictionary lexeme is \dev{cañcu}\,, not 
        \dev{cuñcu}
        as in the Nepalese witnesses.  We translate “spots” following Ḍalhaṇa and
        Gayadāsa on \Su{5.4.15}{571}, where they described a group of spots or
        swellings at the site of the bite. On the history of the word \dev{mālaka},
        see \cite{kief-1996}.} lead to degeneration, and are close together and
        swollen.

\item [17]  

Where there are streaks\q{grammar} with blood, whether it be blue or white, the
physican should recognize that to be “\se{darita}{torn},” having a small
amount of venom.

\item[18]

The physician can recognize the locations of the bites of a person in a
normal state as being free from poison, when the location is not swollen,
and there is little corrupted blood.

\item [19]

The wind of a timid person who has been touched by a snake can get
irritated by fear.  It causes
swelling.\footnote{\label{fright}\citet[69]{wall-1913} remarked on the difficulty 
of separating
    toxicity symptoms from the psychosomatic effects of terror:\begin{quoting}
        The gravity of symptoms due to fright does not appear to me to be 
        sufficiently recognised, though there is no doubt in my mind that fatal cases 
        from this cause are abundant, especially among the timid natives of this 
        country.\end{quoting} Wall went on to give several case studies in which 
        patients experienced syncope or even died as a result of bites from 
        toxicologically harmless creatures.}  That is “hurt
    by a snake's body.”

\item [20]

Locations bitten by sick or frightened snakes are known to have little poison.  
Similarly, a site bitten by very young or old snakes has little poison.

\item [21]

Poison does not progress in a place frequented by
eagles,\footnote{Ḍalhaṇa on \Su{5.4.21}{571} identified the \dev{suparṇa}
    as a \dev{garuḍa}. On the bird called \dev{suparṇa}, \citet[72\,ff,
    514]{dave} too noted that it may be a synonym for Garuḍa,
    and in some contexts may refer to the Golden Eagle, Golden
    Oriole, Lammergeyer, etc. \citet[199\,ff, 492]{dave} noted again that the
    Garuḍa is a mythical bird but may refer to the Himalayan Golden Eagle and
    other species of eagle.  He pointed out that historically,
    \begin{quoting}
        The original physical basis for \dev{garuḍa} as the \dev{nāgāśī}
    (snake-eater) was most probably the Sea-Eagle who picks up sea-snakes
    from the sea or sand-beach and devours them on a nearby tree\ldots\  
    \citep[201]{dave}.
    \end{quoting} Dave
    continued with interesting reference to Śrīharṣa's \emph{Nāgānanda}.}
    gods, holy sages, \diff{spirits}, and saints, or in places full of herbs
    that destroy poison.\footnote{For “spirits” the Nepalese version has
        \dev{bhūta} while the vulgate reads \dev{yakṣa}.}

\subsection{[Types of snake]}

\item [22]

Darvīkara snakes are know to have hoods, to move rapidly, and to have rings, 
ploughs, umbrellas, crosses, and hooks on them.


\item [23]

Maṇḍalin snakes are known for being large and slow-moving.  They are 
decorated with many kinds of circles. 
They are like a flaming fire because of their poisons.


\item [24]

Rājimant snakes are smooth and traditionally said to be, as it were,
mottled with multicoloured streaks across and above.

\subsubsection{[Classes of snake]}

\item[25]

Snakes that are shine like pearls and silver, and that are amber and that
shine like gold, and smell sweet are traditionally thought of as being of
the Brāhmaṇa caste.

\item [26]

Warrior snakes, however, are those that look glossy and get very angry.
The have the mark of the sun, the moon, the earth, an umbrella and
\gls{adrija}.

\item [27]

Merchant snakes may traditionally be black, shine like diamond or have a
red colour or be grey like pigeons.

% got to here

\item [28]

Any snakes that are coloured like a buffalo and a tiger, with rough skin
and different colours are known as servants.\footnote{Presumably
    “different” from the earlier-mentioned castes.
    
    The sequence of the following three verses is slightly different from the
vulgate (\Su{5.4.29--31}{572}).}

\item[31]

All snakes that are variegated (Rājīmats) move about during the first watch
of the night.  The rest, on the other hand, the Maṇḍalins and the
Darvīkaras, are diurnal.\footnote{The readings of the vulgate, that Rājīmats are 
active in the early night, the Maṇḍalins in the later night, and Darvīkaras in the 
day, seem clearer.}

\item[29]

Wind is irritated by all hooded snakes; bile by Maṇḍalins and phlegm by those 
with many  stripes.

\item [30]

Because of the two classes having greater, lesser or equal class, there is the 
characteristic of irritating two humours.  

And he will explain the opposing view that is to be known as a result of the 
non-union of a male and female.\footnote{The 
sense of the last phrase here is quite different from the vulgate, which says 
only that “details” will be explained below.}


%\item[32]
%\item[33]

\subsubsection{[Enumeration of snakes]}
\item[34.1]

In that context, here are the Darvīkaras.
\begin{multicols}{2}
\begin{enumerate}
    \raggedright
    \item \se{kṛṣṇasarpa}{The Black snake};
    \item \se{mahākṛṣṇa}{The Big Black};
    \item \se{kṛṣṇodara}{The Black Belly};
    \item \se{sarvakṛṣṇa}{The All Black};\footnote{Not in vulgate.}
    \item \se{śvetakapota}{The White Pigeon};\footnote{The vulgate adds 
    \se{mahākapota}{The Big Pigeon}.}
    \item \se{valāhako}{The Rain Cloud};
    \item \se{mahāsarpa}{The Great Snake};
    \item \se{śaṃkhapāla}{The Conch Keeper};
    \item \se{lohitākṣa}{The Red Eye};
    \item \se{gavedhuka}{The Gavedhuka};
    \item \se{parisarpa}{The Snake Around};
    \item \se{khaṇḍaphaṇa}{The Break Hood};
    \item \se{kūkuṭa}{The Kūkuṭa};
    \item \se{padma}{The Lotus};
    \item \se{mahāpadma}{The Great Lotus};
    \item \se{apuṣpa}{The Grass Flower};
    \item \se{dadhimukha}{The Curd Mouth};
    \item \se{puṇḍarīkamukha}{The Lotus Mouth};
    \item \se{babhrūkuṭīmukha}{The Brown Hut Mouth};
    \item \se{vicitra}{The Variegated};
    \item \se{puṣpābhikīrṇnābha}{The Flower Sprinkle Beauty};
    \item \se{girisarpa}{The Mountain Snake};
    \item \se{ṛjusarpa}{The Straight Snake};\q{ri- ṛ-?}
    \item \se{śvetadara}{The White Rip};
    \item \se{mahāśīrṣa}{The Big Head}; and
    \item \se{alagarda}{The Hungry Sting};
   
    \end{enumerate}
\end{multicols}

\bigskip

\item[34.2] 

Here are the Maṇḍalins
\begin{multicols}{2}
    \begin{enumerate}
        \raggedright
 \item \se{ādarśamaṇḍala}{The Mirror Ring}; 
 \item \se{śvetamaṇḍala}{The White Ring}; 
 \item \se{raktamaṇḍala}{The Red Ring}; 
 \item \se{pṛṣata}{The Speckled}; 
 \item \se{devadinna}{The Gift of God}; 
 \item \se{pilindaka}{The Pilindaka}; 
 \item \se{vṛddhagonasa}{The Big Cow Snout}; 
 \item \se{panasaka}{The Jackfruit}; 
 \item \se{mahāpanasaka}{The Big Jackfruit}; 
 \item  \se{veṇupatraka}{The Bamboo Leaf}; 
 \item \se{śiśuka}{The Kid}; 
 \item \se{madanaka}{The Intoxicator}; 
 \item \se{pālindaka}{The Morning Glory}; 
 \item \se{tantuka}{The Stretch}; 
 \item \se{puṣpapāṇḍu}{The Pale as a Flower}; 
 \item  \se{ṣaḍaṅga}{The Six Part};
 \item \se{agnika}{The Flame}; 
 \item \se{babhru}{The Brown};
 \item \se{kaṣāya}{The Ochre}; 
 \item \se{khaluṣa}{The Khaluṣa}; 
 \item \se{pārāvata}{The Pigeon}; 
 \item \se{hastābharaṇaka}{The Hand Decoration}; 
 \item \se{tatra}{The Tatra};\footnote{This seems implausible, but 
 otherwise the 
 list of Maṇḍalins would be short.} 
 \item \se{citraka}{The Mark}; 
 \item \se{eṇīpada}{The Deer Foot}.\footnote{The list is short by
     one item.  Perhaps the one of the snakes named in the vulgate, 
     \emph{citramaṇḍala}, \emph{gonasa} or \emph{piṅgala}, should be 
     considered here.}
     
    \end{enumerate}
\end{multicols}
        
        \medskip
        
\item[34.3]

Here are the Rājīmats.\footnote{The following list is one item short.  The 
vulgate text, however, has several names that do not appear in the Nepalese 
Rājīmat list, for example Sarṣapaka and Godhūmaka.}
\begin{multicols}{2}
    \begin{enumerate}
        \raggedright
        \item \se{puṇḍarīka}{The Lotus}; 
        \item \se{rājicitra}{The Stripe Speckle}; 
\item \se{aṅgulirāji}{The Finger Stripe}; 
\item \se{dvyaṅgulirāji}{The Two Finger Stripe}; 
\item \se{bindurāji}{The Drop Stripe}; 
\item \se{kardama}{The Mud}; 
\item \se{tṛṇaśoṣaka}{The Grass Drier}; 
\item \se{svetahanu}{The White Jaw}; 
\item \se{darbhapuṣpa}{The Grass Flower};\footnote{Also in the Darvīkara 
list.}  
\item \se{lohitākṣa}{The Red Eye};\footnote{Also in the Darvīkara 
    list.}  
\item \se{cakraka}{The Ringed}; 
\item \se{kikkisāda}{The Worm Eater};
  
    \end{enumerate}
\end{multicols}

\medskip

\item[34.4]
Here are the Nirviṣas.

\begin{multicols}{2}
    \begin{enumerate}
        \raggedright
\item \se{valāhako}{The Rain Cloud};\footnote{Also in the Darvīkara 
    list.}  
    \item \se{ahipatāka}{Thei Snake Flag}; 
    \item \se{śukapatra}{The White Leaf};
    \item \se{ajagara}{The Goat Swallower}; 
    \item \se{dīpyaka}{The Stimulator}; 
    \item \se{ilikinī}{The Ilikinī}; 
    \item \se{varṣāhīka}{The Year-Snake};
    \item \se{dvyāhika}{The Two-day}; 
    \item \se{kṣīrikāpuṣpa}{The Milk Flower}; 
    \item \se{puṣpasakalī}{The Flower All}; 
    \item \se{jyotīratha}{The Chariot of Light}; 
    \item \se{vṛkṣaka}{The Little Tree};
    \end{enumerate}
\end{multicols}

\medskip

\item[34.5]

The Vaikarañjas originate out of contrary unions amongst the three 
\diff{colours}.\q{varṇa means “colour” elsewhere?}\footnote{The word 
\emph{varṇa} in this chapter normally means “colour” not “class.”  (“Class is 
expressed by “jāti.”)  While \emph{kṛṣṇasarpa} is clearly a colour-type, it is less 
obvious that \emph{gonasī} is a special colour, and \emph{rājimat} is a group of 
snakes.}
    Thus:

%\begin{multicols}{2}
    \begin{enumerate}
        \raggedright
\item The \se{mākuli}{Mākuli}; 
\item The \se{poṭagala}{Poṭa Throat}; 
\item The \se{snigdharāji}{Oil Stripe}; 
\end{enumerate}
%\end{multicols}

Amongst those, the \se{mākuli}{Mākuli};  is born when a male Black Snake
mates with a female \se{gonasa}{Cow Snout}, or the reverse.  The
\se{poṭagala}{Poṭa Throat} is born when a male Rājila mates with a female
\se{gonasa}{Cow Snout} or the reverse.  The \se{snigdharāji}{Oily Stripe}
is born when a male Black Snake mates with a female Rājimat, or the
reverse. Their poison is like that of their father, because it is the
superior one out of the two; but others say it is like the mother.   Thus
eighty of these snakes have been described.


\item[35]
Amongst them, males have large eyes, tongues and heads.\footnote{The 
vulgate includes the snake's mouth in this and the next list.}  Females have 
small 
eyes, tongues and heads. Neuters have both characteristics, and are slow to 
exert themselves or be angry.\footnote{The reading \dev{mandaceṣṭākrodhā} 
is an awkward compound; possibly the original reading was \dev{mandaceṣṭāḥ + 
akrodhā} and sandhi was applied twice.}

\item[36] In that context we shall give instruction in a general way
about the sign of having been bitten by any of the snakes.

For what reason?

Because poison acts quickly, like a fire with an oblation, a honed sword,
or a thunderbolt.  And ignored for even a period of time, it can drag the
patient away. There is not even an opportunity to follow the
literature.\footnote{The idea seems to be that there is no time to
    consult the verbose āyurvedic teachings.  The
    “\sed{vāksamūhārthavistāra}{extensive meaning of the literature}” is
    singled out as one of Āyurveda's virtues in \Su{5.8.142}{594}.} % got to
    % here 2023-10-25
    And when the symptom of being bitten is stated, there will be three ways of 
    treating it because there are three kinds of snake. Therefore we shall
    explain it in three ways. For this is good for patients, does cause
    confusion, and in this alone is every manifestation contained.\q{?}

\item[37]
Thus, the poison of a Darvīkara causes the skin, nails, eyes, mouth, urine, feces, 
and teeth to be black; there is driness, the joints hurt and the head feels heavy; 
the waist, back and neck feel week; there is yawning, the voice becomes faint, 
\ldots

% got to here. 

\item[38]
\item[39]
\item[40]
\item[41]
\item[42]
\item[43]
\item[44]
\item[45]



    
    % 
    %https://global.oup.com/us/companion.websites/9780190200886/student/chapter10/gline/quotation/
\end{translation}