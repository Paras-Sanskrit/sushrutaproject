% !TeX root = incremental_SS_Translation.tex
% Deepro

\chapter{Uttaratantra \diff{65}:  Rules of Interpretation}

\section{Literature} 

Meulenbeld offered an annotated overview of this chapter and a bibliography
of earlier scholarship to 2002.\fvolcite{IA}[331]{meul-hist}  Earlier explorations 
of this topic include \cite{dasg-1952,
    lele-1981,
    mejo-2000,
    nara-1949,
    ober-1967,
    scha-1993,
    sing-2003,
    muth-1976}. 
\cite{mane-2008} gave examples of the use of tantrayuktis in Buddhist 
commentarial literature.

\section{Translation}

\begin{translation}
    
    \item [1] Now we shall explain the chapter called Enunciation of the \se{tantrayukti}{Methods of Disciplines}
    
    \item [3] There are thirty-two methods of disciplines. They are as follows: 
        \begin{itemize}
            \item \se{adhikaraṇa}{topic}
            \item \se{yoga}{logical linkage}
            \item \se{padārtha}{word meaning}
            \item \se{hetvartha}{premise}
            \item \se{samuddeśa}{mention}
            \item \se{nirdeśa}{description}
            \item \se{upadeśa}{prescription}
        \end{itemize}
    
    \item [4] Here one says, what is the purpose of these methods? The answer is: connecting sentences and connecting meanings.
    
    \item [5-6] Here are verses regarding it:
    
    With the methods of the discipline, one prevents/refutes wrong and unsuitable statements, establishes their own statements, and even clarifies the sense insinuated or inexplicit or told in an inverted fashion. 
     
     or,
     
    Through the methods of disciplines, wrong and unsuitable statements are refuted/prevented, one’s own statements are established, and even the senses that are insinuated, inexplicit or told in an inverted fashion are clarified. 
    
    \item [8] Among them, \se{adhikaraṇa}{topic} is the object, concerning which statements are made, such as \se{rasa}{flavour} or \se{doṣa}{humour}. 
    
    \item [9] \se{yoga}{Logical linkagage} is that by which a sentence is coherent, such as connecting the meanings of words which are stated in a reverse/unusual/different order whether placed close or apart.
    
    \dev{tailaṃ pibeccāmṛtavallinimbahaṃsāhvayāvṛkṣakapippalībhiḥ |\\
    siddhaṃ balābhyāñca sadevadāru hitāya nityaṃ galagaṇḍaroge ||\\}
        ``For a benefit in the disease of goitre, one should always drink cooked sesame oil with heart-leaved moonseed (Tinospora cordifolia), neem, the plant walking maidenhair fern (Adiantum lunulatum or Adiantum philippense), long pepper, heart-leaf sida, country marrow', and deodar.'' 
    
    In this verse, Instead of saying first \dev{siddhaṃ pibet} 'one should drink boiled' the word \emph{siddha} 'cooked' is used in the second hemistich. Thus connecting the meanings of words that are placed apart is also a logical linkage.  
     
    
    \item [10] The meaning that is conveyed in an aphorism or a word is called \se{padārtha}{word meaning}. In other words, word meaning is the meaning of one or more words. Word meanings are unlimited. 
    
   Where two or three meanings such as ‘fat’, ‘sweat’ or ‘anointment’ appear to be possible, one should consider the meaning which is valid through logical linkage with prior and subsequent elements. For example, comprehension is doubtful when it is said that “We are going to explain the chapter on the origin of Veda” concerning that the origin of which Veda is to be stated. Vedas are those such as the \emph{Sāmaveda} etc. With regard to prior and subsequent elements, the roots vind and vid have the same meaning. Later a word appears that clarifies that he wants to talk about the origin of Āyurveda. Hence, it (āyurveda) is the word meaning.  
    
    \item [11] That which is stated and becomes the proof of an argument is the \se{hetvartha}{premise}. For example, as a lump of earth is moist by water, the same way a wound becomes moist by substances like milk with black grams and so on.  
    
    \item [12] A brief statement is called \se{samuddeśa}{mention}, such as \se{śalya} {pain-causing agent}. 
    
    \item [13] A detailed statement is \se{nirdeśa}{description}, for example, the pain-causing agents are endogenous or exogenous. 
    
    \item [14] Statements like ``one should do this way is '\se{upadeśa}{prescription}). For example, one should not stay awake at night; you should not sleep during daytime.      
    
    
    
\end{translation}
