% !TeX root = incremental_SS_Translation.tex
% Deepro

\chapter{Uttaratantra \diff{65}:  Rules of Interpretation}

\section{Literature} 

Meulenbeld offered an annotated overview of this chapter and a bibliography
of earlier scholarship to 2002.\fvolcite{IA}[331]{meul-hist}  Earlier explorations 
of this topic include \cite{dasg-1952,
    lele-1981,
    mejo-2000,
    nara-1949,
    ober-1967,
    scha-1993,
    sing-2003,
    muth-1976}. 
\citet{mane-2008} gave examples of the use of tantrayuktis in Buddhist 
commentarial literature.

\subsection{Terminology}



\section{Characteristics of the Manuscript Transmission}

% Deepro

\section{Translation}

\begin{translation}
    
    \item [1] Now we shall explain the chapter called Enunciation of the 
    \se{tantrayukti}{logical methods of the system}.
    
    \item [3] There are thirty-two logical methods of the system. They are as 
    follows: 
        \begin{itemize}
            \item \se{adhikaraṇa}{topic}
            \item \se{yoga}{semantic linkage}
            \item \se{padārtha}{word meaning}
            \item \se{hetvartha}{premise}
            \item \se{samuddeśa}{mention}
            \item \se{nirdeśa}{description}
            \item \se{upadeśa}{prescription}
            \item \se{apadeśa}{statement of reason}
            \item \se{pradeśa}{indication}
            \item \se{atideśa}{prediction}
            \item \se{apavarga}{}
            \item \se{vākyaśeṣa}{}
            \item \se{arthāpatti}{}
            \item \se{viparyaya}{}
            \item \se{prasaṅga}{}
            \item \se{ekānta}{}
            \item \se{anekānta}{}
            \item \se{pūrvapakṣa}{}
            \item \se{nirṇaya}{}
            \item \se{anumata}{}
            \item \se{vidhāna}{}
            \item \se{anāgatāpekṣaṇa}{}
            \item \se{atikrāntāpekṣaṇa}{}
            \item \se{saṃśaya}{}
            \item \se{vyākhyāna}{}
            \item \se{svasaṃjñā}{}
            \item \se{nirvacana}{}
            \item \se{nidarśana}{}
            \item \se{niyoga}{}
            \item \se{vikalpa}{}
            \item \se{samuccaya}{}
            \item \se{ūhya}{}
        \end{itemize}
    
\item [4] Here one says, “what is the purpose of these methods?” The
answer is, “construing sentences and construing
meanings”.\footnote{\Dalhana{6.65.4}{815} explained this as XXX.}\q{fill
        in}
    
    \item [5-6] There are \diff{two} verses about this:
      
\begin{quote}
The logical methods of the system prevent untrue and unsuitable
statements. They also bring about the validity of one’s own
statements.  And they also clarify meanings that are stated back to
front, that are implicit, unclear and any that are partially stated.
\end{quote}    

    \item [8] Among them, \se{adhikaraṇa}{topic} is the object, concerning which 
    statements are made, such as \se{rasa}{flavour} or \se{doṣa}{humour}. 
    
    \item [9] \se{yoga}{semantic linkage} is that by which a sentence is made cohesive, such as unifying the meanings of words which are stated in a reverse/unusual/different order whether placed close or apart.
    
    \dev{tailaṃ pibeccāmṛtavallinimbahaṃsāhvayāvṛkṣakapippalībhiḥ |\\
    siddhaṃ balābhyāñca sadevadāru hitāya nityaṃ galagaṇḍaroge ||\\}
        ``For a benefit in the disease of goitre, one should always drink cooked sesame oil with heart-leaved moonseed (Tinospora cordifolia), neem, the plant walking maidenhair fern (Adiantum lunulatum or Adiantum philippense), long pepper, heart-leaf sida, country marrow', and deodar.'' 
    
    In this verse, instead of saying first \dev{siddhaṃ pibet}... 'one should drink cooked...' the word \emph{siddha} `cooked' is used in the second hemistich. Connecting this way the meanings of words that are placed apart is also semantic linkage.  
     
    
    \item [10] The meaning that is conveyed in an \se{sūtra}{aphorism} or a word is called \se{padārtha}{word meaning}. In other words, word meaning is the meaning of one or more words. Word meanings are unlimited. 
    
   Where two or three meanings such as `fat’, `sweat’ or `anointment’ appear to be possible, one should consider the meaning which is valid through semantic linkage with prior and subsequent elements. For example, comprehension is doubtful when it is said that “We are going to explain the chapter on the origin of Veda” concerning that the origin of which Veda is to be stated. \emph{Sāmaveda} and so on are the Vedas. With regard to prior and subsequent elements, the roots \emph{vind} and \emph{vid} have the same meaning. Later a word appears that clarifies that he wants to talk about the origin of \emph{Āyurveda}. Hence, it (āyurveda) is the word meaning.  
    
    \item [11] The statement that serves as proof of an argument is the \se{hetvartha}{premise}. For example, as a lump of earth is moist by water, the same way a wound becomes moist by substances like milk with black grams and so on.  
    
    \item [12] A brief statement is called \se{samuddeśa}{mention}, such as \se{śalya} {pain-causing agent}. 
    
    \item [13] A detailed statement is \se{nirdeśa}{description}, for example, the pain-causing agents are endogenous or exogenous. 
    
    \item [14] A statements like ``it should be this way'' is '\se{upadeśa}{prescription}. For example, one should not stay awake at night; you should not sleep during daytime.  
    
    \item [15] A statement like “this happens because of this” is a \se{apadeśa}{statement of reason}. For example, in the sentence “Phlegm increases by sweet substances”, the reason is stated.  
    
    \item [16] Substantiation of the subject matter through past evidence is \se{pradeśa}{indication}. For example, he removed the pain-causing substance from Devadatta so he can do it from Yajñadatta. 
    
    \item [17] Substantiation of the subject matter through future event is \se{atideśa}{prediction}. For example, if his wind goes up he would get colic by that.  
    
    
    
\end{translation}
