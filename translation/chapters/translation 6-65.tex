% !TeX root = incremental_SS_Translation.tex
% Deepro

\chapter{Uttaratantra \diff{65}:  Rules of Interpretation}

\section{Literature} 

Meulenbeld offered an annotated overview of this chapter and a bibliography
of earlier scholarship to 2002.\fvolcite{IA}[331]{meul-hist}  Earlier explorations 
of this topic include \cite{dasg-1952,
lele-1981,
mejo-2000,
nara-1949,
ober-1967,
scha-1993,
sing-2003,
muth-1976}. 
\citet{mane-2008} gave examples of the use of tantrayuktis in Buddhist 
commentarial literature.

\subsection{Terminology}



\section{Characteristics of the Manuscript Transmission}

% Deepro

\section{Translation}

\begin{translation}

\item [1] Now we shall explain the chapter called, “the enunciation of the
\se{tantrayukti}{logical methods of the system}.”

\item [3] There are thirty-two logical methods of the system. They are as 
follows: 
\begin{itemize}
\item \se{adhikaraṇa}{topic}
\item \se{yoga}{construing}
\item \se{padārtha}{word meaning}
\item \se{hetvartha}{premise}
\item \se{samuddeśa}{mention}
\item \se{nirdeśa}{description}
\item \se{upadeśa}{prescription}
\item \se{apadeśa}{statement of reason}
\item \se{pradeśa}{indication}
\item \se{atideśa}{prediction}
\item \se{apavarga}{exception}
\item \se{vākyaśeṣa}{ellipis}
\item \se{arthāpatti}{implication}
\item \se{viparyaya}{contraposition}
\item \se{prasaṅga}{recontextualization}
\item \se{ekānta}{invariable statement}
\item \se{anekānta}{variable statement}
\item \se{pūrvapakṣa}{objection}
\item \se{nirṇaya}{determination}
\item \se{anumata}{consent}
\item \se{vidhāna}{itemization}
\item \se{anāgatāpekṣaṇa}{future reference}
\item \se{atikrāntāpekṣaṇa}{past reference}
\item \se{saṃśaya}{}
\item \se{vyākhyāna}{}
\item \se{svasaṃjñā}{}
\item \se{nirvacana}{}
\item \se{nidarśana}{}
\item \se{niyoga}{}
\item \se{vikalpa}{}
\item \se{samuccaya}{}
\item \se{ūhya}{}
\end{itemize}

\item [4] It is said about this, “what is the purpose of these methods?”
The answer is, “construing sentences and construing
meanings”.\footnote{\Dalhana{6.65.4}{815} explained “construing a
    sentence” as “connecting up a sentence that is not connected,” and
    “construing a meaning” as “clarifying  or making appropriate a meaning
    that is implied or inppropriate.”}

\item [5-6] There are \diff{two} verses about this:
  
\begin{sloka}
The logical methods of the system prohibit
statements employed by people who do not speak the truth.
%untrue and unsuitable statements. 
They also bring about the validity of one’s own
statements.  And they also clarify meanings that are stated back to
front, that are implicit, unclear and any that are partially stated.
\end{sloka}

\item [8] Among them, “\se{adhikaraṇa}{topic}” refers to the object, with 
reference to which statements are made, such as \se{rasa}{flavour} or 
\se{doṣa}{humour}.\footnote{The idea here is that “\emph{rasa}” may be the 
topic of a chapter, and statements in that chapter are all understood to be about 
that topic}

\item [9] “\se{yoga}{Construing}” is that by which a sentence
is construed, as when words that are in a reversed order, whether placed close
or apart, have their meanings unified.
%\dev{tailaṃ pibeccāmṛtavallinimbahaṃsāhvayāvṛkṣakapippalībhiḥ |\\
%siddhaṃ balābhyāñ ca sadevadāru hitāya nityaṃ galagaṇḍaroge ||\\}
\begin{quote}
    Sesame oil he should drink, with 
\gls{amṛtavalli}, 
\gls{nimba},
\diff{\gls{haṃsāhvayā}},
\gls{vṛkṣaka}, and
\gls{pippalī}
%with heart-leaved moonseed (Tinospora cordifolia), neem, the plant walking 
%maidenhair fern (Adiantum lunulatum or Adiantum philippense), long pepper, 
%heart-leaf sida, country marrow', and deodar.'' 
\\[2ex]
that is cooked with 
\gls{balā} and \gls{atibalā}, %two mallows
and
\gls{devadāru},
always for a benefit in the case of the disease goitre.
\end{quote}
In this verse, one ought to say, firstly, “one should drink
cooked\ldots.” However, the word “cooked” is used in the second
line.\footnote{The Nepalese version reads \dev{dvitīye pāde} which would
    properly mean the second quarter of the first line; the vulgate reads
    “third quarter” which is more correct.} Unifying the meanings of words in
    this way, even though they are far apart, is construing. 
 

\item [10] The meaning that is conveyed in an \se{sūtra}{aphorism} or a word is 
called \se{padārtha}{word meaning}. In other words, word meaning is the 
meaning of one or more words. Word meanings are unlimited. 

   Where two or three meanings such as `fat’, `sweat’ or `anointment’ appear to be possible, one should consider the meaning which is valid through semantic linkage with prior and subsequent elements. For example, comprehension is doubtful when it is said that “We are going to explain the chapter on the origin of Veda” concerning that the origin of which Veda is to be stated. \emph{Sāmaveda} and so on are the Vedas. With regard to prior and subsequent elements, the roots \emph{vind} and \emph{vid} have the same meaning. Later a word appears that clarifies that he wants to talk about the origin of \emph{Āyurveda}. Hence, it (āyurveda) is the word meaning.  

\item [11] The statement that serves as proof of an argument is the 
\se{hetvartha}{premise}. For example, as a lump of earth is moist by water, the 
same way a wound becomes moist by substances like milk with black grams and 
so on.  

\item [12] A brief statement is called \se{samuddeśa}{mention}, such as 
\se{śalya} {pain-causing agent}. 

\item [13] A detailed statement is \se{nirdeśa}{description}. For example, the 
pain-causing agents are endogenous or exogenous. 

\item [14] \se{upadeśa}{Prescription} refers to statements like ``it should be this way''. For example, one should not stay awake at night; 
you should not sleep during daytime.  

\item [15] \se{upadeśa}{prescription} refers to statements like “this happens because of this”. For example, in the sentence “Phlegm 
increases by sweet substances”, the reason is stated.  

\item [16] Substantiation of the subject matter through past evidence is \se{pradeśa}{indication}. 
For example, he removed the pain-causing substance 
from Devadatta so he can do it from Yajñadatta. 

\item [17] Substantiation of the subject matter through future event is 
\se{atideśa}{prediction}. For example, if his wind goes up he would get colic by that. 

\item [18] A deviation after generalization is \se{apavarga}{exception}. For example, those afflicted by poison should not go through sudorific treatment other than the cases of poisoning by urinary worms.

\item [19] \se{vākyaśeṣa}{Ellipsis} refers to an unstated word that completes a sentence. For example, despite not mentioning the word 'person', when mentioning someone as 'the one having a head, hands, feet, flanks, and abdomen,' it's apparent that the reference is to a person. 

\item [20] \se{Implication}{arthāpatti} refers to an unstated idea that becomes evident through context. For example, when one said, “We will eat rice” it becomes evident from the context that he did not wish to drink gruel. 

\item [21] When there is the reversal of it it is \se{viparyaya}{contraposition}. For example, when it is said, "Weak, dyspneic, and fearful people are difficult to treat," the converse holds true: "Those who are strong and so on are easily treatable." 

\item [22] \se{prasaṅga}{Recontextualization} refers to a concept common to another section. For example, a concept belonging to another section is brought up by mentioning it repeatedly throughout. 

\item [23] \se{ekānta}{Invariable statement} is one that is stated with certainty. For example, \gls{trivṛt} causes purgation; \gls{madana} induces vomiting.

\item [24] \se{anekānta}{Variable statement} is one that is true in one way in some cases and in another way elsewhere. For example, Some teachers identify the main element as substance, others as fluid, some as semen, and some as digestion. 

\item [25] That which is mentioned undoubtedly is \se{pūrvapakṣa}{objection}. (???) For example, how are the four types of diabetes caused by wind inclurable?

\item [26] Its answer is determination. For example, aflicting the body and trickling downwards, it creates urine mixed with fat, fatty tissues, and marrow. Thus, those caused by wind are incurable. 

\item [28] \se{anumata}{Consent} refers to others' opinion that is not rejected. For example, when the assertor says that there are six flavours and that somehow gets accepted with affirmation, it is termed consent.

\item [29] \se{vidhāna}{Itemization} refers to sequentially ordered statements within a chapter. For example, the eleven lethal points of thigh are mentioned sequentially in a chapter.

\item [30] A statement like “Thus will be stated” is \se{anāgatāpekṣaṇa}{future reference} such as when one says in the Sūtrasthāna, “I will mention it in the book of “Cikitsā”. 

\item [31] A statement like “Thus has been stated” is \se{atikrāntāpekṣaṇa}{past reference} such as when one says in the Cikitsāsthāna, "As mentioned in the Sūtrasthāna...". 

\end{translation}
