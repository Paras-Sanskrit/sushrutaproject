% !TeX root = incremental_SS_Translation.tex
% Deepro

\chapter{Uttaratantra \diff{65}:  Rules of Interpretation}

\section{Literature} 

Meulenbeld offered an annotated overview of this chapter and a bibliography
of earlier scholarship to 2002.\fvolcite{IA}[331]{meul-hist}  Earlier explorations 
of this topic include \cite{dasg-1952,
lele-1981,
mejo-2000,
nara-1949,
ober-1967,
scha-1993,
sing-2003,
muth-1976}. 

% \citet{sche-term} discussed the term \emph{yukti} in Buddhist
%literature; see also \cite[444--446]{biar-1964} \cite[343--345]{pret-1991}, while

\citet[105--106, fn.\,109]{prei-2013} provided further references to the
discussion of \emph{yukti} in Buddhist literatures. \citet{mane-2008}
gave examples of the use of tantrayuktis in Buddhist commentarial
literature.

% Meulenbeld HIML IB p.431, note 960 refers to Ruben 1926

\section{Early Sources}

An ancient tradition of enumerating the \textit{tantrayukti}s served as a foundational source not only for medical texts but also for works in various other disciplines, including Arthaśāstra, philosophy, and even grammar. The \textit{Suśruta Saṃhitā} stands as the earliest Āyurvedic text that presents a compilation of a list of \textit{tantrayukti}s followed by their definitions and usage. Mentions to Tantrayuktis are also found in the \Ca{8.12}{} which introduce four additional \textit{tantrayukti}s. However, the \textit{tantrayukti}s remain undefined in the \textit{Caraka Saṃhitā}. The enumeration and definitions of the Tantrayuktis in the \textit{Suśruta Saṃhitā} closely parallel their treatment in the \textit{Arthaśāstra}. For a side-by-side comparison of the Tantrayuktis in the Suśruta Saṃhitā and the Arthaśāstra, please refer to Table \ref{tableSvsA}.


	
\begin{longtable}{m{.25\textwidth} m{.25\textwidth} p{.5\textwidth}}
    
\caption{Tantrayuktis in \textit{Suśruta Saṃhitā} (S) 
    and \textit{Arthaśāstra} (A)} 
    \label{tableSvsA}\\
  	\toprule
    Sequence & Terms	& Definitions \\
	\midrule
	\endfirsthead
    
    \toprule
    Sequence & Terms	& Definitions \\
    \midrule
    \endhead
    
    
	%\rule{0pt}{0.5cm}
    
    (S) 1. & \textit{adhikaraṇa} & \dev{tatra yamarthamadhikṛtyocyate 
    tadadhikaraṇam/} \\
	(A) 1. & \textit{adhikaraṇa} & \dev{yamarthamadhikṛtyocyate tadadhikaraṇa/} \\
	
	\rule{0pt}{0.5cm}(S) 2. & \textit{yoga} & \dev{yena vākyaṃ yujyate sa yogaḥ/} \\
	(A) 3. & \textit{yoga} & \dev{vākyayojanā yoga/} \\
	
	\rule{0pt}{0.5cm}(S) 3. & \textit{padārtha} & \dev{yo'rtho'bhihitaḥ sūtre pade vā sa padārthaḥ/ padasya padayoḥ padānāṃ vā yo 'rthaḥ sa padārthaḥ/ aparimitāś ca padārthāḥ/} \\
	(A) 4. & \textit{padārtha} & \dev{padāvadhikaḥ padārthaḥ/} \\
	
	\rule{0pt}{0.5cm}(S) 4. & \textit{hetvartha} & \dev{yaduktaṃ sādhanaṃ bhavati sa hetvarthaḥ/} \\
	(A) 5. & \textit{hetvartha} & \dev{heturarthasādhako hetvarthaḥ/} \\
	
	\rule{0pt}{0.5cm}(S) 5. & \textit{uddeśa / samuddeśa} & \dev{samāsavacanaṃ samuddeśaḥ/} \\
	(A) 6. & \textit{uddeśa} & \dev{samāsavākyamuddeśaḥ/} \\
	
	\rule{0pt}{0.5cm}(S) 6. & \textit{nirdeśa} & \dev{vistaravacanaṃ nirdeśaḥ/} \\
	(A) 7. & \textit{nirdeśa} & \dev{vyāsavākyaṃ nirdeśaḥ/} \\
	
	\rule{0pt}{0.5cm}(S) 7. & \textit{upadeśa} & \dev{evamityupadeśaḥ/} \\
	(A) 8. & \textit{upadeśa} & \dev{evaṃ vartitavyamityupadeśaḥ/} \\
	
	\rule{0pt}{0.5cm}(S) 8. & \textit{apadeśa} & \dev{anena kāraṇenetyapadeśaḥ/} \\
	(A) 9. & \textit{apadeśa} & \dev{evamasāvāhetyapadeśaḥ/} \\
	
	\rule{0pt}{0.5cm}(S) 9. & \textit{pradeśa} & \dev{prakṛtasyātikrāntena sādhanaṃ pradeśaḥ/} \\
	(A) & \textit{predeśa} & \dev{/} \\
	
	\rule{0pt}{0.5cm}(S) 10. & \textit{atideśa} & \dev{/} \\
	(A) 10. & \textit{atide;sa} & \dev{/} \\
	
	\rule{0pt}{0.5cm}(S) 11. & \textit{apavarga} & \dev{/} \\
	(Aa 22. & \textit{apavarga} & \dev{/} \\
	
	\rule{0pt}{0.5cm}(S) 12. & \textit{vākyaśeṣa} & \dev{/} \\
	(A) 17. & \textit{vākyaśeṣa} & \dev{/} \\
	
	\rule{0pt}{0.5cm}(S) --- & \textit{---} & \dev{---/} \\
	(A) 12. & \textit{upamāna} & \dev{/} \\
	
	\rule{0pt}{0.5cm}(S) 13. & \textit{arthāpatti} & \dev{/} \\
	(A) 13. & \textit{arthāpatti} & \dev{/} \\
	
	\rule{0pt}{0.5cm}(S) 14. & \textit{viparyaya} & \dev{/} \\
	(A) 16. & \textit{viparyaya} & \dev{/} \\
	
	\rule{0pt}{0.5cm}(S) 15. & \textit{prasaṅga} & \dev{/} \\
	(A) 15. & \textit{prasaṅga} & \dev{/} \\
	
	\rule{0pt}{0.5cm}(S) 16. & \textit{ekānta} & \dev{/} \\
	(A) 26. & \textit{ekānta} & \dev{/} \\
	
	\rule{0pt}{0.5cm}(S) 17. & \textit{anekānta} & \dev{/} \\
	(A)-- & \textit{--} & \dev{--/} \\
	
	\rule{0pt}{0.5cm}(S) 18. & \textit{pūrvapakṣa} & \dev{/} \\
	(A) 24. & \textit{pūrvapakṣa} & \dev{/} \\
	
	\rule{0pt}{0.5cm}(S) 19. & \textit{nirṇaya} & \dev{/} \\
	(A) 25. & \textit{uttarapakṣa} & \dev{/} \\
	
	\rule{0pt}{0.5cm}(S) 20. & \textit{anumata} & \dev{/} \\
	(A)18. & \textit{anumata} & \dev{/} \\
	
	\rule{0pt}{0.5cm}(S) 21. & \textit{vidhāna} & \dev{/} \\
	(A) 2. & \textit{vidhāna} & \dev{/} \\
	
	\rule{0pt}{0.5cm}(S) 22. & \textit{anāgatāpekṣaṇa} & \dev{/} \\
	(A) 27. & \textit{anāgatāvekṣaṇa} & \dev{/} \\
	
	\rule{0pt}{0.5cm}(S) 23. & \textit{atikrāntāpekṣaṇa} & \dev{/} \\
	(A) 28. & \textit{atikrāntāvekṣaṇa} & \dev{/} \\
	
	\rule{0pt}{0.5cm}(S) 24. & \textit{saṃśaya} & \dev{/} \\
	(A) 14. & \textit{saṃśaya} & \dev{/} \\
	
	\rule{0pt}{0.5cm}(S) 25. & \textit{vyākhyāna} & \dev{/} \\
	(A) 19. & \textit{vyākhyāna} & \dev{/} \\
	
	\rule{0pt}{0.5cm}(S) 26. & \textit{svasaṃjñā} & \dev{/} \\
	(A) 23. & \textit{svasaṃjñā} & \dev{/} \\
	
	\rule{0pt}{0.5cm}(S) 27. & \textit{nirvacana} & \dev{/} \\
	(A) 20. & \textit{nirvacana} & \dev{/} \\
	
	\rule{0pt}{0.5cm}(S) 28. & \textit{nidarśana} & \dev{/} \\
	(A) 21. & \textit{nidarśana} & \dev{/} \\
	
	\rule{0pt}{0.5cm}(S) 29. & \textit{niyoga} & \dev{/} \\
	(A) 29. & \textit{niyoga} & \dev{/} \\
	
	\rule{0pt}{0.5cm}(S) 30. & \textit{vikalpa} & \dev{/} \\
	(A) 30. & \textit{vikalpa} & \dev{/} \\
	
	\rule{0pt}{0.5cm}(S) 31. & \textit{samuccaya} & \dev{/} \\
	(A) 31. & \textit{samuccaya} & \dev{/} \\
	
	\rule{0pt}{0.5cm}(S) 32. & \textit{ūhya} & \dev{/} \\
	(A) & \textit{ūhya} & \dev{/} \\
			
	\bottomrule



\end{longtable}

	


\section{Terminology}



\section{Characteristics of the Manuscript Transmission}

% Deepro

\section{Translation}

\begin{translation}

\item [1] Now we shall explain the chapter called, “the enunciation of the
\se{tantrayukti}{logical methods of the system}.”

\item [3] There are thirty-two logical methods of the system. They are as 
follows: 
\smallskip


\begin{multicols}{2}
    \raggedright
\begin{enumerate}
\item \se{adhikaraṇa}{topic}
\item \se{yoga}{construing}
\item \se{padārtha}{word meaning}
\item \se{hetvartha}{premise}
\item \se{samuddeśa}{mention}
\item \se{nirdeśa}{description}
\item \se{upadeśa}{prescription}
\item \se{apadeśa}{statement of reason}
\item \se{pradeśa}{indication}
\item \se{atideśa}{prediction}
\item \se{apavarga}{exception}
\item \se{vākyaśeṣa}{ellipis}
\item \se{arthāpatti}{implication}
\item \se{viparyaya}{contraposition}
\item \se{prasaṅga}{recontextualization}
\item \se{ekānta}{invariable statement}
\item \se{anekānta}{variable statement}
\item \se{pūrvapakṣa}{objection}
\item \se{nirṇaya}{determination}
\item \se{anumata}{consent}
\item \se{vidhāna}{itemization}
\item \se{anāgatāpekṣaṇa}{future reference}
\item \se{atikrāntāpekṣaṇa}{past reference}
\item \se{saṃśaya}{doubt}
\item \se{vyākhyāna}{explication}
\item \se{svasaṃjñā}{field-specific term}
\item \se{nirvacana}{interpretation}
\item \se{nidarśana}{illustration}
\item \se{niyoga}{compulsion}
\item \se{vikalpa}{option}
\item \se{samuccaya}{aggregation}
\item \se{ūhya}{deducible}
\end{enumerate}
\end{multicols}
\bigskip

\item [4] It is said about this, “what is the purpose of these methods?”
The answer is, “construing sentences and construing
meanings”.\footnote{\Dalhana{6.65.4}{815} explained “construing a
    sentence” as “connecting up a sentence that is not connected,” and
    “construing a meaning” as “clarifying  or making appropriate a meaning
    that is implied or inppropriate.”}

\item [5-6] There are \diff{two} verses about this:
  
\begin{sloka}
The logical methods of the system prohibit
statements employed by people who do not speak the truth.
%untrue and unsuitable statements. 
They also bring about the validity of one’s own
statements.  And they also clarify meanings that are stated back to
front, that are implicit, unclear and any that are partially stated.
\end{sloka}

\item [8] Among them, “\se{adhikaraṇa}{topic}” refers to the object, with 
reference to which statements are made, such as \se{rasa}{flavour} or 
\se{doṣa}{humour}.\footnote{The idea here is that “\emph{rasa}” may be the 
topic of a chapter, and statements in that chapter are all understood to be about 
that topic}

\item [9] “\se{yoga}{Construing}” is that by which a sentence
is construed, as when words that are in a reversed order, whether placed close
or apart, have their meanings unified.
%\dev{tailaṃ pibeccāmṛtavallinimbahaṃsāhvayāvṛkṣakapippalībhiḥ |\\
%siddhaṃ balābhyāñ ca sadevadāru hitāya nityaṃ galagaṇḍaroge ||\\}
\begin{quote}
    Sesame oil he should drink, with 
\gls{amṛtavalli}, 
\gls{nimba},
\diff{\gls{haṃsāhvayā}},
\gls{vṛkṣaka}, and
\gls{pippalī}
%with heart-leaved moonseed (Tinospora cordifolia), neem, the plant walking 
%maidenhair fern (Adiantum lunulatum or Adiantum philippense), long pepper, 
%heart-leaf sida, country marrow', and deodar.'' 
\\[2ex]
that is cooked with 
\gls{balā} and \gls{atibalā}, %two mallows
and
\gls{devadāru},
always for a benefit in the case of the disease goitre.
\end{quote}
In this verse, one ought to say, first, “one should drink
cooked\ldots.” However, the word “cooked” is used in the second
line.\footnote{The Nepalese version reads \dev{dvitīye pāde} which would
    properly mean the second quarter of the first line; the vulgate reads
    “third quarter” which seems more correct.} Unifying the meanings of words in
    this way, even though they are far apart, is construing. 
 

\item [10] 

The meaning that is conveyed in an \se{sūtra}{aphorism} or a word is 
called \se{padārtha}{word-meaning}. In other words, word-meaning is the 
meaning of one or more words. Word-meanings are unlimited. 

Where two or three meanings such as `fat,’ `sweat’ or `anointment’ appear
to be possible, the valid meaning is the one that construes with prior
and subsequent elements.\footnote{There is a dangling relative clause,
    \dev{yo 'rthaḥ}, in the Nepalese version that is avoided in the vulgate
    recension by the addition of \dev{sa grahītavyaḥ}.} For example, when it
    is said that, “We are going to explain the chapter on the
    \emph{veda}-origin” the mind may be confused about which “\emph{veda}”
    will be spoken about. \emph{Sāmaveda} and so on are the Vedas. Taking
    note of the prior and subsequent elements, the two roots \emph{vind}
    ”find” and \emph{vid} “know” have a single meaning. Subsequently, the
    understanding takes place that there is a wish to talk about the origin
    of āyurveda.  So that is the meaning of the word.\footnote{The Nepalese
        text here is hard to follow, and the vulgate has a significantly
        different reading. But the problem situation seems to be as follows.  The
        \SS\ opens with a statement saying that it will describe the “origin of
        the \emph{veda}” (\emph{vedotpatti}).  The problem is, what does this
        word “\emph{veda}” refer to?  Is it the Veda, as in Sāmaveda?  Or
        something derived from the roots \root vind or \root vid?  Context
        (“prior and subsequent elements”) can help us to know that “\emph{veda}”
        means only “\emph{āyurveda}” and that the \SS\ is talking about the
        origin of ayurveda, specifically.  This same issue is also addressed by 
        Ḍalhaṇa at \Su{1.1.1}{1}.}

\item [11] \se{hetvartha}{The sense of the cause} is a statement that is
a \se{sādhana}{premiss}.  For example, just as a lump of earth is
moistened by water, so a wound is moistened by substances like milk with
\gls{māṣa}.\footnote{The way this principle is expressed here seems to be
    describing the application of a general principle (water makes things
    wet) to a specific context. We can know the moistening of a wound because
    we know the more general case of moistening earth. However,
    etymologically, \dev{hetvartha} does not mean “analogy,” but rather,
    something like “purpose of the reason.”  The phrase “the sense of cause”
    that we have used leans on the use of the term in commentaries on the
    \emph{Aṣṭādhyāyī} (\emph{Kaumudī} on 2.3.23). The vulgate of the \SS\
    rewrites the principle, making it clearer that the principle means
    “clarification by analogy.”  Cf.\ also Cakrapāṇi's discussion at
    \Ca{Si.12.41}{736}, where he explained the principle as using an
    explanation from one situation to clarify another situation. Cf.\
    \emph{Arthaśāstra} 5.1.13 \citep[436]{oliv-2013}, which is also unclear.
    }\q{See also Ḍalhaṇa at \Su{1.1.1}{1}}
\item [12] A \se{samuddeśa}{mention} is a brief statement such as 
“\se{śalya}{spike}”.\footnote{Generally, \dev{śalya} refers to any painful 
foreign body embedded in the flesh that requires surgical removal.}

\item [13] A \se{nirdeśa}{description} is a detailed statement. For example, “in 
the body or exogenous”.\footnote{This is a reference to \Su{1.26.4}{121}
    where \dev{śalya} is described in more detail as being of two kinds.} 

\item [14] “\se{upadeśa}{Prescription}” refers to statements like ``it should be 
this way.'' For example, one should not stay awake at night; 
one should not sleep during the day.  

\item [15] “\se{apadeśa}{Statement of reason}” refers to statements like “this 
happens because of this.” For example, in the sentence “Sweet substances 
increase phlegm,” the reason is stated.\footnote{A techical term also in 
Nyāyaśāstra \citep[54]{jhal-1978}.}  

\item [16] Substantiation of the subject matter through past evidence is
“\se{pradeśa}{indication}.” For example, he pulled out Devadatta's
\se{śalya}{splinter}, therefore he will pull out Yajñadatta's.

\item [17] Substantiation of the subject matter through a future event is
“\se{atideśa}{prediction}.” For example, if his wind moves upwards, that
will cause him to have colic.”\footnote{A techical term also in
    Nyāyaśāstra \citep[6--7]{jhal-1978}.}

% got to here with DW

\item [18] A deviation after generalization is \se{apavarga}{exception}. For example, those afflicted by poison should not go through sudorific treatment other than the cases of poisoning by urinary worms.

\item [19] \se{vākyaśeṣa}{Ellipsis} refers to an unstated word that completes a sentence. For example, despite not mentioning the word 'person', when mentioning someone as 'the one having a head, hands, feet, flanks, and abdomen,' it's apparent that the reference is to a person. 

\item [20] \se{Implication}{arthāpatti} refers to an unstated idea that becomes evident through context. For example, when one said, “We will eat rice” it becomes evident from the context that he did not wish to drink gruel. 

\item [21] When there is the reversal of it it is \se{viparyaya}{contraposition}. For example, when it is said, "Weak, dyspneic, and fearful people are difficult to treat," the converse holds true: "Those who are strong and so on are easily treatable." 

\item [22] \se{prasaṅga}{Recontextualization} refers to a concept common to another section. For example, a concept belonging to another section is brought up by mentioning it repeatedly throughout. 

\item [23] \se{ekānta}{Invariable statement} is one that is stated with certainty. For example, \gls{trivṛt} causes purgation; \gls{madana} induces vomiting.

\item [24] \se{anekānta}{Variable statement} is one that is true in one way in 
some cases and in another way elsewhere. For example, some teachers identify 
the main element as substance, others as fluid, some as semen, and some as 
digestion.\q{See chapter 40 of Sūtrasthāna.}

\item [25]  A \se{pūrvapakṣa}{first point of view} is something stated
with certainty.  For example, how are the four types of diabetes caused
by wind incurable?\footnote{The adverb \dev{niḥsaṃśayam} is problematic:
    the example expresses a query or doubt, the opposite of certainty, which
    is answered in the next passage.  It would seem to make more sense to
    read something like \dev{yas tu sasaṃśayam abhidhīyate sa pūrvapakṣaḥ},
    but our manuscripts are unanimous in their reading.}

\item [26] Its answer is determination. For example, afflicting the body and 
trickling downwards, it creates urine mixed with fat, fatty tissues, and 
marrow.\q{vasā / medas / majjan} Thus, those caused by wind are incurable. 

\item [28] \se{anumata}{Consent} refers to others' opinion that is not rejected. For example, when the assertor says that there are six flavours and that somehow gets accepted with affirmation, it is termed consent.

\item [29] \se{vidhāna}{Itemization} refers to sequentially ordered statements within a chapter. For example, the eleven lethal points of thigh are mentioned sequentially in a chapter.

\item [30] A statement like “Thus will be stated” is \se{anāgatāpekṣaṇa}{future 
reference} such as when he says in the \emph{Sūtrasthāna}, “I will mention it in 
the \emph{Cikitsāsthāna}.” 

\item [31] A statement like “Thus has been stated” is 
\se{atikrāntāpekṣaṇa}{past reference} such as when one says in the 
\emph{Cikitsāsthāna}, “As mentioned in the \emph{Sūtrasthāna}\ldots.” 

\item [32] An indication pointing to causes on both sides is
\se{saṃśaya}{doubt}. For example, a blow to
\sse{talahṛdaya}{sole-heart}\footnote{\dev{talahṛdaya} is one of the
    muscle-group of lethal points mentioned in \Su{3.6.7}{370}.}  is fatal,
    whereas cutting hands and feet is not fatal.

\item [33] An elaborate description is \se{vyākhyāna}{explication}. For example, 
the twenty-fifth entity, \sse{puruṣa}{person}, is being explicated here. Thus, no 
other Āyurvedic texts discuss  entities beginning with matters. \q{Does 
bhūtādi a compound or it means ahaṅkāra or ego?}

\item [34] \se{svasaṃjñā}{Field-specific term} is uncommon in other field of 
studies. The term used in one's own systems is called field-specific term, such as 
in this system, \sse{mithuna}{pair} denotes honey and ghee, and 
\sse{mithuna}{triad} denotes ghee, sesame oil and fat. 

\item [35] A customary potrayal is \se{nirvacana}{interpretation}. For example, one goes along the shade fearing heat. 

\item [36] Providing examples is \se{nidarśana}{illustration}. For example, just as fire spreads rapidly in a dry forest when accompanied by wind, a wound intensifies affected by wind, bile, and phlegm.  

\item [37] A statement like “This is the only way...” ...\se{niyoga}{compulsion}. For example, one should consume only a healthy diet.     

\item [39] A statement like “This and this\ldots” is
\se{vikalpa}{option}. For example, in the section on meat, the major ones
are blackbuck, deer, quail and partridge.\footnote{The example here
    matches \dev{samuccaya} (next text), not \dev{vikalpa}.  There seems to
    have been a metathesis of terms. \citet[1005, footnote
    6]{susr-trikamji1945} notes that this text and the next have been swapped
    in the Calcutta edition that includes Hārāṇacandra's commentary
    \volcite{2}{bhat-1910}, in the same way as in the Nepalese version.}

\item [38] A summarized statement is
\se{samuccaya}{aggregation}.\footnote{As stated in the previous footnote, the 
example here is of \dev{vikalpa}, not \dev{samuccaya}.
%    The term \dev{samāsavacana}, which
%   means more or less the same as \dev{saṃkṣepavacana}, has already been
%    used in tantrayukti \dev{samuddeśa}
    } For example, let there be rice
    with meat broth, rice with milk, or burley with ghee.
    
    \begin{quote}
        A meaningful reading of these two rules would be
    
    39 idaṃ vedaṃ veti vikalpaḥ / yathā rasodanaḥ kṣīrodanaḥ saghṛtā vā 
    yavāgūr bhavatv iti //
    
    38 saṃkṣepavacanaṃ samuccayaḥ / yathā māṃsavarge eṇahariṇalāvatittirāḥ 
    pradhānā iti
    \end{quote}
    

\item [40] What is not explicitly stated but can be understood through discernment is \se{ūhya}{deducible}. For example, in the section on rules of foods and drinks, four types of foods and drinks are mentioned— \se{bhakṣya}{masticable}, \se{bhojya}{edible}, \se{lehya}{suckable}, and \se{peya}{drinkable}. Thus, while four types are needed to be stated, two types are actually mentioned. Here it is deducible that in the section on foods and drinks, by specifically mentioning two types, the four types are also mentioned. Furthermore, a masticable item is not excluded from the category of food because it shares the same characteristic of solidity. A suckable item is not excluded from being classified as a drink because it shares the same characteristic of liquidity. Four types of aliments are rare. They are usually just twofold. Therefore, lord Dhanvantari says “Twofold is popular”.   


\end{translation}
