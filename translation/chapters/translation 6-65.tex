% !TeX root = incremental_SS_Translation.tex
% Deepro

\chapter[Uttaratantra 65:  Rules of Interpretation]{Uttaratantra \diff{65}:  
Rules of Interpretation}

\section{Literature} 

Meulenbeld offered an annotated overview of this chapter and a bibliography
of earlier scholarship to 2002.\fvolcite{IA}[331]{meul-hist}  Other explorations 
of this topic include \cite{dasg-1952,
lele-1981,
mejo-2000,
nara-1949,
ober-1968,
scha-1993,
sing-2003,
muth-1976}. 

% \citet{sche-term} discussed the term \emph{yukti} in Buddhist
%literature; see also \cite[444--446]{biar-1964} \cite[343--345]{pret-1991}, while

\cite{frau-1958} discussed the influence of the \emph{tantrayukti}s in the Sāṅkhya tradition.
\citet[105--106, fn.\,109]{prei-2013} provided further references to the
discussion of \emph{yukti} in Buddhist literatures. \citet{mane-2008}
gave examples of the use of tantrayuktis in Buddhist commentarial
literature. \citet{chev-2009} discusses the translation of the \emph{tantrayukti}s in Tamil literary tradition, with a specific focus on \emph{Tolkāppiyam} and its commentaries.

% Meulenbeld HIML IB p.431, note 960 refers to Ruben 1926

\section{Early Sources}

An ancient tradition of enumerating the \emph{tantrayukti}s served as a 
foundational source not only for medical texts but also for works in various other 
disciplines, including Arthaśāstra, philosophy, and even grammar. The \SS\  stands 
as the earliest Āyurvedic text that presents a compilation of a list of 
\emph{tantrayukti}s followed by their definitions and usage. Mentions to 
Tantrayuktis are also found in the \CS\  \Ca{8.12}{} which introduce four additional 
\emph{tantrayukti}s. However, the \emph{tantrayukti}s remain undefined in the 
\CS. 

\subsection{The \emph{Arthaśāstra}} 
The enumeration and definitions of the \emph{tantrayukti}s in the 
\SS\  closely parallel their treatment in the \emph{Arthaśāstra}. 
\emph{Tantrayukti}s are discussed in the fifteenth and final chapter of the 
\emph{Arthaśāstra}, called the 
\emph{Tantrayukti}.\footnote{\cite[280--283]{kang-1960}} For a side-by-side 
comparison of the \emph{tantrayukti}s in the \SS\ and the \emph{Arthaśāstra}, 
please refer to Table \ref{table-SAV}.

\subsection{The \emph{Yuktidīpikā}} 
\emph{Yuktidīpikā} (circa late sixth to early eighth century), an anonymous 
commentary on Īśvarakṛṣṇa's \emph{Sāṅkhyakārikā}, initiates its discourse with a 
detailed discussion of the characteristics of a scientific treatise, some of which align 
with the \emph{tantrayukti}s.\footnote{See \cite[605--614]{ober-1968} for a 
detailed discussion of the use of the \emph{tantrayukti}s in the 
\emph{Yuktidīpikā}.} 
In the \emph{Yuktidīpikā}, these terms are referred to as \emph{tantraguṇa} or \emph{tantrasampat}. They are: (1) \emph{sūtropapatti} (2) \emph{pramāṇopapatti} (3) \emph{avayavopapatti} (4) \emph{anyūnatā} (5) \emph{saṃśayokti} (6) \emph{nirṇayokti} (7) \emph{uddeśa} (8) \emph{nirdeśa} (9) \emph{anukrama} (10) \emph{saṃjñā} and (11) \emph{upadeśa}.\footnote{
	\begin{verse}
		\dev{sūtrapramāṇāvayavopapattiranyūnatā saṃśayanirṇayoktiḥ/\\
		uddeśanirdeśamanukramaśca saṃjñopadeśāviha tantrasampat//\\}
	\end{verse}
\cite[3]{wezl-1998}}
Apart from these, the \emph{Yuktidīpikā} also exemplifies (12) \emph{utsarga} (general rule), (13) \emph{apavāda} (exception), and (14) \emph{atideśa} (extended application). However, \emph{utsarga} and \emph{apavāda} are not considered \emph{tantrayukti}s in other comprehensive lists. The \emph{Yuktidīpikā} further states that while other \emph{tantrayukti}s can be demonstrated in a similar manner, since they are peripheral topics, the text does not delve into their discussion.\footnote{\dev{evamprakārā anye'pi drastavyāh/ tadyathotsargo'pavādo'tideśa 
		ityādi/...ityevamanyā api tantrayuktayaḥ śakyā iha pradarśayitum/ atiprasaṅgastu prakṛtaṃ tirodadhātīti nivartyate/ siddhaṃ tantrayuktīnāṃ sambandhopapattestantram idam iti/}.\cite[8]{wezl-1998}} 

\subsection{Tamil literature}
Discussions on the \emph{tantrayukti}s are also found in Tamil technical literature, the earliest of which is the \emph{Tolkāppiyam}.\footnote{For a detailed discussion of the treatment of the \emph{tantrayukti}s in the \emph{Tolkāppiyam} see \cite{chev-2009}.} 
A list of 32 \emph{tantrayukti}s, called \emph{utti} or \emph{tantiravutti} in Tamil, are given in the 27\textsuperscript{th} (the final) chapter titled \emph{Marapiyal} “Chapter on conventions” of the last book called \emph{Poruḷ} “Matters” of the \emph{Tolkāppiyam}. There is no consensus regarding the dating of the \emph{Tolkāppiyam}. However, if we endorse Zvelebil's view, which posits that the final redaction of the \emph{Tolkāppiyam} occurred around the fifth century AD, it follows that this section of the \emph{Tolkappiyam} cannot postdate the fifth century. If we follow the dating of Zvelebil, we can safely argue that by that time, Sanskrit \emph{tantrayukti}s had already been translated into Tamil. Nevertheless, determining the correspondence between specific \emph{tantrayukti}s and Tamil \emph{utti}s poses a challenge. A major factor contributing to this challenge is the disagreement between two commentators of the \emph{Tolkāppiyam}, namely Iḷampūraṇar (11th or 12th century) and Pērāciriyar (possibly 13th century), regarding the interpretation of the list of \emph{utti}s. It is still not clear which list of 32 \emph{tantrayukti}s was before the author of the \emph{Tolkāppiyam}. 

After the \emph{Tolkāppiyam}, several other Tamil texts refer to the \emph{tantrayukti}s. Among them the \emph{Yāpparuṅkalam} (possibly 10th century), the \emph{Vīracoḻiyam} (11th century), \emph{Naṉṉūl} (late 12th or early 13th century), and their commentaries hold significant importance in this context.

\subsection{The \emph{Viṣṇudharmottarapurāṇa}}
The third book of the \emph{Viṣṇudharmattarapurāṇa}, believed to have
been composed between the fifth and seventh centuries, includes a chapter
dedicated to the \emph{tantrayukti}s.\footnote{Adhyāya 6,
    \cite[13--14]{shah-1958}.} Unlike the \emph{Arthaśāstra} and the \SS,
    this chapter lacks illustrative examples of the \emph{tantrayukti}s. The
    chapter lists 32 \emph{tantrayukti}s followed by definitions. Notably,
    the list and definitions given here -- we are using the critical
    edition by Priyabala Shah -- in most cases bear a striking resemblance to
    those found in the \SS. Given the striking alignment between the list and
    definitions of \emph{tantrayukti}s, one could suggest that the
    \emph{Viṣṇudharmottarapurāṇa}'s chapter on \emph{tantrayukti}s likely
    draws directly or indirectly from the \SS\ or from a common
    source. The designations and the order of the \emph{tantrayukti}s in the
    \emph{Viṣṇudharmottarapurāṇa} are almost identical. The only differences
    in the order are as follows:

\begin{enumerate}

	\item \emph{Viparyaya} is placed after \emph{vidhāna} whereas in the \SS\  it 
	follows \emph{arthāpatti}.
	
	\item \emph{Anumata} is placed after \emph{vyākhyāna} whereas in the \SS\  
	follows \emph{niṛṇaya}.

	\item \emph{Anāgatāvekṣaṇa} (\emph{anāgatāpekṣaṇa} in the Nepalese 
	version) occurs after \emph{atikrāntāvekṣaṇa} (\emph{atikrāntāpekṣaṇa} in the 
	Nepalese version) whereas the order is reverse in the \SS.
	
\end{enumerate}

For a side-by-side comparison of the \emph{tantrayukti}s in the \SS\  and the 
\emph{Viṣṇudharmottarapurāṇa}, please refer to Table \ref{table-SAV}.


\subsection{The \emph{Saddanīti}}
A list of the 32 \emph{tantrayukti}s accompanied by definitions also appear in the final chapter (\emph{Pariccheda} 28) of the final book (book 3: \emph{Suttamālā}) of the renowned Pali grammar \emph{Saddanīti} composed by Aggavaṃsa in Arimaddanapura (modern Bagan, Burma) in the twelfth-century.\footnote{\cite[920--921]{smit-1930}.} Just as the \emph{Viṣṇudharmottarapurāṇa}, this list also does not provide examples of the \emph{tantrayukti}s. Although written in Pali, the order and the definition of the \emph{tantrayukti}s (\emph{tantiyutti} in Pali) closely resemble those of the \SS. There are, however, a few differences:

	\begin{enumerate}

\item The \emph{tantrayukti} \emph{pradeśa} is referred to as \emph{paṭidesa} (Sanskrit \emph{pratideśa}) and is positioned after \emph{atidesa} (Sanskrit \emph{atideśa}) whereas in the \SS it follows \emph{apadeśa}. 

\item \emph{Atikrāntāpekṣaṇa} is designated as \emph{atītāpekkhana} (Sanskrit \emph{atītāpekṣaṇa}).

\item \emph{Svasaṃjñā} is designated as \emph{anaññā sakasaṃjñā} (Sanskrit \emph{ananyā svasaṃjñā}) and is defined with subtle variations.

\item \emph{Ūhya} is designated as \emph{upānīya}. 
	
	\end{enumerate}

For a side-by-side comparison of the \emph{tantrayukti}s in the \emph{Suśruta Saṃhitā} and the \emph{Saddanīti}, please refer to Table \ref{table-SAV}.




\begin{longtable}{r@{\,}r
		@{\quad\quad}
		m{.25\textwidth} 
		p{.5\textwidth}}
	
	\caption{Tantrayuktis in \SS\  (S), \emph{Viṣṇudharmottarapurāṇa} (V), \emph{Arthaśāstra} (A), and \emph{Saddanīti} (N)} 
	
	\label{table-SAV}\\
	\toprule
	\multicolumn{2}{l}{Sequence} & Terms	& Definitions \\
	\midrule
	\endfirsthead
	
	\toprule
	\multicolumn{2}{l}{Sequence} & Terms	& Definitions \\
	\midrule
	\endhead
	
	
	%\rule{0pt}{0.5cm}
	
	(S) & 1. & \emph{adhikaraṇa} & \dev{tatra yamarthamadhikṛtyocyate 
		tadadhikaraṇam/} \\
	(V) & 1. & \emph{adhikaraṇa} & \dev{tatra yamarthamadhikṛtyocyate 
		tadadhikaraṇam/} \\
	(A) & 1. & \emph{adhikaraṇa} & \dev{yamarthamadhikṛtyocyate 
		tadadhikaraṇa/} \\
	(N) & 1. & \emph{adhikaraṇa} & \dev{tattha yaṃ adhikicca vuccati, taṃ adhikaraṇaṃ/} \\
	
	\rule{0pt}{0.5cm}(S) & 2. & \emph{yoga} & \dev{yena vākyaṃ yujyate sa yogaḥ/ yathā vyatyāsenoktānāṃ sannikṛṣṭaviprakṛṣṭānāṃ padārthānām ekīkaraṇam /} \\
	(V) & 2. & \emph{yoga} & \dev{yena vākyārtho yujyate sa yogaḥ/} \\
	(A) & 3. & \emph{yoga} & \dev{vākyayojanā yogaḥ/} \\
	(N) & 2. & \emph{yoga} & \dev{pubbāparavasena vuttānaṃ sannihitāsannihitānaṃ padānaṃ ekīkaraṇaṃ yogo;/} \\
	
	\rule{0pt}{0.5cm}(S) & 3. & \emph{padārtha} & \dev{yo'rtho'bhihitaḥ sūtre 
		pade vā sa padārthaḥ/ padasya padayoḥ padānāṃ vā yo'rthaḥ sa padārthaḥ/ 
		aparimitāśca padārthāḥ/} \\
	(V) & 3. & \emph{padārtha} & \dev{yo'rtho vidhikṛtaḥ sūtrapade 
		sa padārthaḥ/} \\
	(A) & 4. & \emph{padārtha} & \dev{padāvadhikaḥ padārthaḥ/} \\
	(N) & 3. & \emph{padattha} & \dev{suttapadesu pubbāparayogato yo attho vihito, so padattho/} \\
	
	\rule{0pt}{0.5cm}(S) & 4. & \emph{hetvartha} & \dev{yaduktaṃ sādhanaṃ 
		bhavati sa hetvarthaḥ/} \\
	(V) & 4. & \emph{hetvartha} & \dev{yadanyadyuktimadarthasya sādhanaṃ sa hetvarthaḥ/} \\
	(A) & 5. & \emph{hetvartha} & \dev{heturarthasādhako hetvarthaḥ/} \\
	(N) & 4. & \emph{hetuattha} & \dev{yaṃ vuttatthasādhakaṃ, so hetuattho/} \\
	
	\rule{0pt}{0.5cm}(S) & 5. & \emph{uddeśa / samuddeśa} & \dev{samāsavacanaṃ samuddeśaḥ/} \\
	(V) & 5. & \emph{uddeśa} & \dev{samāsavacanamuddeśaḥ/} \\
	(A) & 6. & \emph{uddeśa} & \dev{samāsavākyamuddeśaḥ/} \\
	(N) & 5. & \emph{uddesa} & \dev{samāsavacanaṃ uddeso/} \\
	
	\rule{0pt}{0.5cm}(S) & 6. & \emph{nirdeśa} & \dev{vistaravacanaṃ 
		nirdeśaḥ/} \\
	(V) & 6. & \emph{nirdeśa} & \dev{vistaravacanaṃ	nirdeśaḥ/} \\
	(A) & 7. & \emph{nirdeśa} & \dev{vyāsavākyaṃ nirdeśaḥ/} \\
	(N) & 6. & \emph{niddesa} & \dev{vitthāravacanaṃ niddeso/} \\
	
	\rule{0pt}{0.5cm}(S) & 7. & \emph{upadeśa} & \dev{evamityupadeśaḥ/} \\
	(V) & 7. & \emph{upadeśa} & \dev{evamevetyupadeśaḥ/} \\
	(A) & 8. & \emph{upadeśa} & \dev{evaṃ vartitavyamityupadeśaḥ/} \\
	(N) & 7. & \emph{upadesa} & \dev{evan ti upadeso/} \\
	
	\rule{0pt}{0.5cm}(S) & 8. & \emph{apadeśa} & \dev{anena kāraṇenetyapadeśaḥ/} \\
	(V) & 8. & \emph{apadeśa} & \dev{anena kāraṇenetyapadeśaḥ/} \\
	(A) & 9. & \emph{apadeśa} & \dev{evamasāvāhetyapadeśaḥ/} \\
	(N) & 8. & \emph{apadesa} & \dev{anena kāraṇenā ti apadeso/} \\
	
	\rule{0pt}{0.5cm}(S) & 9. & \emph{pradeśa} & \dev{prakṛtasyātikrāntena 
		sādhanaṃ pradeśaḥ/} \\
	(V) & 9. & \emph{pradeśa} & \dev{prakṛtasyānāgatena sādhanaṃ pradeśaḥ/} \\
	(A) & 11. & \emph{predeśa} & \dev{vaktavyena sādhanaṃ pradeśaḥ/} \\
	(N) & 10. & \emph{paṭidesa} & \dev{pakatassa anāgatena atthasādhanaṃ paṭideso/} \\
	
	\rule{0pt}{0.5cm}(S) & 10. & \emph{atideśa} & \dev{prakṛtasyānāgatena sādhanam atideśaḥ/} \\
	(V) & 10. & \emph{atideśa} & \dev{atikramaṇena atideśaḥ/} \\
	(A) & 10. & \emph{atideśa} & \dev{uktena sādhanamatideśaḥ/} \\
	(N) & 9. & \emph{atidesa} & \dev{pakatassa atikkantena sādhanaṃ atideso/} \\
	
	\rule{0pt}{0.5cm}(S) & 11. & \emph{apavarga} & 
	\dev{abhipramṛjyāpakarṣaṇamapavargaḥ/} \\
	(V) & 11. & \emph{apavarga} & \dev{abhiprāyānukarṣaṇamapavargaḥ/} \\
	(A) & 22. & \emph{apavarga} & \dev{abhiplutavyapakarṣaṇamapavargaḥ/} \\
	(N) & 11. & \emph{apavagga} & \dev{ativyāpetvā apanayanaṃ apavaggo/} \\
	
	\rule{0pt}{0.5cm}(S) & 12. & \emph{vākyaśeṣa} & \dev{yena 
		padenānuktena vākyaṃ samāpyate sa vākyaśeṣaḥ/} \\
	(V) & 12. & \emph{vākyaśeṣa} & \dev{yenārthaḥ parisamāpyate padenāhāryeṇa sa vākyaśeṣaḥ/} \\
	(A) & 17. & \emph{vākyaśeṣa} & \dev{yena vākyaṃ samāpyate sa 
		vākyaśeṣaḥ/} \\
	(N) & 12. & \emph{vākyadosa} & \dev{yena padena avuttena vākyaparisamāpanaṃ bhavati, so vākyadoso/} \\
	
	\rule{0pt}{0.5cm}(S) \-\- & \-\- & \-\- \\
	(V) \-\- & \-\- & \-\- \\
	(A) & 12. & \emph{upamāna} & \dev{dṛṣṭenādṛṣṭasya sādhanamupamānam/} 
	\\
	(N) \-\- & \-\- & \-\- \\
	
	\rule{0pt}{0.5cm}(S) & 13. & \emph{arthāpatti} & 
	\dev{yadakīrtitamarthādāpadyate sārthāpattiḥ/} \\
	(V) & 13. & \emph{arthāpatti} & 
	\dev{yadakīrtitamarthādāpadyate sārthāpattiḥ/} \\
	(A) & 13. & \emph{arthāpatti} & \dev{yadanuktamarthādāpadyate 
		sārthāpattiḥ/} \\
	(N) & 13. & \emph{atthāpatti} & 
	\dev{yad akittitaṃ atthato āpajjati, sā atthāpatti/} \\
	
	\rule{0pt}{0.5cm}(S) & 14. & \emph{viparyaya} & \dev{yadyasya prātilomyaṃ tadviparyayaḥ/} \\
	(V) & 20. & \emph{viparyaya} & \dev{tasya prātilomyaṃ viparyayaḥ/} \\
	(A) & 16. & \emph{viparyaya} & \dev{pratilomena sādhanaṃ viparyayaḥ/} \\
	(N) & 14. & \emph{vipariyaya} & \dev{yaṃ yattha vihitaṃ, tatra yaṃ tassa paṭilomaṃ, so vipariyayo/} \\
	
	\rule{0pt}{0.5cm}(S) & 15. & \emph{prasaṅga} & \dev{prakaraṇāntareṇa 
		samānaḥ prasaṅgaḥ/} \\
	(V) & 14. & \emph{prasaṅga} & \dev{prakaraṇābhihito'rthaḥ kenacidupodghātena punarucyamānaḥ prasaṅgaḥ/} \\
	(A) & 15. & \emph{prasaṅga} & \dev{prakaraṇāntareṇa samāno'rthaḥ 
		prasaṅgaḥ/} \\
	(N) & 15. & \emph{pasaṅga} & \dev{pakaraṇantarena 
		samāno attho pasaṅgo/} \\
	
	\rule{0pt}{0.5cm}(S) & 16. & \emph{ekānta} & \dev{yadavadhāraṇenocyate 
		sa ekāntaḥ/} \\
	(V) & 15. & \emph{ekānta} & \dev{yathā tathā sa ekāntaḥ/} \\
	(A) & 26. & \emph{ekānta} & \dev{sarvatrāyattamekāntaḥ/} \\
	(N) & 16. & \emph{ekānta} & \dev{sabbathā yaṃ tathā, so ekānto/} \\
	
	\rule{0pt}{0.5cm}(S) & 17. & \emph{anekānta} & \dev{kvacittathā 
		kvacidanyathā so'nekāntaḥ/} \\
	(V) & 16. & \emph{anekānta} & \dev{kvacittathā kvacidanyathā'sāvanekāntaḥ/} \\
	(A)\-\- & \-\- & \-\- \\
	(N) & 17. & \emph{anekānta} & \dev{yo pana katthaci aññathā so anekānto/} \\
	
	\rule{0pt}{0.5cm}(S) & 18. & \emph{pūrvapakṣa} & \dev{yastu 
		niḥsaṃśayamabhidhīyate sa pūrvapakṣaḥ/}\footnote{This definition of 
		\emph{pūrvapakṣa} in the Nepalese version is problematic.}\\
	(V) & 17. & \emph{pūrvapakṣa} & \dev{pratiṣedhavacanaṃ pūrvapakṣaḥ/}\\
	(A) & 24. & \emph{pūrvapakṣa} & \dev{pratiṣeddhavyaṃ vākyaṃ 
		pūrvapakṣaḥ/} \\
	(N) & 18. & \emph{pubbapakkha} & \dev{[yo] tu nissandeham abhidhīyate, so pubbapakkho/}\\
	
	\rule{0pt}{0.5cm}(S) & 19. & \emph{nirṇaya} & \dev{tasyottaraṃ nirṇayaḥ/}\\
	(V) & 18. & \emph{nirṇaya} & \dev{uttaravacanaṃ nirṇayaḥ/}\\
	(A) & 25. & \emph{uttarapakṣa} & \dev{nirṇayavākyamuttarapakṣaḥ/} \\
	(N) & 19. & \emph{niṇṇaya} & \dev{tassa yaṃ uttaraṃ, so niṇṇayo/}\\
	
	\rule{0pt}{0.5cm}(S) & 20. & \emph{anumata} & \dev{paramatamapratiṣiddhamanumatam/} \\
	(V) & 25. & \emph{anumata} & \dev{paramatamapratiṣiddhamanumatam/} \\
	(A) & 18. & \emph{anumata} & \dev{paravākyamapratiṣiddhamanumatam/} \\
	(N) & 20. & \emph{anumata} & \dev{paramatam appaṭisiddhaṃ anumataṃ/} \\
	
	\rule{0pt}{0.5cm}(S) & 21. & \emph{vidhāna} & \dev{prakaraṇānupūrvyādabhihitaṃ vidhānam/} \\
	(V) & 19. & \emph{vidhāna} & \dev{prakaraṇānupūrvaṃ vidhānam/} \\
	(A) & 2. & \emph{vidhāna} & \dev{śāstrasya prakaraṇānupūrvī vidhānam/} \\
	(N) & 21. & \emph{vidhāna} & \dev{pakaraṇānupubbaṃ vidhānaṃ/} \\
	
	\rule{0pt}{0.5cm}(S) & 22. & \emph{anāgatāpekṣaṇa} & \dev{evaṃ 
		vakṣyatītyanāgatāpekṣaṇam/} \\
	(V) & 22. & \emph{anāgatāpekṣaṇa} & \dev{paratra vakṣāmītyanāgatāvekṣaṇam/} \\
	(A) & 27. & \emph{anāgatāvekṣaṇa} & \dev{paścādevaṃ 
		vihitamityanāgatāvekṣaṇam/} \\
	(N) & 22. & \emph{anāgatāpekkhana} & \dev{evaṃ vakkhāmi ti anāgatāpekkhanaṃ/} \\
	
	\rule{0pt}{0.5cm}(S) & 23. & \emph{atikrāntāpekṣaṇa} & 
	\dev{ityuktamityatikrāntāpekṣaṇam/} \\
	(V) & 21. & \emph{atikrāntāpekṣaṇa} & \dev{ityuktamatikrāntāvekṣaṇam/} \\
	(A) & 28. & \emph{atikrāntāvekṣaṇa} & \dev{purastādevaṃ vihitamityatikrāntāvekṣaṇam/} \\
	(N) & 23. & \emph{atītāpekkhana} & \dev{iti vuttan ti atītāpekkhanaṃ/} \\
	
	\rule{0pt}{0.5cm}(S) & 24. & \emph{saṃśaya} & \dev{ubhayahetunidarśanaṃ saṃśayaḥ/} \\
	(V) & 23. & \emph{saṃśaya} & \dev{ubhayato hetudarśanaṃ saṃśayaḥ/} \\
	(A) & 14. & \emph{saṃśaya} & \dev{ubhayato hetumānarthaḥ saṃśayaḥ/} \\
	(N) & 24. & \emph{saṃsaya} & \dev{ubhayahetudassanaṃ saṃsayo/} \\
	
	\rule{0pt}{0.5cm}(S) & 25. & \emph{vyākhyāna} & \dev{tatrātiśayopavarṇanaṃ vyākhyānam/} \\
	(V) & 24. & \emph{vyākhyāna} & \dev{tatrātiśayavarṇanātivyākhyānam/} \\
	(A) & 19. & \emph{vyākhyāna} & \dev{atiśayavarṇanā vyākhyānam/} \\
	(N) & 25. & \emph{vyākhyāna} & \dev{saṃvaṇṇanā vyākhyānam/} \\
	
	\rule{0pt}{0.5cm}(S) & 26. & \emph{svasaṃjñā} & \dev{anyaśāstrāsāmānyā 
		svasaṃjñā/} \\
	(V) & 26. & \emph{svasaṃjñā} & \dev{parairasammataḥ śabdaḥ svasaṃjñā/} \\
	(A) & 23. & \emph{svasaṃjñā} & \dev{parairasamitaḥ śabdaḥ svasaṃjñā/} \\
	(N) & 26. & \emph{anaññā sakasaññā} & \dev{bhūtānaṃ pavattā ārambhacintā anaññā, sassa sādhāraṇā sakasaññā/} \\
	
	\rule{0pt}{0.5cm}(S) & 27. & \emph{nirvacana} & \dev{lokaprathitamudāharaṇaṃ nirvacanam/} \\
	(V) & 27. & \emph{nirvacana} & \dev{loke pratītamudāharaṇaṃ nirvacanam/} \\
	(A) & 20. & \emph{nirvacana} & \dev{guṇataḥ śabdaniṣpattirnirvacanam/} \\
	(N) & 27. & \emph{nibbacana} & \dev{lokappatītam udāharaṇaṃ nibbacanaṃ/} \\
	
	\rule{0pt}{0.5cm}(S) & 28. & \emph{nidarśana} & \dev{dṛṣṭāntavyaktirnidarśanam/} \\
	(V) & 28. & \emph{nidarśana} & \dev{tadyuktinidarśanaṃ dṛṣṭāntaḥ/} \\
	(A) & 21. & \emph{nidarśana} & \dev{dṛṣṭānto dṛṣṭāntayukto nidarśanam/} \\
	(N) & 28. & \emph{nidassana} & \dev{diṭṭhantasaṃyogo nidassanaṃ/} \\
	
	\rule{0pt}{0.5cm}(S) & 29. & \emph{niyoga} & \dev{idameveti niyogaḥ/} \\
	(V) & 29. & \emph{niyoga} & \dev{eveti niyogaḥ/} \\
	(A) & 29. & \emph{niyoga} & \dev{evaṃ nānyatheti niyogaḥ/} \\
	(N) & 29. & \emph{niyoga} & \dev{idam evā ti niyogo/} \\
	
	\rule{0pt}{0.5cm}(S) & 30. & \emph{vikalpa} & \dev{/} \\
	(V) & 30. & \emph{vikalpa} & \dev{idaṃ vedaṃ veti vikalpaḥ/} \\
	(A) & 30. & \emph{vikalpa} & \dev{anena vānena veti vikalpaḥ/} \\
	(N) & 30. & \emph{vikappa} & \dev{idaṃ vā ti vikappo/} \\
	
	\rule{0pt}{0.5cm}(S) & 31. & \emph{samuccaya} & \dev{/} \\
	(V) & 31. & \emph{samuccaya} & \dev{idaṃ cedaṃ ceti samuccayaḥ/} \\
	(A) & 31. & \emph{samuccaya} & \dev{anena cānena ceti samuccayaḥ/} \\
	(N) & 31. & \emph{samuccaya} & \dev{saṃkhepavacanaṃ samuccayo/} \\
	
	\rule{0pt}{0.5cm}(S) & 32. & \emph{ūhya} & \dev{yadanirdiṣṭaṃ 
		buddhigamyaṃ tadūhyam/} \\
	(V) & 32. & \emph{ūhya} & \dev{atra yadanirdiṣṭaṃ yuktigamyaṃ tadūhyam/} \\
	(A) & & \emph{ūhya} & \dev{anuktakaraṇamūhyam/} \\
	(N) & 32. & \emph{upānīya} & \dev{yad aniddiṭṭhaṃ buddhiyā avagamanīyaṃ, tad upānīyan ti/} \\
	
	\bottomrule
	
\end{longtable}


\subsection{Āyurvedic literature}

\subsubsection{Primary texts}
While references to \emph{tantrayukti}s can be found across various disciplines, Āyurveda places a particular emphasis on their discussion, especially evident in key texts of Āyurveda, such as the \emph{Caraka-} and the \emph{Suśruta- saṃhitā}s, as well as the \AS. The \emph{Carakasaṃhitā} and \AS present an identical list of \emph{tantrayukti}s contained in a group of four verses.\footnote{\label{CaAsT}
	\begin{verse}
		\dev{tatrādhikaraṇaṃ yogo hetvartho'rthaḥ padasya ca/ \\
		pradeśoddeśanirdeśavākyaśeṣāḥ prayojanam// \\
		upadeśāpadeśātideśārthāpattinirṇayāḥ/ \\
		prasaṅgaikāntanaikāntāḥ sāpavargo viparyayaḥ// \\
	 	pūrvapakṣavidhānānumatavyākhyānasaṃśayāḥ/ \\
 		atītānāgatāpekṣāsvasaṃjñohyasamuccayāḥ// \\
 		nidarśanaṃ nirvacanaṃ niyogo'tha vikalpanam/ \\
 		pratyutsārastathoddhāraḥ sambhavastantrayuktayaḥ// \\}
	\end{verse}
	\AS \As{6.50.150--153}{959}. \emph{Carakasaṃhitā} \Ca{8.12.41b--45a}{736} reads almost the same. The only two variants are (1) \dev{atītānāgatāvekṣā...} and (2) \dev{nirvacanaṃ saṃniyogo vikalpanam}.
	} 
However, unlike the \SS they lack explicit definitions and examples. This list of the \emph{tantrayukti}s appear in the final chapter of the last book in both \emph{Carakasaṃhitā} (41b--45a, chapter 12, \emph{Siddhisthāna}) and \emph{Aṣṭāṅgasaṅgraha} (150--153, chapter 50, \emph{Uttarasthāna}). The same has been quoted by Aruṇadatta in his commentary \emph{Sarvāṅgasundarī} on the \emph{Aṣṭāṅgahṛdaya} while elucidating the concept of \emph{tantraguṇa} (qualities of the system).\footnote{Aruṇadatta on the \AHS \Ah{6.40.78}{946}.} 
Notably, this list consists of 36 \emph{tantrayukti}s instead of 32 found in the \SS and other texts. The additional four are: \emph{prayojana} (objective), \emph{pratyutsāra} (rebuttal), \emph{uddhāra}, and \emph{sambhava} (origin). 

The presence of identical verses enumerating the \emph{tantrayukti}s in the \AHS, \CS and \emph{Sarvāṅgasundarī} strongly suggests a shared origin. However, a critical issue arises due to the absence of a comprehensive critical edition of the chapter 12 of the \emph{Siddhisthāna} of the \CS, leaving uncertainty about the total number of \emph{tantrayukti}s recognized by Dṛḍhabala in this section.\footnote{We know from internal textual evidence that the Siddhisthāna of the Carakasaṃhitā in which the list of the \emph{tantrayukti}s appear was originally authored by Dṛḍhabala, who lived in a town called Pañcanada sometime between 300 and 500 AD. Cf. 
	\begin{verse}
		\dev{akhaṇḍārthaṃ dṛḍhabalo jātaḥ pañcanade pure/ \\
		kṛtvā bahubhyastantrebhyo viśeṣoñchaśiloccayam// \\
		saptadaśauṣadhādhyāyasiddhikalpairapūrayat/} \\
	\end{verse}
	\Ca{8.12.39--40a}{735}
	} 
The problem arises from different readings of the half-verse that occurs right before the list of 36 \emph{tantrayukti}s. In \MScite{MS Kathmandu NAK 1/1648} (dated 1183 AD, the oldest dated manuscript of the \CS known to us), the reading of this verse is:	
			\textbf{ṣaṭtriṃśadbhi}rvicitrābhirbhū[ṣi]taṃ tantrayuktibhiḥ//}	
This number of 36 \emph{tantrayukti}s perfectly agrees with the following list of the 36 \emph{tantrayukti}s. A similar reading is found in Trikamji's 1933 Carakasaṃhitā edition which contains only the \emph{mūla}-text.\footnote{	
		\dev{\textbf{ṣaṭtriṃśatā} vicitrābhirbhūṣitaṃ tantrayuktibhiḥ//}		
	8.12.70a \parencite[972]{cara-trikamji}.} 
However, although most of the other editions consist of the same reading, a number of editions show quite a lot of discrepancies with the number. For example, \Ca{8.12.41a}{735} reads the same half-verse as 
\dev{\textbf{ṣaḍviṃśatā} vicitrābhir bhūṣitaṃ tantrayuktibhiḥ/}. 
In the same edition, the reading of Cakrapāṇi's \emph{Āyurvedadīpikā} supports the reading: 
\dev{\textbf{ṣaḍviṃśa}ttantrayuktibhirbhūṣitamapūrayaddṛḍhabala iti yojanā/}. 
However, after this verse, the same edition contains the versified list of the 36 \emph{tantrayukti}s and commenting on these verses, the \emph{Āyurvedadīpikā} confirms the total number of the \emph{tantrayukti}s as 36: 
\dev{ityetāḥ \textbf{ṣaṭtriṃśa}ttantrayuktayo vyāhṛtāḥ/}.\footcite[737]{cara-trikamji3} 
Moreover, the edition of Rāmaprasāda Vaidyopādhyāya reads the half-verse as: 
\dev{pañcatriṃśadvicitrābhirbhūṣitaṃ tantrayuktibhiḥ/}.\footnote{\cite[1913]{vaid-1911}, \cite[1020]{}.} 
Rāmaprasāda Vaidyopādhyāya excludes \emph{ūhya}.\footnote{Reading the \emph{tantrayukti} \emph{samuccaya} as \emph{asamuccaya}, he reads the verse where \emph{ūhya} appears as: 
	\dev{atītānāgatāpekṣā svasaṃjñā hyasamuccayāḥ//}. Surely, this reading is erroneous as the plural ending after \emph{samuccaya} do not agree with his reading.} 
The same reading is found in Satīśacandra Śarmā's third edition of the \CS.\footnote{\cite[1020]{sarm-1923}. His first edition, however, reads the half-verse just as the reading in \cite{cara-trikamji}.\parencite[884]{sarm-1904}}
However, adding more troubles to it, Satīśacandra Śarmā, in his Bengali translation, says that there are 34 \emph{tantrayukti}s (even though the main Sanskrit text of his edition counts 35). Then ḥe actually illustrates 36 \emph{tantrayukti}s making a remark that “in Gaṅgadhara's reading there are 36 \emph{tantrayukti}s because he counts \emph{saṃśaya} twice in his commentary. But 35 was reckoned in his \emph{mūla}-text. Another manuscript reckons 34 \emph{tantrayukti}s excluding \emph{apadeśa}. This edition reads thirty-five instead of thirty-four or thirty-six.”\footnote{\textbengli{“গঙ্গাধর পাঠ— তন্ত্রযুক্তি ছত্রিশ প্রকার। তিনি টীকাতে সংশয়কে দুই বার উল্লেখ করিয়া ছত্রিশ প্রকার গণনা করিয়াছেন, কিন্তু তাঁহার মূলে পঁয়ত্রিশ প্রকার আছে; গ্রন্থান্তরে ৩৪ প্রকার আছে; তাহাতে ‘অপদেশ’ ধর্ত্তব্য হয় নাই। এই অনুবাদের মূলে চতুস্ত্রিংশৎ বা ষট্‌ত্রিংশৎ স্থলে পঞ্চত্রিংশৎ লিখিত হইল।”} \cite[1022]{sarm-1923}.} 
In the edition of Narendranātha Senagupta and Balāicandra Senagupta includes Cakrapāṇi's \emph{Āyurvedadīpikā} and Gaṅgādhara's \emph{Jalpakalpataru}, the Sanskrit \emph{mūla} and the \emph{Jalpakalpataru} enumerate 36 \emph{tantrayukti}s. However, in the same edition, the \emph{Āyurvedadīpikā} reads, \dev{\textbf{pañcatriṃśa}ttantrayuktibhirbhūṣitamapūrayaddṛḍhabala iti yojanā/}.\fvolcite{III}{3814}{sena-1928} 
Again, after the illustrations of the 36 \emph{tantrayukti}s it reads, \dev{ityetāḥ \textbf{ṣaṭtriṃśa}ttantrayuktayo vyāhṛtāḥ/}.\fvolcite{III}{3822}{sena-1928} 
In his edition of the \emph{Tantrayuktivicāra}, Muthuswami also mentions that 35 \emph{tantrayukti}s are reckoned in the \CS.\footnote{\dev{‘\textbf{pañcatriṃśa}dvicitrābhirbhūṣitaṃ tantrayuktibhiḥ/’ iti carake/ dvātriṃśaditi suśrutaḥ/}\parencite[fn.2][2]{muth-1976}.} 
Jivānanda Vidyāsagara's edition gives no number at all--- \dev{tathā ca tā vicitrābhirbhūṣitaṃ tantrayuktibhiḥ/}.\footcite[961]{bhat-1877} 

Commentaries on the \CS prior to Cakrapāṇi's \emph{Āyurvedadīpikā}, such as the \emph{Carakanyāsa} of Bhaṭṭāra Hariścandra (c. mid-sixth century) or \emph{Nirantarapadavyākhyā} of Jejjaṭa (c. 7th or 8th century AD) do not help much because the extant portions of these commentaries do not include the concerned section of the 12th chapter of the \emph{Siddhisthāna}.\q{Check if Jejjaṭa is available} However, Hariścandra was possibly not aware of the total number and the list of the \emph{tantrayukti}s in the final chapter of the \emph{Siddhisthāna} because he discussed the \emph{tantrayukti}s right at the beginning of his commentary and showed no indication to his awareness about the discussion on the \emph{tantrayukti}s at the end of the text. Moreover, he discusses 40 \emph{tantrayukti}s instead of 36. It is not yet settled whether or not Hariścandra was aware of Dṛḍhabala's redaction of the \CS. However, Hariścandra's treatment of the tantrayuktis supports the latter.\fvolcite[189]{meul-hist}.  It is clear from Cakrapāṇi's commentary on the \CS that in the version of the text he commented upon contained the four verses that list the 36 \emph{tantrayukti}s. It is, however, not improbable that the four verses that list the 36 \emph{tantrayukti}s were later added to the \CS sometime between the sixth (the date of Hariścandra) and the eleventh century (the date of Cakrapāṇi) and the discrepancy appeared when the previous verse that gives the total number of the \emph{tantrayukti}s was not properly emended by the scribes complying with the following list of 36 \emph{tantrayukti}s. There is a need of a critical edition of the twelfth chapter of the \emph{Siddhisthāna} of the \CS to address these issues definitely.  

\subsubsection{Commentaries}\label{commentaries}
The commentators who extensively delved into the discussion of the \emph{tantrayukti}s are Hariścandra, the author of \emph{Carakanyāsa}, and Aruṇadatta, who authored his commentary \emph{Sarvāṅgasundarī} on the \emph{Aṣṭāṅgahṛdaya} of Vāgbhaṭa. Hariścandra meticulously defined and analyzed 40 \emph{tantrayukti}s at the beginning of his work. The four additional \emph{tantrayukti}s are:  \emph{paripraśna} (question), \emph{vyākaraṇa} (grammatical clarification), \emph{vyutkrāntābhidhāna} (overpassing statement) and \emph{hetu} (means of knowledge).\footnote{This text has only been published once 
	(only until the third chapter of \emph{Sūtrasthāna}) by Masta Ram Shastri from Lahore in 1932/33.\fvolcite{IB}[290]{meul-hist} Unfortunately, it is currently inaccessible to us. Although some fragmented manuscripts of the Carakanyāsa exist, for this section (Chapter 1, \emph{Sūtrasthāna}), we were able to consult only \MScite{MS Jamnagar GAU 114}. This is a recent apograph with several lacunae and corruptions. The list of the tantrayuktis provided in the \emph{Carakanyāsa} is as follows (with some emendations made in the reading): \dev{tantrasya yuktayo'dhikaraṇādyāścatvāriṃśat/... yuktayastāvadadhikaraṇaṃ yogo hetvartha uddeśa upadeśo'padeśo'tideśaḥ pradeśo nirṇayo'rthāpattirvākyaśeṣaḥ prayojanaṃ prasaṅga ekānto'nekānto viparyayo'pavargaḥ pūrvapakṣo vidhānamanumataṃ vyākhyānaṃ paripraśno vyākaraṇamatītāpekṣaṇamanāgatāpekṣaṇaṃ saṃśayaḥ svasaṃjñohyaḥ samuccayo nidarśanaṃ nirvacanaṃ niyogo vikalpaḥ pratyutsāra uddhāraḥ sambhavo vyutkrāntābhidhānaṃ heturiti/}}

Aruṇadatta, while discussing the concept of \emph{tantraguṇa} at the end of the \emph{Aṣṭāṅgahṛdaya}, provided an elaborate description of \emph{tantrayukti}s, considering them as part of a system of ninety-five \emph{tantraguṇa}s. Śrīdāsapaṇḍita (14th century), a commentator on the \emph{Aṣṭāṅgahṛdaya}, echoed Aruṇadatta's exploration of tantrayuktis in the beginning of his commentary, \emph{Hṛdayabodhikā}.\fvolcite{IA}[680]{meul-hist} Thus, both Hariścandra and Śrīdāsapaṇḍita engage with this topic right at the beginning, underscoring the significance they attribute to the subject. Other noteworthy commentators who discussed the topic of \emph{tantrayukti} are Cakrapāṇi (11th century) and Indu (sometime between 8th and 12th century). Cakrapāṇi and Indu defined and illustrated the \emph{tantrayukti}s mentioned in the \emph{Carakasaṃhitā} and the \emph{Aṣṭāṅgasaṅgraha}, respectively. They affirm the inclusion of the four additional \emph{tantrayukti}s in Hariścandra's list. Cakrapāṇi, aligning them with existing concepts, incorporates \emph{paripraśna}, \emph{vyākaraṇa}, and \emph{vyutkrāntābhidhāna} under the \emph{tantrayukti}s \emph{uddeśa}, \emph{vyākhyāna}, and \emph{nirdeśa}, respectively. According to him, \emph{hetu} serves as an overarching term encompassing all \emph{pramāṇa}s (means of knowledge) such as \emph{pratyakṣa} (perception) and others. Indu, however, outlines three possible reasons for not incorporating these \emph{tantrayukti}s into the list: (1) they lack direct mention in the main text, (2) they could be considered as falling within the scopes of already enumerated \emph{tantrayukti}s, or (3) they are not recognized as \emph{tantrayukti}s. \emph{Jalpakalpataru}, a nineteenth-century commentary on the \emph{Carakasaṃhitā} by Gaṅgādhara Kavirāja from Bengal also discusses the \emph{tantrayukti}s, very often quoting from the \SS.  

\subsubsection{Monographs}
two texts authored by Āyurvedic scholars exclusively delve into the topic of \emph{tantrayukti}. The first is the \emph{Tantrayuktivicāra} by a physician named Nīlamegha (also known as Vaidyanātha), while the second is called the \emph{Tantrayukti}, which is a sort of recast of the former by an anonymous author. The anonymous author describes himself as being from the same lineage as Nīlamegha and asserts that Nīlamegha belongs to the same lineage of Bhiṣagārya (also known as Nārāyaṇa Bhiṣaj). Both Nīlamegha and the author of Tantrayukti are likely from Kerala or coastal Karnataka.\footnote{Kolatteri Saṅkaramenon, 
	the first editor of the \emph{Tantrayuktivicāra}, believes that Nīlamegha hails from Kerala. This conclusion is drawn from Nīlamegha's reference to his guru as Sundara, whom Saṅkaramenon identifies as the same individual credited with composing the \emph{Lakṣaṇāmṛta}, a treatise on toxicology. This assertion is plausible because the only known manuscript of \emph{Tantrayuktivicāra} belongs to a member of one of the Aṣṭavaidya families of Kerala, aligning with the Vāgbhaṭa school, to which Nīlamegha also belongs.(\fvolcite{IIA}[143]{meul-hist}) On the other hand, the anonymous author of the \emph{Tantrayukti} associates Nīlamegha with the lineage of Bhiṣagārya, who hails from Uṇṭuru, a village located 3 kilometers from Gokarṇa which is in coastal Karnataka.\parencite[30]{nara-1949}.}
According to Koḷatteri Śaṅkaramenon and Meulenbeld, Nīlamegha flourished in the first half of ninth century.\footnote{Nīlamegha mentions Vāhaṭa (Vāgbhaṭa), Indu, and Jejjaṭa in his work.
	 This places him definitively after the seventh century. The Buddhist influence in the Tantrayukti indicates a date not much later than 800 AD. (\cite[\dev{avatārikā 5–6}]{muth-1976}, \volcite{IIA}[143]{meul-hist}.)} 
The \emph{Tantrayukti} was very likely composed after the sixteenth century.\footnote{From 
	the explicit mention of Nīlamegha and Bhiṣagārya in the work \emph{Tantrayukti}, we can say that the author flourished after them. Determining the date of Bhiṣagārya is problematic. However, since the Kairalī commentary on the \AHS frequently quotes from Bhiṣagārya's \emph{Abhidhānamañjarī}, it indicates that Bhiṣagārya predates the composition of this commentary. Meulenbeld suggests the end of the seventeenth century as the terminus post quem for the Kairalī (\fvolcite{IA}[675]{meul-hist}). Moreover, he views \emph{Abhidhānamañjarī} as a work composed after the sixteenth century, citing details within it that affirm its posteriority to the \emph{Rājanighaṇṭu} and \emph{Bhāvaprakāśa} (\volcite{IIA}[442]{meul-hist}).} 

Nīlamegha's \emph{Tantrayuktivicāra} is a versified text accompanied by an autocommentary. The text comprises eighteen verses plus a hemistich, resulting in a total of 37 hemistichs. Each hemistich serves as a definition for a \emph{tantrayukti}. Nīlamegha enumerates a total of 36 \emph{tantrayukti}s, as mentioned in the \AS and \emph{Carakasaṃhitā}. The additional hemistich defines \emph{aviparyaya}, which, according to Nīlamegha, is sometimes considered instead of \emph{viparyaya}. This substitution occurs when one understands that the negative prefix \emph{a-} is deleted due to a \emph{pūrvarūpa sandhi}--- \emph{sāpavargaḥ + aviparyayaḥ} → \emph{sāpavargo viparyayaḥ} (See footnote \ref{CaAsT}.).

The text of the \emph{Tantrayukti} includes some verses at the beginning and end, where the author discusses the lineage of Nīlamegha. The author explicitly states that his text is a revised version of Nīlamegha's \emph{Tantrayuktivicāra} because the available manuscripts were mostly corrupt.\footnote{\dev{vaidyanāthopasṛṣṭānāṃ lakṣyalakṣaṇavāgjuṣām//\\
	tāsāṃ prāyaḥ prakāśānāṃ durlekhāpaṅkadūṣaṇāt/\\
	kriyate sāmprataṃ kṛcchrāduddhṛtya parimārjjanam//\\}
	\cite[1]{nara-1949}} 
It is evident that there are substantial reproductions of parts of the \emph{Tantrayuktivicāra} and its autocommentary. The total number of \emph{tantrayukti}s and their enumeration remains identical to that of the \emph{Tantrayuktivicāra}. What distinguishes it from the \emph{Tantrayuktivicāra} is the incorporation of a list of other \emph{tantraguṇa}s and 14 \emph{tantradoṣa}s. This list of \emph{tantraguṇa}s includes 15 types of \emph{vyākhyā}, 7 types of \emph{kalpanā}, 20 types of \emph{āśraya}, and 17 types of metaphoric and metonymic devices, such as \emph{tācchīlya} and so on.

\section{\emph{Tantrayukti}-inventories}

It is evident from the discussion on the early sources that all these listings of the \emph{tantrayukti}s in the early sources can be grouped into two categories. For the ease of our following discussion, we name these two inventories as (1) earlier listing and (2) later listing.

\subsection{Earlier Listing}

The four inventories of \emph{tantrayukti}s from the \emph{Arthaśāstra}, \SS, \emph{Viṣṇudharmottarapurāṇa}, and \emph{Saddanīti} belong to this Earlier Listing. The list of the \emph{Tolkāppiyam} also belongs to this group, even though not all of the \emph{utti}s in this list might correspond accurately to the Sanskrit and Pali lists. The common feature of this listing is that every inventory belonging to this listing explicitly mentions the total number of the \emph{tantrayukti}s as thirty-two. Even though there are sometimes different \emph{tantrayukti}s enumerated in different lists, the total number of them always remained 32. As demonstrated in Table \ref{table-SAV}, the Sanskrit and Pali lists are similarly ordered and are always accompanied by similar or identical definitions.

\subsection{Later Listing}

The later listing is the one we find in the \AS, \CS, the commentaries on the \CS, \AS and \AHS which are mentioned in the section on Commentaries and the two monographs, the \emph{Tantrayuktivicāra} and \emph{Tantrayukti}. This list sprung from a single source, a versified list of thirty-six \emph{tantrayukti}s comprising four verses that appear in the \AS, \CS and Aruṇadatta's commentary on the \AHS. It is not clear if these four verses first appeared in the Dṛḍhabala's redaction of the \CS or in Vāgbhaṭa's \AS. Unlike the Earlier Listing, this list does not consist of definitions of the \emph{tantrayukti}s. Definitions and illustrations are given by the authors of the commentaries and monographs as discussed in the previous section. Although Hariścandra's list includes 40 tantrayuktis instead of 36, his enumeration is closer to this listing than the earlier one. 

\section{Terminology}

\label{tantra-trans}
The terms have been translated into English in numerous books and articles. English renditions of the terms can be found in English translations of the \SS such as in \cite[171--172]{sing-1980}, and \volcite{3}[631--639]{shar-1999}; in translations of the \emph{Carakasaṃhitā} such as in \cite[436--444]{shar-2006} and in \cite[1050]{gula-1949}, in the translation of the \emph{Arthaśāstra} such as in \cite[459]{sham-1951}, \cite[593]{kang-1969}, \cite[1103]{unni-2006} and \cite[]{oliv-2013}, and by K. Srikanta Moorthy in \cite[Appendix xi--xxxiv]{muth-1976}. They are also found in various books and articles dedicated to discussing the \emph{tantrayukti}s such as in \cite[601--602]{ober-1968}, \volcite{1}[72]{solo-1976}, \cite[34--155]{lele-1981}, \citeyear[36--150]{lele-2006} and so on. German translations of the terms can be found in \cite[663--664]{meye-1926} (German translation of the \emph{Arthaśāstra}) and in \cite{pret-1991}.


The definitions of \emph{tantrayukti}s exhibit numerous variations across different texts. Here we will discuss each of the \emph{tantrayukti}s that occur in the \emph{Suśruta Saṃhitā} in comparison with their definitions in other texts. As indicated in Table \ref{table-SAV}, the definitions of \emph{tantrayukti}s in the \SS are frequently either identical or nearly identical to those found in the \emph{Arthaśāstra}, \emph{Viṣṇudharmottarapurāṇa} and \emph{Saddanīti}. Therefore, unless the definitions in these two texts notably deviate from those in the \SS, we will not make explicit references to them in the subsequent elucidation of the terms. 


\subsection{1. \emph{adhikaraṇa}}

\emph{Adhikaraṇa} appears as the first \emph{tantrayukti} in all traditional enumerations. It is among those \emph{tantrayukti}s for which there is little disagreement concerning its definition. This tantrayukti functions as a structural and interpretative device. With a tautological expression, the \SS defines \emph{adhikaraṇa} as something, with reference to which statements are made. While defining \emph{adhikaraṇa}, the text employs the same verb, \emph{adhi- kṛ-} (to refer), whence the noun \emph{adhikaraṇa} has been derived. The text supplies examples of \emph{rasa} (taste) and \emph{doṣa} (humour), for which two chapters of the \emph{Uttaratantra}, namely chapter 62 (\emph{Kāyacikitsā} 27) and  chapter 65 (\emph{Kāyacikitsā} 30) are dedicated.\footnote{They are chapters 63 and 66 in the \colorbox{yellow}{vulgate.}.} Clearly, \emph{adhikaraṇa} is the topic or theme. 

Cakrapāṇi defines \emph{adhikaraṇa} in almost the same fashion.\footnote{\dev{yamarthamadhikṛtya pravartate kartā/ yathā “vighnabhūtā yadā rogā”
	ityādi/ atra rogādikamadhikṛtyāyurvedo maharṣibhiḥ kṛta iti ‘rogāḥ}’ \dev{ityadhikaraṇam/} \parencite[736]{cara-trikamji3}.}
Aruṇadatta's definition is similar but he specifies that \emph{adhikaraṇa} can be of a entire discipline \emph{(śāstra)}, or a book \emph{(sthāna)} of it, or a chapter \emph{(adhyāya)}, or a section \emph{(prakaraṇa)}, or even of a sentence \emph{(vākya)}.\footnote{\dev{tatra 
	adhikaraṇaṃ nāma, yadadhikṛtya pravartate śāstraṃ sthānamadhyāyaṃ prakaraṇaṃ vākyaṃ vā/...} \parencite[947]{kunt-1939}.} 
However, in the commentaries of Hariścandra and Indu, we explore two more aspects of the concept of \emph{adhikaraṇa}. According to Hariścandra, \emph{adhikaraṇa} is the reason or ground referring to which the authors direct their discourse. For example, diseases create misery and the authors of Āyurveda began their discussion addressing them.\footnote{\dev{tatrādhikaraṇaṃ nāma yannimittamadhikṛtya pravartate kartā/… uta vā
	vighnabhūtā yadā rogāḥ prādurbhūtāḥ tadidaṃ nimittamadhikṛtya jagadanukampayā maharṣibhirayamāyurveda āgamaḥ/ evamadhikaraṇavyākhyā varṇayitavyā/} MS Jamnagar GAU 114, p.4--5.}.
Thus disease is the \emph{adhikaraṇa} or theme of their discussion. Indu identifies \emph{adhikaraṇa} as a binding force that links ideas. According to him, \emph{adhikaraṇa} as an introductory reference and paraphrases it as something that exposes a general statement to a specific context.\footnote{\dev{adhikaraṇaṃ prastāvaḥ sāmānyenoktamapyarthajātam yadbalādviśeṣe'vasthāpyate tadadhikaraṇam/} \parencite[959]{atha-1980}.} 

In the \emph{Tolkāppiyam}, however, the equivalent expression for this \emph{tantrayukti} remains unclear, as commentators, namely Iḷampūraṇar and Pērāciriyar, list the item differently. In Sastri's translation of the Tolkāppiyam, adhikaraṇa was identified with \emph{atikāra muṟai}, the second element in Iḷampūraṇar's list. Sastri translates this expression as “deciding the extent where one serves as \emph{adhikāra sūtra} or a word or words in a sūtra taken along with the \emph{sūtra}-s that follow.”\footnote{\cite[233]{sast-1936}.} However, Dikshitar, in his brief article on the \emph{tantrayukti}s, equates \emph{adhikaraṇa} with \emph{nutaliyatu aṟital}, the first element in Pērāciriyar's list, and translates it as “that division of a book which centers around a chief topic and deals wholly with that topic.”\footnote{\cite[85]{diks-1930}} Clearly, Dikshitar's interpretation stands close to our definition of \emph{adhikaraṇa}. Sastri's interpretation, on the other hand, corresponds to the concept of \emph{adhikāra} “heading” and \emph{anuvṛtti} “recurrence” in the \emph{sūtra} literature, especially in Pāṇini's \emph{Aṣṭādhyāyī}.\footnote{See \cite[111]{chev-2009}.} 

Oberhammer and Meulenbeld translate \emph{adhikaraṇa} as “topic,” and we have adopted the same translation.\footnote{\cite[601]{ober-1968}, \fvolcite{IB}[431]{meul-hist}.} Sharma translates the term as the “scope of the topic.”\footnote{\volcite{3}[631]{shar-1999}.}

\subsection{2. \emph{yoga}}

This \emph{tantrayukti} typically occupies the second position in most lists, except in the \emph{Arthaśāstra} where it appears third following \emph{vidhāna}. Functioning as a syntactic and semantic tool, a, as defined in the \SS, represents the faculty responsible for the cohesion of a sentence. If we consider the main purpose of the tantrayuktis as narrated in the \SS, namely, cohesion of a sentence (\emph{vākyayojana}) and cohesion of meaning (\emph{arthayojana}), it becomes evident that this \emph{tantrayukti} is one of the fundamental \emph{tantrayukti}s functioning as the device for \emph{vākyayojana}. The \SS further describes \emph{yoga} as a syntactic connection between words, facilitating the linking of words even when they are in reverse order or placed apart. However, this paraphrased statement is absent in the vulgate; instead, it appears in the commentary of Ḍalhana with a minor variant.\footnote{See \Su{6.65.9}{815}.} The definitions of \emph{yoga} in the \emph{Viṣṇudharmottarapurāṇa} and \emph{Arthaśāstra} closely mirror that of the \SS. However, the \emph{Viṣṇudharmottarapurāṇa}'s definition introduces a slight variation by including the term \emph{artha} “meaning”. According to this definition, \emph{yoga} is that by which the meaning of a sentence coheres. The \emph{Arthaśāstra} employs a nominalized verb in a compound noun instead of a relative clause--- \emph{vākyayojanā} “connecting a sentence”. The definition we find in the \emph{Saddanīti} is close to the paraphrased part of the definition of the \SS.\footnote{See Table \ref{table-SAV}.}

In the commentaries of Hariścandra, Indu, Cakrapāṇi and Aruṇa, however, the \emph{tantrayukti} \emph{yoga} is used in a broader sense. In these interpretations, \emph{yoga} serves not only as a device for cohesion within a sentence but also fosters coherence among sentences in a discourse. Hariścandra identifies three alternative interpretations of the term \emph{yoga}.\footnote{\dev{yogo nāma yojanā granthānāṃ yathārthasūtrabhāṣyasūtrayoḥ... pañcalakṣaṇo vā yogaḥ/ pratijñāhetūdāharaṇanigamanāni... yadiha yujyate sa yoga ityeke/} (MS Jamnagar GAU 114, p.5.)} 
Aruṇadatta also interprets \emph{yoga} in a similar fashion but instead of three alternatives he talks about the first two alternatives of Hariścandra. In the first alternative, \emph{yoga} is coherence between the main statement (sūtra) and its gloss (bhāṣya).
Aruṇadatta expands its scope to the coherence between mention (uddeśa) and description (nirdeśa) as well.\footnote{\dev{yogo nāma yojanā, uddeśanirdeśayoḥ sūtrabhāṣyayorvā/} \Ah{6.40.80}{947}.}  
In the second alternative, \emph{yoga} is reasoning (\emph{yukti}) having five types: (1) \emph{pratijñā} “proposition”, (2) \emph{hetu} “reason”, (3) \emph{udāharaṇa} “exemplification” (4) \emph{upanaya} “application”, and (5) \emph{nigamana} “conclusion”, resembling the five-membered syllogism of inference (\emph{anumāna}) in the Nyāya-Vaiśeṣika school.\footnote{\dev{yuktirvā yogaḥ, pratijñā heturdṛṣṭānta upanayo nigamanamiti pañcavidhaḥ/} \Ah{6.40.80}{947}.}  
Hariścandra also notes a different understanding of this \emph{tantrayukti} by some others. In this sense, \emph{yoga} is connectedness. This alternative definition is close to that of the \SS. In Indu's interpretation \emph{yoga} is lexical cohesion as he understand \emph{yoga} as a relation between a word and its meaning or a sentence and its meaning.\footnote{\dev{yogo nāma yogaḥ sambandhaḥ sa ca padārthayorvākyārthayorvā/} \As{6.50.150}{959}.} 
Cakrapāṇi, while defining \emph{yoga} in a fashion similar to the \SS, exemplifies it as a connection between five logical elements, namely \emph{pratijñā}, \emph{hetu}, \emph{udāharaṇa}, \emph{upanaya} and \emph{nigamana}, conflating the definition of \emph{yoga} with Hariścandra's the second alternative i.e. \emph{yoga} is reasoning (\emph{yukti}).\footnote{\dev{yogo nāma yojanā vyastānāṃ padānāmekīkaraṇam/ udāharaṇaṃ tāvadyathā pratijñāhetūdāharaṇopanayanigamanāni/} \Ca{8.12.41}{736}.} 
Nīlamegha defines \emph{yoga} as connecting words one by one coherently.\footnote{.} As he further explains in the autocommentary with examples from the \AHS, it is evident that he understands \emph{yoga} as coherence between a part of a sentence and the discourse.\footnote{\dev{/} \cite[3]{muth-1976}.} 
Neither V. R. Ramachandra Dikshitar nor P. S. Subrahmanya Sastri identified the tantrayukti with any utti mentioned in the Tolkāppiyam.\footnote{\cite[84]{chev-2009}.} 

The word \emph{yoga} derives from the Sanskrit root √yuj “to connect” with the primary suffix \emph{GHaÑ}, which is often used for creating action nouns. In Sanskrit technical literature, the term \emph{yoga} is used in a broad sense to mean any kind of linguistic connection or connectedness. In the \emph{Aṣṭādhyāyī} of Pāṇini, it often refers to the connection with a word or a word-element.\footnote{\cite[64]{josh-1991}.} Hence, it refers to a morphosemantic or syntaco-semantic connection. Patañjali uses this term several times in his Mahābhāṣya. In the \SS the word \emph{yoga} is primarily used to mean the connection between words in a sentence. According to this definition and illustration, it is primarily intra-sentential cohesion. Unlike the later commentators on the works of Caraka and Vāgbhaṭa, it does not extend the scope of this term to inter-sentential cohesion and coherence. Keeping in mind such definition given in the \SS, we translate the term as cohesion even though no other translators of the \emph{tantrayukti}s used this translation. In some other contexts, however, \emph{yoga} can be extended to coherence. Both coherence and cohesion are derived from the Latin verb \emph{cohaere-} (< \emph{con}- “with” \emph{haereō} “cling”) 'to cling together'.  In other translations of the \emph{tantrayukti}s (see \pageref{tantra-trans}), \emph{yoga} is variously translated as employment, arrangement, conjoiner, connecting, concomitance, uniting, union, rational linking, joining and so on. We preferred the term cohesion because the other options are either too narrow or too vague. `Employment' is rather \emph{prayoga}, not \emph{yoga}. `Rational linking' disregards the grammatical aspect of \emph{yoga}. `Conjoiner', `connecting', `union', `uniting' or `arrangement' are vague and they do not reflect the technical import of the term \emph{yoga}.   



\section{Characteristics of the Manuscript Transmission}

% Deepro

\newpage
\section{Translation}

\begin{translation}

\item [1] Now we shall explain the chapter called, “the enunciation of the
\se{tantrayukti}{logical methods of the system}.”

\item [3] There are thirty-two logical methods of the system. They are as 
follows: 
\smallskip


\begin{multicols}{2}
    \raggedright
\begin{enumerate}
\item \se{adhikaraṇa}{topic}
\item \se{\emph{yoga}}{cohesion}
\item \se{padārtha}{word meaning}
\item \se{hetvartha}{premise}
\item \se{samuddeśa}{mention}
\item \se{nirdeśa}{description}
\item \se{upadeśa}{prescription}
\item \se{apadeśa}{statement of reason}
\item \se{pradeśa}{indication}
\item \se{atideśa}{prediction}
\item \se{apavarga}{exception}
\item \se{vākyaśeṣa}{ellipis}
\item \se{arthāpatti}{implication}
\item \se{viparyaya}{contraposition}
\item \se{prasaṅga}{recontextualization}
\item \se{ekānta}{invariable statement}
\item \se{anekānta}{variable statement}
\item \se{pūrvapakṣa}{objection}
\item \se{nirṇaya}{determination}
\item \se{anumata}{consent}
\item \se{vidhāna}{itemization}
\item \se{anāgatāpekṣaṇa}{future reference}
\item \se{atikrāntāpekṣaṇa}{past reference}
\item \se{saṃśaya}{doubt}
\item \se{vyākhyāna}{explication}
\item \se{svasaṃjñā}{field-specific term}
\item \se{nirvacana}{interpretation}
\item \se{nidarśana}{illustration}
\item \se{niyoga}{compulsion}
\item \se{vikalpa}{option}
\item \se{samuccaya}{aggregation}
\item \se{ūhya}{deducible}
\end{enumerate}
\end{multicols}
\bigskip

\item [4] It is said about this, “what is the purpose of these methods?”
The answer is, “cohesion of a sentence and cohesion of
meaning”.\footnote{\Dalhana{6.65.4}{815} explained “cohesion of a
    sentence” as “connecting up a sentence that is not connected,” and
    “cohesion of meaning” as “clarifying  or making appropriate a meaning
    that is implied or inappropriate.”}

\item [5-6] There are \diff{two} verses about this:
  
\begin{sloka}
The logical methods of the system prohibit
statements employed by people who do not speak the truth.
%untrue and unsuitable statements. 
They also bring about the validity of one’s own
statements.  And they also clarify meanings that are stated back to
front, that are implicit, unclear and any that are partially stated.
\end{sloka}

\item [8] Among them, “\se{adhikaraṇa}{topic}” refers to the object, with 
reference to which statements are made, such as \se{rasa}{flavour} or 
\se{doṣa}{humour}.\footnote{The idea here is that “\emph{rasa}” may be the 
topic of a chapter, and statements in that chapter are all understood to be about 
that topic}

\item [9] “\se{yoga}{Cohesion}” is that by which a sentence
coheres, as when words that are in a reversed order, whether placed close
or apart, have their meanings unified.
%\dev{tailaṃ pibeccāmṛtavallinimbahaṃsāhvayāvṛkṣakapippalībhiḥ |\\
%siddhaṃ balābhyāñ ca sadevadāru hitāya nityaṃ galagaṇḍaroge ||\\}
\begin{quote}
    Sesame oil he should drink, with 
\gls{amṛtavalli}, 
\gls{nimba},
\diff{\gls{haṃsāhvayā}},
\gls{vṛkṣaka}, and
\gls{pippalī}
%with heart-leaved moonseed (Tinospora cordifolia), neem, the plant walking 
%maidenhair fern (Adiantum lunulatum or Adiantum philippense), long pepper, 
%heart-leaf sida, country marrow', and deodar.'' 
\\[2ex]
that is cooked with 
\gls{balā} and \gls{atibalā}, %two mallows
and
\gls{devadāru},
always for a benefit in the case of the disease goitre.
\end{quote}
In this verse, one ought to say, first, “one should drink
cooked\ldots.” However, the word “cooked” is used in the second
line.\footnote{The Nepalese version reads \dev{dvitīye pāde} which would
    properly mean the second quarter of the first line; the vulgate reads
    “third quarter” which seems more correct.} Unifying the meanings of words in
    this way, even though they are far apart, is construing. 
 

\item [10] 

The meaning that is conveyed in an \se{sūtra}{aphorism} or a word is 
called \se{padārtha}{word-meaning}. In other words, word-meaning is the 
meaning of one or more words. Word-meanings are unlimited. 

Where two or three meanings such as `fat,’ `sweat’ or `anointment’ appear
to be possible, the valid meaning is the one that construes with prior
and subsequent elements.\footnote{There is a dangling relative clause,
    \dev{yo 'rthaḥ}, in the Nepalese version that is avoided in the vulgate
    recension by the addition of \dev{sa grahītavyaḥ}.} For example, when it
    is said that, “We are going to explain the chapter on the
    \emph{veda}-origin” the mind may be confused about which “\emph{veda}”
    will be spoken about. \emph{Sāmaveda} and so on are the Vedas. Taking
    note of the prior and subsequent elements, the two roots \emph{vind}
    ”find” and \emph{vid} “know” have a single meaning. Subsequently, the
    understanding takes place that there is a wish to talk about the origin
    of āyurveda.  So that is the meaning of the word.\footnote{The Nepalese
        text here is hard to follow, and the vulgate has a significantly
        different reading. But the problem situation seems to be as follows.  The
        \SS\ opens with a statement saying that it will describe the “origin of
        the \emph{veda}” (\emph{vedotpatti}).  The problem is, what does this
        word “\emph{veda}” refer to?  Is it the Veda, as in Sāmaveda?  Or
        something derived from the roots \root vind or \root vid?  Context
        (“prior and subsequent elements”) can help us to know that “\emph{veda}”
        means only “\emph{āyurveda}” and that the \SS\ is talking about the
        origin of ayurveda, specifically.  This same issue is also addressed by 
        Ḍalhaṇa at \Su{1.1.1}{1}.}

\item [11] \se{hetvartha}{The sense of the cause} is a statement that is
a \se{sādhana}{premiss}.  For example, just as a lump of earth is
moistened by water, so a wound is moistened by substances like milk with
\gls{māṣa}.\footnote{The way this principle is expressed here seems to be
    describing the application of a general principle (water makes things
    wet) to a specific context. We can know the moistening of a wound because
    we know the more general case of moistening earth. However,
    etymologically, \dev{hetvartha} does not mean “analogy,” but rather,
    something like “purpose of the reason.”  The phrase “the sense of cause”
    that we have used leans on the use of the term in commentaries on the
    \emph{Aṣṭādhyāyī} (\emph{Kaumudī} on 2.3.23). The vulgate of the \SS\
    rewrites the principle, making it clearer that the principle means
    “clarification by analogy.”  Cf.\ also Cakrapāṇi's discussion at
    \Ca{Si.12.41}{736}, where he explained the principle as using an
    explanation from one situation to clarify another situation. Cf.\
    \emph{Arthaśāstra} 5.1.13 \citep[436]{oliv-2013}, which is also unclear.
    }\q{See also Ḍalhaṇa at \Su{1.1.1}{1}}
\item [12] A \se{samuddeśa}{mention} is a brief statement such as 
“\se{śalya}{spike}”.\footnote{Generally, \dev{śalya} refers to any painful 
foreign body embedded in the flesh that requires surgical removal.}

\item [13] A \se{nirdeśa}{description} is a detailed statement. For example, “in 
the body or exogenous”.\footnote{This is a reference to \Su{1.26.4}{121}
    where \dev{śalya} is described in more detail as being of two kinds.} 

\item [14] “\se{upadeśa}{Prescription}” refers to statements like ``it should be 
this way.'' For example, one should not stay awake at night; 
one should not sleep during the day.  

\item [15] “\sse{apadeśa}{statement of reason}Statement of reason” refers to 
statements like “this 
happens because of this.” For example, in the sentence “Sweet substances 
increase phlegm,” the reason is stated.\footnote{A techical term also in 
Nyāyaśāstra \citep[54]{jhal-1978}.}  

\item [16] Substantiation of the subject matter through past evidence is
“\se{pradeśa}{indication}.” For example, he pulled out Devadatta's
\se{śalya}{splinter}, therefore he will pull out Yajñadatta's.

\item [17] Substantiation of the subject matter through a future event is
“\se{atideśa}{prediction}.” For example, if his wind moves upwards, that
will cause him to have colic.”\footnote{A techical term also in
    Nyāyaśāstra \citep[6--7]{jhal-1978}.}

% got to here with DW

\item [18] A deviation after generalization is \se{apavarga}{exception}. For example, those afflicted by poison should not go through sudorific treatment other than the cases of poisoning by urinary worms.

\item [19] \se{vākyaśeṣa}{Ellipsis} refers to an unstated word that completes a sentence. For example, despite not mentioning the word 'person', when mentioning someone as 'the one having a head, hands, feet, flanks, and abdomen,' it's apparent that the reference is to a person. 

\item [20] \se{Implication}{arthāpatti} refers to an unstated idea that becomes evident through context. For example, when one said, “We will eat rice” it becomes evident from the context that he did not wish to drink gruel. 

\item [21] When there is the reversal of it it is \se{viparyaya}{contraposition}. For example, when it is said, "Weak, dyspneic, and fearful people are difficult to treat," the converse holds true: "Those who are strong and so on are easily treatable." 

\item [22] \se{prasaṅga}{Recontextualization} refers to a concept common to another section. For example, a concept belonging to another section is brought up by mentioning it repeatedly throughout. 

\item [23] \se{ekānta}{Invariable statement} is one that is stated with certainty. For example, \gls{trivṛt} causes purgation; \gls{madana} induces vomiting.

\item [24] \se{anekānta}{Variable statement} is one that is true in one way in 
some cases and in another way elsewhere. For example, some teachers identify 
the main element as substance, others as fluid, some as semen, and some as 
digestion.\q{See chapter 40 of Sūtrasthāna.}

\item [25]  A \se{pūrvapakṣa}{first point of view} is something stated
with certainty.  For example, how are the four types of diabetes caused
by wind incurable?\footnote{The adverb \dev{niḥsaṃśayam} is problematic:
    the example expresses a query or doubt, the opposite of certainty, which
    is answered in the next passage.  It would seem to make more sense to
    read something like \dev{yas tu sasaṃśayam abhidhīyate sa pūrvapakṣaḥ},
    but our manuscripts are unanimous in their reading.}

\item [26] Its answer is determination. For example, afflicting the body and 
trickling downwards, it creates urine mixed with fat, fatty tissues, and 
marrow.\q{vasā / medas / majjan} Thus, those caused by wind are incurable. 

\item [28] \sse{anumata}{consent}Consent refers to others' opinion that is not 
rejected. For example, when the assertor says that there are six flavours and 
that somehow gets accepted with affirmation, it is termed consent.

\item [29] \se{vidhāna}{Itemization} refers to sequentially ordered statements within a chapter. For example, the eleven lethal points of thigh are mentioned sequentially in a chapter.

\item [30] A statement like “Thus will be stated” is \se{anāgatāpekṣaṇa}{future 
reference} such as when he says in the \emph{Sūtrasthāna}, “I will mention it in 
the \emph{Cikitsāsthāna}.” 

\item [31] A statement like “Thus has been stated” is 
\se{atikrāntāpekṣaṇa}{past reference} such as when one says in the 
\emph{Cikitsāsthāna}, “As mentioned in the \emph{Sūtrasthāna}\ldots.” 

\item [32] An indication pointing to causes on both sides is
\se{saṃśaya}{doubt}. For example, a blow to
\sse{talahṛdaya}{sole-heart}\footnote{\dev{talahṛdaya} is one of the
    muscle-group of lethal points mentioned in \Su{3.6.7}{370}.}  is fatal,
    whereas cutting hands and feet is not fatal.

\item [33] An elaborate description is \se{vyākhyāna}{explication}. For example, 
the twenty-fifth entity, \sse{puruṣa}{person}, is being explicated here. Thus, no 
other Āyurvedic texts discuss  entities beginning with matters. \q{Does 
bhūtādi a compound or it means ahaṅkāra or ego?}

\item [34] \se{svasaṃjñā}{Field-specific term} is uncommon in other field of 
studies. The term used in one's own systems is called field-specific term, such as 
in this system, \sse{mithuna}{pair} denotes honey and ghee, and 
\sse{mithuna}{triad} denotes ghee, sesame oil and fat. 

\item [35] A customary potrayal is \se{nirvacana}{interpretation}. For example, one goes along the shade fearing heat. 

\item [36] Providing examples is \se{nidarśana}{illustration}. For example, just as fire spreads rapidly in a dry forest when accompanied by wind, a wound intensifies affected by wind, bile, and phlegm.  

\item [37] A statement like “This is the only way...” ...\se{niyoga}{compulsion}. For example, one should consume only a healthy diet.     

\item [39] A statement like “This and this\ldots” is
\se{vikalpa}{option}. For example, in the section on meat, the major ones
are blackbuck, deer, quail and partridge.\footnote{The example here
    matches \dev{samuccaya} (next text), not \dev{vikalpa}.  There seems to
    have been a metathesis of terms. \citet[1005, footnote
    6]{susr-trikamji1945} notes that this text and the next have been swapped
    in the Calcutta edition that includes Hārāṇacandra's commentary
    \volcite{2}{bhat-1910}, in the same way as in the Nepalese version.}

\item [38] A summarized statement is
\se{samuccaya}{aggregation}.\footnote{As stated in the previous footnote, the 
example here is of \dev{vikalpa}, not \dev{samuccaya}.
%    The term \dev{samāsavacana}, which
%   means more or less the same as \dev{saṃkṣepavacana}, has already been
%    used in tantrayukti \dev{samuddeśa}
    } For example, let there be rice
    with meat broth, rice with milk, or burley with ghee.
    
    \begin{quote}
        A meaningful reading of these two rules would be
    
    39 idaṃ vedaṃ veti vikalpaḥ / yathā rasodanaḥ kṣīrodanaḥ saghṛtā vā 
    yavāgūr bhavatv iti //
    
    38 saṃkṣepavacanaṃ samuccayaḥ / yathā māṃsavarge eṇahariṇalāvatittirāḥ 
    pradhānā iti
    \end{quote}
    

\item [40] What is not explicitly stated but can be understood through discernment is \se{ūhya}{deducible}. For example, in the section on rules of foods and drinks, four types of foods and drinks are mentioned— \se{bhakṣya}{masticable}, \se{bhojya}{edible}, \se{lehya}{suckable}, and \se{peya}{drinkable}. Thus, while four types are needed to be stated, two types are actually mentioned. Here it is deducible that in the section on foods and drinks, by specifically mentioning two types, the four types are also mentioned. Furthermore, a masticable item is not excluded from the category of food because it shares the same characteristic of solidity. A suckable item is not excluded from being classified as a drink because it shares the same characteristic of liquidity. Four types of aliments are rare. They are usually just twofold. Therefore, lord Dhanvantari says “Twofold is popular”.   


\end{translation}
