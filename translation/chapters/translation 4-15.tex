% !TeX root = incremental_SS_Translation.tex

\chapter{Cikitsāsthāna 15:  On Difficult Delivery}

% Vandana Lele

\section{Literature} 

Meulenbeld offered an annotated overview of this chapter on fetal
malpresentation and a bibliography of earlier scholarship to
2002.\fvolcite{IA}[271--272]{meul-hist}  \citeauthor{das-2003} made 
observations about the \se{aparā}{afterbirth} that is mentioned in  
\Su{4.15.17}{432}.\footcite[517]{das-2003}   \citeauthor{selb-2005} 
has explored gyencological 
narratives in ayurveda.\footcite{selb-2005,selb-2005b}

\section{Translation}

\begin{translation}
    

    \item [1]  And now we shall explain the difficult delivery medically treated.
    
    \item [3]   Nothing else is more difficult than the extraction of a
foetus since it has to be performed in the region of vagina, liver,
spleen, intestines and the uterus.  Actions like pushing up, pulling
down, cutting off, incising, removing, pressing and straightening
must be done using one hand, without hurting the foetus or the
pregnant woman, Therefore, having considered that and obtaining
permission, one should proceed with care.
    
    \item [4]  Eight types of the positions of difficult
foetus have earlier been mentioned briefly. Even if, in the natural birth process
also the large / wrong way of the head, shoulders or hips of a foetus
/ child cling firmly in the passage.
    
    \item [5]  In the case of a live foetus, the delivering ladies should
attempt to deliver it. And, during this process, they should be made
to hear the sacred verses repeatedly meant for expulsion of a foetus.
    
    \item [6]  O beautiful woman, may the divine nector and the moon
and the sun and Uccaiśravas reside in your house.
    
    \item [7]  O lady, may this nector extracted from the water release
this tiny foetus of yours. May the fire, wind, sun and Indra together
with the ocean bestow upon you the peace.

    
\end{translation}

