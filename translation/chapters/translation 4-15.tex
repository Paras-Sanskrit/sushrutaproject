% !TeX root = incremental_SS_Translation.tex

\chapter{Cikitsāsthāna 15:  On Difficult Delivery}

% Vandana Lele

\section{Literature} 

Meulenbeld offered an annotated overview of this chapter on fetal
malpresentation and a bibliography of earlier scholarship to
2002.\fvolcite{IA}[271--272]{meul-hist}  \citeauthor{das-2003} made 
observations about the \se{aparā}{afterbirth} that is mentioned in  
\Su{4.15.17}{432}.\footcite[517]{das-2003}   \citeauthor{selb-2005} 
has explored gyencological 
narratives in ayurveda.\footcite{selb-2005,selb-2005b}

\section{Translation}

\begin{translation}
    


    \item [1]  And now we shall explain the difficult delivery medically treated.
    
    \item [3]   Nothing else is more difficult than the extraction of a
foetus since it has to be performed in the region of vagina, liver,
spleen, intestines and the uterus.  Actions like pushing up, pulling
down, cutting off, incising, removing, pressing and straightening
must be done using one hand, without hurting the foetus or the
pregnant woman, Therefore, having considered that and obtaining
permission, one should proceed with care.
    
    \item [4]  Eight types of the positions of difficult
foetus have earlier been mentioned briefly. Even if, in the natural birth process
also the large / wrong way of the head, shoulders or hips of a foetus
/ child cling firmly in the passage.
    
    \item [5]  In the case of a live foetus, the delivering ladies should
attempt to deliver it. And, during this process, they should be made
to hear the sacred verses repeatedly meant for expulsion of a foetus.
  \begin{sloka}
        
    \item [6]  O beautiful woman, may the divine nectar and the moon
and the sun and Uccaiśravas reside icumbhalakan your house.
    
    \item [7]  O lady, may this nectar extracted from the water release
this tiny foetus of yours. May the fire, wind, sun and Indra together
with the ocean bestow upon you the peace.

  \end{sloka}

% 2023-10-12

\item[9] 

And, as mentioned before (3.10.16-20) the medicine should be
administered. In the case of a dead fetus, (the physician) having
inserted (his) hand lubricated with the \emph{dhanvaka} \q{?}, \emph{mṛttikā} 
– soil \q{?},
the \emph{śālmalī}- the \emph{seemul} and ghee into the vagina of a woman 
lying on her
back, whose thighs are bent with the elevated waist with the support of
the cloth of \emph{cumbhalaka} \q{?} should take away the fetus. In the case, 
the
fetus coming out with both the thighs, should be stretched out in a
normal way. If the fetus has reached with only one thigh, spreading out
its other thigh it should be taken out. If the fetus is coming out with
its buttocks portion, squeezing the buttocks upward, spreading the thighs
it should be taken out. A fetus having come in a transverse position like
an oblique (\dev{तिर्यक्चीनस्य} ?) iron club, lifting upward its half of
the lower part from behind, straightening its half of the upper part,
bringing it to the passage of vagina, it should be taken out. The last
two positions of the dead fetus cannot be accomplished. Thus, in this
state, instrument should be employed / surgery should be undertaken.

\item[10] 

But, the live fetus should not be torn apart in any case. As, the live
fetus may kill the mother and self soon.

\item[12]	

Next, assuring safety to the lady, cutting the head of the fetus with the
instrument that has disc on the top (\dev{मण्डलाग्र})or finger shaped
instrument(\dev{अङ्गुलिशस्त्र}); removing the skull, the fetus should be
taken out holding the forceps at its chest and armpit. If the head of the
fetus is not separated, the fetus should be drawn out from its orbital
regions or cheek (with the forceps); if the shoulders are stuck up in the
passage, the fetus should be taken out by cutting its arm / arms\q{(?)} at
the shoulder region; tearing the abdomen when bloated with wind just like
a stretched leather bag used for holding water, casting off the
intestine, the loosened fetus should be taken out. Or else, if its thighs
are adhered to the passage, the bones of the thighs should be cut and
fetus is removed.

\item[13]	The fetus is adhered to the passage from whichever its body part, the 
physician by separating that part should remove the fetus carefully and by all 
means the woman should be protected.

\item[14]
For, irritated wind causes different movements of the fetus. In this situation, the 
wise physician should act intelligently. 

\item[15]	And, the learned physician should not delay even for moment in 
removing the dead fetus as it kills mother in no time like a breathless animal.  

If impacted with hip, the hip bones should be cut and then delivered.

   
\end{translation}

