% !TeX root = incremental_SS_Translation.tex

\chapter{Cikitsāsthāna 15:  On Difficult Delivery}

% Vandana Lele

\section{Literature} 

Meulenbeld offered an annotated overview of this chapter on fetal
malpresentation and a bibliography of earlier scholarship to
2002.\fvolcite{IA}[271--272]{meul-hist}  \citet[517]{das-2003} made 
observations about the \se{aparā}{afterbirth} that is mentioned in  
\Su{4.15.17}{432}. 

\section{Translation}

\begin{translation}
    
    \item [1] And now we shall explain the difficult delivery medically treated.
    
    \item [3] Nothing else is more difficult than the extraction of a foetus here as that has to be performed in the region of vagina, liver, spleen, intestines and the uterus. With the contact of one hand, without hurting the foetus and the pregnant woman (the tasks such as) pushing up, pulling down, cutting off, incision, removing, pressing and straightening etc. (have to be performed). Therefore, having considered that (and) obtaining permission one should go ahead carefully.
    
\end{translation}

