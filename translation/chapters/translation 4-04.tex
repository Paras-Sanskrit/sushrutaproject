% !TeX root = incremental_SS_Translation.tex
% Paras

\chapter{Cikitsāsthāna 4:  On the Treatment of Wind 
    Diseases}

\subsection{Literature} 

Meulenbeld offered an annotated overview of this chapter and a bibliography
of earlier scholarship to 2002.\fvolcite{IA}[265--266]{meul-hist} 

\subsection{Translation}

\begin{translation}
    
    \item [1] 
    Now we shall describe the treatment of wind diseases.
    
    \item [2]

    \item [3]
    When the wind enters the stomach and one vomits as a result, one should sequentially administer the six-bearing (\dev{ṣaḍdharaṇa}) remedy with cool water for seven nights.\footnote{The vulgate has the reading \dev{chardayitvā} which means "after making (him) vomit". Thus, vomiting is a part of the treatment. Whereas in the H manuscript, vomiting is the symptom of the ailment that needs to be cured.}

    \item [4]
    The remedy constituting of \gls{citraka}, \emph{indrayavā}, \gls{pāṭhā}, \emph{kaṭukā}, \emph{ativiṣā}, and \gls{abhayā} cures serious diseases and is called the six-bearing (\dev{ṣaḍdharaṇa}).

    \item [5]
    When the wind has entered the abdomen (\dev{pakvāśaya}), one should treat it with evacuation of the bowels (\dev{virecana}) with an unctuous substance. One should also treat it with cleansing enemas and excessively salty foods.\footnote{In H, the reading \dev{prāsāḥ} should be read as \dev{prāśāḥ} for it to mean "foods". Otherwise, \dev{prāsāḥ} means "throwing/discharging" or "darts/spears".} 
    
\end{translation}
