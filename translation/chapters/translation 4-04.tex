% !TeX root = incremental_SS_Translation.tex
% Paras

\chapter{Cikitsāsthāna 4:  On the Treatment of Wind 
    Diseases}

\section{Literature} 

Meulenbeld offered an annotated overview of this chapter and a bibliography
of earlier scholarship to 2002.\fvolcite{IA}[265--266]{meul-hist} 

\section{Translation}

\begin{translation}
    
    \item [1] 
    Now we shall describe the treatment of wind diseases.
    
    \item [2]

    \item [3]
    When the wind enters the stomach and one vomits as a result, one should sequentially administer the six-bearing (\dev{ṣaḍdharaṇa}) remedy with cool water for seven nights.\footnote{The vulgate has the reading \dev{chardayitvā} which means "after making [him] vomit". Thus, vomiting is a part of the treatment. Whereas in the H manuscript, vomiting is the symptom of the ailment that needs to be cured.}

    \item [4]
    The remedy constituting of \gls{citraka}, \gls{indrayavā}, \gls{pāṭhā}, \gls{kaṭukā}, \gls{ativiṣā}, and \gls{abhayā} cures serious diseases and is called the six-bearing (\dev{ṣaḍdharaṇa}).

    \item [5]
    When the wind has entered the abdomen (\dev{pakvāśaya}), one should treat it with evacuation of the bowels (\dev{virecana}) using an unctuous substance. One should also treat it with cleansing enemas and excessively salty foods.\footnote{In H, the reading \dev{prāsāḥ} should be read as \dev{prāśāḥ} for it to mean "foods". Otherwise, \dev{prāsāḥ} means "throwing/discharging" or "darts/spears".}\q{This is a change we should make in the edition.} 

    \item [6]
    Once the wind has entered the lower belly, a cleansing enema is recommended. And, on the wind having entered the ears, etc., the wind-slayer sequence should be executed.\footnote{In the H manuscript reading "\dev{śrotādi}\ldots," there appears to be a double sandhi. %The base word (\emph{prātipadika}) for "ear" is "\dev{śrotas}". First, the "\dev{s}" disappears making the word "\dev{śrota}". Then the "\dev{a}" at the end combines with the "\dev{ā}" in the word "\dev{ādi}". Thus, the final form becomes "\dev{śrotādi}" as in H.
    See \textit{Nidānasthāna} Ch. 1 verse 12 for another example of double sandhi.
    Furthermore, the syllable in H after "\dev{cānila}" is not clear. It could be "\dev{hya}" or "\dev{hā}" or perhaps something else. The reading in the vulgate for this syllable is "\dev{hā}". Thus, the complete word becomes "\dev{anilahā}" which means "the slayer of wind". This makes proper sense in this verse. We have considered this reading ("\dev{anilahā}") for our translation.}\q{You need not give all the grammatical details about śrotādi.  Assume you are talking to knowledgeable Sanskrit scholars.}    

    \item [7]
    On the wind having entered the skin, flesh, and blood, one should rub oil on the body (\dev{abhyaṅga}), apply a poultice on the body (\dev{upanāha}), massage the body (\dev{mardana}), smear ointments on the body (\dev{ālepana}), and do blood-letting (\dev{asṛgvimokṣaṇa}). 

    \item[8]
    On the wind having entered the ligaments, joints, and bones, the wise [physician] should employ the application of an unctuous poultice (\dev{snehopanāha}), cauterization (\dev{agnikarma}), binding (\dev{bandhana}), and massage.

    \item [9]
    On the wind being concealed within the bones, it (wind) should be beaten by churning those body parts with hands. A strong physician should then insert a narrow tube within the bone and suck out the wind completely from the bone.\footnote{The H manuscript has the reading \dev{asthīni} which is the accusative plural form of \dev{asthi}. The accusative case does not make sense here.  The vulgate has the reading \dev{asthani}, the locative singular form of \dev{asthi}. This reading makes proper sense in the verse. Therefore, we have accepted the vulgate reading \dev{asthani} for translating this verse.} 
    % Ask Dominik Sir about the vipula types of anustup metre. Refer Ch. 106 of the Gita Darpana book of Svami Ramasukhadasa. It has many vipula variants, etc. of the anustup.        
    % My earlier comment in footnote: Also, the metre of this hemistich of the verse is not according to the \emph{anuṣṭup} metre. This gives us more reason to think that the reading \dev{asthīni} is corrupted.   

    \item[10]
    On the wind having entered the semen, one should perform the treatment for the defects of the semen.\footnote{Ḍalhaṇa comments \citep[421]{vulgate} that this treatment for the defects of the semen is mentioned [earlier] as the \dev{śukraśoṇitaśuddhi}, the purification of the semen and the blood. This is the \emph{Śārīrasthāna} Ch. 2, \dev{śukraśoṇitaviśuddhi}. The second hemistich of this verse is not a part of this sentence but is a part of the sentence in the next verse. That is because the remedies described in this hemistich are appropriate for the disease described in the first hemistich of the next verse.}

    \item[11]
    The intelligent physician should conquer the wind situated within the whole body by immersion, \textit{kuṭī}, \textit{karṣa}, \textit{prastara}, oil massage, enema, and blood-letting.\footnote{In H, the last syllable \dev{ni} of the compound word does not make sense. The vulgate has the compound word ending with \dev{bhiḥ} which makes proper sense. For making a meaningful translation, we have accepted the vulgate reading here. Furthermore, Ḍalhaṇa describes the treatments \textit{kuṭī}, \textit{karṣū}, and \textit{prastara} in his commentary in \citep[421]{vulgate}. Regarding blood-letting, he comments there that because the verse has the plural form \dev{sirāmokṣaiḥ}, five blood vessels have to be drained of blood if the wind is not pacified by oil massage, etc.} Or, in case of wind situated in one part of the body and contained within it, the intelligent physician should cure it with horns. 

    \item[12]
    On the wind having mingled with phlegm, bile, and blood, the physician should treat it with non-hostile remedies. However, on the wind being inactive, the physician should perform blood-letting many times. 

    \item[13]
    [On the wind being inactive], one should also lick the milk of the \emph{pancamūlī} accompanied with salt and \gls{āgāradhūma}\footnote{\emph{Āgāradhūma} seems to be a plant as seen in Monier Williams' Sanskrit dictionary.} mixed with oil, and one should indeed consume meat soup made sour with fruit.\footnote{The vulgate reading \dev{dihyāt} (should apply) totally changes the meaning.}

    \item[14-15]
    Or, one should consume cereal soup with a good amount of ghee, or the food that is beneficial and that curtails the wind. However, \gls{kākolī}, etc.\footnote{For grammatical accuracy, there needs to be a \emph{visarga} at the end of the word \dev{kākolyādi}.} with a wind-removing remedy combined with all sour substances and with the meat from a water body along with lots of unction, lukewarm\footnote{Perhaps \dev{sukhoṣṇam} is an indeclinable. But, it could also be a grammatical inaccuracy where it should have a \textit{visarga} at the end: \dev{sukhoṣṇaḥ}.} and salty, is well known as \textit{Sālvala}.    

    \item[16ab]
    For patients with diseases of the wind, one should always apply this (\textit{sālvala}) as a poultice.
    
    \item[16cd-18ab]
     Whether a body part has become contracted or bent, is troubled by a [wind] disease, or has become numb, one should tightly bind it with a long strap made of tree bark, cloth, or wool [after applying the \textit{sālvala} poultice]. Or, after massaging the affected body part and applying the \textit{śālvala}\footnote{This seems to be the correct spelling as against the unclarity in the earlier verses.} poultice on it, one should insert it into a sack made of the hide of a cat, mongoose, \textit{udra}\footnote{some aquatic animal}, or deer.

    \item[18cd-19]
    Vomiting and \textit{nasya} done under the supervision of an expert physician alleviates the wind that has entered the chest, loins, shoulders, or the nape of the neck. \textit{Śirobasti} and blood-letting alleviate the wind situated in the head. 

    \item[20-21ab]
    In that (\textit{śirobasti}), the oil should be held carefully for a duration of one thousand \textit{mātrās}. Enema (\dev{basti}) alone curtails the wind that is situated throughout the whole body or in one part. This is just as the wind [curtails] its force.\footnote{The last four words in H, \dev{tasya vegam ivānilaḥ} do not make sense in the context.} 

    \item[21cd-26]
    Oils, perspiration, oil massage, enema, unctuous purging of the bowels, \textit{śirobasti}, oiling the head, unctuous smoke, gargling with lukewarm water, \textit{nasya}, unctuous paste\q{Perhaps \textit{kalka} here could also mean the \textit{Terminalia Bellerica} (\dev{vibhītaka}).}, milks, meats\footnote{The plural indicates milk and meat from various animals.}, soups, oils\footnote{This is the second occurrence of the word \dev{snehāḥ} in this sentence. This seems to be an anomaly.}, any unctuous substance, unctuous and salty meals that are made sour by fruits, bathing with lukewarm water, massages, saffron, \gls{aguru}, \gls{patra}, \gls{kuṣṭha}, \gls{elā}, \gls{tagara}, garments made of silk, wool, and fur, soft cotton garments, inner rooms with sunlight, no wind flow, and a soft bed, taking the warmth of fire, and celibacy, etc. are to be collectively employed for patients with wind diseases.    

    \item[27]
    One should take \textit{akṣa} quantities of unguent pastes\footnote{\dev{kalka} also means an unguent paste. Refer to Apte's dictionary.} of \gls{trivṛt}\footnote{In H, perhaps it should have been \dev{trivṛd} instead of \dev{tṛvṛt}}, \gls{dantī}, \gls{śaṅkhinī}, \gls{suvarṇṇakṣīrī}, \gls{triphalā}, and \gls{viḍaṅga}, a \gls{bilva} fruit equivalent measure of \gls{tilvaka}-root and \gls{kampilya}, two \textit{pātra} quantities of both \textit{triphalā-}decoction\footnote{\dev{triphalārasa} is here taken to mean a decoction of \textit{triphalā}.} and yogurt, and one \textit{pātra} measure of ghee.\footnote{The exact measurements of \textit{akṣa} and \textit{pātra} are given in Ḍalhaṇa's commentary in \cite[422]{vulgate}.} One should mix these ingredients all at once and cook the mixture properly. This (resultant) is \gls{tilvaka}-ghee. Unctuous purging of bowels is prescribed for treating wind disorders.\footnote{It should be understood here that the unctuous substance to be used for purging the bowels is the \gls{tilvaka}-ghee.}\\
    This procedure of making \gls{tilvaka}-ghee should also be referred for making \gls{aśoka}-ghee and \gls{ramyaka}-ghee.\footnote{\dev{aśoka} and \dev{ramyaka} are the Ashoka and Chinaberry respectively.}

    \item[28]
    One should collect the wooden logs of the instruments that have been used for a long time for extracting oil from sesame seeds. One should then have them chopped into very tiny pieces and then pound those pieces. Next, one should put them in a big vessel, submerge them in water, and boil them. Thereafter, one should collect the oil from the surface of the water with a goblet or by hand. Thereafter, one should properly cook wind-alleviating herbs with this oil that was effectively cooked.\footnote{In H, the word \dev{dantapratīvāyaṃ} in the compound word \dev{vātaghnauṣadhadantapratīvāyaṃ} does not appear to make sense. Perhaps the syllable \dev{ya} should be \dev{pa}, thus making the word \dev{pratīvāpaṃ} that refers to an admixture of substances to medicines either during or after decoction. Refer to Monier-Williams's Sanskrit dictionary.} This is the \textit{anutaila} (\dev{anutaila})\footnote{The \dev{n} should be read \dev{ṇ}.} that is mentioned in wind disorders. It is called \textit{anutaila} because it is produced from tiny oily objects.\footnote{The word \dev{anu} in the compound word \dev{anutailadravyebhyaḥ} should be read \dev{aṇu}.} 

    \item[29]
    Alternatively, one should burn a great amount of \gls{mahāpañcamūlī}-wood on the ground for one night. When the fire gets extinguished the ash should be removed. Then, the ground that is relieved of the fire should be soaked with a hundred pots of oil cooked with \gls{vidāri}, \gls{gandha}, and other herbs, and left in that condition for one night. Thereafter, one should take all the earth that is oily\footnote{In H, the word \dev{yāvan} should have been \dev{yāvān}.} in a big vessel and totally cover it with water.\footnote{The reading in H, \dev{kaṭāhebhyaḥ siṃcet}, does not make sense here. Thus, we have accepted the vulgate reading \dev{kaṭāhe 'bhyāsiṃcet} for the translation.} The oil that rises up in that vessel should be taken out with both hands and kept nicely covered. Thereafter, one should properly cook that oil for as long as possible\footnote{The phrase "\dev{yāvatā kālena śaknuyāt paktum}" appears as a part of a new sentence in H. But, we should take it to be a part of the earlier sentence for it to make proper sense.} with one thousand parts of each of the following---a decoction of wind-alleviating herbs, meat soup, milk, and \textit{kāñjika}\footnote{Ḍalhaṇa comments \citep[423]{vulgate} that the word \dev{amla} here means \dev{kāñjika} which is the water drained after boiling rice and is a little fermented. Refer Monier Willams's Sanskrit Dictionary.}---and thus prepare the \textit{sahasra-pāka} (that which is cooked with thousands). The admixture added to the oil contains the \textit{hemavata} herbs\footnote{The word should be \dev{haimavatāḥ} as in the vulgate. It means "the herbs of the snowy mountains". Ḍalhaṇa comments \citep[423]{vulgate} that \dev{haimavatāḥ} refers to the herbs that grow in the northern region.}, herbs of the southern region, \gls{aśvagandhā}, and other wind-alleviating herbs.\\
    While the oil is being cooked, conchshells should be blown loudly, umbrellas should be held, huge drums should be resounded, and whisk fans should be waved.\footnote{These activities are a symbolic way of showing reverence.} Thereafter, the perfectly cooked oil should be poured into a golden or silver pot and stored. This \textit{sahasra-pāka} is the oil possessing undiminishing potency and is fit for kings.\\
    Thus, that which is cooked with a thousand parts is called \textit{sahasra-pāka}.

    \item[30]
    One should collect fresh leaves of \gls{gandharvahasta}, \gls{aṭa}, \gls{ruṣkaka}, \gls{muṣkaka}, \gls{naktamāla}, \gls{pūtikā}, and \gls{citraka}.\footnote{In H, the ending \dev{nām} should be \dev{ṇām} due to sandhi.} These leaves should be completely pounded along with salt in a mortar. This mixture should be put in a pot filled with oil\footnote{\dev{snehaghaṭa} can also mean a pot filled with ghee}. It (pot) should be smeared\footnote{The H or vulgate do not specify with words that it is the pot to be smeared. But, it is to be understood.} with cow-dung. Thereafter, the pot should be heated.\footnote{The word \dev{dāhayet} usually refers to burning, but sometimes it can refer to heating.} This (resultant) is the \textit{patra-lavaṇa} (leaf-salt) that is mentioned in wind disorders. 

    \item[31]
    In the same way, one should pound the stalks of \gls{snuhā}\q{Euphorbia Antiquorum (Antique spurge)} and eggplants smeared with salt and fill a pot with it.\footnote{In H, there should be a \textit{visarga} after \dev{lavaṇā}.} In that pot, one should add ghee, oil, fat, and marrow. Then, one should smear it\footnote{As earlier, the pot should be smeared with cow-dung.} and heat it as earlier. This (resultant) is the \textit{sneha-lavaṇa} (fat-salt) that is mentioned in wind disorders.

    \item[32]
    One should collect the fresh fruits, roots, leaves, and branches of all the twenty [herbs]: \gls{gaṇḍīra}, \gls{palāśa}, \gls{kuṭaja}, \gls{bilva}, \gls{arka}, \gls{snuhā}, \gls{pāmārga}, \gls{pāṭalī}, \gls{pāribhadraka}, \gls{nādeyī}, \gls{kṛṣṇagandhā}, \gls{nīpa}, \gls{nirdahī}, \gls{ānyāṭa}, \gls{rūṣaka}, \gls{naktamāla}, \gls{pūtīka}, \gls{bṛhatī}, \gls{kaṇṭakārikā}, \gls{bhallātaka}, \gls{aśoka}, \gls{vaijayantī}. One should then mix them with salt and heat them as earlier.\footnote{It is to be understood that all these fresh branches, leaves, fruits, and roots of the herbs should be completely pounded together with salt. The mixture should then be put into a pot filled with oil or ghee. The pot should be smeared with cow-dung and then heated.} The oil on top should be poured out completely with the salty mixture intact [at the bottom]. This mixture should be cooked thoroughly. The admixture added to it consists of \gls{pippalī}, etc. This (resultant) is the salt called \textit{kalyāṇaka} that is mentioned in wind disorders and in meals and drinks for the patients troubled by \textit{plīhāgnisaṃga}, indigestion, loss of appetite, and piles.\\ 


    Thus ends the fourth chapter on the treatment of wind diseases. 
    
\end{translation}
