% !TeX root = incremental_SS_Translation.tex
% Paras

\chapter{Cikitsāsthāna 4:  On the Treatment of Wind 
    Diseases}

\section{Literature} 

Meulenbeld offered an annotated overview of this chapter and a bibliography
of earlier scholarship to 2002.\fvolcite{IA}[265--266]{meul-hist} 

\section{Translation}

\begin{translation}
    
    \item [1] 
    Now we shall describe the treatment of wind diseases.
    
    \item [2]

    \item [3]
    When the wind enters the stomach and one vomits as a result, one should sequentially administer the six-bearing (\dev{ṣaḍdharaṇa}) remedy with cool water for seven nights.\footnote{The vulgate has the reading \dev{chardayitvā} which means "after making (him) vomit". Thus, vomiting is a part of the treatment. Whereas in the H manuscript, vomiting is the symptom of the ailment that needs to be cured.}

    \item [4]
    The remedy constituting of \gls{citraka}, \gls{indrayavā}, \gls{pāṭhā}, \gls{kaṭukā}, \gls{ativiṣā}, and \gls{abhayā} cures serious diseases and is called the six-bearing (\dev{ṣaḍdharaṇa}).

    \item [5]
    When the wind has entered the abdomen (\dev{pakvāśaya}), one should treat it with evacuation of the bowels (\dev{virecana}) using an unctuous substance. One should also treat it with cleansing enemas and excessively salty foods.\footnote{In H, the reading \dev{prāsāḥ} should be read as \dev{prāśāḥ} for it to mean "foods". Otherwise, \dev{prāsāḥ} means "throwing/discharging" or "darts/spears".}\q{This is a change we should make in the edition.} 

    \item [6]
    On the wind having entered the lower belly, a cleansing enema is ordained. And, on the wind having entered the ears, etc., the wind-slayer sequence should be executed.\footnote{In the H manuscript reading "\dev{śrotādi}\ldots," there appears to be a double sandhi. The base word (\emph{prātipadika}) for "ear" is "\dev{śrotas}". First, the "\dev{s}" disappears making the word "\dev{śrota}". Then the "\dev{a}" at the end combines with the "\dev{ā}" in the word "\dev{ādi}". Thus, the final form becomes "\dev{śrotādi}" as in H. Also see Nidānasthāna Ch. 1 verse ?? for another example of double sandhi.
    Furthermore, the syllable in H after "\dev{cānila}" is not clear. It could be "\dev{hya}" or "\dev{hā}" or perhaps something else. The reading in the vulgate for this syllable is "\dev{hā}". Thus, the complete word becomes "\dev{anilahā}" which means "the slayer of wind". This makes proper sense in this verse. We have considered this reading ("\dev{anilahā}") for our translation.}\q{You need not give all the grammatical details about śrotādi.  Assume you are talking to knowledgeable Sanskrit scholars.}    

    \item [7]
    On the wind having entered the skin, flesh, and blood, one should rub oil on the body (\dev{abhyaṅga}), apply poultice on the body (\dev{upanāha}), massage the body (\dev{mardana}), smear ointments on the body (\dev{ālepana}), and do blood-letting (\dev{asṛgvimokṣaṇa}). 

    
\end{translation}
