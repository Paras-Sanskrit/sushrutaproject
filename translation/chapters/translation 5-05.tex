% !TeX root = ../incremental_SS_Translation.tex
\chapter{Kalpasthāna 5: Therapy for those bitten by Snakes}

\section{Introduction} 

\section{Literature}

A brief survey of this chapter's contents and a detailed assessment of
the existing research on it to 2002 was provided by
Meulenbeld.\footnote{\volcite{IA}[294--295]{meul-hist}. In addition to the
    translations mentioned by \tvolcite{IB}[314--315]{meul-hist}, a translation
    of this chapter was included in \volcite{3}[35--45]{shar-1999}.} 
    
\section{Translation}

\begin{translation}
    \item [1]
    Now we shall explain the \se{kalpa}{procedure} that is the therapy for 
    someone bitten by a snake.\footnote{On \dev{kalpa}, see note 
    \ref{arunadatta:kalpa}.}
    
    \item[3] For a person bitten on a limb by any snake, one should
first of all make a strong binding, at four fingers measure above the
bite.\footnote{Application of a tourniquet is deprecated by
    modern establishment medicine, which relies on antivenom medications
    \citep[e.g.,][150--151 et passim in the literature]{pill-2013}.
    
    The vulgate introduces the word \dev{ariṣṭā} at this point.  This may be a 
    borrowing fro \Ca{Ci.23.251cd}{582}.}
    
    \item[4]
    
    % got to here
    
    %%%%%%%%%%%%
    
    \item[34] \footnote{After this verse, the vulgate text adds twelve
    verses, 35--46, that do not appear in the Nepalese version.}
    
     \item[78] \footnote{After this verse, the vulgate text adds five
        verses, 79--83, that do not appear in the Nepalese version.}
\end{translation}    