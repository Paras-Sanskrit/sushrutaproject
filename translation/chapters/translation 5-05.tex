% !TeX root = ../incremental_SS_Translation.tex
   
\chapter{Kalpasthāna 5: Therapy for those Bitten by Snakes}

\section{Introduction} 

\section{Literature}

A brief survey of this chapter's contents and a detailed assessment of
the existing research on it to 2002 was provided by
Meulenbeld.\footnote{\volcite{IA}[294--295]{meul-hist}. In addition to the
    translations mentioned by \tvolcite{IB}[314--315]{meul-hist}, a translation
    of this chapter was included in \volcite{3}[35--45]{shar-1999}.} 
    
    
    \newpage
    
\section{Translation}

\emph{Passage numbers refer to the canonical numbering of the vulgate edition  
\citep{vulgate}.
}
\begin{translation}
    \item [1]
    Now we shall explain the \se{kalpa}{procedure} that is the therapy for 
    someone bitten by a snake.\footnote{On \dev{kalpa}, see note 
    \ref{arunadatta:kalpa}.}
    
    \item [2]  \ \cbdelete 
       
    \item[3] For a person bitten on a limb by any snake, one should first
of all make a strong binding, at four fingers measure above the
bite.\footnote{Application of a tourniquet is deprecated by modern
    establishment medicine, which relies on antivenom medications
    \citep[e.g.,][150--151 et passim in the literature]{pill-2013}.
    
    The vulgate introduces the word \dev{ariṣṭā} at this point.  This may be a 
    borrowing from \Ca{Ci.23.251cd}{582}.}
    
    
    \item[4]
    
    Poison does not move around into the body if it is prevented by
bandages (\emph{ariṣṭā})\sse{ariṣṭā}{bandage} or by any other soft
items of \se{plota}{cloth}, \se{carmānta}{leather} or
bark.\footnote{It is hard to translate the word \dev{ariṣṭā}
    otherwise than “bandage,” as referred to by \dev{badhnīyāt} in the
    previous verse, and apparently similar to items of cloth etc., and
    called a \dev{bandha} in the next verse.  But in general Sanskrit
    literature, including medical literature, the word (in masc.\ gender)
    means either “an alcoholic tonic” or “an omen of death,”
    (\Su{1.30.3}{137}), or is a plant name.  This raises a question mark
    over its unique meaning in the present context.  The \AH\
    (\Ah{Utt.36.42cd}{910}) seems to be a gloss on \dev{ariṣṭā}, saying
    “An expert in mantras may bind using a braid made of silk etc.,
    empowered with mantras” (see also \Su{5.5.8}{575}).}
    
\item[5] Where a \se{bandha}{bandage} is not suitable, one should
\diff{raise the bite up} and then cauterize it.\footnote{The vulgate
    reads \dev{utkṛtya} “having excised” rather than translate \dev{uddhṛtya}
    “having raised up.”} Suction, cutting and cauterizing are recommended in
    all cases.

\item[6] Suction will be good after filling the mouth with
\diff{\se{pāṃśu}{earth}}.\footnote{The vulgate recommends cloth, not
    earth (\Su{5.5.6}{574}).}  Alternatively, the snake should be bitten
    \diff{by the person who knows} that they have just been
    bitten.\footnote{The syntax is odd here, and the vulgate has removed the
        difficulties. \Dalhana{5.5.6}{574} noted that one should hold the snake
        firmly and give a good bite to its head and tail
        (\dev{hastābhyāmupasaṃgṛhya pucche vaktre ca sarpaḥ samyag daṣṭavyaḥ}). 
        Our colleague Dr Madhu K. Paramesvaran reports that this procedure is
        known in Malayalam \emph{viṣavaidya} treatises and is practiced in
        Kerala, though rarely: “this practice has been described as one of the
        first-response cares for snakebite in most of the Malayalam texts of
        Vishavaidya. I have never seen this happening in real life and my
        teachers used to consider it to be a method (albeit a bit outrageously
        dangerous) for self-reassurance by the patient.” \citep{para-2023}. Cf.\
        the Viṣavaidya text edited by \citet{maha-1958}.}


\item[7] 

Now, one should in no way cauterize someone bitten by a Maṇḍalin. Because
of the over-abundance of \se{pittaviṣa}{poison in the bile}, that bite will
\diff{be lethal} as a result of cauterization.\footnote{Verses 5.4.29,
    and 37 above note that the venom of Maṇḍalins particularly irritates the
    bile.}

\subsection{[The application of mantras]}

\item [8]

An expert in mantras should tie on a \se{ariṣṭā}{bandage} too, with
mantras.  But they say that a bandage that is tied on with cords and so
on causes the \diff{poison to be purified}.\footnote{\Dalhana{5.5.8}{575}
    clarified that on the one hand the bandage must be accompanied with
    mantras, but on the other hand, it may also be used without mantras.  The
    verse seems to put two points of view.}

\item[9]

Mantrās prescribed by gods and \se{brahmarṣi}{holy sages}, that are
imbued with truth and \se{tapas}{religious power} are inexorable and they
rapidly destroy intractable poison.

\item [10]

Drugs cannot eliminate poison as quickly as the application of mantras
imbued with \se{tapas}{religious power} and imbued with truth,
\se{brahma}{holiness} and religious power.\footnote{\Dalhana{5.5.10}{575}
    noted that mantras like “kurukullā” and “bheruṇḍā” are explained in other
    treatises and therefore not explained further in his commentary. These
    two mantras are the names of tantric Śaiva and Buddhist goddesses. For a
    study on this specific subject see \citet{slou-2016b}. \volcite{IIB}[151,
    n.\,344]{meul-hist} provides a bibliography to 2002 of studies on
    Kurukullā, who is mentioned in Māhuka's \emph{Haramekhalā}, and
    \cite[30--34]{meul-2008b} includes discussion of Bheruṇḍa as a bird, with
    related terms.} 
    % and  further  studies such as \cite[ch.\,22]{shaw-2006}.

\item [11] The mantras should be received by a person who is
abstaining from women, meat and \se{madhu}{mead}, who has a
\diff{restricted} diet, and who is pure and lying on a bed of \gls{kuśa}.

\item [12]

For the mantras to be successful, one should diligently worship the
\se{devatā}{deity} with perfume, garlands, and \se{upahāra}{oblations},
as well as \se{bali}{sacrificial offerings}, and with \se{japa}{mantra
    repetition} and rituals.\footnote{\Dalhana{5.5.12}{575} noted that
    \dev{upahāra} includes incense, while \dev{bali} refers to sacrifice
    with an animal (\dev{sapaśunaivedya}).}


\item [13]

But mantras pronounced illicitly or that are deficient in
\se{svara}{accents} and letters do not give success.  So
\se{agada}{antitoxic} procedures need to be employed.

\subsection{[Blood letting]}

\item [14]

A skilled physician should puncture \diff{a \se{sirā}{duct} which is
\se{śākhāśrayā}{located on the limb}, and comes from the bite and 
the general area.} If the poison has spread, one on the forehead should be
pierced.

\item [15] 

\diff{The blood being drawn out draws away all the poison}.\footnote{The
    Nepalese version uses a present passive participle construction here,
    that is less common than the vulgate's locative absolute.  The Nepalese 
    version states that it is the blood coming out of the patient that carries away 
    the venom; the vulgate text says merely that the venom emerges while the 
    blood comes out.} Therefore one
    should cause blood to flow, for that is his very best procedure.

\item [16]  

After scarifying the area around the bite, one should smear it with
\se{agada}{antidotes} and sprinkle it with water infused with
\gls{candana} and \gls{uśīra}.\footnote{\dev{pracchāna} is the second of
    the two methods of blood letting  described in \SS\ \Su{1.14.25}{64}
    (this verse does not appear in the Nepalese version of the \SS).}

\item [17]

One should make him drink \se{agada}{antidotes} including milk, honey and 
ghee. If they are unavailable, the earth of black ants can be good. 


\item [18]

Alternatively, he should consume \gls{kovidāra}, \gls{śirīṣa} and
\gls{arka} with \gls{kaṭabhī}.

    
    %%%%%%%%%%%%
    \strut
    \bigskip
    
    \item[34] \footnote{After this verse, the vulgate text adds twelve
    verses, 35--46, that do not appear in the Nepalese version.}
    
     \item[78] \footnote{After this verse, the vulgate text adds five
        verses, 79--83, that do not appear in the Nepalese version.}
\end{translation}    