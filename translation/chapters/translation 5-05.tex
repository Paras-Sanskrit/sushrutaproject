% !TeX root = ../incremental_SS_Translation.tex
   
\chapter{Kalpasthāna 5: Therapy for those Bitten by Snakes}

\section{Introduction} 

\section{Literature}

A brief survey of this chapter's contents and a detailed assessment of
the existing research on it to 2002 was provided by
Meulenbeld.\footnote{\volcite{IA}[294--295]{meul-hist}. In addition to the
    translations mentioned by \tvolcite{IB}[314--315]{meul-hist}, a translation
    of this chapter was included in \volcite{3}[35--45]{shar-1999}.} 
    
    
    \newpage
    
\section{Translation}

\emph{Passage numbers refer to the canonical numbering of the vulgate edition  
\citep{vulgate}.
}

\begin{translation}
    \item [1]
    Now we shall explain the \se{kalpa}{procedure} that is the therapy for 
    someone bitten by a snake.\footnote{On \dev{kalpa}, see note 
    \ref{arunadatta:kalpa}.}
    

       
    \item[3] For a person bitten on a limb by any snake, one should first
of all make a strong binding, at four fingers measure above the
bite.\footnote{Application of a tourniquet is deprecated by modern
    establishment medicine, which relies on antivenom medications
    \citep[e.g.,][150--151 et passim in the literature]{pill-2013}.
    
    The vulgate introduces the word \dev{ariṣṭā} at this point.  This may be a 
    borrowing from \Ca{Ci.23.251cd}{582}.}
    
    
    \item[4]
    
    Poison does not move around into the body if it is prevented by
bandages (\emph{ariṣṭā})\sse{ariṣṭā}{bandage} or by any other soft
items of \se{plota}{cloth}, \se{carmānta}{leather} or
bark.\footnote{It is hard to translate the word \dev{ariṣṭā}
    otherwise than “bandage,” as referred to by \dev{badhnīyāt} in the
    previous verse, and apparently similar to items of cloth etc., and
    called a \dev{bandha} in the next verse.  But in general Sanskrit
    literature, including medical literature, the word (in masc.\ gender)
    means either “an alcoholic tonic” or “an omen of death,”
    (\Su{1.30.3}{137}), or is a plant name.  This raises a question mark
    over its unique meaning in the present context.  The \AH\
    (\Ah{Utt.36.42cd}{910}) seems to be a gloss on \dev{ariṣṭā}, saying
    “An expert in mantras may bind using a braid made of silk etc.,
    empowered with mantras” (see also \Su{5.5.8}{575}).  On problems that 
    can arise from tying a bandage too tightly, see \Su{5.5.56}{577} below.}
    
\item[5] Where a \se{bandha}{bandage} is not suitable, one should
\diff{raise the bite up} and then cauterize it.\footnote{The vulgate
    reads \dev{utkṛtya} “having excised” rather than translate \dev{uddhṛtya}
    “having raised up.”} Suction, cutting and cauterizing are recommended in
    all cases.

\item[6] Suction will be good after filling the mouth with
\diff{\se{pāṃśu}{earth}}.\footnote{The vulgate recommends cloth, not
    earth (\Su{5.5.6}{574}).}  Alternatively, the snake should be bitten
    \diff{by the person who knows} that they have just been
    bitten.\footnote{The syntax is odd here, and the vulgate has removed the
        difficulties. \Dalhana{5.5.6}{574} noted that one should hold the snake
        firmly and give a good bite to its head and tail
        (\dev{hastābhyāmupasaṃgṛhya pucche vaktre ca sarpaḥ samyag daṣṭavyaḥ}). 
        Our colleague Dr Madhu K. Paramesvaran reports that this procedure is
        known in Malayalam \emph{viṣavaidya} treatises and is practiced in
        Kerala, though rarely: “this practice has been described as one of the
        first-response cares for snakebite in most of the Malayalam texts of
        Vishavaidya. I have never seen this happening in real life and my
        teachers used to consider it to be a method (albeit a bit outrageously
        dangerous) for self-reassurance by the patient.” \citep{para-2023}. Cf.\
        the Viṣavaidya text edited by \citet{maha-1958}.}


\item[7] 

Now, one should in no way cauterize someone bitten by a Maṇḍalin. Because
of the over-abundance of \se{pittaviṣa}{poison in the bile}, that bite will
\diff{be lethal} as a result of cauterization.\footnote{Verses 5.4.29,
    and 37 above note that the venom of Maṇḍalins particularly irritates the
    bile.}

\subsection{The application of mantras}

\item [8]

An expert in mantras should tie on a \se{ariṣṭā}{bandage} too, with
mantras.  But they say that a bandage that is tied on with cords and so
on causes the \diff{poison to be purified}.\footnote{\Dalhana{5.5.8}{575}
    clarified that on the one hand the bandage must be accompanied with
    mantras, but on the other hand, it may also be used without mantras.  The
    verse seems to put two points of view.}

\item[9]

Mantrās prescribed by gods and \se{brahmarṣi}{holy sages}, that are
imbued with truth and \se{tapas}{religious power} are inexorable and they
rapidly destroy intractable poison.

\item [10]

Drugs cannot eliminate poison as quickly as the application of mantras
imbued with \se{tapas}{religious power} and imbued with truth,
\se{brahma}{holiness} and religious power.\footnote{\Dalhana{5.5.10}{575}
    noted that mantras like “kurukullā” and “bheruṇḍā” are explained in other
    treatises and therefore not explained further in his commentary. These
    two mantras are the names of tantric Śaiva and Buddhist goddesses. For a
    study on this specific subject see \citet{slou-2016b}. \volcite{IIB}[151,
    n.\,344]{meul-hist} provides a bibliography to 2002 of studies on
    Kurukullā, who is mentioned in Māhuka's \emph{Haramekhalā}, and
    \cite[30--34]{meul-2008b} includes discussion of Bheruṇḍa as a bird, with
    related terms.} 
    % and  further  studies such as \cite[ch.\,22]{shaw-2006}.

\item [11] The mantras should be received by a person who is
abstaining from women, meat and \se{madhu}{mead}, who has a
\diff{restricted} diet, and who is pure and lying on a bed of \gls{kuśa}.

\item [12]

For the mantras to be successful, one should diligently worship the
\se{devatā}{deity} with perfume, garlands, and \se{upahāra}{oblations},
as well as \se{bali}{sacrificial offerings}, and with \se{japa}{mantra
    repetition} and rituals.\footnote{\Dalhana{5.5.12}{575} noted that
    \dev{upahāra} includes incense, while \dev{bali} refers to sacrifice
    with an animal (\dev{sapaśunaivedya}).}


\item [13]

But mantras pronounced illicitly or that are deficient in
\se{svara}{accents} and letters do not give success.  So
\se{agada}{antitoxic} procedures need to be employed.

\subsection{Blood letting}

\item [14]

A skilled physician should puncture \diff{a \se{sirā}{duct} which is
\se{śākhāśrayā}{located on the limb}, and comes from the bite and 
the general area.} If the poison has spread, one on the forehead should be
pierced.

\item [15] 

\diff{The blood being drawn out draws away all the poison}.\footnote{The
    Nepalese version uses a present passive participle construction here,
    that is less common than the vulgate's locative absolute.  The Nepalese 
    version states that it is the blood coming out of the patient that carries away 
    the venom; the vulgate text says merely that the venom emerges while the 
    blood comes out.} Therefore one
    should cause blood to flow, for that is his very best procedure.

\item [16]  

After \se{pracchāna}{incising} the area around the bite, one should smear
it with \sse{agada}{antidote}antidotes and sprinkle it with water infused with
\gls{candana} and \gls{uśīra}.\footnote{\dev{pracchāna} is the second of
    the two methods of blood letting  described in the vulgate text of the \SS\ at 
    \Su{1.14.25}{64}; this verse does not appear in the Nepalese version of the 
    \SS.\label{pracchana}}\label{5.5.16}

\subsection{Internal medications}

\item [17]

One should make him drink various \sse{agada}{antidote}antidotes together
with milk, honey and ghee. If they are unavailable, the earth of black
ants can be good.\footnote{This refers to earth taken from an anthill. 
    In South Asia, there is a long tradition of considering such earth to be
    beneficial and even holy \citep[e.g.,][]{irwi-1982}.}


\item [18]

Alternatively, he should consume \gls{kovidāra}, \gls{śirīṣa}
and \gls{arka} or \gls{kaṭabhī} too.  He should not drink \gls{taila} or
\gls{kaulattha}, nor wine or  \gls{sauvīraka}.


\item [19]

But after drinking any other liquid at all, he should throw up after drinking it.  For 
on the whole, poison is easily removed by means of vomiting. 

\subsection{Therapies at each pulse of toxic reaction}

\item [20]

In the case of hooded snakes, when there is a \se{vega}{toxic reaction}
first one should let blood.  At the second, \diff{one} should make him drink an 
\se{agada}{antidote} together with honey and ghee.\footnote{This
    section reproduces some of the therapies from  \SS\ \Su{5.2.40--43}{566}
    on the stages of \se{dūṣīviṣa}{slow poisoning} by plant poisons; see
    translation on p.\,\pageref{dusivisa} above.}
    
    % Dūṣīviṣa: HIML  
    % IA, 69, 70, 291, 297, 466, 468, 470, 580  [dūṣīviṣa seems in AS to become a 
    %kind of poison]
    % IB, 359 
    % IIA, 519

\item [21]

At the third one should use errhines and \se{añjana}{collyrium} that destroy
poison.\footnote{%
%    Strictly, the \dev{viśanāśanam} “that destroy poison” does
%    not agree with the dual “errhines and collyrium”; it is supported by both
%    Nepalese manuscripts but the vulgate altered it to an easier reading
%    agreeing with the dual compound.  
The rare word \dev{nastaḥ} “from or
    into the nose” in \dev{nastaḥkarma} “errhine” is supported by both
    Nepalese manuscripts.  The term is more common in the \CS, occurring
    eleven times, e.g., at \Ca{1.20.13}{114}, \Ca{2.1.36}{203}, \emph{et
    passim}.  
    
    The \CS\ describes how collyriums, especially \dev{rasāñjana}, cause
phlegm to flow, thus clearing the eyes (\Ca{1.5.14--19}{38--39}).
This could be appropriate in expelling poisons.} At the fourth, when he
has vomited, the physician should make him drink a \se{yavāgū}{gruel}
that destroys poison.

\item [22]

At the fifth and sixth \sse{vega}{toxic reaction}toxic reactions one should make 
the
person drink something that aids cooling, that is cleansing
and \se{tīkṣṇa}{sharp}, and a well-regarded gruel too.


    
\item [23]

\diff{But at the seventh, one should \se{\root śodh}{purge} his head with a sharp
    sternutatory.}\footnote{The vulgate adds a half-verse here recommending
    the application of a \se{añjana}{collyrium} to a cut made on the patient's 
    head.}


\subsubsection{In the case of Maṇḍalins}

\item [24] Amongst Maṇḍalins, the earliest \se{vega}{toxic reaction}
should be treated in the same way as with
Darvīkaras.\footnote{\label{crowsfoot}The vulgate again adds a half-verse
    here, recommending the “crow's foot” incision on the patient's head. On
    this procedure, described in  \CS\ \Ca{6.23.66--67}{574}, see
    \cite[145]{wuja-2003}. This text is not supported here, as it was not in
    the Nepalese text at \SS\ \Su{5.2.43}{566} either. See footnote
    \ref{kakapada}, p.\,\pageref{kakapada} above.  As stated there, it
    appears that this procedure was known in the tradition of the \CS, but
    not in the earliest text of the \SS.}
    
\item [25]

\diff{At the second, one should make him drink ghee and honey and then 
make him vomit}.\footnote{Again, the vulgate text differs substantively, adding 
another half-verse.  But the general idea of the treatment is the 
similar.}

\item [26]

At the third, one should give the purged patient healthy gruel. At the fourth and 
the fifth too, one should do the same as for the Darvīkara.

\item [27]

At the sixth, wholesome things from the group of plants starting with
\gls{kākolī} should be drunk and a sweet antidote.\footnote{The “group of
    seventeen plants beginning with \gls{kākolī}” (\dev{kākolyādi gaṇa}) is
    described at \SS\ \Su{1.38.35--36}{167}. These plants pacify the bile,
    blood and wind and increase phlegm, body-weight, semen and breastmilk.}
    And at the seventh, a wholesome antidote that destroys poison in a
    \se{avapīḍa}{sternutatory}.\footnote{The \dev{avapīḍa} is described at
        \SS\ \Su{4.40.44--45}{556}, where it is also recommended for victims of
        snakebite.  It is a type of head-evacuant. Commenting on that passage,
        Ḍalhaṇa cited “other treatises” as saying that \dev{avapīḍa} treatment
        was suitable for restoring the consciousness of those who have been
        poisoned.  He also quoted a text by an authority called Videha, that says
        the same.  Videha was an author known to Dṛḍhabala (according to
        Cakrapāṇidatta) and often cited in the \emph{Madhukośa} on the topic of
        eye diseases \pvolcite{IA}[132 \emph{et passim}]{meul-hist}. See also
        \volcite{1}[62--63]{josi-maha}.}

\subsubsection{In the case of Rājimats}

\item[28]

\diff{Now, Amongst Rājimats, one should let blood at the first toxic
    shock}.\footnote{The vulgate text says that the blood-letting should be
    done with a \gls{alābu}.  It also has an extra half-verse here,
    prescribing an antitoxin to be drunk together with honey and ghee.}

\item [29]

At the second, a patient who has vomited should be made to drink an antidote 
that destroys poison.   At the third, fourth and fifth, the rule that applies to the 
Darvīkara is suitable. 



\item [30] At the sixth, use a very sharp \se{añjana}{collyrium}, and at
the seventh a \se{avapīḍa}{sternutatory}.  There is a prohibition
    on using blood-letting for pregnant women, children and the elderly.

\item [31ab]

In those who are in pain because of poison, it is advised that the prescribed 
procedures be applied gently. 

\item [31ab]

\subsubsection{In animals}
In goats and sheep, bleeding and collyriums are the same as for people. 

\item [32cd]

In cows and horses, that is twice as much; three times as much for buffalos and 
camels, four times for elephants and \se{kevala}{simply} for all birds.\footnote{ 
\Dalhana{5.5.32}{576} explained “simply for all birds” as meaning that birds 
should receive just drugs, and not blood-letting or collyriums.  See 
p.\,\pageref{bird-pulse} for the toxic reactions in birds and other 
animals.}\q{write note on pariṣekān pradehāṃś}\footnote{The vulgate includes 
several verses after this
    sentence that give a recipe and also a list of specific items like place
    and constitution that should be given careful consideration.
    \Dalhana{5.5.33}{576} cited the opinions of Gayadāsa and Jejjaṭa on this
    recipe but stated that he preferred to follow the contrasting opinions of
    Vṛddhavāgbhaṭa (\As{1.25.24cd--25ab}{184}) and Suśruta
    (\Su{4.31.29cd--30ab}{511}) on this topic, as well as several citations
    “another work” (\dev{tantrāntara}) that is unidentified.}


\subsection{Subsequent therapies}

\item[34]

One should consider carefully with one's intellect the location,
\se{prakṛti}{constitution}, \se{sātmya}{suitability}, the season, the
poison, and the strength or weakness of the toxic reaction and then
proceed with therapy.\footnote{The vulgate here has twelve verses not
    found in the Nepalese version.  These verses explicitly switch subject
    away from assesments according to toxic reactions and to the treatment of
    both mobile and immobile poisons, starting from physical symptoms such as
    swelling and discolouration as well as humoral theory. At the point where
    the vulgate summarizes the extra verses, saying that cases should be
    treated “according to their humors” (\dev{yathādoṣaṃ}), the Nepalese
    witnesses have “as is appropriate” (\dev{yathāyogaṃ},
    \Su{5.5.49cd}{577}).  This suggests that the text has been edited to fit
    the insertion of the verses referring to humoral therapy.  These verses also 
    include therapies such as the crow's foot treatment (see footnotes
    \ref{kakapada} and \ref{crowsfoot}, pp.\,\pageref{kakapada}, 
    \pageref{crowsfoot} above) and the beating of drums 
    that have been smeared with antidotes, as discussed in \SS\ 
    \Su{5.6}{580--582} (see p.\,\pageref{dundubhi} below).}

\item [47--48ab] One should eliminate this poison completely.  It is
extremely hard to overcome. For even a small amount remaining can
strongly bring about a toxic reaction.\footnote{The word
    \dev{avatiṣṭhaṃ} “remaining” is hard to parse.  It cannot be a
    \dev{ṇamul} formation (Pāṇini 3.4.22\,ff), because of the root's
    reduplication, and should not be a present participle because it is
    not neuter.  However, lack of gender concord is not unknown in Epic
    Sanskrit; several of the examples cited by
    \citet[\S\,10.2.1]{ober-2003} even involve present participles without
    gender concord. Cf.\ \volcite{1}[\S\,6.12]{edge-1953} for examples in
    BHS.}

\item[48cd--49] Or it may lead to dejection, pallor, fever, cough and
headaches, dessication, swelling, catarrh, poor vision,
\se{aruci}{disinterest in food} or
\diff{\se{jāḍyatā}{rigidity}}.\footnote{\Dalhana{5.5.49ab}{577} reported
    a reading from Jejjaṭa of \dev{staimitya} “immobility” instead of
    \dev{pratiśyāya} “catarrh.”} And in such cases one should apply the cure
    \diff{as appropriate}.\footnote{The vulgate introduces \dev{doṣa} theory
        here, which is absent in the Nepalese version.}

\item[50--51ab]

One should also treat the \se{upadrava}{secondary
    ailments} of a poisoned patient each as appropriate.

Now, after the  \se{ariṣṭā}{bandage} has been removed and after the place
marked by it has been quickly \se{pracchāna}{incised} one may see poison
that has leaked out there, and a toxic reaction may strongly result.


\subsubsection{Treatment of secondary ailments}
\item[52.1]

Once the poison has disappeared  one can conquer irritated wind using items 
that restrain the wind.\footnote{This half-verse is is not present in the vulgate, 
but has broadly the same sense as \Su{5.5.52cd}{577}, that is not present in 
the Nepalese version.}

\item [53] One can conquer bile using substances that remove
\se{pittajvara}{bile-fever}, with decoctions, oleation and purges,
combined with substances that remove poison,
with the exception of \se{taila}{sesame oil}, \diff{wine},
\gls{kulattha}, and \gls{amla}.\footnote{The vulgate  reads “fish” in
    place of “wine.”}


\item[54]

One can conquer phlegm with the group that starts with \gls{āragvadha},
together with honey.\footnote{The \dev{āragvadhagaṇa} is listed at \SS\
    \Su{1.38.6}{164}. These herbs are there explicitly said to pacify phlegm
    and to remove poison, etc.\ (\Su{1.38.7}{164}).}

\bigskip


\item[56]
\begin{sloka}
If the the \se{ariṣṭā}{bandage} is bound tightly, or if it is
\se{pracchita}{incised} with sharp ointment or \diff{with the remnants of
    the poison}, then, when the limb swells up, the flesh weeps, smells a 
    great deal and is \diff{is \se{śīrṇa}{putrid}, it is designated  
“\se{viṣapūti}{poison-stink}.”}\footnote{\SS\ \Su{5.5.16}{575} 
(p.\,\pageref{5.5.16} above) suggests smearing an incised area with 
antidotes.}
\end{sloka}

\item [57--58ab]

\begin{sloka}
    One may be certain that a person has been \diff{struck by something
    \se{digdha}{poisoned}} if their wound immediately starts to suppurate
has black blood that flows and is inflamed, as well as having black,
weeping and exceptionally foul-smelling flesh coming out of the wound
and also someone who has thirst, \se{mūrcchā}{fainting}, fever and a
temperature.\footnote{The Nepalese witnesses describe someone who has
    been struck or hurt (\dev{kṣata, āhata}), while the vulgate describes
    someone who is pierced (\dev{viddha}). \Dalhana{5.5.58ab}{576}
    interpreted the latter wording as being struck by a poison-smeared arrow.}
\end{sloka}

\item[58.1--60]

\begin{sloka}
    
    One who is known to have these exact symptoms may have poison
    in their wound  that is \dag\ given by mistake.\dag\ And they may have 
    a wound that  has been hit by something
    \se{digdha}{poisoned} and is full of poison. And others
    are sick because of a wound that stinks because of poison.  
    The wise person debrides the excess flesh of such people and then, after
    removing the blood by means of leeches and after removing the
    humours from above and below, he should irrigate with cold bark
    decoctions from milky trees.   And he should apply items that destroy poison 
    such as cloths containing 
    ointments together with cold liquids\sse{dravya}{liquid} mixed with ghee.
        
\end{sloka}
    
\item[61ab]    

\begin{sloka}
    When the bone is \diff{injured} by poisons, the very same rule should be 
    followed as for bile poison.
\end{sloka}

\subsubsection{Antitoxin drugs}
\item[61cd--63ab] 
\begin{sloka}
The following items are powdered, mixed with honey and put in a horn:
%
\gls{trivṛt}, \gls{viśalyā}, \gls{madhuka}, the two kinds of
\gls{haridrā}, \gls{mañjiṣṭhā} and \diff{\gls{vakra}},\footnote{There
    is no \dev{mañjiṣṭhā} group, but there is a plant \dev{vakra}.} and all
    kinds of salt.\footnote{There is a \dev{lavaṇavarga},
        (\Su{1.46.313--321}{236--237}).}  This antidote, taken with drinks,
        \se{añjana}{collyrium}, \se{abhyañjana}{oil rubs}, errhines and drugs,
        destroys poison.

With its relentless \se{vīrya}{potency} and as a destroyer of the
\se{vega}{toxic reaction} to poison, it is called “The Great Antidote”
and has great power.
\end{sloka} 

\item[63cd--65ab]

\begin{sloka}
Very fine \gls{viḍaṅga}, \gls{pāṭhā}, \gls{triphalā}, \gls{ajamodā},
and \gls{hiṅgu}, as well as \gls{vakra} and \gls{trikaṭu}, the whole
group of salts, together with \gls{citraka} and honey should be placed
in a cow's horn and covered with something made of cow's horn.  It
should be set aside for two weeks. %
This antidote is called “Unbeaten” because it conquers  both
stationary and mobile poisons.
\end{sloka}

\item [65cd--68ab]

\begin{sloka}
One should make a fine powder of the following items and place them in a 
horn, together with honey: %67cd
\gls{prapauṇḍarīka},
\gls{suradāru},
\gls{rāsnā},
\gls{kālānusārī},
\gls{kaṭurohaṇī},
\gls{sthauṇeyaka}, 
\gls{dhyāmaka},
\gls{padmaka}, 
\gls{punnāga},
\gls{tālīsa}, 
\gls{suvarcikā},
\gls{kuṭannaṭa}, 
\gls{elā},
blue \gls{sinduvāra}, 
\gls{śaileyaka},
\gls{kuṣṭha},
\gls{tagara},
\gls{priyaṅgu},
% 67ab
\gls{lodhra},
\gls{guggula}, 
\gls{gairika}, 
\gls{saindhava}, %dual?
\gls{pippali}, 
and 
\gls{nāgara}.  This \se{agada}{antidote} is identified as 
“\se{tārkṣya}{Garuḍa}.” It can even destroy the poison of  \se{takṣaka}{the 
snake 
prince Takṣaka}.
\end{sloka}


\item [69cd--72ab] 

\begin{sloka}
% 5.5.70cd
One should make powder of the following items and place it in a 
horn:
% 5.5.69cd
\gls{māṃsī},
\gls{hareṇu},
\gls{triphalā},
\gls{muruṅgī},
\gls{mañjiṣṭhā},
\gls{yaṣṭī},
\gls{padmaka},
% 5.5.69ab
\gls{viḍaṅga},
\gls{tālīsa},
\gls{sugandhikā},
\gls{elā},
\gls{tvak},
\gls{kuṣṭha},
\gls{vakra}, 
\gls{candana}, 
\gls{bhārgī}, 
\gls{paṭolī},
\gls{kiṇihī}, 
\gls{pāṭhā},
\gls{mṛgādanī},
\gls{kroṣṭakamekhalā}, 
% 5.5.70
\gls{pālindī}, 
\gls{aśoka}, 
\gls{kramuka}, 
\gls{surasī},
and the flower that is the \se{prasūna}{blossom} born from the 
\gls{āruṣkara}.\footnote{\Dalhana{5.5.70}{579} glossed \dev{prasūna} more 
specifically as \dev{tulasīpuṣpa} “the Tulasi flower.”} 
The bile derived from boars, monitor lizards, peacocks, and 
porcupines is to be added, with honey, and the products 
of \gls{mārjāra}, \gls{pṛṣata} and \gls{nakula}.\footnote{All three animals 
produce musk.   \Dalhana{5.5.71}{579} remarked that some people thought 
\dev{śikhī} was a cock, not a peacock.  He also here glossed \dev{pṛṣata} as 
\dev{cittala}.}  

This properly-prepared antidote is called “Bull.” Someone who 
has it in the house is called “Bull Amongst Men.”  
% 72 ab
There will be no snakes there, nor even insects: they lose their potency and 
their toxins too. 
\end{sloka}

\item [72cd--73ab]

\begin{sloka}
Drums\sse{bherī}{drum} and tabors\sse{paṭaha}{tabors} smeared with
this rapidly destroy poison when they are sounded.
Smeared flags flags\sse{patāka}{flag} being looked upon easily and quickly
overcome poison.
\end{sloka}

\item [73ab--75ab] 

\begin{sloka}
    One should make a powder of the following items and
place the collection in a cow's horn, mixed with \gls{rajanī}, and mingled with 
honey and ghee.  As before, there is a cover: \gls{lākṣā},
    the two \glspl{hareṇu}, \gls{nalada}, \gls{priyaṅgu}, \gls{mañjiṣṭhā},
    \gls{yaṣṭī} and \gls{pṛthvīkā}. 
      \diff{It should then be used
        with \se{añjana}{collyrium}, drinks and errhines.}  This antidote is
    called “\se{sañjīvana}{Resuscitator}” because it  brings to life the 
    dead whose breath is almost gone.
\end{sloka}
\item [75cd--76ab]

\begin{sloka}
    The best antidote for the poisons of Darvīkaras and Rājilas is
\gls{śleṣmātakī},\footnote{\Dalhana{5.5.75}{579} noted the common name
    \dev{bahuvāra} for \dev{śleṣmātakī}.}
\gls{kaṭphala},
\gls{mātuluṅga},
\gls{śvetā}, % cannot both be white clitoria
\gls{girihvā}, % cannot both be white clitoria
\gls{kiṇihī},
and \gls{sitā}, 
taken with \gls{taṇḍulīya}.\footnote{\dev{rājila} appears to be a synonym
    for \dev{rājimat}, a “striped” snake. \Dalhana{5.5.76ab}{579} once 
    again gives interesting local synonyms for these plant names.}
\end{sloka}

\item[76cd--78ab]

\begin{sloka}
    The best antidote for the poison of Maṇḍalins is
\gls{drākṣā},
\gls{aśvagandhā},
\gls{gajavṛttikā},
ground \gls{śvetā},
combined in equal amounts and given with 
two parts of the leaves of 
\gls{surasa}, and those from
\gls{kapittha},
\gls{bilva} 
and 
\gls{dāḍima}, 
as well as \diff{one measure} from those of
white \gls{sinduvāra}
\gls{aṅkolla} seed
as well as 
\gls{gairika}.\footnote{After this passage, the vulgate has five and a half verses
    that do not appear in the Nepalese version.}
\end{sloka}

\item[84ab--86]
\begin{sloka}
    
    The following group is known as the \diff{\se{ekarasa}{One
        Essence}}:\footnote{The vulgate reads \dev{ekasara}, “one run.”
    \Dalhana{5.5.86}{580} also read \dev{ekasara} and glossed it as
    the proper name of a \dev{gaṇa}.} \gls{śyāmā}, \gls{ambaṣṭhā},
    \gls{tālapatrī}, and \gls{āmra}, as well as \gls{aśmantaka},
    \gls{maṇḍūkaparṇī}, \gls{varuṇa}, \gls{saptalā}, \gls{punarnavā},
    \gls{coraka}, \gls{nāgavinnā}, and \gls{sarpagandhā} as well;
    \se{bhūmī}{black earth},\footnote{A hapax in this meaning
        \volcite{1}[582]{josi-maha}.  So glossed by \Dalhana{5.5.86}{580}:
        \dev{bhūmiḥ kṛṣṇamṛttikā}} and \gls{kuravaka}. Whether used
        separately or in pairs, it  removes poison.

\end{sloka}

    
% got to here 
    \strut
    \bigskip
    
       
  
\end{translation}    