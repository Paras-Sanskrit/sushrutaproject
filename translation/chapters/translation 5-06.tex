% !TeX root = ../incremental_SS_Translation.tex
\chapter{Kalpasthāna 6: Beating Drums}
\label{dundubhi}

\section{Introduction}

\subsection{Literature}

A brief survey of this chapter's contents and a detailed assessment
of the existing research on it to 2002 was provided by
Meulenbeld.\footnote{\volcite{IA}[295]{meul-hist}. In addition to the
    translations mentioned by \tvolcite{IB}[314--315]{meul-hist}, a
    translation of this chapter was included in
    \volcite{3}[61--66]{shar-1999}.}

\section{Translation}

\begin{translation}
    
    \item[1] Now I shall explain the \se{kalpa}{procedure} that is the playing of 
    drums.
    
    \item[3] 
     
     \gls{dhava},
     \gls{aśvakarṇa},
     \gls{tiniśa}, 
     \gls{picumarda}, 
     \gls{pāṭalī}, 
     \gls{pāribhadraka},\footnote{\label{drum-detox}The ingredients to this point
         are similar to the water-detoxifier described in \SS\ \Su{5.3.9}{568}, 
         p.\,\pageref{water-detox1} above.}
     \gls{udumbara}, 
     \gls{karaghāṭaka}, 
     \gls{arjuna},
     \gls{sarjja}, 
     \gls{kapītana}, 
     \gls{śleṣmātakā}, 
     \gls{aṅkoṭha},
     \gls{kuṭaja},
     \gls{śamī}, 
     \gls{kapittha},
     \gls{aśmantaka},
     \gls{arka},
     \gls{ciribilva}, 
     \gls{mahāvṛkṣa}, 
     \gls{arala}, 
     \gls{madhuka},
     \gls{madhukaśigru}, 
     \gls{śāka},
     \gls{gojī}, 
     \gls{bhūrja}, 
     \gls{tilvaka},
     \gls{ikṣuraka},
     \gls{gopaghoṇṭā}, 
     \gls{arimeda},
     
     \gls{pippalī}, 
     \gls{pippalīmūla},
     \gls{taṇḍulīyaka},
     \gls{varāṅga}, 
     \gls{coraka},
     \gls{mañjiṣṭhā},
     \gls{karañjikā}, 
     \gls{hastipippalī},
     \gls{viḍaṅga},
     soot\sse{gṛhadhūma}{soot},
     \gls{ananta},
     soma,\q{Roots page number.}\footnote{The literature on the identification of 
     Soma is large \citep[passim]{wuja-2003}. To the cited literature, a useful 
     historical discussion by \citet[449--]455{gvdb} gives special attention to the 
     āyurvedic literature.}
     \gls{sarala},
     \gls{bāhlīka},
     \gls{kuśa},
     \gls{āmra},
     \gls{sarṣapa},
     \gls{varuṇa},
     \gls{plakṣa},
     \gls{nicula},
     \gls{vardha},
     \gls{mānavaṇjalaputra},
     \gls{śreṇī},
     \gls{saptaparṇṇa},
     \gls{ṭuṇṭuka},
     \gls{elavāluka},
     \gls{nāgadantī},
     \gls{ativiṣā},
     \gls{bhadradāru},
     \gls{marica},
     \gls{kuṣṭha},
     \gls{vacā},
    
\end{translation}
