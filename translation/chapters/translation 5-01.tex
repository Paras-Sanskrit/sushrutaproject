% !TeX root = incremental_SS_Translation.tex
\chapter{Kalpasthāna 1: Protecting the King from Poison}

\section{Introduction}

The first chapter of the Kalpasthāna of the \emph{Suśrutasaṃhitā} addresses the 
topic of protecting a king from those who would assassinate him using poison.  
The king's kitchen is presented as the site of greatest vulnerability.  The staff in 
the kitchen must be vetted carefully and watched for signs of dissimulation.  The 
description of the body-language that tells a poisoner (verses 18--25) are 
engaging and vivid.  These verses are closely parallel in sense to a passage in the 
\emph{Arthaśāstra} that says, 
\begin{quote}
    The signs of a poisoner, on the other hand, are as follow: dry and
    dark look on the face, stuttering speech, excessive perspiration and yawning, 
    trembling, stumbling, looking around while speaking, agitation while
    working, and not remaining in his place.\footnote{\emph{Arthaśāstra} 1.21.8 
    \citep[1, 30]{kang-1969}, 
    translation by \citet[97]{oliv-2013}.} 
\end{quote}

Next, the text discusses the signs of poison in toothbrushes, in food, drink,
massage oil and other items that are likely to come into physical contact with the
king.  In passages that are again paralleled in the \emph{Arthaśāstra} the work
describes how poisoned food kills insects and crackles in a fire, flashing blue
and  the reactions of various birds to poison are described.\footnote{Cf.\
\emph{Arthaśāstra} 1.21.6, \emph{ibid.}, \citet[96]{oliv-2013}.}


The work then moves on to the various symptoms experienced by the king after 
being poisoned, and remedies appropriate to each case.  Poison exhibits 
characteristic signs when added to milk and other drinks.\footnote{Cf.\
\emph{Arthaśāstra} 1.21.6 again.} Further forms of poisoning, their symptoms 
and treatments are described  and finally the king is advised to live amongst 
trusted friends and to protect his heart by drinking various ghee compounds.  He 
should eat the meat and soup made from various animals, including peacock, 
mongoose, alligator, deer.  The chapter ends with the description of an emetic.

\section{Literature}

A brief survey of this chapter's contents and a detailed assessment of
the existing research on it to 2002 was provided by
Meulenbeld.\footcite[IA, 289--290]{meul-hist} Translations of this
chapter since Meulenbeld's listing have appeared by
\textcites[131--139]{wuja-2003}[3,
1--15]{shar-1999}{srik-2002}.\footnote{For a bibliography of translations
    to 2002, including Latin (1847), English (1877), Gujarati (1963) and
    Japanese (1971), see \cite[IB, 314--315]{meul-hist}.}


\section{Manuscript notes}

\begin{itemize}
    \item \MScite{Kathmandu NAK 5-333} has foliation letter numerals, for example
on f.\,323a, that are similar to \MScite{Cambridge Add.\ 1693},\footnote{Scan
at 
\href{https://cudl.lib.cam.ac.uk/view/MS-ADD-01693/1}{cudl.lib.cam.ac.uk/view/MS-ADD-01693/1}.}
 dated to 1165\,\CE\, noted in Bendall's chart of Nepalese letter-numerals \cite[Lithograph V, 
after p.\,225]{bend-budd}
\end{itemize}

\newpage

\section{Translation}

\begin{translation}
 \item[1--2]  And now I shall explain the procedures for safeguarding food and
drink, as were declared by the Venerable Dhanvantari.\footnote{MS H adds in the
margin \dev{atha khalu vatsa suśrutaḥ} “Now begins Vatsa Suśruta.”  This phrase
has been copied here by the scribe from the beginning of the \SS\ chapter in the
\emph{sūtrasthāna} on the rules about food and drink (\Su{1.46.3}{214}).  The
scribe presumably felt, not unreasonably, that this section had common subject
matter with the present chapter.  Further, SS 1.46.3 is the only place in the Nepalese 
transmission of the \SS\ that names Dhanvantari and integrates him into the narrative of the 
\SS\ as the teacher of Suśruta. 
  
 The mention of Dhanvantari here is the only other time in the Nepalese
transmission that this authority is cited as the source of Ayurvedic teaching, and the unique 
occurrence of this actual phrase, “as was declared by the Venerable Dhanvantari.”
See the discussion by \citet[28--32]{kleb-2021b}, who concludes that the earliest
recoverable recension of the \SS\ may have had the phrase only at this point and
not elsewhere in the work. See the further discussion by \citet{birc-2021}.}
 
 \item[3] 

 Divodāsa, the king of the earth, was the foremost supporter of religious
discipline and virtue. With unblemished instruction he taught his students, of
whom Suśruta was the leader.\footnote{This is a quite different statement from
the vulgate which has Dhanvantari as the teacher, and calls him the
\se{kāśipati}{Lord of Kāśī} \citep[559]{vulgate}.  Ḍalhaṇa followed the vulgate
but explicitly noted the reading before us with small differences: \dev{divodāsaḥ
kṣitipatistapodharmaśrutākaraḥ} “Divodāsa, the king of the earth, was a mine of
traditions about discipline and virtue.”}

\subsection{[Threats to the king]}

\item[4--5]  

Evil-hearted enemies who have plucked up their courage, may seek to harm the king,
who knows nothing of it.  He may be assailed with poisons by or by his own people
who have been subverted, wishing to pour the poison of their anger into any
vulnerability they can find.\footnote{Verses about the use of Venemous Virgins as a weapon
do not appear in the Nepalese manuscripts. Cf.\ \cite[81\,f., 132]{wuja-2003}.  This material 
is present in the commentary of Gayadāsa.} 

\item[6] Therefore, a king should always be protected from poison by a physician.

%A king may be cunningly assailed with poisons by evil-hearted enemies who
%have plucked up their courage, or even by his own people turned traitor,
%wishing to pour the poison of their anger into any chink they can find. Or
%sometimes by women using various concoctions, hoping to make him love
%them.\footnote{On how women of ill-character mix their nail-clippings or
%menstrual blood, etc.\ with the king's food, see
%p.\,\pageref{dusyodara}.} Or again, if a Venomous Virgin is used, a man can
%lose his life instantly.\label{visakanya}
%% \footnote{\label{visakanya}On the `Venomous
%% Virgin', see p.\,\pageref{intro:visakanya}.}

\item [7] 

The racehorse-like fickleness of men's minds is well known. And for this reason, a
king should never trust anyone.\footnote{The verb $\surd$ śvas is conjugated as a
first class root in the Nepalese manuscripts.}

\item [8--11]

He should employ a doctor in his \se{mahānasa}{kitchen} who is respected by experts, who 
belongs to a good family, is orthodox, sympathetic, not emaciated, and always busy.

\item [12--13]

The kitchen should be constructed at a recommended location and orientation.  It should
have a lot of light,\footnote{We read \dev{mahacchuciḥ} with the Nepalese manuscripts and 
against the vulgate's \dev{mahacchuci}.  We understand \dev{śucis} as a neuter noun 
meaning “light” following \citet[1050a]{apte-prac}.} have clean utensils and be staffed by 
men 
and
women who have been vetted.\footnote{Verses detailing the ideal staff are omitted in the 
Nepalese manuscripts. 
Cf.\ \cites[560]{vulgate}[132]{wuja-2003}.}


\item[17--18ab]

The chefs, \se{voḍhāra}{bearers}, and makers of boiled rice soups and cakes and whoever
else might be there, must all be under the strict control of the
doctor.\footnote{The word \dev{saupodanaikapūpika} “chefs for the boiled rice soups
and cakes” is grammatically interesting.  The term \dev{sūpodana} (as opposed to
\dev{sūpaudana}) is attested in the \emph{Bodhāyanīya\-gṛhyasūtra} 2.10.54 
\citep[68]{shas-1920}.  More pertinently, perhaps, \dev{sūpodana} is attested in
the Bower Manuscript, part II, leaf 11r, line 3 \citep[vol.\,1,
p.\,43]{hoer-bowe}.} 
% 2.11.54 supodana in the Bodh. (from Einoo's cards)
% sūpodana kṣīrodana
% Bower MS 328
% Kāty  otoṣthayoḥ samāse vā.

\item[18cd--19ab]

An expert  knows people's \se{iṅgita}{body language} 
through abnormalities
in voice, movement and facial expression. He should be able to identify 
a poisoner by the following signs.\q{Cf.\ Arthaśāstra 1.21.8.}


\item[19cd--23]

Wanting to speak, he gets confused, when asked a question, he never arrives at an
answer, and he talks a lot of confused nonsense, like a fool.  He laughs for no
reason, cracks his knuckles and scratches at the ground. He gets the shakes and
glances nervously from one person to another. His face is drained of colour, he is
\se{dhyāma}{grimy} and he cuts at things with his nails.\footnote{The word
\dev{dhyāma} is glossed by Ḍalhaṇa (in a variant reading) as someone who is the
colour of dirty clothes \Su{5.1}{560}.}  A poisoner goes the wrong way and is
absent-minded.

\item[25--27]

I shall explain the signs to look for in toothbrush twigs, in food and drink as
well as in \se{abhyaṅga}{massage oil} and \se{avalekhana}{combs}; in
\se{utsādana}{dry rubs} and showers, in \se{kaṣāya}{decoctions} and 
\se{anulepana}{massage ointment};
in \se{sraj}{garlands}, clothes, beds, armour and ornaments; in slippers and footstools, and
on the backs of elephants and horses; in \se{snuff}{nasya}, \se{dhūma}{inhaled
    smoke}, \se{añjana}{eye make-up}, etc., and any other things which are commonly 
    poisoned. Then, I shall also explain the remedy.

\item[28]

% My old Susruta.tex translation has \bird and \animal commands for making 
% indexes.  Convert them to the \se{}{} command that we're using in the 
% present document.
\newcommand\animal[4]{\se{#2}{#1}} 
\let\bird=\animal 

%28
Flies or crows or other creatures that eat 
a poisonous \se{bali}{morsel} served 
from the king's portion, die on the spot. 

\item [29] 

Such food makes a fire crackle violently, and gives it an overpowering colour like
a peacock's throat.

\item[30--33]

%Its flames sputter, it has acrid smoke, and before long it goes out. 

After a chukar partridge %\animal{chukar partridge}{cakora}{Alectoris
% chukar}{Collins 45}
looks at food which has poison mingled with it, its eyes are promptly drained of
colour; a peacock pheasant %\animal{peacock pheasant}{jīvajīvaka}{Polyplectron
% bicalcaratum}{Dave BSL 270, 273,
%274, 281}
drops dead.  A koel %\animal{koel}{kokila}{Eudynamys scolopacea}{Collins 66}
changes its song and the common crane %\animal{common crane}{kroñca}{Grus
% grus}{Collins 47}
rises up excitedly.\footnote{The verb \dev{arcchati} “rises up” is a rare form
best known from epic Sanskrit \citep[see][212, \S 7.6.1]{ober-2003}.   The
transmitted form \dev{kroñca} is obviously a colloquial version of Sanskrit
\dev{krauñca}.  Commenting on \Su{1.7.10}{31}, Ḍalhaṇa interestingly gives the
colloquial versions of several Sanskrit bird names, even singling out
pronunciation in the specific location of Kānyakubja.  For \dev{krauñca} he says
that people pronounce it \dev{kurañja} and \dev{koṃci}.  The form \dev{koñca}
is found in Pāli (see \cite[731]{cone-dict}, who notes that Ardhamāgadhī has the
same form). Elsewhere, Ḍalhaṇa calls the bird \dev{krauñcira},  \dev{krauñci}, and 
\dev{kaicara}
(\Su{1.46.105}{223}, \Su{6.31.154}{684} and
(\Su{6.58.44}{790} respectively).}  It will excite a peacock 
%\bird{peacock}{mayūra}, %{Pavo cristatus}{Collins 39}
and the terrified parakeet %\saneng{parakeet}{śuka}%{Psittacula krameri\slash
% eupatria\slash
%cyanocephala}{Collins 64}
and the hill myna %\ssaneng{hill myna}{sārikā}%{Acridotheres tristis tristis, L.,
% etc.}{Ali \#1006,
%\citet[28\,ff.]{Dave}, \citet[119]{Collins}}
screech. The swan %\animal{swan}{haṃsa}{?}{?}
trembles very much, and the racket-tailed drongo %\animal{racket-tailed
% drongo}{bhṛṅgarāja}{Dicrurus paradiseus}{Collins 123}
churrs.\footnote{Ḍalhaṇa seemed confused about the \sed{bhṛṅgarāja}{racket-tailed
drongo}.  He called it a generic \sed{bhramaraka}{drongo}, a word that can also mean 
“bee,” \citep[62]{dave}, and then said that it is like the
\sed{dhūmyāṭa}{black drongo} \citep[for a nice explanation of this name,
see][62--63]{dave} and that people call it “the king of birds.”} The chital deer
%\saneng{pṛṣaṭa}{chital}  
sheds tears and the
monkey releases excrement.\footnote{\MScite{Kathmandu KL 699} reads 
“\sed{vṛṣabha}{bull}" for
“\sed{pṛṣata}{Chital deer}.”  The latter may perhaps be mistaken for the former in
the Newa script, although the reading of \MScite{Kathmandu KL 699} is hard to 
read at this point.}

\item[34cd]

Vapour\sse{bāṣpa}{vapour} rising from tainted food gives rise to a pain in the 
heart,
it makes the eyes roll, and it gives one a headache.\footnote{ “Tainted” translates
\dev{upakṣipta}.  The word's semantic field includes “to hurl, throw against,” and
especially “to insult verbally, insinuate, accuse.”  The commentator Ḍalhaṇa
glossed the term as, “spoiled food given to be eaten” (\dev{vidūṣitasyānnasya
bhoktuṃ dattasya}), but he noted that some people read “\dev{ukhākṣipta}” or
“thrown into a pan.”  Other translators have commonly translated it as “served,” perhaps
influenced by Ḍalhaṇa's “\sed{datta}{given}.”}


\item[35, 36cd] 

In such a case, an errhine and a collyrium that are costus, \gls{lāmajja}, 
\gls{nalada} and \se{madhus}{honey};\footnote{The vulgate 
    supplies another phrase and verb at this point that is not present in the Nepalese 
    transmission, but that makes the text flow more easily.}  a paste of sandalwood
on the heart may also provide relief.\footnote{\citet[350]{sing-1972} discussed
the difficulties in identifying \dev{lāmajja}, a plant cited more often in the
\SS\ than in the \CS; Ḍalhaṇa adopted the common view that it is a type of \emph{uśīra} or 
vetiver grass.  The grammatical neuter form 
\dev{madhus}  “sweetness” of the Nepalese
manuscripts is less common than neuter \dev{madhu} “honey, sweetness,
liquorice.”}

\item[37]

Held in the hand, it makes the hand burn, and the nails fall out. In such a case,
the \se{pralepa}{ointment} is \gls{śyāmā}, %{Callicarpa macrophylla,
% Vahl.}{AVS 1.334,    NK \#420},
\gls{indragopa}, %{Kerria lacca
% (Kerr.)}{http://www.icar.org.in/ilri/de fault.htm},
soma and \gls{utpala}.%
%{Nymphaea stellata, Willd.}{GJM 528, IGP 790; Dutt 110, NK \#1726}
\footnote{\label{beautyberry}“Beautyberry” (\emph{Callicarpa macrophylla} 
Vahl.) is one
identification of \dev{śyāmā}, but vaidyas and commentators have different ideas
about the plant's identity (see glossary).  
\par 
On translating \dev{indragopa} as “velvet-mite,”
see \cite{lien-1978}. Ḍalhaṇa's remarks show that he had a reading
\dev{indrāgopā} before him, and he tries to explain \dev{indrā} and \dev{gopā} as
separate plants.  But he also says that some people read \dev{indragopa}. 
\par
 Ḍalhaṇa
curiously parsed the name \dev{somā} (f.) out of the compound; this feminine noun
is almost unknown to Ayurvedic literature.  Some dictionaries and commentators
consider it a synonym for \dev{guḍūcī}, others for \dev{brāhmī} or
\dev{candrataru}.  Ḍalhaṇa also mentioned that some people think the word refers to
the \sed{somalatā}{soma creeper}, which might explain his choice to take the word as
feminine.  But the compounded word is far more likely to be \dev{soma} (m.), the
well-known mystery plant \citep[see][76--78, 125]{wuja-2003}.  If this can be
taken as rue (\emph{Ruta graveolens}, L.), as some assert, one can point to a
pleasing passage in Dioscorides where rue plays an antitoxic role: “\ldots it is a
counterpoison of serpents, the stinging of Scorpions, Bees, Hornets and Wasps; and
it is reported that if a man be anointed with the juice of the Rue, these will not
hurt him; and that the serpent is driven away at the smell thereof when it is
burned; insomuch that when the weasel is to fight with the serpent she armeth
herself by eating Rue, against the might of the serpent” \parencites[cited 
from][262]{wren-1956}[not found in][]{osba-dios}.}
     
     \item [38--39] If he eats that food, through inattention or by mistake, then
his tongue will feel like a \se{aṣṭhīlā}{pebble} and it will lose its sense
of taste. It stings and %\sskt{stings}{tudyate},
burns, and his \se{śleṣman}{saliva}\label{saliva} dribbles out.\footnote{The word
\dev{aṣṭhīlā} is normally feminine.   The Nepalese manuscripts read it with a
short \dev{a-} ending.  Gayadāsa noticed that some manuscripts read 
\dev{aṣṭhīla}
with a short \dev{-a} ending (\MScite{Bikaner RORI 5157}, f.\,5v:7--8) and 
Ḍalhaṇa
reproduced his observation.  The vulgate reading “\sed{cāsyāt}{from his 
mouth}” is
more obvious (\emph{lectio facilior}), but is not attested in the Nepalese
manuscripts.} In such a case, he should apply the treatment recommended above 
for
\se{bāṣpa}{vapour}, and what will be stated below under “toothbrush
twigs”.\footnote{Poisoned toothbrushes are discussed in verses 48\,ff.\ below.}
     
     \item[40]
     
     On reaching his stomach, it causes \se{mūrcchā}{stupor}, vomiting, the hair
stands on end, there is distension, a burning feeling and an impairment of
the senses.\footnote{I translate \dev{mūrcchā} in the light of the metaphors
discussed by \citet{meul-2011}, that include thickening and losing
consciousness.}

     \item[41] 
     
In this case, vomiting must quickly be induced using the fruits of
\gls{madana}, %{Randia dumetorum, Lamk.}{NK \#2091},
\gls{alābu}, %{Lagenaria vulgaris, Seringe.}{NK \#1419},
\gls{bimbī}, %{Coccinia indica, W. \& A.}{PVS 1994.4.715; NK 534}
and \gls{koṣītakī}, %{Luffa cylindrica, (L.) M. J. Roem.
% \textnormal{or}
% L. acutangula, (L.) Roxb.}{ADPS 252, NK \#1514 etc.}
taken with milk and \gls{udaśvit}, or alternatively with
rice-water.
     
     \item[42]
     
    
 Reaching the \se{pakvāśaya}{intestines}, it causes a burning feeling, stupor,
diarrhoea, thirst, impairment of the senses, \se{āṭopa}{flatulence} and it makes
him pallid and thin.
    
    % % % % % % % % % % % % % % % % % % % % %
    
      \item [43]
In such a case, purgation with the fruit of \se{nīlī}{indigo}, 
       %{Indigofera tinctoria, L.}{NK \#1309},
together with ghee, is best.  And  `\se{dūṣīviṣāri}{slow-acting poison antidote}'
should be drunk with honey and \se{dadhi}{curds}.\footnote{The `slow-acting
poison' is discussed at \Su{5.2.25\,ff.}{565}.}
     
     \item[44]
     
     When poison is in any liquid substances such as milk, wine or water, there are
     various streaks, and foam and bubbles form.  

     \item[45]
     
     \q{I'm still unhappy about this verse.} And no reflections are visible or,
however, if they can be seen once more, they are distorted, fractured, or
tenuous and distorted too.\footnote{Both Nepalese witnesses read
\dev{vikṛta} ({distorted}) twice, which is tautologous.  In the first occurrence
both read \dev{vikṛtā} without proper termination.  One might read the sandhi
in the second occurrence as \se{vāvikṛtā}{or not distorted}, but this gives
no better sense. The scribe of \MScite{Kathmandu NAK 5-333}, apparently the
original hand,  added in the margin the alternate reading
“\se{yamalā}{double}” as in the vulgate. Perhaps the scribe too was troubled
by the tautology.  It is also evidence that he was aware of a witness with
variant readings similar to the vulgate. We emend for grammar but retain the
\emph{lectio difficilior}.} \q{Mention this in the introduction as an example
    of the scribe knowing the vulgate.}
     
\item[46]

Vegetables, soups, food and meat are soggy and tasteless.  They seem to go stale
suddenly, and they have no aroma.\q{fn about sadyas+}  

\item[47] 

All edibles lack aroma, colour or taste.  Ripe fruits rapidly \se{pra$\surd$kuth}{rot} and 
unripe ones ripen.\footnote{The root $\surd$\dev{kuth} “stink, 
    putrify, rot” 
    is apparently known only from its few uses in the \SS.}

\item[48]

When a toothbrush twig has poison on it, the bristles are corroded and the
flesh of the tongue, gums and lips swells up.\footnote{Gayadāsa and Ḍalhaṇa 
pointed out that “\sed{dantaveṣṭa}{enclosure of a tooth}” and 
“\sed{dantamāṃsa}{flesh of the tooth}” have the same meaning 
(\Su{2.16.14--26}{331--332}).}

\item[49]

 Then, once his swelling is 
 lanced, one should \se{pratisāraṇa}{rub} it with
 \gls{dhātakī} flowers
 %{Woodfordia fruticosa (L.) Kurz}{AVS 5.412, NK \#2626}
 %{Terminalia chebula Retz.}{NK \#2451}, 
 \gls{jambū},
 %{Syzygium cumini, (L.) Skeels}{ADPS 188, NK \#967, Potter 168}
 \gls{āmra} stones and
 \gls{harītakī}
 fruit mixed with honey.\footnote{This recipe is different from the vulgate.}
 
 \item[50] Alternatively, the \se{pratisāraṇa}{rubbing} can be done with either
 the roots of \gls{aṅkolla}, the bark
of \gls{saptachada} or \gls{śirīṣamāṣaka}.\footnote{The 
    spelling of
the name \dev{aṅkolla} varies \dev{aṅkoṭa, aṅkoṭha, aṅkola} 
\citep[5]{gvdb};
Ḍalhaṇa noted that the form  \dev{aṅkolla} is a colloquialism
(\Su{1.37.12}{161}).  The sentence is awkward and we have emended
\dev{śirīṣamāṣaka} to be a plural, as in the vulgate, rather than the ablative 
singular of 
the Nepalese witnesses.  We follow Ḍalhaṇa in interpreting the compound to refer 
to the distinctive bean-like siris seeds, rather than to \gls{māṣaka} 
(\Su{5.1.50}{562}).}

\item[51ab] 
 
One should give advice about a poisoned tongue-scraper or 
\se{kavala}{mouthwash} in the
same way as  for a toothbrush twig.

\item[51cd]

Massage oil that has been laced with poison is slimy, thick and discoloured.   

\item[52]

When the massage oil has been contaminated with poison, boils arise,
pain, a \se{srāva}{discharge}, inflammation of the skin, and
sweating.\footnote{The feminine \dev{sphoṭā} for “boils” is unattested.}
    And the flesh splits open.

\item[53--54]

In such a case, sandalwood, \gls{tagara}, \gls{kuṣṭha}, and 
\gls{uśīra}, 
\gls{veṇupatrikā}, 
\gls{somavallī}
and 
\gls{amṛtā}, 
\gls{śvetā}, 
\gls{padma}, and 
\gls{kālīyaka} should be made into an
\se{anulepana}{ointment} for the patient, who has been sprinkled with cold 
water.
That is also recommended as a drink with the juice and leaves of
\gls{kapittha}.\footnote{This compound could be interpreted as 
“wood
apple juice and \gls{patra}.”  Note that this recipe is differs
from that of the vulgate, which requires urine.}
 
 \item[55]
 
In the case of a \se{utsādana}{dry rub}, a \se{parīṣeka}{shower}, an infusion, a
\se{anulepana}{massage ointment}, or in beds, clothes, or armour, the physician 
should understand that it is the same as for 
\se{abhyaṅga}{oil massage}.\footnote{See verse 52 above.}
 
 \item[56--58]
 
 When a comb has poison in it, the hair falls out, the head aches and blood
oozes from the \se{kha}{follicles} and \se{granthi}{lumps} appear on the
head. In such a case, one should repeatedly apply an ointment of black earth
soaked with \diff{bear's bile},\q{Bear's bile instead of deer's bile.}
\label{fluidbile}\footnote{Ḍalhaṇa comments here that `bile is that fluid
    which goes along inside the tube attached to the liver'
    (\dev{kālakhaṇḍa\-lagna\-nalikā\-madhya\-gata\-jalaṃ pittam})
    \Su{5.1.57}{562}.} ghee, \gls{śyāmā},\footnote{See note \ref{beautyberry}.}
        \gls{pālindī},
        and \gls{taṇḍulīyaka}.
        Good alternatives are either the fluid extract of cow-dung, or the juice of
        \gls{mālatī}, 
        the juice of \gls{mūṣikakarṇī},
        or household soot.\footnote{The plant identifications in this passage follow
            Ḍalhaṇa's glosses, although he noted a difference of opinion on the identity
            of \gls{mūṣikakarṇī} (lit.\ “mouse-ear”). \par The expression \dev{dhūmo
            vāgārasaṃjñitaḥ} `\ldots or the smoke termed ``house''\,' is commonly
            interpreted by translators and in Ayurvedic dictionaries as `household soot,'
            and this does seem to be the meaning, in context.  The term was
            comprehensively discussed by \citet[443]{meul-2008}. Cf.\ note
            \ref{grhadhuma}, p.\,\pageref{grhadhuma}.\label{soot}}
 
 
 \item[59]
 
 If either massage oil for the head, or a helmet for the head, in a wash, turban, or 
 garlands that are contaminated with poison, then one should treat it in the same 
 way as a comb.
 
 \item[60--61]
 
 When face make-up is poisoned, the face becomes dark and has  the symptoms 
 found
with poisoned massage oil. It is covered with \se{kaṇṭaka}{spots} that are like
\se{padminīkaṇṭaka}{lotus-spots}.\footnote{See the description of this condition
at \Su{2.13.40}{323}, where the skin on the face is characterized as having pale
circular patches that are itchy and have spots.}  In this case, the drink is
honey and ghee, and the \se{pralepa}{ointment} is sandalwood %{Santalum
%album, L.
% %}{ADPS 111, NK
%%\#2217}
with ghee, curds, honey, \gls{phañjī}, %{Clerodendrum
%serratum, L.}{AVS 2.121, ADPS 87},
\gls{bandhujīva} %{Pentapetes phoenicea, L.}{NK \#1836},
and \gls{punarnavā}.\footnote{The common plant-name 
\dev{punarnavā} is
read as \dev{punarṇṇavā} in both Nepalese witnesses.  This unusual form is
technically-speaking legal according to Pāṇini 8.4.3, but is not attested in
published texts.  \dev{punarṇavā} is found rarely in some other Nepalese
manuscripts such as the \emph{Brahmayāmala} (a.k.a.\ \emph{Picumata}, 44.81,
transcription thanks to Shaman Hatley), and elsewhere (e.g., in
\cite[20]{gana-1920}, where it is the name of a constellation.}\q{punarṇṇavā in
    the N \& K MSS} %{Boerhaavia
%diffusa, L.}{ADPS 387, AVS
%1.281,NK \#363}.
 
\item[62--63ab] 

Elephants and the like become ill and they dribble saliva. And the rider gets
\se{sphoṭa}{spots} and a discharge on his scrotum, penis, and rectum. In this
case, one prescribes the same therapy as for poisoned massage oil for both the
rider and the mount.

\item[63cd--65ab]

When there is poison in \se{nasya}{snuff} or smoke, the \se{liṅga}{symptom} is
blood coming out of the \se{kha}{apertures of the head}, a headache, a flow of
\se{kapha}{mucus} and impairment of the senses.

In such a case, 
ghee of cows etc., boiled up\q{śrita for śṛta} with their milk
and \gls{ativiṣā},
%{Aconitum heterophyllum, Wall. ex     Royle}{AVS 1.42, NK \#25},
is prescribed, with
%\se{śvetā}{white clitoria}
%{Clitoria ternatea, L.}{AVS 2.129, NK \#621}
\gls{madayantikā},
%{Lawsonia inermis, L.}{AVS 3.303, NK \#1448, Potter 151},
 as a cold drink or errhine.

\item[65cd--66]

Flowers lose their fragrance and colour, and wilt. On smelling them, he gets a
headache and his eyes fill with water.  In this case, the treatment is what was
proposed above for \se{bāṣpa}{vapour} and that which is traditional for face
make-up.


\item[67--68]

When it is in ear-oil, there is  degeneration in the ear, and painful swelling.
There is also a discharge from the ear and in such a case it needs to be
\se{pratipūraṇa}{irrigated} promptly with ghee and honey.  
\se{svarasa}{Extracted
    juice} of \gls{bahuputrā} and  very cold juice of
\gls{somavalka} are also  recommended as 
something good.\footnote{The syntax of the Nepalese version is slightly unclear, 
but the vulgate has smoothed out the difficulties.}\q{explain more}

%\se{Asparagus racemosus, Willd.}{wild asparagus}{bahuputrā}{ADPS 441, 
%AVS 1.218, NK \#264, IGP 103,
%    IMP 4.2499ff., Dymock 482ff.}
%\se{svarasa}{juice} and ghee, mixed with honey. Very cold
%\se{Acacia polyacantha, Willd.}{white cutch tree}{somavalka}{AVS 1.30, IGP
%    7, GJM 602, IMP 2.935; \emph{pace} NK \#1038} juice is another desirable
%remedy.

\item[69]


When poison is mixed in with \se{añjana}{eye make-up}, he gets tears and
\se{upadeha}{rheum}, with a burning feeling, pain, \se{dṛṣtivibhrama}{faulty
    vision}, and possibly even blindness.\footnote{The term translated as  “faulty
vision” could also mean “rolling eyes.” “Eye make-up” is normally made of  
\gls{añjana}.}

\item[70--71]

In this case, one must immediately drink ghee and have it also
in an \se{tarpaṇa}{eyewash} with
\gls{māgadha}.\q{Medical difference from Sharma.}
%\se{Piper longum, L.}{long pepper}{māgadha} {NK \#1928; but cf.\ AVS    
%3.245},
%
% \se{Jasminium auriculatum, Vahl.}{needle-flower 
%jasmine}{māgadha}{AVS
% 3.245, but cf.\ NK \#1928 etc.}
One should have an \se{añjana}{eye ointment} of the juice of
\gls{meṣaśṛṅga}
%{Gymnema sylvestre (Retz.) R. Br.}{AVS
%    3.107, NK \#1173}, 
and have the \se{niryāsa}{extract} of 
\gls{varuṇa},
%{Crataeva magna (Lour.)  DC.}{AVS 2.202; ADPS 500; cf.\, NK \#696}.
% synonym of Crataeva nurvala Buch. Ham. 
\gls{kapittha} and 
%{Limonia acidissima, L.}{AVS 3.327, NK \#1021}, 
\gls{meṣaśṛṅga}
%{Gymnema sylvestre (Retz.)
%    R. Br.}{AVS 3.107, NK \#1173}, 
and the flower of 
\gls{bhallātaka}.\q{example where the vulgate clarifies that 
these should be used separately; appears to be a gloss inserted into the vulgate 
text.}
%{Semecarpus anacarium, L.}{NK\#2269, AVS 5.98},

\item[72--73]

Because of poisoned slippers there will definitely be a
swelling, \se{svāpa}{numbness}, a \se{srāva}{discharge} and an
outbreak of \se{sphoṭa}{spots} on the feet. One should \se{pra$\surd$
sādh}{clean} footstools together with slippers.

\item[74]

Ornaments lose their lustre, and they do not shine as they used to.
They damage their respective locations with  burning, 
\se{pāka}{sepsis}, and \se{avadāraṇa}{fissuring}.\footnote{The reading 
\dev{avadāruṇa} in MS Kathmandu KL 699 is not attested elsewhere in Sanskrit 
literature.  On “sepsis” for \dev{pāka}, see \cite[xlv--xlvi]{wuja-2003}.}

\item[75ab]

One should apply the stated procedure for \se{abhyaṅga}{massage oil} to
poisoned slippers and ornaments.

\item[75cd--76]  In the case of the \se{upasarga}{affliction} by poison which has
been described above, starting from `vapour' and ending with `ornaments,' the
physician should observe the \se{upadrava}{side-effects} and then prescribe the
therapy called the \se{mahāsugandha}{Great Fragrance} antidote,  which I shall
describe.\footnote{This antidote is indeed described later, in dramatic terms, at
\Su{5.6.14--27}{581}.  A recipe with eighty-five ingredients including cow's bile,
it is praised as chief of all antidotes, one that can drag the patient back from
the very jaws of death, from even the poisonous fangs of Vāsuki.}


% got to here

\item [77--78ab] He should prescribe it in drinks, \se{ālepana}{liniments},
\se{nasya}{errhines}, and in \se{añjana}{eye ointment}.  Also, he should use 
sharp
purgatives and emetics.  If bleeding is present, he should have the
indicated\q{The two uses of prāpta are hard to translate.  prāptāḥ $\rightarrow$
    kṣipraṃ is an example of the vulgate banalizing the Sanskrit text to make 
    sense of
    a difficult passage.} veins pierced.\q{$\surd$ vyadh not $\surd$ vedh (also elsewhere
    and for the ears), causative optative.}


\item[78cd--79ab]

If either \gls{mūṣikā} %{Jatropha curcas, L.}{AVS 3.261, NK
%\#1374}
or a \gls{ajaruhā} is tied on to the King's wrist, then all food that is mixed
with poison will be rendered free of poison.\footnote{In early Ayurvedic
    literature, the plant \dev{ajaruhā} is mentioned only here and its identity is
    unknown.  It may be a fern of the Nephrodium family, according to
    \citet[7]{gvdb}.  Ḍalhaṇa, on \Su{5.1.78}{563}, cited a description of the two
    plants from the little-known authority Uśanas \citep[IA, 660 et
    passim]{meul-hist} who described \dev{ajaruhā} as a white root with spots on
    it that looks like collyrium when it is split; when drunk with sandalwood it
    causes poison to be digested.}

%\emph{kandaḥ śvetaḥ
% sapiḍako bhede
% cāñcanasannibhaḥ/ gandhalepanapānais
%tu viṣaṃ jarayete nṛṇām//}

\item[79cd--80]

He should always guard his heart when amongst \diff{people who are not
    his friends}.\footnote{The \emph{Carakasaṃhitā} described “protecting the
    heart” (\dev{hṛdayāvaraṇa}) as drinking several sweet, oily drinks to
    surround the heart and keep it safe (\Ca{6.23.46}{574}).
    \Dalhana{5.1.79--81}{563} explained it as taking a number of anti-toxic
    medicines, including those listed in the present passage, in order to
    cover or hide (\dev{pracchādana}) the heart.  Note that the Nepalese
    version reads the opposite of the vulgate: one should guard one's heart
    when amongst enemies, not friends.  This is far more logical; it is also
    the reading known to the \As{1.8.89}{79}.} Before eating, he should drink
    the kinds of ghee called \sse{ajeya}{Invincible}\sse{amṛta}{Immortal}
    “Invincible” and “Immortal”.\footnote{These ghee compounds are described
        in later chapters: see \Su{5.2.47--49}{566} and \Su{5.6.13}{581}.} He
        should drink \se{sarpiṣ}{ghee}, \gls{kṣaudra}, \se{dadhi}{curds},
        \se{payas}{milk}, or cold water.


\item[81]

He should consume monitor lizard, peacock, \gls{nakula}, \gls{pṛṣata},
and \gls{hariṇa} too, that destroy poison, and their juices.

\item [82]

As discerning person should add well-crushed
\gls{pālindī},\footnote{\Dalhana{5.1.82}{563} equated this with
    \gls{trivṛt}.} \gls{madhuka}, and sugar to the meats of \gls{godhā},
    \gls{nakula} and \gls{hariṇa} too. 
\item[83]

Add sugar and \gls{ativiṣā} to peacock flesh, together with
\gls{mahauṣadha}. And for meat from a \gls{pṛṣata}, he should add
\gls{pippalī}, with \gls{mahauṣadha}.

\item[84ab]
\diff{A cold neem} broth with honey and ghee is wholesome too. 

\item [84cd]

A discerning person should partake of hard and soft foods that counteract
poison.\footnote{On this expression, see \cite{yagi-1994}.}
    
\item [85]

If poison might have been drunk, a person who has protected his heart
should make himself vomit using \gls{pippalī}, \gls{madhuka},
\gls{kṣaudra}, \gls{śarkara}, \gls{ikṣu} juice, and water.

\bigskip

The first chapter in the Kalpas. 

    \end{translation}

   
   

