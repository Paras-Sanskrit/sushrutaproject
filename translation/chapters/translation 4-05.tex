% !TeX root = incremental_SS_Translation.tex
% Paras

\chapter{Cikitsāsthāna 5:  On the Treatment of Serious Wind 
Diseases}

\section{Literature} 

Meulenbeld offered an annotated overview of this chapter and a bibliography
of earlier scholarship to 2002.\fvolcite{IA}[266]{meul-hist} 

\section{Translation}

\begin{translation}
    
    \item [1]
    Now we shall describe the treatment of serious wind diseases.
    
    \item [2]

    \item [3]
    One group says that the blood afflicted by wind (wind-blood) (\textit{vāta-rakta}) is of two types: spreading out over a surface (\dev{uttāna}) and deep (\dev{avagāḍha}).\footnote{Ḍalhaṇa comments \citep[424]{vulgate} that \dev{uttāna} refers to being situated in the skin and flesh, and \dev{avagāḍha} refers to being situated internally.} However, this is not correct.\footnote{In H, the word \dev{tan} should be \dev{tat}.} Why? Just as leprosy, after spreading over a surface it (afflicted blood) becomes deeply situated. Therefore, its being of two different types is refuted.  

    \item[4]
    When the wind is aggravated by fighting a strong person, etc.\footnote{These factors that aggravate the wind are mentioned in \textit{Nidānasthāna}, Ch. 12, text 6.}, one's corrupted blood caused by eating heavy or hot food before the last meal is digested blocks the path of the aggravated wind. It then combines with the wind and simultaneously creates pain due to the wind-blood. This [condition] is called wind-blood (\textit{vāta-śoṇita}). At first, it is situated in the hands and feet.\footnote{In H, the word \dev{tan} should be \dev{tat}.} Later, it spreads throughout the body. Its early forms are pricking pain, burning, itching, ulcer, trembling\footnote{In H, there should not have been the \dev{s} after \dev{stambha}.}, roughness of the skin, pulsation in the blood vessels, tendons, and tubular vessels\footnote{In addition to blood vessels, it would also include the nerves.}, weakness of the thighs, as well as the sudden appearance of dark brown, tawny, or red spots on the soles of the feet, fingers, ankles, and wrists. The disease becomes fully manifest in the person who does not undertake the means to revert the disease or applies a wrong treatment. Its symptoms have been mentioned. Among them, weakness occurs for the one who does not counter the disease.

    \item[5]
    Generally, wind-blood occurs in those who are very delicate, those who eat the wrong foods and enjoy improperly, those who are fat, and even in those who indulge in pleasure.  

    \item[6]
    In that regard, one should treat the patient who is not degenerating due to wasting of life air, thirst, fever, unconsciousness, dyspnea, trembling, and loss of appetite, is not oppressed by the contraction [of limbs], is strong, composed, and has the means.

    \item[7]
    In the treatment, at the beginning itself one should do blood-letting of the wind-affected body part little by little and more than once. That (slow blood-letting) is because of the danger of further aggravation of wind. One should avoid doing blood-letting of the part hardened or weakened by excessive wind.\footnote{In H, the reading \dev{amlāna} does not make sense given the context. Therefore, we have accepted the vulgate reading \dev{mlāna} for the translation.} Thereafter, one should make the patient do the remedies of vomiting, etc. If the wind that is mixed [with blood] or separated is very aggravated then one should make him consume aged ghee or goat-milk. Or, [one can give him] half a measure of oil added with an \textit{akṣa} of \gls{madhuka} and cooked with \gls{pṛśniparṇī}\footnote{Ḍalhaṇa glosses \citep[425]{vulgate} \emph{śṛgālavinnā} as \emph{pṛśniparṇī}.}, or the oil that is sweetened by sugar and honey and cooked with \gls{śuṇṭhī} and \gls{kaśeru}. Or, one should boil milk with an eight times volume of the decoction of the following herbs: \gls{śyāmā}, \gls{rāsnā}, \gls{suṣavī}, \gls{pṛśniparṇī}\footnote{According to Ḍalhaṇa, \emph{śṛgālavinnā} is \emph{pṛśniparṇī}.}, \gls{pīlu}, \gls{śatāvarī}, \gls{śvadaṃṣṭrā}, and \gls{dvipañcamūla}. This milk should then be used to cook oil with the admixture of pastes of \gls{meṣaśṛṅgī}, \gls{śvadaṃṣṭrā}, \gls{madhu}, \gls{nāgabalā}, \gls{bhadradāru}, \gls{vacā}, and \gls{surabhi}. This (resultant) should be utilised in drinks, etc. Or, one should use the oil that is cooked with a decoction of \gls{śatāvarī}, \gls{apāmārga}\footnote{Ḍalhaṇa glosses \citep[425]{vulgate} \textit{mayūraka} as \textit{apāmārga}.}, \gls{dhavaka}, \gls{madhuka}, \gls{kṣīravidārī}, \gls{balā}, \gls{atibalā}, and \gls{tṛṇapañcamūlī}\footnote{Ḍalhaṇa comments \citep[425]{vulgate} that \gls{kuśa}, \gls{kāśa}, \gls{nala}, \gls{darbha}, \gls{kāṇḍa}, and \gls{ikṣuka} are called \textit{tṛna} (grass).}, with the admixture of \gls{kākolī}, etc. Or, one should use the \gls{balā}-oil that is cooked as \emph{śatapāka}.\footnote{\emph{Śatapāka} seems to be an oil that is prepared with a hundred parts of some things similar to \textit{sahasrapāka} that is prepared with one thousand parts of some herbs. Refer \textit{Cikitsāsthāna} Ch. 4 text 29 for the preparation of \textit{sahasrapāka}.} Or, [the affected body part] should be moistened with milk that is boiled with the roots of wind-alleviating herbs, or it should be moistened with sour things.\footnote{Ḍalhaṇa comments \citep[425]{vulgate} that the sour things (\textit{amla)} are \gls{surā}, \gls{sauvīraka}, \gls{tuṣa}-water, etc. \textit{Surā} is some kind of liquor, \textit{sauvīraka} is perhaps the fruit of the jujube tree, and \textit{tuṣa} is perhaps Terminalia Bellerica (\dev{vibhītaka}).} In that regard, five remedies prepared with milk are described. For preparing a poultice, milk should be cooked in ghee, oil, fat, marrow, and \emph{dugdha}\footnote{In the \SS, the word for milk is \textit{kṣīra} or \textit{payas} but not \textit{dugdha}. Therefore, the word \textit{dugdha} here can mean the sap of plants or something that is extracted.} separately with each of these powdered grains or pulses---barley, wheat, sesame, \gls{mudga}, or \gls{māṣa}---that is mixed with unctuous pastes of \gls{kākolī}, \gls{kṣīrakākolī}, \gls{jīvaka}, \gls{ṛṣabhaka}, \gls{balā}, \gls{atibalā}, \gls{pṛśniparṇī}\footnote{\emph{śṛgālavinnā}}, \gls{meṣaśṛṅgī}, \gls{piyāla}, sugar, \gls{kaśeru}\footnote{For \emph{kaśerukā}}, \gls{surabhi}, and \gls{vacā}. Or, the essence of unctuous fruits\footnote{Ḍalhaṇa comments \citep[425]{vulgate} that the unctuous fruits mentioned here are sesame, castor, \gls{atasī}, \gls{vibhītaka}, etc.} can be used as a poultice. Or, a \textit{veśavāra}\footnote{In H, the reading \dev{vaiśavāro} does not make sense. It should have been \dev{veśavāro}, as shown in the vulgate, which is the reading we have accepted here.\\ \textit{Veśavāra} is boneless meat minced, steamed, and added with spices, ghee, etc. Refer to 'Ayurveda Medical Dictionary' by Ranganayakulu Potturu.\\Perhaps the word \dev{vaiśavāra} is an earlier form of the word \dev{veśavāra}.} prepared from the flesh of a fat \textit{cilicima} fish\footnote{H has the compound word \dev{nalapīnamatsya}. \dev{nalamīna} is a particular fish known as \textit{cilicima} (\dev{cilicimaḥ}). See \textit{Amarakośa}. Also, if the name is \dev{nalamatsya} then the word \dev{pīna} (fat) within the name is not according to proper Sanskrit. But, it can be allowed because the word \dev{matsya} (fish), instead of being a part of the name, can be considered to mean fish in general and thus the word \dev{pīna} becomes its modifier. Thus, \dev{nalapīnamatsya} can mean "a fat fish that is a \dev{nala} (\textit{cilicima})".\\ Ḍalhaṇa says in his comment \citep[425]{vulgate} that \dev{nalamīna} is a type of \dev{rohita} (\textit{rohita}). Monier Williams says that \textit{rohita} is a kind of fish: Cyprinus Rohitaka. Regarding the \textit{rohita} fish, there is a \textit{subhāṣita}: \dev{agādhajalasañcārī na garvaṃ yāti rohitaḥ | aṅguṣṭhodakamātreṇa śapharī pharpharāyate ||} This indicates that \textit{rohita} is a deep water fish.}\q{The webpage https://hindi.shabd.in/vairagya-shatakam-bhag-acharya-arjun-tiwari/post/117629 says that this verse belongs to the \textit{Nītiratna}. I could not find this text.} can be used instead. Or, [one can use] the poultice containing \gls{bilva}-rind\footnote{The word \dev{pesikā} in H should be read \dev{peśikā}.}, \gls{tagara}, \gls{devadāru}, \gls{saralā}, \gls{rāsnā}, \gls{hareṇu}, \gls{kuṣṭha}, \gls{śatapuṣpa}, liquor, yogurt, and whey. Or, [one can use] the ointment prepared by mixing \gls{mātuluṅga}, \emph{amla}\footnote{Perhaps it could mean vinegar or sour curds. Refer to Monier Williams Sanskrit Dictionary.}, salt, and ghee with honey and \gls{śigru}-root. Or else, [one can use] the unctuous sesame paste. 

    \item[8]
    When the [condition of wind-blood] has a predominance of bile, the patient should be made to drink a decoction of grapes, \gls{revataka}-fruit, \gls{payasyā}, \gls{madhuka}, \gls{candana}, and \gls{kāśmarī}. This decoction is sweetened with honey and sugar before consumption. Or, the decoction of \gls{śatāvarī}, \gls{paṭola}, \gls{patra}, \textit{triphalā}, \gls{kaḍurohiṇī}, and \gls{guḍūcī} should be given. [The patient should be administered] ghee that is prepared with sweet, bitter, and astringent [remedies].\footnote{Ḍalhaṇa comments \citep[425]{vulgate} that the sweet remedies are \gls{kākolī}, etc., bitter remedies are \gls{paṭola}, etc., and astringent remedies are \textit{triphalā}, etc.} 
    
    [The patient] should be sprinkled with a decoction of \gls{bisa}, \gls{mṛṇāla}, \gls{bhadraśriya}, and \gls{padmaka} mixed with goat-milk\footnote{The compound word ending with \dev{kaṣāyeṇa} is taken to be a \textit{bahuvrīhi} for \dev{ajākṣīreṇa} (goat-milk).}, or with rice water that is mixed with milk, sugarcane juice, honey, and sugar, or with whey and sour rice gruel mixed with a decoction of grapes and sugarcane. Or else, [the patient] should be sprinkled with ghee that is prepared with \textit{jīvanīya}\footnote{\textit{Jīvanīya} seems to be a group of medicinal herbs. There is an Ayurvedic preparation called \textit{jīvanīya-ghṛta}. Refer to the \textit{Āyurvedīya Śabdakośa} vol. 1.} or sprinkled with ghee that is purified for one hundred times.

    The poultice [to be applied] should be made of rice flour or of the paste of sour rice gruel mixed with \gls{nala}, \gls{vañjula}, \gls{tālīśa}\footnote{\dev{tālīsa} should be read \dev{tālīśa}}, \gls{śṛṅgāṭaka}, \gls{kālodyā}, \gls{gaurī}, \gls{śaivāla}, \gls{padma}, etc. The poultice should be mixed with ghee.

    \item[9]
     The [condition of wind-blood] with a predominance of blood should be treated in the same way. Also, blood-letting should be done repeatedly.

     \item[10]
    However, when the [condition of wind-blood] has a predominance of phlegm, the patient should be made to consume a decoction of \gls{āmalaka} and \gls{haridrā} that is sweetened with honey, or a decoction of \textit{triphalā}, or a paste of \gls{madhuka}, \gls{śṛṅgavera}, \gls{harītakī}, and \gls{tiktarohiṇī}. He should be made to drink \gls{harītakī} with water mixed with a little urine. He should be sprinkled with oil, urine, salty water, and liquor that are acidic\footnote{Reading the word \dev{sukta} in H as \dev{śukta}}. Or, he should be sprinkled with a decoction of \gls{āragvadha}, etc. 

    The patient should be massaged with ghee cooked with sour cream, urine, liquor, \gls{śuktā}\footnote{Monier Williams states Rumex Vesicarius for \textit{śuktā}}, \gls{madhuka}, \gls{śārivā}\footnote{DCS has this entry: Cryptolepsis buchananii Roem. et Schult. (Surapāla (1988), 453) Decalepis hamiltonii Wight et Arn. (Surapāla (1988), 453)}, and \gls{padmaka}.
    
     The poultice should be made of either the paste of white mustard, or the paste of sesame and \gls{aśvagandhā}, or the paste of \gls{dāru}\footnote{According to V. S. Apte, \dev{dāru} can mean \dev{devadāru}.}, \gls{śelu}, and \gls{kapittha}, or the paste of honey, \gls{śigru}, and \gls{punarnavā},\footnote{H has a short \dev{a} at the end instead of the long \dev{ā}.} or the paste of dry ginger, long pepper, black pepper,\footnote{\dev{vyoṣatiktā} refers to the group of these three pungent spices. Also see \textit{Sūtrasthāna} 14.35.} \gls{pṛthakparṇī}, and \gls{bṛhatī}.\footnote{In H, the Sanskrit syntax does not match up with what the author is trying to say. The name of the fifth paste should also have been in the nominative case, as the other four pastes.}\q{The provisional edition should be modified accordingly.} These five poultices are prepared with salty water. Thus, they have been described.

     \item[11] 
     In case of combined aggravation of two humours or simultaneous aggravation of all three humours, the stated methods of treating those aggravations should be combined.\footnote{It means that the respective methods of treating the aggravation of individual humours should be combined.}  

     \item[12] 
    In all [aggravations], one should consume \gls{harītakī} with jaggery. Or, one should have a diet of rice cooked in milk for ten days and should drink a mixture of \gls{pippalī}s crushed in milk, with increasing by five \gls{pippalī}s each night. Then one should reduce them again by the order of five more [each night].\footnote{In H, the letter \dev{ñ} in \dev{bhūyañca} should have been \dev{ś}.} In this way, one should [reduce] all the \gls{pippalī}s. This is called \textit{Pippalīvarddhamānakam} (Increasing Long Peppers). It indeed cures wind-blood, intense fever,\footnote{Perhaps \dev{viṣamajvara} could mean irregular fever.} loss of appetite, jaundice, abdominal affection, piles, heavy breathing, cough, wasting disease, weak digestion, and heart disease. 

    The poultice is a paste of \gls{sahā}, \gls{candana}, \gls{mūrvā}, \gls{piyāla}, \gls{śatāvarī}, \gls{kaśeru},\footnote{H has \dev{kaśerukā}.} \gls{sahadevā}, \gls{padmaka}, \gls{madhuka}, \gls{śatapuṣpā}, \gls{kuṣṭhā}, \gls{niśaileya}, \gls{kāṭarūṣaka}, \gls{balā}, \gls{atibalā}, and \gls{jīvantī} mixed with milk. Or it is a paste of \gls{kāśmarī}, \gls{madhuka}, and \gls{tarpaṇa} mixed with ghee and cream. Or it is olibanum cooked with milk that is mixed with \gls{madhūcchiṣṭa}, \gls{śārivā}, \gls{sarjarasa}, \gls{madhuka} and the group of sweet herbs. 

    Old ghee that is cooked with \gls{āmalaka} and \gls{sarala} and sweetened with sugar and honey is for drinking. Old ghee that is cooked with \textit{jīvanīya} or that is cooked with a decoction of \gls{suṣavī} is for sprinkling. Cooked \gls{balā} oil is for sprinkling, bathing, enema, and eating\footnote{Perhaps it means that one should eat foods cooked in that oil.}. One should eat food preparations made of rice, \gls{ṣaṣṭika}, barley and wheat accompanied with milk, meat soup, or \gls{mudga} soup that is not sour. Blood-letting also [should be done]. The treatments of vomiting, purging of bowels, enema, and oily enema should be conducted when the humours are highly aggravated.

    \item[13] 

    \item[14] There are verses in this regard.\footnote{The word \dev{bhavati} in H should have been \dev{bhavanti}.}
    \begin{sloka}
    There is immediate relief by the application of remedies such as these by which the physicians cure the chronic condition of wind-blood. 
    \end{sloka}

    \item[15-16]
    \begin{sloka}
    Poultice, sprinkling [oil], plaster, oil massage,\footnote{In H, the part \dev{vyajanānilāḥ} does not make proper sense in the verse. Emending it to \dev{vyajanāni ca} could be a consideration, but fanning (\dev{vyajana}) a patient with wind-blood is not good, as understood from the recommendation that such a patient should stay in a non-windy room. Therefore, we have accepted the vulgate reading for the first half of this verse.} spacious and comfortable rooms\footnote{In H, read the \dev{sa} \dev{saraṇāni} as \dev{śa}.} with no wind, soft pillows, comfortable beds, and soft massages are recommended in the condition of wind-blood.
    \end{sloka}

    \item[17]
    \begin{sloka}
     Exercise, mating, anger, eating hot, sour, or salty foods, sleeping during the day, and food that is slimy or heavy should be avoided.
\end{sloka}

    \item[18]
    One should treat the person who is affected with spasmodic contraction,\footnote{In H, the reading \dev{apatākinam} should have been \dev{apatānakinam}.} who does not have droopy eyes and crooked eyebrows, whose fingers have not become rigid, who is not perspiring or trembling, who is not in a state of delirium, who is not bed-ridden,\footnote{V. S. Apte has \dev{khaṭvayati}. The \textit{Āyurvedīya \'{S}abdakośa} has the entry \dev{khaṭvāpātin} which means \enquote{one who is inclined to fall from bed.} Perhaps the reading in H has an error of the letter \dev{yā} which should have been \dev{pā}.} and who is not restrained externally. There at the beginning itself,\footnote{In H, \dev{prāgaiva} should have been \dev{prāgeva}.} after rubbing the patient with oil and making him perspire, one should treat him with a strong \textit{avapīḍa}\footnote{The \textit{Āyurvedīya \'{S}abdakośa} has the entry \dev{avapīḍa} that means administering an oily paste through the nose. Refer \textit{SS Cikitsāsthāna} Ch. 40 text 44 for a better understanding of \textit{avapīḍa}.}\q{There, Ḍalhaṇa comments that deliberation on \textit{avapīḍa} had been done earlier when it was mentioned. Find that description to know more details.} in order to clear his head. Then, the patient should be made to drink filtered ghee that is properly cooked with a decoction of \gls{vidārigandhā} and other herbs, sugarcane juice, milk, and yogurt. In that way, the wind does not spread exceedingly. 

    Thereafter, one should gather wind-alleviating herbs such as \gls{bhadradāru}, etc. and other constituent parts, along with \gls{yava}, \gls{kola}, and \gls{kulattha}, and the flesh of a freshwater aquatic creature all at one place and prepare a decoction of them. One should take this decoction and mix it properly with sour substances and milk, and then cook the \textit{pratīvāpa}\footnote{It refers to an admixture of substances to medicines either during or after decoction. Refer to Monier-Williams's Sanskrit dictionary.} of \gls{madhuka} in this mixture along with ghee, oil, body fat, and bone marrow. This is \textit{trivṛt} that should be recommended in treatments of sprinkling, oil massage, applying a poultice, oral consumption, oily enema, and errhine for patients having spasmodic contractions.

    The patient should then be made to sweat by the methods described earlier. If the wind is stronger then the patient should be immersed in [a vessel] filled with lukewarm fluid used for sprinkling (\textit{trivṛt}). Or he should be kept in the hot fireplace of a blacksmith.\footnote{H has the reading \dev{rathākāracullyām} that means \enquote{fireplace shaped like a chariot}, but the vulgate reading \dev{rathakāracullyām} makes more sense here. Thus, we have accepted it.} Or else he should be made to sweat by [a mixture of] \gls{kṛśara}, \textit{veśavāra},\footnote{Refer the above text no.7 for \textit{veśavāra}. In H, the syllable \dev{vai} should have been \dev{ve}.} and milk.
 
    Oil cooked with the juice of \gls{mūlaka}, \gls{uruvūka}, \gls{sphūrjaka}, \gls{saptalā}, and \gls{śaṅkhinī} should be used in sprinking, etc. for patients with spasmodic contractions.\footnote{The word \dev{tailam} is not present in H but is present in the vulgate. We have accepted it.} Sour yogurt mixed with \gls{marica} and drunk on an empty stomach alleviates spasmodic contractions. Or else, ghee, oil, body fat, or bone marrow [can be consumed on an empty stomach]. 
    
    This procedure of treatment thus described is for spasmodic contractions caused only by wind. When mixed humours cause it then the treatment should also be mixed. And when the spasms subside the patient should be given \textit{avapīḍa}-s. One should also consider the fats of cock, crab, black fish, and porpoise.\footnote{H has the reading \dev{rasān} which means \enquote{juices}. It seems unrealistic that juice would be extracted by crushing these whole animals. Vulgate has the reading \dev{vasāḥ} instead of \dev{rasān} which appears to be the more probable reading. Thus, we have accepted it.} Milk prepared with wind-alleviating medicines. Gruel prepared with barley, \gls{kola}, \gls{kulattha}, \gls{mūlaka}, yogurt, ghee, and oil. 

    One should treat this recurring spasm for ten nights with oil massage, purging of bowels, enemas, and oily enemas. One should also consider the treatment of diseases caused by wind. One should also undertake preventive measure.    
    
\end{translation}
