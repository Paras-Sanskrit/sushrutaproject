%!TeX root = incremental_SS_Translation.tex

\chapter{Nidānasthāna 1: The Diagnosis of Diseases Caused 
by Wind}

% Harshal Bhatt

\section{Literature}

Meulenbeld offered an annotated overview of this chapter and a
bibliography of earlier scholarship to 2002.\fvolcite{IA}[234]{meul-hist}

%Existing research on this chapter to 2002:
%\volcite{IB}[354--369]{meul-hist}.

\section{Translation}

\begin{translation}
    
\item [1] And now we shall explain the chapter about diagnosis of
diseases caused by wind.\footnote{ Appropriate word for vātavyādhi?
    Diseases caused by wind or rheumatism?}
    
\item[2]
    
\item[3] Dhanvantari, the foremost of the upholders of righteousness, who
emerged with nectar, Suśruta asks after touching/holding his
feet.\q{subject-verb-object}
    
\item[4] O King!\footnote{ H and N both mss read \dev{भूपते }instead of
    \dev{कौपनैः} in the vulgate.} (Perhaps divodāsa)\q{Don't put guesses in
        the main text. Footnote them.} the best of the orators! Let us know about
    the naturalized and disordered form of wind, its places in the body and
    types of the diseases caused by its contamination.
        
\item[5--9ab] On hearing his words, the venerable sage replied that being
independent, constant and omnipresent this wind is revealed as self-born
and supreme being. It is situated in the form of life-force in all beings
and worshiped by all worlds. It is the cause of origin, continued
existence and destruction of beings. It is unmanifest though manifests
in/through action, cold, dry, light in weight, variable, moving
horizontally with two attributes i.e., sound and
tangibility\footnote{According to Ḍalhaṇa, it has power to divide
    humours, fluids, feces etc. moving inside the body and it is the cause to
    the disease in the limbs. It carries humours, chyle, semen/7 fluids? and
    feces further in the body. The wind which is moving outside is holding
    the earth and body. (\dev{सा चास्य शक्तिः
    शरीरदोषमूत्रपुरीषादिविभागोऽवयवसंस्थानका(क)रणं दोषधातुमलसंवहनादिश्च,
    शरीराद्बहिस्तु संचरतो धरणीधारणादिः  }      Su 1938:257)}. Having all
    chief qualities which are sattva, rajas and tamas but predominated by
    rajas. It has inconceivable power. It is inducer of humours\footnote{
        Ḍalhaṇa suggests \dev{नेता=प्रेरक} (Su 1938:257)} and distinguished in
        the group of diseases\footnote{ Ḍalhaṇa suggests \dev{राट्=राजते} not 
            \dev{राजा}}. \emph{\textbf{It move}\textbf{s quickly, moves again and
                    again, }}stays in stomach and intestine.
                    
\item[9cd] Now, listen to the description of wind which moves inside the
body.
                    
\item[10] Unvitiated wind makes possible objects of senses connect with
intellect. It maintains a state of equilibrium between the humours,
semen/7 fluids? and Gastric fluid and actions done by body, speech and
intellect bring to one's right place.\footnote{ According to Ḍalhaṇa,
    \dev{सम्पत्तिः=सम्पन्नता} at 1.6.3 (Su1938:23). Ḍalhaṇa commented that
    Gayadāsa reads \dev{`इन्द्रियार्थोपसंप्राप्तिं'} but not written here
    because of being detailed. (\dev{गयदासाचार्यस्तु इमं श्लोकं
    `इन्द्रियार्थोपसंप्राप्ति' इत्यादि कृत्वा पठति, स 
    च विस्तरभयान्न लिखितः)} But H and N MSS suggest  
    `\dev{इन्द्रियार्थोपसम्पत्तिः}'}

\item[11] Just as the five types of bile have been described based 
on their
name, place and their actions, similarly, one type of air is of five
types based on name, place, action and diseases.

\item[12] Five types of wind:

\begin{enumerate}
    \def\labelenumi{\arabic{enumi}.}
    \item
    \begin{quote}
        Vital wind (\emph{prāṇa})
    \end{quote}
    \item
    \begin{quote}
        \emph{udāna }
    \end{quote}
    \item
    \begin{quote}
        \emph{samāna}
    \end{quote}
    \item
    \begin{quote}
        \emph{vyāna}
    \end{quote}
    \item
    \begin{quote}
        \emph{apāna}
    \end{quote}
\end{enumerate}

\begin{quote}
    above five types of wind remain in their equilibrium and hold the
body\footnote{ Ḍalhaṇa suggests \dev{स्थान=साम्य, यापयन्ति=धारयन्ति}
    (The manuscripts all read \dev{प्राणोदानः समानश्च व्यानोपानस्तथैव च ।
    }       against the vulgate's \dev{प्राणोदानौ समानश्च व्यानश्चापान एव
    च । } I think \dev{प्राणोदानौ, व्यानापानौ} or \dev{व्यानश्चापान एव च
    } should be read)}.
    \end{quote}
    
\item[13--14ab] The wind that flows through the mouth is called the
vitality (prāṇa), which holds the body. It propels down food inside the
stomach and engages with the gastric fluid\footnote{ Ḍalhaṇa suggests
    head, chest, throat and nose as locations of prāṇa. (Sus1938:259)
    Gayadāsa suggests \dev{अग्नि} for \dev{प्राण}.}. Unvitiated Vital wind
    mostly causes hiccups, asthma etc. diseases.
        
\item[14cd--15] The wind which flows upwards in the body, the best among
all five winds is called udāna. Singing, speech etc. individual things
done by the same wind. Unvitiated udāna wind mostly causes diseases above
the collar bone e.g., nose, eyes, head and ears\footnote{ Ḍalhaṇa
    suggests it also causes diseases like cough etc. (\dev{चकारादन्यादपि
    प्राणोदानौ, व्यानापानौ कासादीन् करोति ।})}.
            
\item [16--17ab] The samāna wind flows in stomach and duodenum. It helps
gastric fluids in the digestion of food and separates the substances
produced from it e.g., chyle, impurities, urine and feces. Unvitiated
samāna wind causes diseases like a chronic enlargement of spleen (gulma),
weak digestion, and diarrhea.
            
\item[17cd--18] The vyāna wind moves inside the whole body and circulates
chyle and expels sweat and blood outside the body. It helps in the
movements of limbs in every way. Contaminated vyāna wind causes all
diseases occurring in the body.
            
\item[19--20ab] Staying in the abdomen, the apāna wind propels wind of
body, feces, urine, semen, womb and menstruation to come out of the body
at their proper time. Contaminated apāna wind causes terrible diseases
that occur in the bladder and anus.

\item[20cd--21ab] Contaminated vyāna and apāna wind causes defect of 
semen 
and gonorrhea, while simultaneous contamination of all the five winds surely 
leads to death.  

% The diseases caused by contaminated wind staying in different places of the 
%body are being described.

\item[21cd--22ab] I shall therefore describe all the diseases caused by the 
contamination of winds staying in the various places of the body.

\item[22cd--24ab] Contaminated wind in the stomach causes disease like
vomiting, loss of consciousness, fainting, thirst, heart-seizure, pain in
lateral sides of stomach. It also causes rumbling of the bowels, acute
pain, inflated belly, pain while discharging urine and feces, suppression
of urine and pain in the loins.

\item[24cd]Contaminated wind residing in the ear causes loss of function of the 
senses.

\item[25--29] Residing in the skin,\footnote{Ḍalhaṇa and Gayadāsa both 
suggest \dev{त्वक्=रस}. Gayadāsa explained that chyle stays in the skin and 
therefore, in the verse \dev{त्वक्स्थ} should be read as \dev{रसस्थ} as we read 
secondary meaning in the sentences like \dev{गङ्गायां घोषः}.} contaminated 
wind causes discoloration of skin, throbbing of parts of the body, dryness, 
numbness, itching, pricking pain, swelling. It being inherent in the flesh of body 
causes swelling with pain and being inherent with the fat of the body causes 
swelling with slight pain but do not become wound.\footnote{The MS H does 
not 
read \dev{व्रणांश्च रक्तगो ग्रन्थीन् सशूलान् मांससंश्रितः ।} against the vulgate. 
\citep[261]{vulgate}.}

Residing in the artery it causes acute pain, contraction and filling up of the 
artery.\footnote{According to Ḍalhaṇa \dev{सिराकुञ्चनं} is also known as 
\dev{कुटिला सिरा} \citep[262]{vulgate}} It stuns, vibrates and 
destroys\footnote{Ḍalhaṇa and Gayadāsa both suggest the meaning of 
\dev{हन्ति} for being not capable of both stretching and contraction. 
\dev{सन्धिगतः संधीन् हन्ति प्रसारणाकुञ्चनयोरसामर्थ्यं करोति} \citep[262]{vulgate} 
...} 
the muscle tissues by residing in the muscle. Residing in the joints it causes 
pain 
and swelling. Residing in the bone it causes fracture and dryness of bones 
which 
also cause to acute pain and, in the marrow, it dries up marrow which may 
never 
be cured. Residing in the semen it causes non-production and distorted 
production of semen.\footnote{Ḍalhaṇa and Gayadāsa both suggest that a 
distorted production \dev{विकृतां प्रवृत्तिम्} is too fast, too slow, knotty and 
discolored.} 


\item[30--31ab] Contaminated wind moves from the hand, foot, head, then it 
may be omnipresent or pervade the entire body of men and causes stiffness, 
convulsion, numbness and acute pain.

\item[31cd--32ab] Wind (5 types) mixed with other doṣas (bile etc.) in the 
places mentioned above produces mixed types of pains.

% Symptoms of diseases combined with other humours (bile etc.) of prāṇa 
%wind.
\item[34cd--35ab] Prāṇa wind surrounded by bile causes vomiting and burning 
sensation, by phlegm it causes weakness, exhaustion, laziness and bad taste. 

% Symptoms of diseases combined with other humours (bile etc.) of udāna 
%wind.
\item[35cd--36ab] Udāna wind surrounded by bile causes loss of 
consciousness, stupor, dizziness and fatigue, by phlegm it causes absence of 
perspiration, slowness of digestion, sensation of coldness.

% Symptoms of diseases combined with other humours (bile etc.) of samāna 
%wind.
\item[36cd--37ab] Samāna wind surrounded by bile causes perspiration, a 
burning sensation, heat and stupor, association with phlegm it causes erection 
in 
urine, feces and limbs.  

% Symptoms of diseases combined with other humours (bile etc.) of apāna 
%wind.
\item[37cd--38ab] Apāna wind associated with bile causes a burning sensation, 
heat and the voiding of blood with urine, with phlegm it causes a feeling of 
heaviness in the lower part of the body and coldness.

% Symptoms of diseases combined with other humours (bile etc.) of vyāna 
%wind.
\item[38cd--39ab] Vyāna wind surrounded by bile causes a burning sensation, 
tossing of the limbs and fatigue, by phlegm it causes stiffening limbs, 
uddaṇḍaka? and pain in the swelling.

\item[40]



\end{translation}
