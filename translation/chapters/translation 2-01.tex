%!TeX root = incremental_SS_Translation.tex

\chapter{Nidānasthāna 1: The Diagnosis of Diseases Caused 
by Wind}

% Harshal Bhatt

\section{Literature}

Meulenbeld offered an annotated overview of this chapter and a
bibliography of earlier scholarship to 
2002.\footnote{\volcite{IA}[234]{meul-hist}. \citep{rube-1954b} studied the 
wind doctrines in the \CS.}

%Existing research on this chapter to 2002:
%\volcite{IB}[354--369]{meul-hist}.

\section{Subject matter}

The present chapter describes the diseases caused by vitiated wind and
wind's mixing with other humours. Contemporary ayurvedic physicians 
consider these diseases to include rheumatism. 

\section{Translation}

\begin{translation}

\item [1] And now we shall explain the chapter about the aetiology of 
wind diseases.



\item[3] After holding the feet of Dhanvantari, the foremost of the upholders of 
righteousness who emerged out of nectar, Suśruta makes this 
enquiry.\footnote{Explain the nectar myth.}\q{add footnote here}

\item[4] 

O King! O best of orators! Explain the location and types of diseases
of the wind, whether in its natural state or disordered.\footnote{MSS H and N 
both read \dev{bhūpate} instead of \dev{kopanaiḥ} in the vulgate:  instead of 
addressing the king, the vulgate is saying “by irritations of the wind\ldots.”  
The vulgate also has Suśruta asking about \dev{karma}, whereas in the 
Nepalese version he asks only about the types of diseases. Note that 
Dhanvantari is here addressed as king, a title associated elsewhere with 
Divosdāsa.}\q{add refs to Divodāsa as king. }
%\footnote{https://doi.org/10.20935/AL2992}
.

\item[5--9] On hearing his words, the venerable sage spoke.  This
lordly wind is declared to be self-born because it is independent,
constant and omnipresent. It is worshipped by the whole world.  
Amongst all beings, it is the self of all. During creation, continued existence 
and destruction, it is the cause of beings. 

It is unmanifest though its actions are manifest; it is cold, dry,
light, and mobile.  It moves horizontally, has two attributes and is
full of \se{rajas}{dust}.\footnote{According to \Dalhana{2.1.8}{257},
    the two qualities are sound and tangibility.  The word \dev{rajas}
    could also refer to the quality of activity in the three-quality
    (\emph{guṇa}) theory, which is how Ḍalhaṇa interpreted it.}
%    
%\footnote{    
%    it has power to divide
%humours, fluids, feces etc. moving inside the body and it is the cause to
%the disease in the limbs. It carries humours, chyle, semen/7 fluids? and
%feces further in the body. The wind which is moving outside is holding
%the earth and body. (\dev{sā cāsya śaktiḥ
%śarīradoṣamūtrapurīṣādivibhāgo'vayavasaṃsthānakā(ka)raṇaṃ 
%doṣadhātumalasaṃvahanādiśca,
%śarīrādbahistu saṃcarato dharaṇīdhāraṇādiḥ  }  Su 1938:257)}
%
It has inconceivable power. It is the leader of the
humours\footnote{\Dalhana{2.1.8}{257} interpreted \dev{netā} “leader”
    as \dev{preraka} “impeller.”} and the ruler of the multitude of
    diseases.
    
It moves fast, it moves constantly, it is in the stomach, in the rectum,
travelling in the body.  

Learn its characteristics from me.
     % got to here. 
     
     

\item[9cd] Now, listen to the description of wind which moves inside the
body.

\item[10] Unvitiated wind makes possible objects of senses connect with
intellect. It maintains a state of equilibrium between the humours,
semen/7 fluids? and Gastric fluid and actions done by body, speech and
intellect bring to one's right place.\footnote{ According to Ḍalhaṇa,
\dev{sampattiḥ=sampannatā} at 1.6.3 (Su1938:23). Ḍalhaṇa commented that
Gayadāsa reads \dev{`indriyārthopasaṃprāptiṃ'} but not written here
because of being detailed. (\dev{gayadāsācāryastu imaṃ ślokaṃ
`indriyārthopasaṃprāpti' ityādi kṛtvā paṭhati, sa 
ca vistarabhayānna likhitaḥ)} But H and N MSS suggest  
`\dev{indriyārthopasampattiḥ}'}

\item[11] Just as the five types of bile have been described based 
on their
name, place and their actions, similarly, one type of air is of five
types based on name, place, action and diseases.

\item[12] Five types of wind:

\begin{enumerate}
\def\labelenumi{\arabic{enumi}.}
\item
\begin{quote}
Vital wind (\emph{prāṇa})
\end{quote}
\item
\begin{quote}
\emph{udāna }
\end{quote}
\item
\begin{quote}
\emph{samāna}
\end{quote}
\item
\begin{quote}
\emph{vyāna}
\end{quote}
\item
\begin{quote}
\emph{apāna}
\end{quote}
\end{enumerate}

\begin{quote}
above five types of wind remain in their equilibrium and hold the
body\footnote{ Ḍalhaṇa suggests \dev{sthāna=sāmya, yāpayanti=dhārayanti}
(The manuscripts all read \dev{prāṇodānaḥ samānaśca vyānopānastathaiva ca .
}   against the vulgate's \dev{prāṇodānau samānaśca vyānaścāpāna eva
ca . } I think \dev{prāṇodānau, vyānāpānau} or \dev{vyānaścāpāna eva ca
} should be read)}.
\end{quote}

\item[13--14ab] The wind that flows through the mouth is called the
vitality (prāṇa), which holds the body. It propels down food inside the
stomach and engages with the gastric fluid\footnote{ Ḍalhaṇa suggests
head, chest, throat and nose as locations of prāṇa. (Sus1938:259)
Gayadāsa suggests \dev{agni} for \dev{prāṇa}.}. Unvitiated Vital wind
mostly causes hiccups, asthma etc. diseases.

\item[14cd--15] The wind which flows upwards in the body, the best among
all five winds is called udāna. Singing, speech etc. individual things
done by the same wind. Unvitiated udāna wind mostly causes diseases above
the collar bone e.g., nose, eyes, head and ears\footnote{ Ḍalhaṇa
suggests it also causes diseases like cough etc. (\dev{cakārādanyādapi
prāṇodānau, vyānāpānau kāsādīn karoti .})}.

\item [16--17ab] The samāna wind flows in stomach and duodenum. It helps
gastric fluids in the digestion of food and separates the substances
produced from it e.g., chyle, impurities, urine and feces. Unvitiated
samāna wind causes diseases like a chronic enlargement of spleen (gulma),
weak digestion, and diarrhea.

\item[17cd--18] The vyāna wind moves inside the whole body and circulates
chyle and expels sweat and blood outside the body. It helps in the
movements of limbs in every way. Contaminated vyāna wind causes all
diseases occurring in the body.

\item[19--20ab] Staying in the abdomen, the apāna wind propels wind of
body, feces, urine, semen, womb and menstruation to come out of the body
at their proper time. Contaminated apāna wind causes terrible diseases
that occur in the bladder and anus.

\item[20cd--21ab] Contaminated vyāna and apāna wind causes defect of 
semen 
and gonorrhea, while simultaneous contamination of all the five winds surely 
leads to death.  

% The diseases caused by contaminated wind staying in different places of the 
%body are being described.

\item[21cd--22ab] I shall therefore describe all the diseases caused by the 
contamination of winds staying in the various places of the body.

\item[22cd--24ab] Contaminated wind in the stomach causes disease like
vomiting, loss of consciousness, fainting, thirst, heart-seizure, pain in
lateral sides of stomach. It also causes rumbling of the bowels, acute
pain, inflated belly, pain while discharging urine and feces, suppression
of urine and pain in the loins.

\item[24cd]Contaminated wind residing in the ear causes loss of function of the 
senses.

\item[25--29] Residing in the skin,\footnote{Ḍalhaṇa and Gayadāsa both 
suggest \dev{tvak=rasa}. Gayadāsa explained that chyle stays in the skin and 
therefore, in the verse \dev{tvakstha} should be read as \dev{rasastha} as we 
read 
secondary meaning in the sentences like \dev{gaṅgāyāṃ ghoṣaḥ}.} 
contaminated 
wind causes discoloration of skin, throbbing of parts of the body, dryness, 
numbness, itching, pricking pain, swelling. It being inherent in the flesh of body 
causes swelling with pain and being inherent with the fat of the body causes 
swelling with slight pain but do not become wound.\footnote{The MS H does 
not 
read \dev{vraṇāṃśca raktago granthīn saśūlān māṃsasaṃśritaḥ .} against the 
vulgate. 
\citep[261]{vulgate}.}

Residing in the artery it causes acute pain, contraction and filling up of the 
artery.\footnote{According to Ḍalhaṇa \dev{sirākuñcanaṃ} is also known as 
\dev{kuṭilā sirā} \citep[262]{vulgate}} It stuns, vibrates and 
destroys\footnote{Ḍalhaṇa and Gayadāsa both suggest the meaning of 
\dev{hanti} for being not capable of both stretching and contraction. 
\dev{sandhigataḥ saṃdhīn hanti prasāraṇākuñcanayorasāmarthyaṃ karoti} 
\citep[262]{vulgate} 
...} 
the muscle tissues by residing in the muscle. Residing in the joints it causes 
pain 
and swelling. Residing in the bone it causes fracture and dryness of bones 
which 
also cause to acute pain and, in the marrow, it dries up marrow which may 
never 
be cured. Residing in the semen it causes non-production and distorted 
production of semen.\footnote{Ḍalhaṇa and Gayadāsa both suggest that a 
distorted production \dev{vikṛtāṃ pravṛttim} is too fast, too slow, knotty and 
discolored.} 


\item[30--31ab] Contaminated wind moves from the hand, foot, head, then it 
may be omnipresent or pervade the entire body of men and causes stiffness, 
convulsion, numbness and acute pain.

\item[31cd--32ab] Wind (5 types) mixed with other doṣas (bile etc.) in the 
places mentioned above produces mixed types of pains.

% Symptoms of diseases combined with other humours (bile etc.) of prāṇa 
%wind.
\item[34cd--35ab] Prāṇa wind surrounded by bile causes vomiting and burning 
sensation, by phlegm it causes weakness, exhaustion, laziness and bad taste. 

% Symptoms of diseases combined with other humours (bile etc.) of udāna 
%wind.
\item[35cd--36ab] Udāna wind surrounded by bile causes loss of 
consciousness, stupor, dizziness and fatigue, by phlegm it causes absence of 
perspiration, slowness of digestion, sensation of coldness.

% Symptoms of diseases combined with other humours (bile etc.) of samāna 
%wind.
\item[36cd--37ab] Samāna wind surrounded by bile causes perspiration, a 
burning sensation, heat and stupor, association with phlegm it causes erection 
in 
urine, feces and limbs.  

% Symptoms of diseases combined with other humours (bile etc.) of apāna 
%wind.
\item[37cd--38ab] Apāna wind associated with bile causes a burning sensation, 
heat and the voiding of blood with urine, with phlegm it causes a feeling of 
heaviness in the lower part of the body and coldness.

% Symptoms of diseases combined with other humours (bile etc.) of vyāna 
%wind.
\item[38cd--39ab] Vyāna wind surrounded by bile causes a burning sensation, 
tossing of the limbs and fatigue, by phlegm it causes stiffening limbs, 
uddaṇḍaka? and pain in the swelling.

\item[40--41] Persons who are of delicate nature, follow faulty diet and lifestyle, ? also afflicted with intoxicating drinks, 	sexual enjoyment, exercise causes vitiation of wind and blood.??

\item[42] Riding elephant, horse and camel, lifting great weights, consuming vegetables which are pungent, hot, sour, alkali and being frequently distressed situation causes contamination of wind. 

\item[43--44] Blood flowing in the body blocks the passage of contaminated wind which moves quickly in the body. Excessively irritated wind--being contaminated by wind and dominance of wind, it is called \dev{वातरक्त} Gout\footnote{In the medical term \dev{वातरक्त} is known as Gout. Cakrapāṇi called it \dev{आढ्यरोगः} Carakasaṃhitā sū.14.18 and ci.28.66}.

\item[45-46] Vātarakta causes -- pricking pain, dryness, loos of sensation in the feet. Contaminated Bile mixed with blood causes sharp burning sensation, excessive heat and soft swelling with red color in the feet. Contaminated Phlegm mixed with the blood causes itching in the feet. It makes feet white, cold, dry, thick and hard. All defects \footnote{Gayadāsa suggests \dev{सर्वे दुष्टाः शोणितं चापि} nominative plural instead of locative singular.} in the blood contaminated by humours (wind, bile, phlegm) manifest their symptoms in the feet.

\item[48] This disease spreads all over the body like rat poison by staying in feet or sometimes hands.

\item[49] Gout spreads in the knee and the skin bursts and starts bleeding makes it incurable. It is mitigatable if it is of a year’s old.

\item[50--51] When vitiated wind enters in the all arteries it causes
quickly convulsions again and again and because of frequent
\se{ākṣepa}{contractions} it is called \se{ākṣepaka}{convulsions}.

%types of \dev{अपतानक} are being described.
\item[52--56] Because in this situation a person often sees darkness and
fall, it calls \se{apatānaka}{spasmodic contraction} \footnote{Gayadāsa
    accepted the Nepalese reading \dev{ताम्यते} which vulgate does not read.
    Gayadāsa gives definition of \dev{अपतानक} as \dev{येनापताम्यते} means a
    situation in that a person sees the dark.} . If wind mixed with phlegm
    stays excessively in the arteries, it stiffs body like a staff and it is
    called \dev{दण्डापतानकः} epilepsy with convulsions. Vitiated wind entered
    in the arteries and bends the body like a bow, it is called
    \dev{धनुःस्तम्भ} Tetanus. When vitiated wind accumulated in the regions
    of finger, ancle, abdomen, heart, chest, and throat swiftly attack on the
    group of vain and ligaments, it gets a person’s eyes stuck, chin stuns,
    side breaks and vomiting phlegm he moves inwards like a bow and this
    situation is known as \se{antarāyāma}{emprosthotonos}. When vitiated wind
    attacks on outside ligaments, body of a person will stretch forward like
    a bow. In this situation, if the chest, hip or thigh break, wise men call
    it incurable.

\item[58] Aggravated phlegm and bile mixed with wind or only vitiated wind causes fourth convulsive disease due to trauma.

\item[59] Convulsions due to miscarriage, excessive bleeding, and injury are 
incurable \footnote{According to Ḍalhaṇa \se{ākṣepaka}{convulsion} is also 
known as 
\dev{अपतानक} (Su 1938:266). He further mentions that even if fortunately, it is 
cured, it cripples the limb.}.

\item[60--62] When excessively agitated and strong wind flows in the arteries 
which spread downward, upward, and sideways, it loses the joints and kills the 
other side of body. The best of physicians calls it \se{pakṣāghāta}{paralysis}. 
\footnote{In the ca.6.28.55 \dev{पक्षाघात} is described as 
\se{ekāṅgaroga}{monoplegia}. In that case it damages one of the limbs.  In the 
medical terms \se{apakṣāghāta}{paralysis} is known as hemiplegia.} Then half 
of his entire body becomes inefficient and unconscious. Afflicted by wind he 
suddenly falls or dies.

\item[62.1] Bile integrates with wind causes burning sensation, affliction, and infatuation. When it integrates with phlegm causes coldness, morbid swelling, and heaviness. \footnote{This verse is not available in vulgate. It deals with the symptoms when bile and phlegm mix with the wind. It is already discussed in su.2.1.38.}. 

\item[63] A \se{pakṣāghāta}{paralysis} caused by wind \footnote{Here the term 
\dev{शुद्धवात} suggests the meaning of the wind that is devoid of bile and 
phlegm.} is curable with most difficulty. It becomes curable when caused by bile 
and phlegm mix with the wind. It becomes incurable when caused by the loss of 
bodily constituents.

\item[64--66] Verses from 64--66 are not found in the Nepalese manuscripts. 
These verses discuss the term \se{āpatantraka}{spasmodic contradiction} 
which is the same as \dev{अपतानक}. Ḍalhaṇa commented on ni.1.64-66 (Su 
1938:267) that because of having the similar condition in both situations, some 
scholars do not read the \dev{अपतन्त्रक}. In the verse ni.1.59 Ḍalhaṇa 
commented that the \dev{आक्षेपक} and \dev{अपतानक} is same (Su 1938:266) and 
again he suggested that the \dev{अपतानक} and \dev{अपतन्त्रक} both are similar 
condition. Therefore, \dev{आक्षेपक}, \dev{अपतानक} and \dev{अपतन्त्रक} should be 
the same. Gayadāsa further commented that the Caraka has not read 
\dev{आक्षेपक} as \dev{अपतानक} and therefore described the \dev{अपतन्त्रक} 
separately (Su 1938:267).

\item[67] This verse also not found in the Nepalese Manuscripts. The verse 
describes \se{manyāsthambha}{rigidity of neck}. According to Ḍalhaṇa, rigidity 
of neck is a prior symptom of spasmodic contradiction. 

% \se{अर्दित}{spasm of the jaw-bones}
\item[68--72] By speaking very loudly, eating hard foods, excessively laughing 
and yawning, lifting heavy loads and sleeping in an awkward position, vitiated 
wind lodges into face painfully and produces \se{ardita}{paralysis of the 
jaw-bones} disease. In that case, half of the face and neck become curved, head 
trembles, speech hindrances, deformity occurs in the eys, eyebrows and 
cheeks.\footnote{Ḍalhaṇa suggests \dev{नेत्रादीनाम् इत्यादि शब्दात् भूगण्डादि 
उपसङ्ग्रहः}} Experts in diseases call this disease \se{ardita}{spasm of the 
jaw-bones}. 

\item[73] Spasm of the jawbones cannot be cured when it stays in a person for three years, who is very weak, stays without blinking, trembles, and constantly speaks gibberish.

\item[74] Arteries of Heel and toes stricken by vitiated wind prevents stretching 
of thighs. This disease is known as \se{gṛdhrasī}{sciatica}.

\item[75] Arteries which run to the tips of fingers from behind the roots of the 
upper arm affected by vitiated wind terminates all activities of arms and back. 
This disease is called \se{viśvañci}{paralysis of arms and back}. \footnote{Both 
the MSS N and H read \dev{विश्वञ्चि} instead of the vulgate reading 
\dev{विश्वाची}. There is no such word found in other Āyurveda texts.}

\item[76] Vitiated wind and blood in the joint of knee causes 
\se{kroṣṭukaśīrṣa}{synovitis of knee join}. In this extremely painful situation, the 
shape of swelling in knee joints seems like a head of Jackal. 

\item[77] Vitiated wind resides in the waist attacks on the arteries of thigh 
causes \se{khañja}{limpness} and when it attacks on both the thighs a person 
becomes \se{paṅgu}{lame}.

\item[78] A person who trembles at the beginning of walking or walks limping 
and whose foot joint has become loose is called \se{kalāyakhañja}{lathyrism}.

\item[79] Vitiated wind residing in the ankle-joint causes pain when one steps on uneven ground. This disease occurs is called \dev{वातकण्टक}.

\item[80] Vitiated wind mixed with bile and blood cause burning sensation in feet. 
It should be declared as \se{pādadāha}{burning sensation in feet}.

\item[81] A person whose feet tingle and become insensible due to vitiation of phlegm and wind is called \dev{पादहर्ष}.

\item[82] Vitiated wind lying in the shoulder dries the shoulder joints and it is called \dev{अंसशोष}. It also bends the arteries of shoulder, and this disease is called \dev{अवबाहुक}. \footnote{Ḍalhaṇa and Gayadāsa both have defined two diseases i.e., \dev{अंसशोष} and \dev{अवबाहुक} respectively.}

\item[83] Vitiated wind singly or mixed with phlegm cover the channel of ears causes deafness.

\item[84] Vitiated wind saturated with phlegm covering the arteries which 
conduct the sound of speech makes a person \se{akriya}{inactive}, 
\se{mūka}{dumb}. He \se{mimmira}{mumbles} through the nose and 
\se{gadgad}{stammers}.\footnote{Nepalese Manuscripts read \dev{मिर्म्मिर} 
instead of the Vulgate’s reading \dev{मिन्मिण}. Dictionary of MW suggests the 
meaning of \dev{मिर्म्मिर} = having fixed unwinking eyes which is not relevant to 
the disease of tongue.}

\item[85] Vitiated wind penetrating into the cheekbones, temporal bones, head 
and neck causes piercing pain in the ears. It is called 
\se{karṇaśūla}{ear-ache}.\footnote{In the medical terms, this disease is known 
as Otitis.}

\item[86--87] The pain that arises from the bladder or feces goes down as if it were breaking the rectum and…… ? is called \dev{तूनी}, whereas the pain, rising upward from the rectum extending up to the region of the intestines, is called \dev{प्रतितूनी}.

\item[88--89] Retention of vitiated wind inside abdomen causes distension of the 
stomach and flatulence and intense pain and rumbling inside, is called 
\se{ādhmāna}{tympanites}. Vitiated wind mixed with phlegm causes 
\dev{प्रत्याध्मान}. It rises in the stomach anda causes pain in the heart and 
sides. \footnote{There’s an addition in MS N. \dev{नाभेरधस्तात् संजातः संचारी यदि 
वाऽचलः}}

\item[90--91] A knotty stone-like tumour caused by wind appearing in the stomach having an elevated shape and stretched upward direction which obstructing the passage of faeces and urine should be known as \dev{वाताष्ठीला}. A tumour of similar shape rose obliquely in the abdomen obstructing the passage of wind, faeces and urine should be known as \dev{प्रत्यष्ठीला}. 


Names of diseases discussed in the chapter 2.1

\se{vātarakta}{Gout}
\se{ākṣepaka}{convulsion}
\se{pakṣāghāta}{paralysis of one side}
\se{ardita}{paralysis of the jaw-bones}
\se{gṛdhrasī}{sciatica}
\se{viśvañci}{paralysis of arms and back}
\se{kroṣṭukaśīrṣa}{synovitis of knee join}
\se{kalāyakhañja}{lathyrism}
\se{vātakaṇṭaka}{}
\se{avabāhuka}{}
\se{tūnī}{}
\se{pratitūnī}{}
\se{ādhmāna}{tympanites}
\se{pratyādhmāna}{}
\se{vātāṣṭhīlā}{}
\se{pratyaṣṭhīla}{}




\end{translation}