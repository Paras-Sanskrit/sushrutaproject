% !TeX root = incremental_SS_Translation.tex

\chapter{Śārīrasthāna 2:  On Semen and Menstrual Fluid}

% Jane Allred

\section{Literature} 

Meulenbeld offered an annotated overview of this chapter and a bibliography
of earlier scholarship to 2002.\fvolcite{IA}[244--246]{meul-hist}  \citet[chs 
6--8]{das-2003} also studied topics of this chapter. 

\section{Conceptual background}


\citet[ch.\,13]{das-2003} provides an overview of the conceptual background of 
ayurveda on the topics discussed in this adhyāya.  In brief \ldots 

\section{Translation}

\begin{translation}
    
    \item [1] We shall now explain the purification of \saneng{śukra}{sperm} and 
    \saneng{śoṇita}{blood}.\q{JG in the light of your reflections, I removed 
    “women's fertile”.}
    
    
    \item [3]  \saneng{retas}{Semen}\footnote{The Nepalese version has
    \dev{-retāṃsi} “semen” (in the plural) as the subject of the sentence: “seeds 
    are
    unable to produce offspring\ldots.”  In the vulgate, \dev{-retasaḥ}
    is a masculine bahuvrīhi, making “men whose semen has\ldots” the
    subject of the sentence.} is incompetent to produce offspring if it
    is [characterized by] wind, bile, phlegm,
    \saneng{śoṇita}{blood},\footnote{Note that the list begins with the
        four entities, wind, bile, phlegm and blood, perhaps hinting at a
        four-humour system \citep[see][485--486]{wuja-2000}.}
        \saneng{kuṇapa}{decomposition},
        \saneng{granthi}{lumps},\footnote{Contemporary medicine understands
            that normal ejaculate contains coagula which, however, dissolve after
            about half an hour.  But coagula that do not dissolve may sometimes
            be a sign of an underlying disorder.}\q{JG could you provide a standard 
            citation reference for this information?} \saneng{pūtipūya}{stinking
                pus}, \saneng{kṣīṇa}{low volume}, urine, or feces.
    
\end{translation}