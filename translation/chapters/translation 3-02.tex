% !TeX root = incremental_SS_Translation.tex

\chapter{Śārīrasthāna 2:  On Semen and Menstrual Fluid}

% Jane Allred

\section{Literature} 

Meulenbeld offered an annotated overview of this chapter and a
bibliography of earlier scholarship to
2002.\fvolcite{IA}[244--246]{meul-hist}  \citet[chs 6--8]{das-2003} also
studied topics of this chapter and in chapter 13 provided an overview of
the conceptual background of ayurveda on the topics discussed in this
chapter.  

\section{Translation}

\begin{translation}
    
    \item [1] We shall now explain the anatomy that is the purification of 
    \saneng{śukra}{sperm} and 
    \saneng{śoṇita}{blood}.
%    \q{JG in the light of your reflections, I removed 
%    “women's fertile”. I've put śārīram back in.}
%    
    
    \item [3]  \saneng{retas}{Semen}\footnote{The Nepalese version has
    \dev{-retāṃsi} “semen” (in the plural) as the subject of the
    sentence: “seeds are unable to produce offspring\ldots.”  In the
    vulgate, \dev{-retasaḥ} is a masculine bahuvrīhi, making “men
    whose semen has\ldots” the subject of the sentence.} is
    incompetent to produce offspring if it is [characterized by] wind,
    bile, phlegm, \saneng{śoṇita}{blood},\footnote{Note that the list
        begins with the four entities, wind, bile, phlegm and blood,
        hinting at a four-humour system \citep[see][485--486]{wuja-2000}.}
        \saneng{kuṇapa}{decomposition},
        \saneng{granthi}{clumps},\footnote{\label{granthi}Modern
            Establishment Medicine (MEM) understands that normal ejaculate
            contains coagula which, however, dissolve after about half an
            hour.  But coagula that do not dissolve may sometimes be a sign of
            an underlying disorder (see, e.g., 
            \volcite{2}[614--615]{lamm-1990}; \cite{cohe-1990}).}
            \saneng{pūtipūya}{stinking pus}, \saneng{kṣīṇa}{low volume},
            urine, or feces.
    
 \subsection{Diagnosis by humours}

 \item[4]
 
  When the disfunction is caused by wind, there is a colour and a type of
pain
  that typically goes with wind problems. If  caused by bile the colour and the 
  pain are typical of bile afflictions.  If caused by phlegm the discoloration and 
  suffering are characteristic for phlegm disease.  And if caused by  
  \saneng{śoṇita}{blood} there will be a coloration due to blood and a 
  sensation of a bile affliction. Moreover, when caused by 
  \saneng{rakta}{blood} there is the \se{kuṇapa}{smell of
  decomposition}.\footnote{Note that the text mentions both \dev{śoṇita} and 
  \dev{rakta}.  This raises the question of whether the author considered these 
  to be different, or whether it is an artefact of textual transmission.}  Phlegm 
  with wind causes the appearance of clumps,
  bile with \saneng{śoṇita}{blood}  causes the appearance of 
  \saneng{pūtipūya}{foul-smelling pus}. 
   Bile with \se{māruta}{wind} cause a weakening of semen.
      \se{sannipāta}{Humoral colligation} causes the smell of urine and 
      feces.\footnote{The expression “humoral colligation,” translating 
      \dev{sannipāta}, refers to the simultaneous 
      disorder of three humors at the same time, a condition that is difficult to 
      treat \citep[see][38 \emph{et 
      passim}]{wuja-2016}.}
      
Cases of foul-smelling sperm, sperm with clumps, and when it reeks of
pus are hard to treat.
  
  
   However, when sperm contains urine or faeces there is no
treatment\sse{asādhya}{incurable}.\footnote{Note that the above
    characterizations presuppose the direct inspection of an ejaculate. 
    The process of collection is not described in the sources in this
    chapter.}
 
 \item[5]
 
 Moreover, \se{ārtava}{seasonal blood} too can become
\se{upasṛṣṭa}{afflicted}, \se{abīja}{seedless} because of the three
humours, and blood as the fourth, taken individually, in pairs or
triples or all together.\footnote{This translates the text of the oldest
    surviving witness, N, and the vulgate.  But MS H, that normally follows
    K very closely, has a negative particle, \dev{na}, reversing the sense
    of the sentence.}
 
 This can also be known by means of the humour, colour and pain.
 
 In these cases, that which displays \se{kuṇapa}{decomposition}, clumps 
 and the putrid smell of pus is \se{asādhya}{incurable}. And otherwise it is 
 \se{sādhya}{curable}.
 
 
%  %
%   Rather it is  the pain 
%  caused or the discoloration of the sperm itself that suggest one of these 
%  afflictions. 
  Among these, the kind which shows decomposition, or coagula, or 
  putrid pus is incurable. The other types, however, can be treated.  
 
 \item[6]
 
 And there is a verse on this. 
 
 \begin{sloka}
      An expert should overcome the first three of these sperm
\sse{doṣa}{pathology}pathologies with special
treatments\sse{kriyā}{treatment} such as unction and sweating,
as well as by means of a \se{uttarabasti}{urethral
    instillation}\sse{basti}{instillation}.\label{uttarabastyantam}%
\footnote{\Dalhana{3.2.6}{345} noted that “unction and sweating”
    indicates the “five treatements”: \dev{vamana, virecana, anirūha,
    anuvāsana} and \dev{uttarabasti}.  He noted that the explicit mention
    of urethral enema in the verse was for the purpose of highlighting its
    priority. However, a natural reading of the verse does not suggest
    that these distinctions were in the author's mind.}\q{find out about
        uttarabasti}
    
 \end{sloka}
 
 \subsubsection{Therapies by humour}
 
 \item[6a] In that context, when the sperm is of the nature of wind, there is
a \se{āsthāpana}{tisane}
consisting of \gls{bilva} and \gls{vidārī}.
% 
One may use an oily preparation in the instillations, with well-cooked
\gls{madhūka}, \gls{rāsnā}, \gls{devadāru}, and \gls{sarala}.
%
One can also make the patient drink clarified butter cooked with
\gls{dāḍima}, \gls{mātuluṅga} fruit, \gls{saindhava}, a
\saneng{kṣāra}{caustic}, and \gls{vasukavasira}.
 
 
 
 \item[6b]
 
 When the sperm is of the nature of bile, there is  a
  \se{āsthāpana}{tisane} consisting of the cooked milky sap of 
  \gls{śrīparṇī} and \gls{madhuka} with milk.
  %
  One should also apply a \se{kalka}{paste} of a
  \gls{sarja} and \gls{dhava}
  %axlewood and sala
  in the vagina.  
 %
 One should apply an \se{anuvāsana}{oily enema} of sesame oil cooked with 
 \gls{madhuka}; and it should only be applied as an upper enema.\footnote{By 
 specifying 
 “upper enema” the author is clarifying that this is not a rectal enema.}

  One should make him swallow ghee cooked with 
  \gls{kāṇḍekṣu},
  \gls{śvadaṃśtra},
  \gls{guḍūcī},
  \gls{madhuparṇī},
  \gls{bhṛṅga},
  and 
  the \gls{pañcamūla}.
  
 \item[6c]
 
  When the sperm is of the nature of phlegm, there is  a
\se{āsthāpana}{tisane} consisting of a \se{kaṣāya}{decoction}  of
\gls{rājavṛkṣa}.  And one should also apply an \se{anuvāsana}{oily
    enema} of sesame oil cooked with \gls{pippalī}, \gls{viḍaṅga} and
honey; and it should only be applied as an upper enema. 
He should be given to drink a ghee cooked with 
\gls{pāṣāṇabheda},
\gls{kāśmaryā},
\gls{āmalaka},
\gls{pippalī},
\gls{vasuka}, and 
\gls{vasira}.
 
 
 \item[3.2.6d]  And there are verses about this.
 
 \item [3.2.7]
 
 \begin{sloka}
     When there is blood in the sperm, the physician should give the person ghee 
 cooked with 
 flowers of the \gls{dhātakī},
 \gls{khadira},
 \gls{dāḍima},
 and \gls{arjuna}.
 \end{sloka}
 
 \item [3.2.8]
 
\begin{sloka}
     When it smells like a corpse, he should drink ghee cooked with the
\gls{śālasārādi}. %
\dag When clumps appear, it is cooked with stones, or also in ash from a
\gls{palāśa}.\footnote{The Nepalese text and translation of this sentence
    are uncertain. The vulgate text reads, \Su{3.2.8}{345}: \dev{granthibhūte
    śaṭīsiddhaṃ pālāśe vā 'pi bhasmani} “If clumps appear, it is cooked with
    \emph{śaṭī} or in ash from a \emph{palāśa}.”  The vulgate edition notes in a 
    footnote that some vulgate manuscripts add an extra line, \dev{snehādiśca 
    kramaḥ 
    ṣaṭsvetāsu vijānatā}. The Nepalese manuscripts read this line two verses 
    further 
    down.}


 

 \item[9]
 
And also, when it resembles pus, it is treated with items such as
\gls{parūṣaka} and \gls{vaṭa}.  When the sperm is deficient it should be
treated as was stated before and also as will be
described.\footnote{\Dalhana{3.2.9}{345} noted that “what was stated
    before” refers to the \dev{svayonivardhana} section, i.e., \SS\
    \Su{1.15.10}{69}, and that “what will be described” refers to \SS\
    \Su{4.26}{496}, the chapter on weakness and strength
    (\dev{kṣīṇabalīya}).}
 
 \item [10]
 
 When it looks like feces, he should be made to drink ghee together with
\gls{citraka}, \gls{uśīra} and \gls{hiṅgu}.
 
\item[10a] In these six cases, the wise person should carry out the therapies 
starting with oleation.

\end{sloka} 

\item[10aa]

From

\subsection{Therapies for menstrual blood}

\item [12cd]

For purifying the menstrual blood one should follow the procedure, the
last of which is a \se{uttarabasti}{urethral instillation}.\footnote{The
    “procedure ending with a urethral instillation” probably refers to verse
    6 above (see page \pageref{uttarabastyantam}).}
    
    
 \item[13]  
 
 One should use a \se{kalka}{paste} as well as cloths and a salutary
\se{ācamana}{lavages}.\footnote{The word \dev{ācamana}, normally
    “sipping water from the palm” is here translated “lavage” following the
    context and \Dalhana{3.2.13}{345}, who described it as “water for
    washing the vagina” (\dev{yoniprakṣālanodaka}).  This treatment may be
    intended for the condition mentioned in 12cd, but in the vulgate text
    there is a preceding half verse stating that the treatment is for the
    “four disorders of menstrual blood.”}
    
\item[14]

In case of a bad smell and the appearance of pus, or the appearance of
marrow in the blood.
\item [15]

He should drink a \se{kvātha}{decoction} of \gls{bhadraśrī} or a decoction of 
red \gls{candana}.\footnote{The name \dev{candana} may refer to several 
types of sandalwood; presumably one is meant here that is different from 
white sandalwood, i.e., perhaps Pterocarpus santalinus Linn.\ f.  The vulgate 
has an extra half-śloka here.}
 
\item[14ab]
 
 When \se{granthi}{clumps} appear, he should drink \gls{pāṭhā}, 
 \gls{tryūṣaṇa}, and \gls{vṛkṣaka}.\footnote{On \dev{granthi}, see note 
 \ref{granthi}.}
 
 \item[14a] 
 
 He should drink a a \se{niḥkvātha}{decoction} that is the
\se{surasa}{extracted juice} of  a \se{kṣāra}{caustic}, \gls{nāgara},
and \gls{hiṅgu}.
 
\item[3.2.24]

Thus a man has unblemished semen and a woman has pure menstrual blood. 
 
 \subsection{During menstruation}
 
  \begin{tt}
     \bigskip
     \raggedright
     %\item[6H]
 
% \item[9]
% 
%  In case the sperm shows signs of decomposition, one should make the patient 
%  drink a medicated fluid containing dhātaki flowers, cutch-tree, pomegranate 
%  and arjuna tree bark.
% 
% \item[8]
% 
%  In case of apparent disintegration of the sperm, he should drink clarified 
%butter 
%  with heart of sāl. Moreover in case of lumps and clots, he should even eat a 
%  preparation of ashes obtained after burning of a fig-tree.
 
 \item[9]
 
  In case the sperm appears purulent, a mixture of mangrove canon ball in some 
  food leftovers or anything else should be prepared. When the sperm is 
  depleted, one should perform these instructions straight away as soon as they 
  have been explained.
 
 \item[10]
 
  One should make the patient drink ghee with citra, koshira and hingu by way of 
  an antidote. A wise person should then perform one by one the six oleation 
  processes on his own body.
  
  %%%%%%%%%%%%%%%%%%
  
  
\item[10A]  By not engaging in sexual activities with women for a long time and 
similarly 
  through the use of expedients and instruments
  
\item[10B]

By intense use of astringent, pungent or bitter substances
  
\item[10C]

Like an acid or a salty, oily or fermented solution, that has just stood 
  somewhere for some time,
  
\item[10D]

deteriorates both by the effect of time going by and by interaction with 
  yogi’s*  
  
\item[10E]

similarly in case of affection of the female system one should prescribe 
  oleation and other  similar treatment.
  
\item[10F]

Precisely as formulated for external use, one could prescribe exactly 
  the same for internal administration as well.
  
\item[10G]

In case of disease resulting from trouble with Vāyu, then the patient 
  must drink a beverage consisting of clarified butter, prepared with split cedar 
  and Kāśmarya fruit.
  
\item[10H]

One should prescribe either an intravaginal solution of Payasyā, 
  Kāśmarya fruit, Kṣīravidārī and Udaka sap or lumps drenched in diluted milk.
  
\item[10I]

A sip from the palm of the hand of Madhukamunga and astringent betel
  
\item[10J]

Furthermore, in case of trouble caused by bile and in the menstrual 
  cycle.
  
\item[10K]

The patient should drink milky Kākolī sap and a decoction of Vidārī to 
  which some candied sugar is added
  
\item[10L]

and one should insert an intravaginal solution of Madhuka flowers and 
  Kāśmarya fruit mixed with sugar cane juice or a paste with santal sap.
  
\item[10M]

and a pinch of astringent Paan*
  
  
\item[10N]

When disease is caused by phlegm, he must drink an astringent 
  sandalwood and acrid Christmas rose solution.
  
\item[10O]

… or a paste of young tree sprouts soaked in a plant juice.
  
\item[10P]

he must lick or sip bits of Tinduka, wood apple, slime apple and sandal 
  powder or Kṣandra.
  
\item[10Q]

 Sarjadhava paste should be placed inside the vagina
  
\item[10R]

a pinch of Ladhra and astringent Tinduka
  
\item[10S]

In case of clots and lumps in the sperm he should take Sringavera and 
  Pāṭhā to which is added some sandal powder as well as white Surasa.
  
\item[10T]

One should insert into the vagina a salve* with Kustha and cedar 
  extracts.
  
\item[10U]

and add just a pinch of something astringent
  
\item[10V]

in case the sperm is clearly in decomposition, the patient must be 
  prescribed to drink a astringent beverage of Manjista and astringent sandalwood
  
\item[10W]

or of Kuṭaja fruit, sandalwood and sandal sweetened with sugarcane 
  candy.
  
\item[10X]

or in case there is obvious pus, this is exactly what the patient should 
  be given
  
\item[10Y]

 and the lady should be prescribed to place  inside the vagina a salve of 
  cachou and arjuna
  
\item[10Z]

and for both a pinch of something astringent is indicated is just right.
  
\item[10A1]

in case of …………*, he should ingest an astringent solution of false 
  black pepper, coral tree and Manjista.
  
\item[10A2]

one should introduce into the vagina a paste of Surastastra (?)* and 
  Rocana as well as  a salve of Bhadrasriya.
  
\item[10A3]

Both must receive just a sip of something astringent.
  
\item[10A4]

Here is more.
  
\item[11]


  
\item[12]

Generally speaking, at the end of the period one should apply an internal 
  cleansing from menstrual discharge*
  
  
\item[13]

and one should definitely prepare both cotton-plant paste and salutary 
  beverages.
  
\item[14]

in case the sperm spreads a foul-smelling stench and definitely when 
  there is blood
  
\item[15]

the patient should drink decoctions of sandal or sandalpaste.*
  
\item[14AB]

and in case of lumps and clots in his sperm he should consume a 
  salutary Tryusana with coral swirl fruits.
  
\item[14A]

and drink a beverage of acrid dry ginger, Hing and holy basil.
  
\item[16]


\item[23]


  
\item[24]

That being said, when sperm is not causing disease, even then all of this 
  may serve to purify the female system*
  
\item[25]

From the first day onwards* when the period starts, she  should shun 
  young celibates, when they are alone, bathing, anointing, decorating and 
  scratching themselves and she should also suppress day-sleep, put collyrium to 
  her eyes, weeping, be frightened or cut her nails, run hither and tither, laugh or 
  speak or listen to lots of talk or exert herself. Why should her partner arrive 
  late during the day? It is claimed that if she puts a collyrium, a child will be born 
  blind, if she is weeping, he will have abnormal vision, by bathing and anointing 
  he will be depressive, by smearing oil on her limbs, he ‘ll be born a leper, by 
  holding her nail downwards he will be a child with ugly nails, by being 
  continuously busy he will become a restless and troubled character. When she 
  behaves accordingly, she well be the best of future mothers. She should take 
  rest on a layer of Kuśa  grass,  take Haviśya-food so pure that it is fit for a 
  sacrifice**, eat varied food from the palm of her hand or from a plate made of 
  leaves and she should keep herself then from her husband from the third day 
  onwards. However, on the fourth day, first she should take a ritual, put on a 
  new untorn dress and some jewels, the make an auspicious happy recitation 
  and then confidently hug her husband. What is the purpose of all this?
  
\item[26]

Once she has taken a ritual bath after her period, a woman should put 
  eyes on her husband before anyone else.
  
\item[27]

Then the priest conducts the rites for procuring a son* and at the end of 
  i, the husband should be seen to closely observe the following.
  
\item[28]

 In order to beget a male child he should eat both clarified butter and 
  milk as well as śāli-rice* boiled in water. After observing a month of sexual 
  abstinence, the wife should lubricate herself in oil from the very best Māṣa 
  pulse and he should approach her at night. After gaining her complete trust by 
  gentle words he should then make his move on the fourth, sixth, eighth, tenth 
  and twelfth day (of the cycle)** successively.
  
\item[31]

Each month again she should be approached sexually.* 
  
\item[32]

And when conception has occurred in this way during one of these 
  nights, it is claimed that she should press three or four drop of juice from 
  Lakṣmaṇā, Vaṭaśuṅgā, Sahadevā, Viśvadevānā or any other drug and then 
  administer them in the right nostril if she desires a son and in the left if she 
  wants a girl, and not spit nor sneeze them out. 
  
\item[32a]

Here are some more verses.
  
\item[11cd]

On top of that those around her want to see her smelling sweet as 
  honey, sparkling like a crystal, agile and active, smooth and sweetly perfumed, 
  
\item[12ab]

bright with splendour equally due to the smell of honey as to the 
  smoothness of oil. 
  
\item[17]

It is a token of good health when the menstrual blood is red like a hare’ s 
  blood or like the shine of red lac and when its colour stains can be removed.
  
\item[18]

Metrorrhagia or abnormal uterine bleeding is diagnosed when there is 
  either excessive bleeding, untimely or irregular bleeding or when symptoms are 
  the opposite of what occurs in a normal menstrual cycle. 
  
\item[19]

 Excessive uterine bleeding is always accompanied by aching limbs and 
  with pain. In case blood loss is extremely abundant, symptoms may be 
  weakness, …………………. (bhramamūrcchā), fatigue,…
  
\item[20]

… fever, lamenting pain, anaemia*, tiredness and others signs of 
  disturbance of Vāta. A minor concomitant disease may easily set in motion 
  ………….. (taruṇyā).
  
\item[21cd]

Because these afflictions have a recurrent character, the woman 
  becomes amenorrhoeic. 
  
\item[22]

In such a case a diet is indicated including meat, Kulattha-pulses, sour 
  Tila-seeds, Māṣa-beans and whine and for drinks (cow)urine, whey and sour 
  curd.
  
\item[23]

In case of thin or scanty menses with features that cannot be treated 
  with drugs, other measures indicated in case of uterine metrorrhagia must be 
  taken. 
  
\item[29]


  eṣūttarottaraṃ vidyādāyurārogyameva ca || prajāsaubhāgyamaiśvaryaṃ balaṃ 
  ca divaseṣu vai ||
  
\item[30]


  ataḥ paraṃ pañcamyāṃ saptamyāṃ navamyāmekādaśyāṃ ca strīkāmaḥ; 
  trayodaśīprabhṛtayo nindyāḥ ||
  
\item[33]

 When the four ingredients for the embryo are combined, i.e. the right 
  womb to grow in, the right seed to descend from, the propitious life juices to be 
  fed upon and a lucky constellation of stars, according to age-old tradition, the 
  newborn will grow into a child of unshakeable health. 
  
\item[34]

Conceived and developed in this way, they become beautiful, of noble 
  character and they live a long life. Although, beings sons, they have obligations 
  to fulfil towards their parents, they can take care of these and thus honourably 
  discharge themselves.
  
\item[35]

On the one hand there those who claim that it is the Tejas-element 
  which lies at the base of the different types of complexions, on the other there 
  are those who say that it is the colour of the food the mother eats while 
  pregnant that dictates the complexion. The normal complexion of the foetus 
  therefore is fair. But when earth (as a source of food) is the main determining 
  element, complexion will shift tod ark. When a mix of earth and sky are the 
  main elements (in the food), it turns towards the dark bluish. By analogy, some 
  say it is the colour of the food the woman eats while pregnant that fixes the 
  complexion of her offspring. There are arguments in favour of both theories. 
  In so far as in dark, yellow and white ( kṛṣṇapītasvetāsu)  earths ( bhūmiṣu) 
  snakes, trees, ….. and so on (sarppavṛścikagalagoṇādayaḥ) are essential 
  elements (satvāḥ), they are black, yellow and white (kṛṣṇapītasveta). (uncertain 
  hence in italics)
  When the Tejas-principle fails, the child is born blind. Similarly, when 
  penetration into the blood of this disturbance results in a newborn with 
  blood-shot red eyes; penetration into the phlegm makes for a pale-eyed 
  newborn; penetration into the bile makes for a yellow-eyed baby; penetration 
  into the wind results  the in a with eyes that have poor vision. 
  
\item[36]

Here are some more verses.
  
\item[36a]

He whose eyes are entered by a pure wind
  
\item[36bj]

will have oblong downcast eyes, dark or bright.
  
\item[36c]

When bile with phlegm both are present unsullied in a man’ s eye 
  
\item[36d]

then in that man’s eye the yellow, green and reddish-brown will all light 
  up together. 
  
\item[36e]

when phlegm has shaken off all kinds of bodily secretions from 
  someone’s eyes,
  
\item[36f]

then both irises of that person will light up brightly shining. 
  
\item[36g]

Whenever blood with phlegm move around in a someone’s eyes,
  
\item[36h]

he will appear to have either bluish-dark or blood-shot pupils. 
  
\item[36i]

Just as a lump of ghee melts when placed near a fire,\footnote{Cf.\ the attempt 
by \citet[222--241]{das-2003} to identify the \SS's descriptions with the physical 
processes involved as known to Modern Establishment Medicine, and also the 
self-contradictions in the ancient āyurvedic medical models.}
  
\item[36j]

so a woman’ s propensity to ovulate glides into receptivity in contact 
  with a man.  
  
\item[37]

When sperm is divided in the uterus by the wind into two beings, twins 
  are born conditioned by the former good and evil deeds.  
  
\item[37.1]

When in the mixture there is an excess of male sperm, a fertile woman 
  will create two male children*.
  
\item[37.2]

whereas when there is an excess of female semen then similarly the 
  woman engenders two girls. But there is no certainty.*
  
\item[37.3]

 A child born from a man who has but a poor sperm to give his wife is 
  called āsevyaḥ (impotent).*
  
\item[38cd]

When a man has a fellatio he does not have to doubt his penis will get 
  up erect.*
  
\item[38]

 A boy born from a father with poor sperm becomes an āsekya.
  
\item[39]

 He who is born in a sordid vagina is commonly known as a Saugandhika. 
  Such a person becomes aroused only after smelling a vagina or a scrotum. *
  
\item[40]

 When a man first had same-sex anal coitus because of a period of 
  sexual abstinence from women and then turns towards his regular partners* 
  again, he should be known as a Kumbhīka. And now get it right about what an 
  īrṣyakaṃ is:
  
\item[41]

 somebody who has to watch sexual intercourse of others before being 
  able to his own sexual activities should be known as an  īrṣyakaḥ.* He who 
  turns towards copulation**
  
\item[42]

 during the fertile days of the cycle* but out of pure sexual ignorance 
  ejaculates on the breasts of his virgin wife** will create boys who also exhibit 
  feminine character traits.
  
\item[42]

see 3.2.41
  
\item[43]

 If a woman in her fertile days* throws herself at the feet of males 
  around her and she begets a girl, she will also have character traits of a man. 
  *** 
  
\item[44]

Men who do produce sperm but have a pathology can be identified as 
  āsekya, Sugandhi, Kumbhika or īrṣyaka. Men who do not produce any sperm 
  are called saṇḍha.*
  
\item[45]

The sperm ducts that lead the sperm in both groups of men should be 
  …………. (viprakṛtyā) of these (teṣāṃ) (?). This will help instore a slow evolution 
  towards satisfactory erections. 
  
\item[46]

 It is to be expected that what the mother eats will reflect in how 
  children behave both in the uterus and after birth.*
  
\item[47]

 Now when two women are having sex and somehow succeed in making 
  fit both their sperm contributions then a boneless being is born.
  
\item[48]

A woman could even get carried away and reach an orgasm in a dream 
  following her ritual bath. The Vāyu then transporting her fertilized egg into the 
  uterus, results in her belly …
  
\item[49]

 … showing the obvious signs of pregnancy month by month in the 
  pregnant lady.
  
\item[50]

In addition, it should be known that monster-like creatures looking like 
  serpents, scorpions and pumpkin-gourd shaped foetus, are born frequently from 
  the womb as a consequence of sins committed.
  
\item[51]

When a pregnant woman’s wishes are not respected due to a deranged 
  condition of the Vāta, the child stands in danger of being born a humpback or 
  ……….. (kūnipaṇgur) or dumb,
  
\item[52]

and when the parents are atheistic or due to the aggravation of Vāyu are 
  under the effect of misdeeds in former lives, the newborn may develop 
  malformations.
  
\item[53]

 Due to the scantiness of bodily excretions, itself due to a disabling of 
  Vāyu with respect to processing of food, the foetus, whilst in the womb, 
  produces (almost)* no urine nor stools, 
  
\item[54]

 and because of this dwindling away of the Vāyu in mouth and throat, in 
  the bowels and especially in the small intestine, these all get wrapped up in 
  phlegm resulting in impediment of intestinal transit; moreover the foetus does 
  not weep all the time …*
  
\item[55]

Furthermore, the ups-and-downs of the foetal respiratory movements 
  during its sleep are coordinated with the ups-and-downs of the respiratory 
  movements of the mother.
  
\item[56]

 The adjustment of the limbs of the body to its bodily constraints, both 
  the appearance and the falling out of teeth, the disappearance of hair from the 
  palms of hands and soles, all of this follows intrinsic laws of nature.
  
\item[57]

 Men who have uninterruptedly entered one previous existence after 
  another and who have a vast understanding of the scriptures, do remember 
  their own previous births.
  
  This was the second chapter of the śārīrāsthana.
  
\end{tt}
\end{translation}
