% !TeX root = incremental_SS_Translation.tex

\chapter{Śārīrasthāna 2:  On Semen and Menstrual Fluid}

% Jane Allred

\section{Literature} 

Meulenbeld offered an annotated overview of this chapter and a bibliography
of earlier scholarship to 2002.\fvolcite{IA}[244--246]{meul-hist}  \citet[chs 
6--8]{das-2003} also studied topics of this chapter. 

\section{Conceptual background}


\citet[ch.\,13]{das-2003} provides an overview of the conceptual background of 
ayurveda on the topics discussed in this adhyāya.  In brief \ldots 

\section{Translation}

\begin{translation}
    
    \item [1] We shall now explain the anatomy that is the purification of 
    \saneng{śukra}{sperm} and 
    \saneng{śoṇita}{blood}.\q{JG in the light of your reflections, I removed 
    “women's fertile”. I've put śārīram back in.}
    
    
    \item [3]  \saneng{retas}{Semen}\footnote{The Nepalese version has
    \dev{-retāṃsi} “semen” (in the plural) as the subject of the sentence: “seeds 
    are
    unable to produce offspring\ldots.”  In the vulgate, \dev{-retasaḥ}
    is a masculine bahuvrīhi, making “men whose semen has\ldots” the
    subject of the sentence.} is incompetent to produce offspring if it
    is [characterized by] wind, bile, phlegm,
    \saneng{śoṇita}{blood},\footnote{Note that the list begins with the
        four entities, wind, bile, phlegm and blood, perhaps hinting at a
        four-humour system \citep[see][485--486]{wuja-2000}.}
        \saneng{kuṇapa}{decomposition},
        \saneng{granthi}{lumps},\footnote{Contemporary medicine understands
            that normal ejaculate contains coagula which, however, dissolve after
            about half an hour.  But coagula that do not dissolve may sometimes
            be a sign of an underlying disorder.}\q{JG could you provide a standard 
            citation reference for this information?} \saneng{pūtipūya}{stinking
                pus}, \saneng{kṣīṇa}{low volume}, urine, or feces.
    
 %%%%%%%%%%%
 
\subsection{JG translation}             
 
 \item[1]
 
  We shall now discuss male and female reproductive function and anatomy.
 
\item[2]
 
  This is how Dhanvantari was teaching.
 
 \item[3]
 
  Sperm becomes unable to result in offspring when it is under a negative effect 
  of wind, bile, phlegm or blood, of decomposition, lumps, purulent matter or real 
  pus, of volume depletion or of the presence of urine or faeces.
 
 \item[4]
 
  When the disfunction is caused by wind, there is a colour and a type of pain 
  that typically goes with wind problems; if  caused by bile the colour and the 
  pain are typical of bile afflictions; if caused by phlegm the discoloration and 
  suffering are characteristic for phlegm disease; and if caused by some female 
  bleeding there will be a discoloration due to blood and a sensation similar to 
  that when there is a bile affliction. Moreover when caused by blood and 
  decomposition, or if the affection is caused by both phlegm and wind 
  disfunction, or when the sperm is characterized by the presence of lumps and 
  clots, and if caused by both bile and female bleeding problems, the sperm 
  becomes foul-smelling; if caused by both bile and wind troubles the volume gets 
  depleted; when there is some episode of despair a smell of urine and faeces will 
  occur. Some of these sperm abnormalities can be treated, e.g. cases of 
  foul-smelling sperm, sperm containing an abnormal amount of clotting lumps, 
  and when it reeks of pus and causes excruciating pain. However, when sperm 
  contains urine or faeces there is no treatment.
 
 \item[5]
 
  Moreover, in the period of about ten days following the onset of the menses - 
  when the woman is receptive to becoming pregnant - the sperm can be vitiated 
  by any of the three pathologies that may occur during the first quarter of the 
  menstrual cycle, either separately or by two or three of them or even all three 
  together but this will not necessarily lead to subfertility. Rather it is  the pain 
  caused or the discoloration of the sperm itself that suggest one of these 
  afflictions. Among these, the kind which shows decomposition, or coagula, or 
  putrid pus is incurable. The other types, however, can be treated.  
 
 \item[6]
 
 Such are the facts. A smart professional getting the most out of his 
 professional competence will, normally speaking, be able to treat the first three 
 among these sperm pathologies. What is needed therefore can be either 
 lubrification, or making the tissues exude or any other tricks of the trade, such 
 as something like an enemas or an instillation.
 
 \item[6A]
 
  When the sperm is negatively affected by wind disorders, one should applicate 
  an oily enema containing \gls{bilva} and \gls{vidārī}.\q{I have replaced the 
  plant-names with entries from my plant database.} 
 
 \item[6B]
 
  One could also consider administering an oily preparation, well-cooked and 
  medicated with simple \gls{devadāru} drenched in honey, in the form of an 
  enema. 
 
 \item[6C]
 
  One can also make the patient drink clarified butter finished with \gls{dāḍima}, 
  \gls{mātuluṅga} fruit, \gls{saindhava}, a \saneng{kṣāra}{caustic}, and 
  \gls{vasukavasira}.
 
 % got to here with \gls
 
 \item[6D]
 
   When sperm disfunction is due to bile issues, one can prescribe application of 
   a preparation based on the milky juice of plants cooked with honey or else 
   sharply tasting betel leaves in milk or curd.
 
 \item[6E]
 
  One could apply also a salve of axlewood and sal into the vagina.
 
 \item[6F]
 
  Or apply externally an oily preparation of well-cooked honey.
 
 \item[6G]
 
  Of course that oily preparation could also be applied in the form of an enema.
 
 \item[6H]
 
  One can also make him swallow a beverage of clarified butter finished with the 
  “five roots”: nightshade, betel, moonseed in honey, dog’s tooth and sugarcane 
  stalks. 
 
 \item[6I]
 
  If the sperm is afflicted because disturbances in phlegm, one can consider an 
  oily ghee-based preparation with adstringent leaves of the golden shower tree.
 
 \item[6J]
 
  The oil processed as a medicated decoction of long pepper, honey and false 
  black pepper should be administered as an anointment but similarly also in the 
  form of an enema. 
 
 \item[6K]
 
  One should try a solution of cooled-down clarified butter, compounded with the 
  juice of basil, Indian gooseberry, long pepper and stone-breaker plant in case of 
  kidney gravel disease.
 
 \item[6L]
 
  Here are some more verses.
 
 \item[7]
 
  In case the sperm shows signs of decomposition, one should make the patient 
  drink a medicated fluid containing dhātaki flowers, cutch-tree, pomegranate 
  and arjuna tree bark.
 
 \item[8]
 
  In case of apparent disintegration of the sperm, he should drink clarified butter 
  with heart of sāl. Moreover in case of lumps and clots, he should even eat a 
  preparation of ashes obtained after burning of a fig-tree.
 
 \item[9]
 
  In case the sperm appears purulent, a mixture of mangrove canon ball in some 
  food leftovers or anything else should be prepared. When the sperm is 
  depleted, one should perform these instructions straight away as soon as they 
  have been explained.
 
 \item[10]
 
  One should make the patient drink ghee with citra, koshira and hingu by way of 
  an antidote. A wise person should then perform one by one the six oleation 
  processes on his own body.   
    
\end{translation}