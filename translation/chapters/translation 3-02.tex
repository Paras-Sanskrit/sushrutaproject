% !TeX root = incremental_SS_Translation.tex

\chapter{Śārīrasthāna 2:  On Semen and Menstrual Fluid}

% Jane Allred

\section{Literature} 

Meulenbeld offered an annotated overview of this chapter and a
bibliography of earlier scholarship to
2002.\fvolcite{IA}[244--246]{meul-hist}  \citet[chs 6--8]{das-2003} also
studied topics of this chapter and in chapter 13 provided an overview of
the conceptual background of ayurveda on the topics discussed in this
chapter.  

\section{Translation}

\begin{translation}
    
    \item [1] We shall now explain the anatomy that is the purification of 
    \saneng{śukra}{sperm} and 
    \saneng{śoṇita}{blood}.
%    \q{JG in the light of your reflections, I removed 
%    “women's fertile”. I've put śārīram back in.}
%    
    
    \item [3]  \saneng{retas}{Semen}\footnote{The Nepalese version has
    \dev{-retāṃsi} “semen” (in the plural) as the subject of the
    sentence: “seeds are unable to produce offspring\ldots.”  In the
    vulgate, \dev{-retasaḥ} is a masculine bahuvrīhi, making “men
    whose semen has\ldots” the subject of the sentence.} is
    incompetent to produce offspring if it is [characterized by] wind,
    bile, phlegm, \saneng{śoṇita}{blood},\footnote{Note that the list
        begins with the four entities, wind, bile, phlegm and blood,
        hinting at a four-humour system \citep[see][485--486]{wuja-2000}.}
        \saneng{kuṇapa}{decomposition},
        \saneng{granthi}{clumps},\footnote{\label{granthi}Modern
            Establishment Medicine (MEM) understands that normal ejaculate
            contains coagula which, however, dissolve after about half an
            hour.  But coagula that do not dissolve may sometimes be a sign of
            an underlying disorder (see, e.g., 
            \volcite{2}[614--615]{lamm-1990}; \cite{cohe-1990}).}
            \saneng{pūtipūya}{stinking pus}, \saneng{kṣīṇa}{low volume},
            urine, or feces.
    
 \subsection{Diagnosis by humours}

 \item[4]
 
  When the disfunction is caused by wind, there is a colour and a type of
pain
  that typically goes with wind problems. If  caused by bile the colour and the 
  pain are typical of bile afflictions.  If caused by phlegm the discoloration and 
  suffering are characteristic for phlegm disease.  And if caused by  
  \saneng{śoṇita}{blood} there will be a coloration due to blood and a 
  sensation of a bile affliction. Moreover, when caused by 
  \saneng{rakta}{blood} there is the \se{kuṇapa}{smell of
  decomposition}.\footnote{Note that the text mentions both \dev{śoṇita} and 
  \dev{rakta}.  This raises the question of whether the author considered these 
  to be different, or whether it is an artefact of textual transmission.}  Phlegm 
  with wind causes the appearance of clumps,
  bile with \saneng{śoṇita}{blood}  causes the appearance of 
  \saneng{pūtipūya}{foul-smelling pus}. 
   Bile with \se{māruta}{wind} cause a weakening of semen.
      \se{sannipāta}{Humoral colligation} causes the smell of urine and 
      feces.\footnote{The expression “humoral colligation,” translating 
      \dev{sannipāta}, refers to the simultaneous 
      disorder of three humors at the same time, a condition that is difficult to 
      treat \citep[see][38 \emph{et 
      passim}]{wuja-2016}.}
      
Cases of foul-smelling sperm, sperm with clumps, and when it reeks of
pus are hard to treat.
  
  
   However, when sperm contains urine or faeces there is no
treatment\sse{asādhya}{incurable}.\footnote{Note that the above
    characterizations presuppose the direct inspection of an ejaculate. 
    The process of collection is not described in the sources in this
    chapter.}
 
 \item[5]
 
 Moreover, \se{ārtava}{seasonal blood} too can become
\se{upasṛṣṭa}{afflicted}, \se{abīja}{seedless} because of the three
humours, and blood as the fourth, taken individually, in pairs or
triples or all together.\footnote{This translates the text of the oldest
    surviving witness, N, and the vulgate.  But MS H, that normally follows
    K very closely, has a negative particle, \dev{na}, reversing the sense
    of the sentence.}
 
 This can also be known by means of the humour, colour and pain.
 
 In these cases, that which displays \se{kuṇapa}{decomposition}, clumps 
 and the putrid smell of pus is \se{asādhya}{incurable}. And otherwise it is 
 \se{sādhya}{curable}.
 
 
%  %
%   Rather it is  the pain 
%  caused or the discoloration of the sperm itself that suggest one of these 
%  afflictions. 
  Among these, the kind which shows decomposition, or coagula, or 
  putrid pus is incurable. The other types, however, can be treated.  
 
 \item[6]
 
 And there is a verse on this. 
 
 \begin{sloka}
      An expert should overcome the first three of these sperm
\sse{doṣa}{pathology}pathologies with special
treatments\sse{kriyā}{treatment} such as unction and sweating,
as well as by means of a \se{uttarabasti}{urethral
    instillation}\sse{basti}{instillation}.\label{uttarabastyantam}%
\footnote{\Dalhana{3.2.6}{345} noted that “unction and sweating”
    indicates the “five treatements”: \dev{vamana, virecana, anirūha,
    anuvāsana} and \dev{uttarabasti}.  He noted that the explicit mention
    of urethral enema in the verse was for the purpose of highlighting its
    priority. However, a natural reading of the verse does not suggest
    that these distinctions were in the author's mind.}\q{find out about
        uttarabasti}
    
 \end{sloka}
 
 \subsubsection{Therapies by humour}
 
 \item[6a] In that context, when the sperm is of the nature of wind, there is
a \se{āsthāpana}{tisane}
consisting of \gls{bilva} and \gls{vidārī}.
% 
One may use an oily preparation in the instillations, with well-cooked
\gls{madhūka}, \gls{rāsnā}, \gls{devadāru}, and \gls{sarala}.
%
One can also make the patient drink clarified butter cooked with
\gls{dāḍima}, \gls{mātuluṅga} fruit, \gls{saindhava}, a
\saneng{kṣāra}{caustic}, and \gls{vasukavasira}.
 
 
 
 \item[6b]
 
 When the sperm is of the nature of bile, there is  a
  \se{āsthāpana}{tisane} consisting of the cooked milky sap of 
  \gls{śrīparṇī} and \gls{madhuka} with milk.
  %
  One should also apply a \se{kalka}{paste} of a
  \gls{sarja} and \gls{dhava}
  %axlewood and sala
  in the vagina.  
 %
 One should apply an \se{anuvāsana}{oily enema} of sesame oil cooked with 
 \gls{madhuka}; and it should only be applied as an upper enema.\footnote{By 
 specifying 
 “upper enema” the author is clarifying that this is not a rectal enema.}

  One should make him swallow ghee cooked with 
  \gls{kāṇḍekṣu},
  \gls{śvadaṃśtra},
  \gls{guḍūcī},
  \gls{madhuparṇī},
  \gls{bhṛṅga},
  and 
  the \gls{pañcamūla}.
  
 \item[6c]
 
  When the sperm is of the nature of phlegm, there is  a
\se{āsthāpana}{tisane} consisting of a \se{kaṣāya}{decoction}  of
\gls{rājavṛkṣa}.  And one should also apply an \se{anuvāsana}{oily
    enema} of sesame oil cooked with \gls{pippalī}, \gls{viḍaṅga} and
honey; and it should only be applied as an upper enema. 
He should be given to drink a ghee cooked with 
\gls{pāṣāṇabheda},
\gls{kāśmaryā},
\gls{āmalaka},
\gls{pippalī},
\gls{vasuka}, and 
\gls{vasira}.
 
 
 \item[3.2.6d]  And there are verses about this.
 
 \item [3.2.7]
 
 \begin{sloka}
     When there is blood in the sperm, the physician should give the person ghee 
 cooked with 
 flowers of the \gls{dhātakī},
 \gls{khadira},
 \gls{dāḍima},
 and \gls{arjuna}.
 \end{sloka}
 
 \item [3.2.8]
 
\begin{sloka}
     When it smells like a corpse, he should drink ghee cooked with the
\gls{śālasārādi}. %
\dag When clumps appear, it is cooked with stones, or also in ash from a
\gls{palāśa}.\footnote{The Nepalese text and translation of this sentence
    are uncertain. The vulgate text reads, \Su{3.2.8}{345}: \dev{granthibhūte
    śaṭīsiddhaṃ pālāśe vā 'pi bhasmani} “If clumps appear, it is cooked with
    \emph{śaṭī} or in ash from a \emph{palāśa}.”  The vulgate edition notes in a 
    footnote that some vulgate manuscripts add an extra line, \dev{snehādiśca 
    kramaḥ 
    ṣaṭsvetāsu vijānatā}. The Nepalese manuscripts read this line two verses 
    further 
    down.}


 

 \item[9]
 
And also, when it resembles pus, it is treated with items such as
\gls{parūṣaka} and \gls{vaṭa}.  When the sperm is deficient it should be
treated as was stated before and also as will be
described.\footnote{\Dalhana{3.2.9}{345} noted that “what was stated
    before” refers to the \dev{svayonivardhana} section, i.e., \SS\
    \Su{1.15.10}{69}, and that “what will be described” refers to \SS\
    \Su{4.26}{496}, the chapter on weakness and strength
    (\dev{kṣīṇabalīya}).}
 
 \item [10]
 
 When it looks like feces, he should be made to drink ghee together with
\gls{citraka}, \gls{uśīra} and \gls{hiṅgu}.
 
\item[10a] In these six cases, the wise person should carry out 
oleation and succeeding therapies.

\end{sloka} 

\item[10aa]

From

\subsection{Therapies for menstrual blood}

\item [12cd]

For purifying the menstrual blood one should follow the procedure, the
last of which is a \se{uttarabasti}{urethral instillation}.\footnote{The
    “procedure ending with a urethral instillation” probably refers to verse
    6 above (see page \pageref{uttarabastyantam}).}
    
    
 \item[13]  
 
 One should use a \se{kalka}{paste} as well as cloths and a salutary
\se{ācamana}{lavages}.\footnote{The word \dev{ācamana}, normally
    “sipping water from the palm” is here translated “lavage” following the
    context and \Dalhana{3.2.13}{345}, who described it as “water for
    washing the vagina” (\dev{yoniprakṣālanodaka}).  This treatment may be
    intended for the condition mentioned in 12cd, but in the vulgate text
    there is a preceding half verse stating that the treatment is for the
    “four disorders of menstrual blood.”}
    
\item[14]

In case of a bad smell and the appearance of pus, or the appearance of
marrow in the blood.
\item [15]

He should drink a \se{kvātha}{decoction} of \gls{bhadraśrī} or a decoction of 
red \gls{candana}.\footnote{The name \dev{candana} may refer to several 
types of sandalwood; presumably one is meant here that is different from 
white sandalwood, i.e., perhaps Pterocarpus santalinus Linn.\ f.  The vulgate 
has an extra half-śloka here.}
 
\item[14ab]
 
 When \se{granthi}{clumps} appear, he should drink \gls{pāṭhā}, 
 \gls{tryūṣaṇa}, and \gls{vṛkṣaka}.\footnote{On \dev{granthi}, see note 
 \ref{granthi}.}
 
 \item[14a] 
 
 He should drink a a \se{niḥkvātha}{decoction} that is the
\se{surasa}{extracted juice} of  a \se{kṣāra}{caustic}, \gls{nāgara},
and \gls{hiṅgu}.
 
\item[\ldots] 
 
\item[24]

Thus a man has unblemished semen and a woman has pure menstrual blood. 
 
 \subsection{During menstruation}
 
 \item[25]
 
During the \se{ṛtu}{season}, starting from the first day onwards, the
\se{brahmacāriṇī}{chaste woman} foregoes bathing, anointments,
ornaments and \se{vilekhana}{grooming}.\footnote{The word \dev{ṛtu}
    “season” in āyurvedic texts can, according to context, refer either to
    the period of menstruation or else to the period of fecundity
    following menstruation \citep[15\,ff., note 27, \emph{et
    passim}]{das-2003}. \Dalhana{3.2.25}{347} noted that the woman's
    abstention should last three days from the first appearence of her
    menses.} She should abstain from sleeping during the day, collyriums,
    \se{aśrupāta}{weeping tears}, massages, cutting her nails, taking
    showers, laughing, telling stories, hearing too much noise and from
    exertion.\footnote{On the similar prohibitions relating to a
        menstruating woman as described in Dharmaśāstra literature, as well as
        the similar defects accruing from disobedience         
        \citep[see][284--287]{lesl-1989}.}
        
For what reason?  By sleeping during the day, the fetus becomes
\diff{deaf}.\footnote{Here, the vulgate reads \dev{svapnaśīlaḥ} “he
    tends to sleep.”} From collyrium he becomes blind.  From weeping, his
    vision is impaired. From bathing and anointing, he becomes badly
    behaved. From massage with oil he gets a \se{kuṣṭha}{pallid skin
        disease}.\footnote{On translating \dev{kuṣṭha} in Āyurvedic texts, see
        \cite[96\,ff]{emme-1984}.} From cutting the nails he gets
        \se{kunakha}{ugly nails}.  From smearing an unguent he becomes bald.
        From habitually exercising in the open air he goes mad. For this
        reason one should avoid these.
    
For three days of ritual food, the husband should \se{\root rakṣ}{protect} the
woman.  She lies on a layer of \gls{darbha}, and eats a different kind of food
from the palm of her hand, or from a plate or from a leaf.\footnote{This
    sentence is hard to construe because \dev{haviṣyaṃ} “ritual food” cannot 
    agree
    with \dev{-bhojinīṃ}.}

On the forth day, one should  show to the husband the woman 
who has
had a purifying bath, is wearing unstitched clothes, is ornamented and who has
chanted a benediction and recited a blessing.\footnote{See \cite[58 and
    fn.\,167]{wuja-2023}.}
    
    What is the reason for that?
     
     
     \item[26]
     And there is a verse on this.
     
     \begin{sloka}
         
         A woman has a bath after her period.  The type of man she sees after 
         that determines the type of son to whom she will give birth. She may then 
         show her son to her husband.     
           
     \end{sloka}
        
        \item[27]
        
\begin{sloka}
            Next, the \se{upādhyāya}{priest} should perform the appropriate 
            ritual 
        for producing a son.  At the end of the ritual, the \se{vicakṣaṇa}{expert} 
        should anticipate the following procedure. 
 
\end{sloka}       

\item [28]
        Next, after the man has eaten a rice porridge with ghee and milk in the
afternoon, having been celibate for a month, at night he should sexually
approach the woman who has had a diet rich in oil and mung beans.  He then
soothes her in a friendly way and he may go to her optionally on the fourth, 
sixth, eighth, tenth or twelfth day. 
        
\q{29, 30 missing?}
        
 \item [31]

Henceforth, he should approach after a month

[At this point there is a misplaced folio in MS N]
  
\item[32] 

\textcolor{red}{And when conception has occurred in this 
way}\q{\textcolor{red}{Problematic 
passage in the edition.}}

During one of these nights, the pregnant woman should press three or
four drops of juice from one or other of the following:
\gls{lakṣmaṇā}, \gls{vaṭa}, \gls{śuṅgā}, \gls{sahadevā},
\gls{viśvadevā}. Then she should administer them in the right nostril if she
desires a son and in the left if she wants a girl, and she should not
sneeze them out.\footnote{There is a textual problem at the start of this 
passage.}


%\newpage
%
%\begin{tt}
%    \bigskip
%    \raggedright
%    
%
%3.2.32a 
%Here are some more verses.
%
%3.2.11cd 
%On top of that those around her want her smelling sweet as honey, sparkling 
%like a crystal, agile and active, smooth and sweetly perfumed, 
%
%3.2.12ab 
%bright with splendour equally due to the smell of honey as to the smoothness 
%of oil. 
%
%3.2.17 
%It is a token of good health when the menstrual blood is red like a hare’ s 
%blood or like the shine of red lac and when its colour stains can be removed.
%
%3.2.18
%Metrorrhagia or uterine blood loss is diagnosed when there is either 
%excessive 
%bleeding, untimely or irregular bleeding or when symptoms are the opposite 
%of 
%what occurs in a normal menstrual cycle. *
%
%3.2.19 
%Excessive uterine bleeding is always accompanied by aching limbs and with 
%pain. In case blood loss is extremely abundant, symptoms may be weakness, 
%…………………. (bhramamūrcchā), fatigue,
%
%3.2.20 
%And (ca) fever, lamenting pain, anemia, tiredness (tandrā), diseases (rogāś) 
%due to disturbance of Vāta (vātajāḥ), set in motion (hita) by an inhabiting 
%(sevinyā) (disease?) that has just begun (taruṇyā) could become (bhavet) a 
%small (alpa-) disease (-dravam) in a person already labouring under another 
%disease (upa-).
%
%3.2.21cd 
%Because these afflictions have a recurrent character, the woman becomes 
%amenorrhoeic*. 
%
%3.2.22 
%One should impel in such cases in the food meat, Kulattha-pulses, sour 
%Tila-seeds, beans and wine and for drinks (cow)urine, whey and sour curd.*
%
%3.2.23
%In case of thin or abundantly flowing ? menses with features that cannot be 
%treated with drugs, other measures* indicated in case of uterine 
%metrorrhagia 
%must be taken. 
%
%3.2.29 
%The desire is always increasing of knowledge, …… ( dāyura ) and health 
%definitely, of success and power for the husband as well as strength over the 
%days indeed.
%
%3.2.30 
%So then, (a visit to the wife should be made) subsequently on the fifth, 
%seventh, 
%ninth and eleventh day (if) desirous for a female offspring; from the 
%thirteenth 
%day onwards he is to blame.

\item[33]

\begin{sloka}
    For certain, in the presence of these four, a fetus that follows
the rules will come into being, just like a sprout is from a
combination of field, seed, water and grass.\footnote{The Nepalese
    version reads \dev{kṣettrabījodakatṛṇām} “of field, seed, water
    and grass” in contrast to the vulgate's \dev{ṛtukṣetrāmubījānām}
    “of season, field, water and seed.” This gives the two versions
    quite different meanings. In the Nepalese version, the author is
    referring to the four plants mentioned in the previous verse,
    \gls{lakṣmaṇā}, \gls{vaṭa}, \gls{śuṅgā}, \gls{sahadevā}, and
    \gls{viśvadevā}.  Then the author presents a simple agricultural
    simile.  In the vulgate version, the words of the compound each
    have a double meaning: they can refer to the agricultural simile,
    but they can also be construed to mean “menstrual season, womb,
    nourishing bodily fluids, and male and female semen,” a
    parallelism not present in the Nepalese transmission.   This is
    how Ḍalhaṇa interpreted the verse.}
\end{sloka}

\item[34] 

Children born in this manner are beautiful, of noble character and
enjoy long lives.\footnote{We translate \dev{mahāsattvāḥ} as “noble
    character;” Ḍalhaṇa, commenting on the vulgate reading
    \dev{sattvavantaḥ}, refers to the \dev{guṇas}, interpreting the
    expression as “not strongly influenced by \dev{rajas} and
    \dev{tamas}.”}  They provide release from \se{ṛṇa}{obligation} and
    they themselves have children, benefitting their parents.\footnote{Children 
    born in this manner
        fulfil their parent's obligation to have children and they themselves
        have children, thus continuing the family.  The three debts are
        normally understood as being to the gods, the ancestors and to sages.
        But Ḍalhaṇa's phrasing is odd in that he says
        \dev{pitṝṇāmṛṇatrayamokṣaṇaśīlāḥ} “behaving so as to provide release
        from the three debts to the ancestors.”} 

\item[35]

In that context, the element of \se{tejas}{heat} is the most important
factor where \se{varṇa}{complexion} is concerned. That being granted,
at the moment the fetus is formed, when the food has water as its
chief element, then the fetus is fair.\footnote{The food of the
    mother, that is.}  When earth is the predominant element, it is
    \se{kṛṣṇa}{dark}. When earth and ether are the chief elements, it is
    \se{śyāma}{dark brown}.\footnote{The terms \dev{kṛṣṇa} and 
    \dev{śyāma}
        often mean more or less the same, a dark blue or black colour. The
        latter can shade into brown or dark green.}  Some people say that the
        \se{prasava}{newborn} has the same colour as the colour of the food
        that the pregnant woman commonly eats. Similarly, creatures like
        snakes, scorpions and \glspl{galagoḍikā} that inhabit black,
        yellow or white land are black, yellow or 
        white.\footnote{\label{galagodika}Cf.\ also n.\,\ref{godheraka}, 
        p.\,\pageref{godheraka}.}
    
  
% got to here 2024-05-24 

%When a mix of earth and sky are the main 
%elements (in the food), it turns towards the dark bluish. By analogy, some 
%say 
%it is the colour of the food the woman eats while pregnant that fixes the 
%complexion of her offspring. There are arguments in favour of both theories. 
%For instance animals living in black, yellow or white soils, such as snakes, 
%scorpions,  …. (gala-) (?) or oxen and others are reported to be black, yellow 
%or white. Thus the Tejas-principle makes a newborn irreversibly blind from 
%birth onwards. Exactly so, when it enters the blood it makes the newborn 
%red-eyed; entering the phlegm it makes it white-eyed; entering the bile, it 
%makes it yellow-eyed and when entering the wind, is responsible for the eyes 
%being diseased. 
%
%3.2.36 
%Here are some more verses.
%3.2.36a 
%He whose eyes are entered by a pure wind,
%
%3.2.36bj
%will have oblong downcast eyes, dark or bright.
%
%3.2.36c 
%When bile with phlegm both are present unsullied in a man’ s eye,
%
%3.2.36d 
%then in that man’s eye the yellow, green and reddish-brown will all light up 
%together. 
%
%3.2.36e 
%When phlegm has shaken off all kinds of bodily secretions from someone’s 
%eyes,
%
%3.2.36f 
%then both irises of that person will light up brightly shining. 
%
%3.2.36g 
%Whenever blood with phlegm move around in a somebody’s eyes,
%
%3.2.36h 
%he will appear to have either bluish-dark or blood-shot pupils. 
%
%3.2.36i 
%Just as a lump of ghee melts when placed near a fire,
%
%3.2.36j 
%so a woman’ s sexual receptivity gives way in contact with a man.  
%
%3.2.37 
%When sperm is divided in the uterus by the wind into two beings, twins are 
%born conditioned by their former good and evil deeds.  
%
%3.2.37.1 
%When in the mixture there is an excess of male sperm, a fertile woman will 
%create two male children.*
%
%3.2.37.2 
%and when indeed there is a mixture with an excess of female sperm, then the 
%woman produces two female children; in this there is no certainty.
%
%3.2.37.3 
%A child born from a man who has but a poor sperm to give his wife is called 
%āsevyaḥ (impotent).*
%
%3.2.38cd 
%When a man has a fellatio (?) he does not have to doubt his penis will get up 
%erect.*
%
%3.2.38 
%A boy born from a father with poor sperm becomes an āsekya.
%
%3.2.39 
%He who is born in a sordid vagina is commonly known as a saugandhika. 
%Such 
%a person becomes aroused only after smelling a vagina or a scrotum.*
%
%3.2.40 
%When a man first had same-sex anal intercourse because of a period of 
%sexual 
%abstinence from women and then turns towards his regular partners* again, 
%he 
%should be known as a kumbhīka. And now get it right about what an 
%īrṣyakaṃ 
%is:
%
%3.2.41 
%somebody who has to watch sexual intercourse of others before being able 
%to 
%his own sexual activities should be known as an īrṣyakaḥ.* He who turns 
%towards copulation**
%
%3.2.42 
%during the fertile days of the cycle* but out of pure sexual ignorance 
%ejaculates 
%on the breasts of his virgin wife** will create boys who also exhibit feminine 
%character traits.
%
%3.2.43 
%If a woman in her fertile days* throws herself at the feet of males around her 
%and if she begets a girl, she will also have character traits of a man. *** 
%
%3.2.44 
%Men who do produce sperm but have a pathology can be identified as 
%āsekya, 
%sugandhi, kumbhika or īrṣyaka. Men who do not produce any sperm are 
%called 
%saṇḍha.*
%
%3.2.45 
%The sperm conducting channels of both these (men) should be changed? 
%retaliated? opposed?  By an erection towards openness thus there should be 
%a 
%going towards a rising of the flag. 
%
%3.2.46 
%Just as by similar eating habits both boys and girls (are) connected, just so 
%also 
%foetuses should definitely be(come) in it (the womb).
%
%3.2.47 
%However, whenever a woman and another woman through sexual 
%intercourse 
%succeed in accomodating … yopa (?) each other’ s sperm, then a boneless 
%(being) is born.
%
%3.2.48
%Furthermore a woman could even be carried away into a sexual climax in a 
%dream after having taken her ritual bath. The Vāyu, having taken the egg 
%into 
%the uterus, makes in the belly surely…
%
%3.2.49 
%Month by month in the pregnant woman the signs of pregnancy become 
%apparent. From such a woman a kalala* is born, on account of the absence 
%of 
%paternal qualities.**
%
%3.2.50 
%However, disfigured creatures such as serpents, scorpions and 
%pumpkin-gourd 
%shaped foetuses, they, verily as well as others, should be known (to be) very 
%frequently born from the womb the consequence of sins.
%
%3.2.51
%A womb of a mother (who is) disrespected because of an excess of wind 
%(results in a child that) could be in danger (to become) humpbacked or 
%rather  
%…… (kūnipaṇgur) or dumb.
%
%3.2.52 
%Because of being without religious teacher and (ca) because of misfortunes 
%of 
%the parents or due to the excessiveness of the wind-eaters the child  could 
%obtain disfigurement.
%
%3.2.53 
%Due to the scantiness of bodily excretions, itself due to a disabling of Vāyu 
%with respect to processing of food, the foetus, whilst in the womb, produces 
%(almost)* no urine nor stools. 
%
%3.2.54 
%Due to the dwindling of the Vāyu in the face, the covered parts and the 
%narrowest parts (all) wrapped up by phlegm (kapha-), the foetus while it is in 
%the womb because of obstruction of the going does not weep all the time*.   
%
%3.2.55 
%Thus the foetus, provided with the movements of inhalation and of 
%exhalation, 
%goes on; the coming together of (its) moments of sleep with the movements 
%of 
%inhalation  and of exhalation of the mother.
%
%3.2.56 
%The (adjustment?) of the (limbs of the) body and both the appearance  and 
%the 
%falling of the teeth, even (ca) the very non-appearance of hairs  in the palms 
%of 
%the hands and the soles of the feet, (goes) according to its intrinsic nature ||
%
%3.2.57 
%Men who have uninterruptedly entered one previous existence after another 
%and who have a vast understanding of the scriptures, do remember their own 
%previous births.
%
%This was the second chapter of the śārīrāsthana.
%  
%\end{tt}

\end{translation}
