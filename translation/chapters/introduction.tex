% !TeX root = ../incremental_SS_Translation.tex

\chapter{Introduction}

What follows is a translation of selected chapters of the
\emph{Compendium of Suśruta} (\emph{Suśrutasaṃhitā}).  This differs
from former translations, being based on the Nepalese
version of the text.\footnote{See \cite{wuja-2023} for an
    introduction to the Nepalese text and \cite{wuja-2021b} for
    background on the Suśruta Project, 2021--2024.}  The Nepalese version of the 
    work has been reconstructed on the basis of three manuscripts from 
    Kathmandu, 
    \begin{enumerate}
        \item \MS{Kathmandu KL 699} (siglum K),
        \item \MS{Kathmandu NAK 1-1079} (N), and
        \item \MS{Kathmandu NAK 5-333} (H).
    \end{enumerate}
The first of these MSS is the oldest, dated to
\CE~878.\footcite[15]{kleb-2021b}  It covers most of the \SS, but
lacks the \emph{Nidānasthāna} and the \emph{Śārīrasthāna} (see 
Fig.\,\ref{fig:mss-1-visual-paradigm}).  The
second is undated but is datable on palaeographical grounds to the
twelfth or thirteenth centuries.\footcite[17--18]{kleb-2021b} It
contains the \emph{Sūtrasthāna} and \emph{Nidānasthāna} but breaks
off shortly afterwards.  The third manuscript is the most complete,
covering the whole of the \SS. It is dated \CE~1513.\footnote{I
    follow the arguments of \citet[21--26]{kleb-2021b} on the
    interpretation of the colophon although, as he pointed out, some
    interpret the date as \CE\ 1573.}  
    
The text of this manuscript follows K very closely but is probably
not a direct apograph.  I conjecture that it was either copied from
an intermediary that followed K very closely or from a ancestor of
K.\footnote{“\ldots as neither my own research \ldots\ nor the study
    undertaken in Harimoto \ldots\ could determine any linear connection
    between any of the Nepalese manuscripts of the SS, one may assume
    that [there exists] an older common ancestor of both of the
    manuscripts K and H.” \citep[21]{kleb-2021a}.}
        
        \begin{figure}[t]
            \centering
            \includegraphics[width=\textwidth]{"media/MSS 1 visual paradigm.art"}
            \caption{Coverage of the text by MSS K, N and H.}
            \label{fig:mss-1-visual-paradigm}
        \end{figure}
        
    
    
The translation follows the methods of rigorous philological care and 
modern principles of translation theory.\footnote{See 
\cite[intro.]{wuja-2003} and \cite[81--83]{wuja-2021} for an overview.}  Major 
differences in sense from the vulgate text are marked with \diff{in this manner}.