% !TeX root = main.tex

\section{Sūtrasthāna, adhyāya 14:  On Blood}


\subsection{Previous scholarship}

Meulenbeld offered both an annotated summary of this chapter as well as a
 study specifically on the place of blood
in Ayurvedic theory.\footnote{\volcite{IA}[209--201]{meul-hist}  and  \cite{meul-1991}.  Meulenbeld's footnotes on this chapter in \volcite{IB}[325\,ff.]{meul-hist} refer often to ``Hoernle's note.''  This appears to be a reference to Hoernle's copious notes to his translation of this chapter \citep[87--98]{hoer-1897}.}



%leeches and their application.\footcite[IA, 209; IB,
%324, n.\,131]{meul-hist}




\subsection{Translation}

\begin{translation}    
\item [1] Here is the first passage

   \begin{enumerate}
  
       \item 
    And now we shall explain \ldots \se{karma}{fate}

\item 

    \cite[33]{adri-1984} -> Author 1999: 33
\item     
    \citep[33]{adri-1984} -> (Author 1999: 33)
\item     
    \citet[33]{adri-1984} -> Author (1999: 33) 
\item 
    As Ḍalhaṇa remarked on \Su{1.14.22}{64}, \ldots  % quick way to refer to vulgate passages and pages
\item \SS
\item \CS
   \end{enumerate}
    

\end{translation}

