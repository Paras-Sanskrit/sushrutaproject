% !TeX root = main.tex

\section{Sūtrasthāna, adhyāya 14:  On Blood}


\subsection{Previous scholarship}

Meulenbeld offered both an annotated summary of this chapter as well as a
 study specifically on the place of blood
in Ayurvedic theory.\footnote{\volcite{IA}[209--201]{meul-hist}  and  \cite{meul-1991}.  Meulenbeld's footnotes on this chapter in \volcite{IB}[325\,ff.]{meul-hist} refer often to ``Hoernle's note.''  This appears to be a reference to Hoernle's copious notes to his translation of this chapter \citep[87--98]{hoer-1897}.}



%leeches and their application.\footcite[IA, 209; IB,
%324, n.\,131]{meul-hist}




\subsection{Translation}

\begin{translation}    
\item [1] Now we shall speak the chapter describing blood.

\item [2]

    %\cite[33]{adri-1984} -> Author 1999: 33
    
\item [3]    
    The most intangible/subtle essence of the food that is of four types\footnote{Ḍalhaṇa says that the four types of food refer to eatable, breakable?, lickable, and drinkable.}, endowed with the six tastes, made of the five elements\footnote{Earth, water, fire, air, space}, having either two or eight potencies, endowed with many qualities\footnote{\emph{Aneka-guṇopayukta} can also be understood as “suitable because of possessing many qualities” or “suitable and possessing many qualities”.}, and properly transformed, is called chyle. It is of the nature of a quality of the fire element. It is situated in the heart. From the heart, it enters into the twenty-four arteries—ten upward arteries, ten downward, and four sideways—and doing so day after day owing to the reaction of past activities that is caused by the invisible\footnote{\emph{Adṛṣṭa} (unseen): Doing any righteous or unrighteous action produces good merit and demerit respectively. This good merit and demerit are called \emph{adṛṣṭa} (invisible) because they cannot be directly known but can only be assumed through logical deduction.}, it satisfies the entire body, enlivens it, prolongs it\footnote{In the sense of prolonging its lifespan}, and makes it grow. The speed of the entity that flows throughout the body should be understood by inference. That speed causes deterioration and growth. 
    
    The inquiry into the chyle that flows through all the limbs, humors, body tissues, and excretory organs of the body is of the form “Is it gentle or fiery?” On its being mobile due to fluidity, it is understood to be gentle\footnote{Ḍalhaṇa interprets "gentle" to mean "similar to phlegm". Monier Williams glosses \emph{saumya} as cool and moist (opp. to \emph{āgneya}, 'hot and dry').} due to attributes such as lubrication, enlivening, satisfaction, holding, etc. 
    
    %\citep[33]{adri-1984} -> (Author 1999: 33)
\item [4]  
That watery chyle is then reddened after reaching the liver and spleen.

    %\citet[33]{adri-1984} -> Author (1999: 33) 
\item [5]
Here is a verse regarding it.

The untransformed fluid that is reddened by the fire element in its proper state within the body of living beings is known as blood. %[by them??].
% Transcription of vulage has the verse divided into two IDs. They have to be combined into one.

%\SS
\item [6]

It is only due to chyle that women's blood called menses exists. It increases from the twelfth year and decreases after the fiftieth year. 

%\CS

\item [7]

The menstrual blood, however, is called fiery\footnote{Ḍalhaṇa comments that this is to distinguish the menstrual blood from regular blood that is gentle.}.  

\item [8]

That is due to the embryo being fiery and gentle.\footnote{\emph{Agnīṣomīya} is a particular Vedic sacrifice which is related to the deities of fire (\emph{agni}) and moon (\emph{soma}). Ḍalhaṇa comments that the embryo is called such because the menstrual blood is fiery and the semen is gentle (\emph{saumya}). The word \emph{saumya} is derived from the word \emph{soma}, where it means that which has the qualities of the moon, i.e. that which is gentle.}
Others state the embryo as constituted of the five elements and the preceptors call it as the living blood. 

Here are verses regarding it.

\item [9]

That is because the qualities stinkiness, fluidity, redness, pulsation, and thinness of earth, etc. are seen in blood.

\item [10]

Blood is formed from chyle, flesh from blood, lymph from flesh, bone from lymph, marrow from bone, semen from marrow, and progeny from semen. 

\item [11]

There, the essence (chyle) of food and drink is the satisfier of these body tissues.
Here is a verse in this regard.

\item[12]

A living being should be known as born from chyle. One should diligently preserve\footnote{All three manuscripts have 'rakṣeta' which is an incorrect form. 'rakṣet' is the correct form.} chyle by administering food and drink, being nicely disciplined with food\footnote{'āhāreṇa' - The third case is used. The semantic property of the third case used here is unclear. Unclear regarding the Aṣṭādhyāyī rule justifying this usage.}.

\item[13]

The verbal root \emph{rasa} means movement.\footnote{\cite[109]{kunj-1907}} Because it keeps moving day after day, it is called \emph{rasa} (chyle).\footnote{In the list of verbal roots of Pāṇini, the verbal root \emph{rasa} means taste and moistening. It does not mean movement.}    

\item[14]

Chyle stays in every body tissue for 2548 ((25*100)+48) kalās and 9 kāṣṭhas. As such, it becomes semen after a month. For women, it becomes menses.  

\item[15ab-cd] Here are verses regarding it.
According to similar and dissimilar treatises, the quantity of kalās in this group\footnote{duration of chyle in all the body tissues as a whole} is 18,090.

\item[15ef-gh]

This is the particular transformation period regarding chyle that lasts for a person with mild fire. For a person with developed fire, one should know it to last for the exact same time.

%This particular transformation in chyle comes to an end due to the mild fire. Also, in the same duration of time, that\footnote{The end of transformation of chyle} is to be known due to the grown/big/manifest fire.

%OR

%This particular transformation in chyle forms the end due of the mild fire. Also, in the same duration of time, that of the intense fire is to be known.

\item[16]




\end{translation}

