% !TeX root = main.tex

\section{Sūtrasthāna, adhyāya 14:  On Blood}


\subsection{Previous scholarship}

Meulenbeld offered both an annotated summary of this chapter as well as a
 study specifically on the place of blood
in Ayurvedic theory.\footnote{\volcite{IA}[209--201]{meul-hist}  and  \cite{meul-1991}.  Meulenbeld's footnotes
on this chapter in \volcite{IB}[325\,ff.]{meul-hist} refer often to ``Hoernle's note.''  This appears to be a reference
to Hoernle's copious notes to his translation of this chapter \citep[87--98]{hoer-1897}.}



%leeches and their application.\footcite[IA, 209; IB,
%324, n.\,131]{meul-hist}




\subsection{Translation}

\begin{translation}    
\item [1] Now we shall speak the chapter describing blood.

\item [2]

    %\cite[33]{adri-1984} -> Author 1999: 33
    
\item [3]    
    The most intangible/subtle essence of the food that is of four types\footnote{Ḍalhaṇa says that the four types of food
    refer to \emph{bhojya}, \emph{bhakṣya}, \emph{lehya}, and \emph{peya} (eatable, breakable??, lickable, and drinkable).}, endowed with
    the six tastes, made of the five \emph{bhūtas}\footnote{Earth, water, fire, air, space}, having either two or eight potencies, endowed
    with many qualities\footnote{\emph{Aneka-guṇopayukta} can also be understood as “suitable because of [possessing] many qualities”
    or “suitable and possessing many qualities”.}, and properly transformed, is called as \emph{rasa}. It is of the nature of a quality
    of \emph{tejas}\footnote{The fire element}. It is situated in the heart. From the heart, it (\emph{rasa}) enters into the twenty-four
    arteries—ten upward arteries, ten downward, and four sideways—and doing so day after day owing to the reaction of past activities
    that is caused by \emph{adṛṣṭa}\footnote{\emph{Adṛṣṭa} (unseen): Doing any righteous or unrighteous action produces \emph{puṇya}
    (good merit) and \emph{pāpa} (demerit) respectively. This good merit and demerit are called \emph{adṛṣṭa} (unseen) because they
    cannot be directly known but can only be assumed through \emph{arthāpatti} (\emph{kāryānyathānupapatti}: non-substantiation of the
    effect otherwise).}, it satisfies the entire body, enlivens it, prolongs [its life], and makes it grow. The movement/speed of that
    flowing [\emph{rasa}] throughout the body should be understood by inference. That movement/speed causes deterioration and growth. 
    
    The inquiry into the \emph{rasa} that flows through all the limbs, humours, body tissues, and excretory organs of the body is of
    the form “Is it gentle or fiery?” On [its] being mobile due to fluidity, it is understood to be gentle due to attributes such as
    sticking, that (\emph{rasa})

    
    %\citep[33]{adri-1984} -> (Author 1999: 33)
\item [4]  

    %\citet[33]{adri-1984} -> Author (1999: 33) 
\item [5]

%    As Ḍalhaṇa remarked on \Su{1.14.22}{64}, \ldots  % quick way to refer to vulgate passages and pages

\item [6]
%\SS
\item [7]
%\CS
   
\end{translation}

