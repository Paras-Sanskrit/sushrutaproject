\section{Uttaratantra, adhyāya 16 (17 in the vulgate)}

\subsection{Literature}

Survey of this chapter and the existing research 
on it to 2002: \cites[IA, 305--306]{meul-hist}.

History of couching in India: \cite{wujad-2019,
    desh-2000,
    desh-1999,
    bret-1826,
    leff-2020,
    shas-kaly,
    jack-1884,
    scot-1817,
    elli-1918,
    hend-1895}.


\subsection{Translation}

\begin{translation}
    
    \item[1]  Now I shall explain the \se{pratiṣedha}{counteraction} of diseases
    located in the \se{dṛṣṭi}{pupil}.
    
    \item[2]  There are three \se{sādhya}{curable}, three \se{asādhya}{incurable},
    and six \se{yāpya}{mitigatible} diseases located in peoples eyes. Among these,
    three are \se{sādhya}{curable}. Amongst these three, %\item[3]
    the \se{pratīkāra}{remedy} has been stated for the one called
    “\se{dhūmadarśin}{seeing smoke}”.\footnote{This disease and its cure are 
    described earlier (SS.6.7.39 and SS.6.10.16 
    \citep[609 and 614]{vulgate} respectively). The latter part of this verse is
    hard to construe and the text here may have been altered at an early period.}
    
    \item[3--5ab]  When the eye is \se{vidagdha}{inflamed} by bile and when it is 
    inflamed 
    by phlegm, one should apply the method for removing bile and phlegm, using 
    \se{nasya}{nasal medicines}, 
    \se{seka}{irrigation},
    \se{añjana}{application of collyrium},
    \se{ālepa}{liniment}, and 
    \se{puṭapāka}{medicines cooked in a crucible},
    together with a \se{tarpaṇa}{balm},\footnote{These therapies are described in 
    SS.6.18 
    \citep[633--640]{vulgate}.} but not 
    \se{śastrakṣata}{cutting with a blade}.\q{where is cutting with a knife related to 
        removing bile or phlegm.}\footnote{Dalhaṇa interprets this as 
    \se{sirāvedha}{blood-letting}, which is discussed in SS.1.14 \citep[]{vulgate}.}
    
    One should drink \se{sarpis}{ghee} prepared with the \se{triphalā}{three fruits} 
    and in the first [case where the problem is bile], and \se{traivṛta}{prepared with 
        turpeth} in the 
    latter [case, of phlegm].
    
    And ghee \se{tailvaka}{prepared with tilvaka} is wholesome in both cases, or else 
    aged ghee on its own.
    % Viburnum nervosum D. Don.  Chunekar 185-186.
    
    %gairika Hellwig 140-141
    \item[5cd--7ab]
    
    In a collyrium, these four \se{yoga}{compounds} are beneficial in both cases:
    \begin{itemize}
        
        \item \se{gairika}{ochre}, \se{saindhava}{Sind salt}, \se{kṛṣṇā}{long pepper} 
        and the \se{maṣī}{black soot} from cow's teeth;\q{maṣī burned charcoal. Find 
            refs.}
        
        \item \se{gomāṃsa}{Cow's flesh}, \se{marica}{black pepper}, \se{śirīṣa}{siris}
        and \se{manaḥśilā}{red arsenic};
        
        \item 
        \se{vṛnta}{stalk} from a
        \se{kapittha}{wood apple} with
        \se{madhu}{honey};\footnote{\se{kapittha}{wood apple} in this verse is ablative 
        singular or accusative plural, neither of which construe obviously.}
        
        \item
        or the the fruits of the \se{svayaṃgupta}{velvet bean}.
    \end{itemize}
    
    \item [8]
    
    The physician should make a collyrium with ground up 
    \se{kupyaka}{metal},\footnote{A metal other than gold or silver, according to 
    \cite[1.217]{josi-maha}.  Perhaps lead, which is used in making contemporary 
    collyrium.}
    \se{aśoka}{Asoka tree},
    \se{śālā}{Sal tree},
    \se{amra}{mango},
    \se{priyaṃgu}{beautyberry},
    \se{nalina}{Indian lotus},
    \se{utpala}{blue lotus}, 
    together with
    \se{hareṇu}{hareṇu},
    \se{āmalaka}{emblic},
    \se{pathyā}{myrobalan},
    \se{pippali}{long pepper}.  
    It should be combined with ghee and 
    \se{kṣaudra}{honey}.
    
    
    \item[9--10]
    
    Also, when bile and phlegm have developed, the physician should apply
    \se{hareṇu}{hareṇu} with the \se{svarasa}{expressed juice} of the 
    flowers
    from \se{amra}{mango} and \se{jambū}{Jambu} trees.
    
    Then this collyrium, \se{vipakva}{matured} with ghee and \se{kṣaudra}{honey},
    should then be applied.
    
    \item[10--11ab]
    
    \se{kiñjalka}{Filaments} of \se{nalina}{Indian lotus} and \se{utpala}{blue lotus},
    with \se{gairika}{ochre}, and the juice of \se{gośakṛt}{cow-dung} are a collyrium
    in the form of a  \se{guḍikā}{pill}.  This is good for both day and night
    blindness.
    
    \item[11cd--12ab]
    
    \se{rasāñjana}{Elixir-salve}, \se{kṣaudra}{honey}, ghee, \se{tālīśa}{scramberry},
    together with gold and ochre, with the \se{gośakṛt}{juice of cow-dung} are for an
    eye afflicted with bile.
    
    \item [12cd--13] 
    
    Alternatively, wise physician should first grind together \se{śīta}{elixir-salve}
    and \se{sauvīraka}{stibnite}, \se{bhāvita}{infused} with \se{rasa}{the
        blood of birds and animals}.\footnote{This is Ḍalhaṇa's preferred interpretation
    of \emph{rasa} “juice” in this context.  He also notes that some take
    \se{śīta}{elixir-salve} to be camphor.}  Then he mixes it with the bile of a
    tortoise or with \se{rauhita}{extract of rohu carp}.  It should always be used
    with powdered collyrium to quell the bile.
    
    \item[14]
    
    Thus, a collyrium of \se{kārśmarī}{white teak} flowers, \se{madhuka}{liquorice},
    \se{dārvī}{tree turmeric}, \se{lodhra}{lodh tree} and \se{rasāñjana}{elixir salve}
    is always good as a collyrium in this case.
    
    
    \item [15]
    
    Alternatively, for those who cannot see during the day, this \se{guḍikā}{pill},
    with sandalwood, is recommended: \se{nadīja}{salt}, conch shell and the three
    spices, collyrium, \se{manaḥśilā}{realgar}, the two
    \se{rajana}{turmerics}\footnote{Turmeric (Curcuma longa \textit{Linn.}) and tree
    turmeric (Berberis aristata DC).  The term \emph{rajana} is unusual; the normal
    term is \emph{rajanī}. \emph{Rajana} occurs in \emph{Suśrutanighaṇṭu} 158 in the
    sense of Ferula asafoetida, Linn.} and \se{yakṛdrasa}{liver
        extract}.\footnote{This verse appears as no.\,27 in the vulgate.}
    
    \item [16] One should grind up \se{srotoja}{kohl},\footnote{Glossed by Ḍalhaṇa as
    a kind of collyrium.  Cf.\ \cite[2.M13]{nadk-1954} and \cite[197--198]{shar-1982}}
    and \se{saindhava}{Sind salt} and long pepper and also \se{hareṇu}{hareṇu}.  
    Such wicks with goats urine are good in a collyrium for
    \se{kṣaṇadāndhya}{night blindness}.
    
    \item [17--18ab] Alternatively, in such a case, grind together
    \se{kālānusāriva}{Indian sarsaparilla}\footnote{There are two forms of
    \emph{sārivā} mentioned widely in Āyurvedic literature, the white and the black.
    Ideas on the identity of the black form are particularly fluid.  See
    \citet[434--438]{adps} for a clear discussion.} long pepper, \se{nāgara}{dried
        ginger} and honey, the leaf of the \se{tālīśapatra}{scramberry}, the two
    \se{rajana}{turmerics}, a conch shell and \se{yakṛdrasa}{liver extract}. %
    Then shade-dried wicks take away \se{ruj}{illness}.
    
    
    \item[18cd--19ab]
    
    Wicks made of \se{manaḥśilā}{red arsenic},
    \se{abhayā}{chebulic myrobalan},
    \se{vyoṣa}{the three spices}.
    \se{sāriva}{Indian sarsaparilla}, 
    \se{samudraphena}{cuttlefish bone}, 
    combined with goat's milk are good.
    
    \item[19cd--21ab]
    
    One should cook a \se{kṣaudrāñjana}{honey collyrium} either in the juices of
    \se{gomūtra}{cow's urine}, and bile, \se{madirā}{spirits}, \se{yakṛt}{liver}, and
    \se{dhātrī}{emblic} or else in the juice of the \se{yakṛt}{liver} of something
    different, or else with the extract of the \se{triphalā}{three fruits}. %
    One of these should be mixed with cow urine, ghee and \se{arṇavamala}{cuttle
        fish}\footnote{At SS 6.12.31, Ḍalhaṇa glossed \emph{arṇavamala} as
    \se{samudraphena}{cuttlefish bone}. It may be worth considering whether the
    unusual term \emph{arṇavamala} “ocean-filth” might refer to ambergris.} with long
    pepper, honey and \se{kaṭphala}{box myrtle}. It is placed in sea salt and stored
    in a bamboo tube.
    
    \item[21cd--22]
    
    One should cook the liver of a sheep, the ghee of a goat, with long pepper and Sindh 
    salt, honey and the juice of emblics.  Then one should store it properly in a catechu 
    box.  Prepared thus, the honey collyrium is good. 
    
    
    \item[23]
    
    Alternatively, a collyrium that is \se{hareṇu}{hareṇu} mixed with
    \se{māgadhī}{long pepper}, the bone and the marrow of a goat, 
    \se{elā}{cardamom}
    and liver, together with liver extract, is good for eyes afflicted by
    phlegm.\footnote{On the identities of \emph{elā} and \emph{hareṇu}
    \citet[511\,ff]{watt-comm} described the former as “true” or “lesser” or “Malabar”
    cardamom, Elettaria cardamomum, Maton \& White, in contrast to the “greater”
    cardamom is Amomum subulatum (that he discusses on p.\,65) that is commonly 
    used
    as an inferior substitute for E. cardamomum. \citet[467\,f]{sing-1972} provided an
    interesting discussion of \emph{hareṇu}, noting that the term refers to two
    substances, first the \emph{satīna} pulse (Pisum sativum, Linn.), and second an
    unknown fruit such as perhaps a Vitex. They noted, “None of the text commentators
    have attempted to disclose the nature of its source plant,” although Ḍalhaṇa
    described it as aromatic and identical to \emph{reṇukā} (SS.ci.2.75).%
    % Sanskrit: {hareṇu}, Latin: {Amomum subulatum, Roxb.?}, References: {PVS
    %Caraka 2.734, AVS 1.128, NK #154} %
    }
    
    \item[24] 
    
    Over a fire, one should cook the \se{yakṛt}{liver} of a \se{godhā}{monitor lizard} 
    prepared
    with \se{antra}{entrails} and stuffed with \se{māgadhi}{long pepper}.  As is well 
    known, \se{yakṛt}{liver} which is \se{niṣevita}{used} with collyrium certainly 
    destroys night blindness.
    
    
    \item [25] 
    
    After preparing both a \se{plīhan}{spleen} and a liver on a spit, one should eat them
    both with ghee and oil.\footnote{We read the locative as if an instrumental; if the
    locative were intended then it would be the spit that would be coated with oil and
    ghee.}
    
    \item [25cd--26ab]
    
    As is well known, there are  six diseases that can be \se{yāpya}{alleviated};
    \se{tatra}{in those cases} one should release the blood by bloodletting.
    
    And for the sake of wellbeing one should also purge using aged ghee 
    \se{upahita}{combined} with 
    purgative \se{aṅga}{aids}. 
    
    \item[26cd--27]
    
    When an eye-disease is \se{pavanodbhava}{caused by wind} they say that
    \se{pañcāṅgulataila}{castor oil} mixed with milk is good.\footnote{Ḍalhaṇa says
    that the unexpressed topic of this recipe is \se{timira}{partial blindness}.} In
    the case of diseases of \se{śonita}{blood} and \se{pitta}{bile}, one should drink
    ghee with the three fruits; it is particularly
    cleansing.\footnote{\se{śonita-pitta, rakta-pitta}{Blood-bile} is a
    widely-recognized disease in ayurveda, but the compound here is definitely dual,
    which rules out that interpretation. One would expect blood-bile because the
    previous verse } In the case of phlegm, a purgative by means of
    \se{trivṛt}{turpeth} is recommended. In the case of all three humours,
    \se{sugandhi}{sandal} in oil is prepared with it (turpeth).\footnote{The expression 
    “\se{tailasugandhi}{the fragrant one in oil}” is puzzling. The word \emph{sugandhi} 
    has different referents in the \emph{Nighaṇṭu} literature but is not common as a 
    noun in the extant literature. “Sandal” is just one of its possible meanings.}
    % Ḍalhaṇa (commentary on SS.ci.24.7) equates  
    %\emph{trivarga}, i.e., \emph{triphalā}, with \emph{trisugandhi}.
    
    \item[28]
    
    In cases of \se{timira}{partial blindness}, aged ghee is recommended.  It is good if 
    it is kept in an iron vessel.
    
    \item [28cd--29ab]
    
    One should know that ghee with the three mylobalans is always good, and it is made
    with what is called \se{meṣaviṣāṇa}{periploca of the woods}.
    
    A man who is suffering from partial blindess should lick the finely-ground three fruits 
    mixed with ghee \se{sapāṇa}{off his hand}.\footnote{“Off his hand” translates the 
    adverbial \emph{sapāṇam}, an unusual word. Ḍalhaṇa reproduces a reading close 
    to the Nepalese recention but says that Jejjaṭa rejects it and so he also does 
    \citep[627]{susr-trikamji3}.}
    
    \item[29cd]
    
    Alternatively, someone afflicted by phlegm should apply them (the three fruits)
    mixed with oil and \se{pragāḍha}{steeped} in honey.
    
    \item[30]
    
    The very best oil, well-cooked with a decoction of cow-dung, is good in cases of 
    partial blindness, taken as an errhine.
    
    In cases caused by bile, ghee by itself is good, as is oil when it arises from wind and 
    blood.
    
    \item[31]
    And in the case of wind one should apply
    \se{trivṛt}{turpeth} based on
    \se{atibalā}{strong mallow},
    and \se{balā}{country mallow}
    in an \se{nasya}{errhine}.\footnote{“Based on” translates \emph{-āśrita} 
    “depending on” which does not construe easily here.  The vulgate has \emph{śṛta} 
    “cooked” which makes easier sense but is not supported by the Nepalese MSS.}
    
    Ghee which has been extracted from milk cooked with the meat of aquatic 
    creatures and those from marshlands should be prescribed.
    
    \item[32]
    
    \dag An \se{puṭākhya}{enclosed roasting} with  Sindh salt and the product of the
    meat of a \se{kravyabhuj}{carnivore} and a \se{eṇa}{deer}, is combined with 
    honey
    and ghee.\footnote{Ḍalhaṇa notes \citep[628a]{vulgate} that \emph{puṭāhvaya} 
    (see
    verse 35 below) is a synonym for \emph{puṭapāka}, and that the process is
    described in the \emph{Kriyākalpa} chapter, i.e., SS.6.18.33--38
    \citep[635]{vulgate}.  On the \emph{puṭa} process in the 
    \emph{Suśrutasaṃhitā},
    which is earlier and different than that of \emph{rasaśāstra} literature, see the
    discussion by  \citet[83]{wujad-2019}:
    \begin{quote}
        The term ‘enclosed roasting’ (\emph{puṭapāka}) does occur in
        the \emph{Suśrutasaṃhitā} in the context of eye treatments, but
        designates a method of obtaining juice from substances by
        wrapping them in leaves pasted with earth and cooking
        the bolus on charcoal to finally extract a juice.
    \end{quote}
    } % end of footnote
    
    \se{vasā}{Fat} from a horse, a vulture, a snake, and a \se{tāmracūḍa}{cock},
    combined with \se{madhūka}{mahua} is always good in a 
    collyrium.\dag\footnote{This
    verse contain irresolvable difficulties. There are no significant variants in the
    Nepalese MS transmission, but the text is ungrammatical.   The vulgate reads
    substantially differently but we have nevertheless made some emendations in line
    with it and read the verse as two sentences.}
    
    \item [33]
    
    Having 
    \se{niṣevita}{prepared}
    a collyrium made of 
    \se{srotas}{kohl}
    and gradually combine it with
    \se{rasa}{juices}, milk and ghee.\footnote{Ḍalhaṇa specifies that the juices are 
    meat soups of various animals \citep[628]{vulgate}.}
    
    For thirty days, this collyrium is put in the mouth of a black snake that is
    covered with \se{kuśa}{kuśa grass}.
    
    \item [34]
    
    Next, a collyrium that is milk containing \se {māgadhī}{long pepper},
    \se{kṣāraka}{lye} and \se{saindhava}{Sindh salt} that has been repeatedly 
    prepared
    with the mouth of a black snake, is good in the case of \se{rāgin
        timira}{bloodshot blindness}.\footnote{Ḍalhaṇa describes this blindness as a type
    of \emph{kāca} disease caused by wind  \citep[628]{susr-trikamji3}. The expression
    “bloodshot blindness” is an attempt to capture the idea of a blind eye that is
    dyed or coloured (not colour-blindness). This verse is quite different from the
    vulgate and also syntactically challenging.}
    
    \item [35]
    
    They say that ghee may be produced from that and combined with sweet herbs is 
    good as an errhine for eye-diseases caused by bile.
    
    And here,  a \se{tarpaṇa}{balm} is good that is a combination that is the flesh of
    wild animals \se{puṭāhvaya}{taken hot}.\footnote{The expression
    \se{puṭāhvaya}{taken hot} is a guess.  }
    
    
    \item[36]
    
    And 
    \se{manaḥśilā}{realgar} mixed with 
    \se{rasāñjana}{elixir salve} and honey is a 
    \se{dravāñjana}{liquid collyrium} which is, in this case, combined with 
    \se{madhūka}{mahua}.\footnote{The expression \se{dravāñjana}{liquid collyrium} is
    only known from Ḍalhaṇa's comments on SS.6.17.11ab \citep[626]{vulgate}.  The
    recipe in the present collyrium is different from that discussed by Ḍalhaṇa.}
    
    Alternatively, experts on this say that finely ground \se{tuttha}{blue vitriol}
    extracted from a gold mine is the “\se{samāñjana}{same collyrium}”.\footnote{The
    expression “\se{samāñjana}{same collyrium}” is a hapax legomenon glossed
    inexplicably by Ḍalhaṇa as “a collyrium with an equal amount of fermented barley”
    (\emph{tulyasauvīrāñjana}) \citep[628]{vulgate}.}
    
    \item [37]
    
    Conch mixed with equal parts of sheep's horn and \se{añjana}{stibnite} removes the
    impurity of the \se{kāca}{glassy opacity} because of the \se{añjana}{application
        of collyrium}.\footnote{The ablative “from collyrium” is hard to construe, but 
    Ḍalhaṇa uses this term and phrase in his commentary on \Su{6.17.41ab}{629}.}
    
    
    
    The \se{rasa}{extracts} produced from 
    a\se{palāśa}{flame of the forest},
    \se{rohīta}{Rohīta tree},\footnote{Probably \emph{Soymida febrifuga} A. Juss.}
    \se{madhūka}{mahua},
    ground with the \se{agra}{supernatant layer} of the \se{madira}{spirits} is 
    applied. 
    
    \item[38]
    
    Alternatively, one should cook an errhine with 
    \se{uśīra}{cuscus grass},
    \se{lodhra}{lodh tree},
    \se{triphalā}{the three fruits},
    \se{priyaṅgu}{beauty berry} to pacify eye diseases caused by 
    phlegm.\footnote{Ḍalhaṇa invokes a \se{paribhāṣā}{general rule} to indicate that 
    this mixture should be cooked with sesame oil. }
    
    
    
    One should apply smoke of the bark of
    \se{vidaṅga}{embelia},
    \se{pāthā}{velvet leaf},
    \se{kinihī}{white siris},
    and
    \se{iṅgudī}{desert date}; and
    \se{uśīra}{cuscus grass} alone.
    
    \item[39] 
    
    A ghee that is \se{bhāvita}{cooked} from a decoction of a
    \se{vanaspati}{non-flowering tree}\footnote{These are fig trees.  The 
    \emph{Sauśrutanighaṇṭu} (252) specifies the Uḍumbara. Cf.\ the classification in
    CS.1.1.71--72, 1.8, \emph{et passim}.}
    % cross-reference the description in Su.su.
    as well as 
    \se{haridrā}{turmeric}
    and 
    \se{nalada}{spikenard}
    is good in a 
    \se{tarpaṇa}{balm}. 
    
    
    % \coffeestainD{0.4}{0.5}{90}{3 cm}{4 cm}
    
    Alternatively, one may have an \se{puṭapāka}{enclosed roasting} done with
    \se{jāṅgala}{arid-land animals}\footnote{On this term, see SS.1.35.42 
    \citep[157]{vulgate} and the discussion by
    \citet[25--31]{zimm-1999}.} 
    and a plentiful amount of \se{māgadha}{long
        pepper}, Sindh salt and honey.
    
    \item[40] %
    
    A \se{kriyā}{treatment} with 
    \se{manaḥśilā}{realgar}, the three spices, conch, honey, along with Sindh salt, 
    \se{kāsīsa}{green vitriol} and \se{rasāñjana}{elixir salve}.\footnote{Ḍalhaṇa 
    glosses  \se{kriyā}{treatment} specifically as \se{rasakriyā}{inspissation} 
    \citep[629]{vulgate}.}
    
    They say that an 
    \se{rasāñjana}{elixir salve} combined with 
    myrobalans, 
    treacle and 
    dried ginger
    is good.\footnote{We emend \emph{hite} to \emph{hitam}, against the MSS.}
    
    \item[41]
    
    Alternatively, a collyrium that has been prepared many times in the eight types of
    urine\footnote{See SS mūtravarga}\q{find ref.} is put into water with the three 
    fruits. Having
    stored it in the mouth of a \se{niśācara}{nocturnal creature}\footnote{Ḍalhaṇa
    glosses \se{niśācara}{nocturnal creature} as “vulture,” although elsewhere in the
    SS it is more commonly interpreted as a spirit or demon.  In the present context, 
    following verses 33 and 34, it is probably a snake.} one should place it in a
    \se{salilotthita}{conch} for two months.\footnote{We interpret
    “\se{salilotthita}{water-born}” as “conch” in line with \emph{jalodbhava}, but the
    term is uncertain.}
    
    
    \item[42]
    
    One should apply that collyrium 
    together with the flowers of
    \se{madhūka}{mahua}
    and
    \se{śigru}{horseradish tree}
    when [the disease] is caused by all [the humours].
    
    But alternatively, all treatments apply when blood is the cause.  The procedure
    that removes bile is good when there is \se{mlāyin}{blue dot
        cataract}.\footnote{The vulgate follows Ḍalhaṇa in glossing \emph{mlāyin} as
    \emph{parimlāya}.  The description of this condition at SS.6.7.27--28 appears to
    refer to “blue dot” or “cerulean” cataract.  \emph{$\surd$mlai} derivatives can 
    mean “dark” or “black.”), which is normally a different ailment.}\q{Check
        out these refs.}
    
    \item[43] For one who has a humour, the physician should consider the rule in
    all humoral cases and then smear the ointment on the face.\footnote{The vulgate
    edition omits part of this verse (ab) combining earlier and later passages.}
    
    The treatment that is good for removing \se{syanda}{watery eye} should be 
    properly
    applied in all these humoral cases, according to the individual.\footnote{The term
    \se{syanda}{watery eye} refers to the specific disease \emph{abhiṣyanda}.  See
    SS.6.6.5, 1.46.51, etc.}
    
    \item[44] % tr. 47, p. 201
    
    The physician should not employ substances in errhines etc., when the humours intensify, 
    and also when disease spreads.  And further, in the \emph{Kalpa}, there is a good deal 
    more 
    said about collyriums, and that should be considered and then applied.\footnote{Ḍalhaṇa 
    notes that \emph{Kalpa} means the Uttaratantra adhyāya 18 \citep[633\,ff]{vulgate}.}
    
    \item[45] 
    Someone who uses matured ghee, the three fruits, \se{śatāvarī}{wild asparagus}, 
    as well as
    \se{mudga}{mung beans}, emblic and barley
    has nothing to fear from cases of severe \se{timira}{blindness}.
    
    \item[46] Blindness is dispelled by milk prepared with wild asparagus or in
    emblics, or again \se{yavaudana}{cooked barley} followed by the water of three
    fruits with plenty of ghee.
    
    \item[47]
    
    When there is \se{rāgiṇi timire}{bloodshot blindness}, the wise physician should
    not cut a vein.  A humour \se{utpīḍita}{injured} by the instrument rapidly
    destroys vision.
    
    \item[48] 
    
    \se{araga timira}{Non-bloodshot blindness} in the first \se{paṭala}{layer} is
    treatable.  And \se{rāgiṇi timire}{bloodshot blindness} in the second layer, with
    difficulty.
    %\footnote{Although the SS says \se{kṛcchra}{with difficulty}, the
    %implication is that it is \se{asādhya}{impossible to treat} (cf.\ SS 6.17.2
    %above).} 
    And in the third layer it is \se{yāpya}{mitigable}.
    
    
    \item [49]
    
    I shall explain the therapy for success when there is a \se{liṅganāśa}{cataract}
    caused by phlegm.   It may be white, like a full moon, an umbrella, a
    \se{muktā}{pearl} or a \se{āvarta}{spiral}.
    
    \item [50] Or it may be uneven, thin in the middle, streaked or have excessive
    \se{prabha}{shine}. A \se{doṣa}{humour} in the pupil may be characterized as being
    painful or having blood.\footnote{In the vulgate, and in parallel passages in the
    AS, the reading “\se{bhavet}{it may be}” is replaced with the negative “\se{na
    ced}{if, then not}” (cf.\ \AS{utt.17.1--3}{712}).   These characteristics are then
    read as conditions that preclude surgery;  for the Nepalese recension, they are
    simply descriptions of the appearance of a cataract.}
    
    \item [51--52]
    
    At a time that is neither too hot or too cold, 
    the patient who has been oiled and sweated
    is restrained and seated, looking symmetrically at his own nose.
    
    The wise physician should \se{muktvā}{separate} two white sections from the
    \se{kṛṣṇa}{black part} and from the \se{apāṅga}{outer corner of the eye}. Having
    \se{pressed}{pīḍ-} properly into the eye,\footnote{We 
    understand the
    locative \emph{nayane} as the place of pressing; other interpreters take it as an
    accusative dual.  The idea is that the eye is held steady by the surgeon.} at the
    \se{daivakṛte}{naturally occurring} \se{chidra}{hole} with the \se{śalākā}{probe}
    made of copper or iron, with a tip like a barley-corn that is held by a steady
    hand with the middle finger, forefinger and thumb,
    the left one with the right hand and the other one contrariwise.
    
    When the piercing is done, there is the simultaneous issue of a drop of liquid and a 
    sound.\footnote{Ḍalhaṇa interprets \se{samyak}{simultaneous} rather as “proper,” 
    referring to the proper kind of incision.}
    
    \item [55]
    
    The expert should moisten the exact place of piercing  with a woman's breast-milk.  Then 
    he 
    should scratch the \se{dṛṣṭimaṇḍala}{circuit of the pupil} with the tip of the 
    \se{śalākā}{probe}.\footnote{The anatomy of the eye is described in 
    \Su{6.1.14--16}{596}  
    The disks or 
    \emph{maṇḍala}s are the circuits or disks of the eye.}
    
    \item[56]
    Without injuring, gently pushing the phlegm in the circuit of the pupil against the nose, he 
    should remove it by means of \se{ucchiṅgana}{sniffing}.\footnote{Ḍalhaṇa describes 
    \se{ucchiṅgana}{sniffing} at 
    \Su{6.19.8}{641}, clearly intending inward sniffing.}
    
    
    % fill in later
    
    \item[57]
    
    
    
    Whether the humour is \se{styāna}{solid} or \se{cala}{liquid}, one should apply
    sweating to the eye externally, with \se{bhaṅga}{leaves} that remove wind, after
    fixing the \se{sūcī}{needle} properly.\footnote{We interpret \emph{bhaṅga} as
    leaves, following the usage elsewhere in this sthāna \Su{4.32.9, 6.11.5}{513, 614}
    where \emph{bhaṅga} means \se{pallava}{shoots}. A similar procedure is
    described at \AS{6.17.25}{716a}, where sweating of the eye is done by means of the
    leaves of a castor-oil plant.}
    
    
    \item[58]
    
    But if the humour cannot be destroyed or if it comes back, one should apply the
    \se{vyadha}{piercing} once again, with appropriate oils and so on.
    
    
    \item[59]
    
    %athābhramukte ca harir yathā dṛṣṭiḥ prakāśate 
    % athābhramukto va harir
    
    Now the \se{dṛṣṭi}{pupil} shines like the \se{hari}{sun} in a cloudless sky; then, 
    when objects become visible, one may slowly remove the 
    \se{śalākā}{probe}.\footnote{There are many problems with the MS readings and 
    interpretation of this half-verse.  We have inferred “sky” and emended from 
    “\se{agramukta}{free from the point}” to “\se{abhramukta}{free from clouds}”.  The 
    latter 
    meaning is supported (in different words) by the vulgate and occurs elsewhere in Sanskrit 
    literature.}
    
    \item[60]
    
    Having smeared ghee on the eye, one should cover it with a bandage.  Then, he must lie 
    down supine in a house free from disturbances.\footnote{Ḍalhaṇa explains disturbances 
    specifically as dust, smoke, drafts and sunlight \Su{6.17.67}{631a}.} 
    
    \item[61]
    At that time, he should not belch, cough, sneeze, spit or shiver.  Afterwards there should be 
    \se{yantraṇā}{restrictions} as in the case of someone who has drunk 
    oil.\footnote{Ḍalhaṇa glosses “\se{yantraṇā}{restrictions}” as having a controlled diet 
    and 
    the other restrictions appropriate to someone who is taking oil as a preparation before 
    further 
    therapy (\Su{6.17.68}{631}). These restrictions are also described at \Su{6.18.28}{635} 
    and   \Ah{1.16.25cd}{249}.}
    
    
    \item[62]
    
    Every three days one should wash it with 
    \se{kaṣāya}{decoctions} that remove wind.
    After three days, one should sweat the eye externally because of the danger of wind. 
    
    \item[63]
    Having restrained himself in this way for ten days he should thereafter take a beneficial 
    \se{karma}{regimen} that clears the \se{dṛṣṭi}{pupil} and also he should take light food 
    in 
    measure.
    
\end{translation}

\subsubsection{[Complications]}

\begin{translation}
    
    \item[64] 
    
    When there is a \se{vilocana}{misshapen eyeball}, the eye may fill because of the
    release of blood from a vein.\footnote{The condition of “misshapen eye” is referred
    to briefly in \Su{6.61.9}{800}, where Ḍalhaṇa glosses it as
    “\se{vakrabhrūnetra}{bent brow and eye}.”  The vulgate's reading of “\se{śonitena}{with 
    blood}” is easier to construe.}
    
    % many verses missing in Nepalese recension.
    
    A hard probe leads to \se{śūla}{shooting pain}, a thin to
    \se{doṣapariplava}{unsteadiness of the humours},\footnote{There is a medically
    significant difference here from the vulgate, which reads “a \se{khara}{rough}
    probe” not a “thin” probe.}
    
    \item[65]
    
    a thick-tipped probe leads to a large wound, and a sharp one may cause harm in many 
    ways;
    a very irregular one may cause a discharge of water, a \se{sthirā}{rigid} one brings about 
    a \se{kriyāsaṅga}{loss of function}.\footnote{This translation of \se{kriyāsaṅga}{loss of 
    function} is given on the basis of Ḍalhaṇa's gloss of \emph{kriyāsaṅgakarin} 
    as “\se{gamanādikriyāvināśakarī}{causing the destruction of actions such as moving}”
    at \Su{3.8.19}{382}.}
    
    \item[66]
    
    Therefore, one should make a good probe that is free from these defects.
    
\end{translation}

\subsubsection{[Characteristics of the probe]}

\begin{translation}
    
    \item[\empty ] 
    The probe should be eight finger-breadths long and in the middle it is wrapped with thread 
    and is as thick as a thumb joint.  It is shaped like a bud at both \se{vaktra}{ends}.
    
    \item[67]
    
    A commendable probe should be made of silver, iron or 
    \se{śātakumbhī}{gold}.\footnote{The vulgate reads “\se{tāmra}{copper}” in place of 
    “silver.”}
    
    
\end{translation}

\subsubsection{[Complications]}

\begin{translation}
    \item[\empty ] 
    
    Redness, swelling, lumps, \se{coṣa}{driness},
    \se{budbuda}{bubbling},\footnote{Ḍalhaṇa glosses “\se{budbuda}{bubbling}” as
    “\se{māṃsanirgama}{prolapse} that looks like bubbles.”} \se{sūkarākṣitā}{pigs'
        eye},\footnote{The expression “pigs' eye” appears to be a \emph{hapax}.  It is
    glossed as “\se{adhodṛṣṭitva}{downward vision}” by Ḍalhaṇa.},
    \se{adhimantha}{irritation}, etc.\ and other diseases arise from faults in the
    piercing,
    
    \item[69--70]
    
    or even from bad behaviour. One should treat them each accordingly.
    
    Listen to me once again about compounds for painful red eyes.
    
    \se{gairikaḥ}{Red chalk}, 
    \se{śārivā}{Indian sarsaparilla},
    \se{dūrvā}{panic grass},
    and ghee ground with barley.
    
    \item[71]
    
    This face ointment is to be used for quelling pain and redness.  Or else it
    may be taken combined with the juice of \se{mātuluṅga}{citron} with sesame gently
    fried, mixed with \se{siddhārthaka}{white mustard}.\footnote{On the adverbial use
    of \se{mṛdu}{gently}, see \cite{gomb-1979}.}  This is immediately beneficial when
    someone is looking for relief.
    
    \item[72]
    
    A paste with 
    \se{payasyā}{Holostemma},\footnote{The identity of \emph{payasyā} is debated
    \citep[538]{sing-1972}, and was already in doubt at the time of Ḍalhaṇa but  likely 
    candidates may be those suggested by Ḍalhaṇa,   who suggests either
    \emph{arkapuṣpī} or \emph{kṣīrakākolī}, that may be \emph{Holostemma adakodien} 
    Schult.\ 
    and \emph{Leptadenia reticulata}  (Retz.) Wight \& Arn.\ 
    \citep[195-196]{adps}.  The \emph{Sauśrutanighaṇṭu} glosses it as \emph{kṣīrikā} or 
    \emph{arkapuṣpikā} \citep[v.\,307]{suve-2000}.} 
    %
    \se{śārivā}{Indian sarsaparilla}, \se{patra}{cassia cinnamon},
    \se{mañjiṣṭhā}{Indian madder}, and \se{madhukair}{liquorice} stirred with goat's
    milk, pleasantly warmed, is said to be healthy.\footnote{The expression
    “\se{ajākṣīrārdita}{stirred with goat's milk}” is difficult.  It may be connected
    with the rare root \emph{ard} documented by \citet[15]{whit-root}. Cf.\
    $\surd$\emph{ard gatau} (\emph{Dhātupāṭha} 1.56).}
    
    \item[73]
    
    Alternatively, it can be made in this way with Himalayan cedar, \se{padmaka}{Himalayan 
    cherry} and dried 
    ginger.
    Or, in the same way, with grapes, liquorice and the Lodh tree mixed with Sindh salt.
    
    \item[74]
    
    Alternatively, goats' milk with the Lodh tree, Sindh salt, red grapes and liquorice, cooked, 
    should be used in irrigation because it removes pain and redness. 
    % śritam for śṛtam
    
    \item[75]
    
    Having cooked it with liquorice, water-lily, and costus, mixed with \se{drākṣā}{grapes},
    \se{lākṣā}{lac},
    \se{sitā}{white sugar}, 
    % 
    with wild asparagus, \se{pṛthakparṇī}{Hare Foot Uraria},\footcite[18]{suve-2000}
    \se{mustā}{nutgrass},
    liquorice,
    \se{padmaka}{Himalayan cherry},
    and Sindh salts, 
    one should apply it [irrigation] gently warm.
    
    
    \item[76cd--77ab]
    
    Ghee that has been cooked in four times the amount of milk that has itself
been cooked with drugs that destroy wind.\footnote{Ḍalhaṇa mentions that these
drugs include \se{bhadradāru}{Deodar} and other wind-destroying drugs.  The
\emph{vātasaṃśamana} group is listed in \SS\ \emph{sūtrasthāna} 1.39.7.}
%bhadradārvādivarga ...
This has an admixture of \se{kākolī}{cottony jujube} etc., should be
prescribed in all treatments.\footnote{Ḍalhaṇa notes that this would include
errhines, ointments, etc.}
    
    \item[77cd--78ab]
    
    If pain does not end in this way, one should administer blood-letting to the vein of 
    someone who has previously been oiled and sweated.  Then the wise physician should 
    apply cauterization in the advised manner.\footnote{The vulgate reads \emph{vāpi} for 
    \emph{cāpi}, so Ḍalhaṇa sees blood-letting and cautery as alternatives, not a sequence of 
    treatments.  Ḍalhaṇa lists the places that cauterization may be applied, such as  the brow, 
    forehead, etc.}
    
    \item[78cd--80ab]
    
    Now listen to two excellent collyriums for making the pupils clear.  
After grinding the flowers of \se{meṣaśṛṅga}{perploca of the woods},
\se{śirīṣa}{siris}, \se{dhava}{axelwood} 
\se{jātī}{royal jasmine}, pearl and \se{vaiḍūrya}{beryl} with goat's milk, 
one should put it in a copper pot for seven days.

\item[80cd--81]

Having made it into \se{vartti}{wicks}, the physician should apply it as a collyrium.\q{or a 
dual?}  
Alternatively, one should make 
\se{srotoja}{kohl}, 
\se{vidruma}{coral},
\se{phena}{cuttlefish bone}, and 
\se{manaḥśilā}{realgar}
and peppers into wicks as before.  One should apply these wicks, which are good in a 
collyrium, to steady the pupil. 

\item[82]

I shall again discuss the foremost collyriums at length in the \emph{Kriyākalpa} section. 
Those various methods may be applied here too. 
    
    
    \end{translation}
    
    
    
    