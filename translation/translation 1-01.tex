    \section{Sūtrasthāna, adhyāya 1}

% \coffeestainB{0.7}{1}{-30}{18 pt}{-50 pt}

\begin{translation}
    
    \item[1] Now I shall narrate the chapter on the origin of this
    knowledge.\footnote{Ḍalhaṇa understood the word "\se{veda}{knowledge}" as
    specifically "medical knowledge." He said that the word "longevity"
    (\emph{āyur}) \ssaneng{āyur}{life, longevity} had been elided.
    %    
    %    Notes Dec 8:
    %    Dominik: N's ādhyāyaṃ corruption of H's nāmādhyāyaṃ: possible evidence that N was 
    %created 
    %after H
    %    Check: Ācārya 1931 footnote on vedotpattim
    %    Commentary Ḍalhaṇa (c. 1200 CE) notes āyur dropped from veda in vedotpattim
    %    
    %    
    After this opening statement, later manuscripts and commentaries include
    the attribution, "as the venerable Dhanvantari stated."  The absence of this
    statement in the early Nepalese manuscripts is highly significant because it 
    removes
    the outer narrative frame of the \SS\
    \parencites[148]{wuja-2013}[\S\,3.1.2]{kleb-2021b}{rai-2019}{birc-2021}.  On 
    the 
    figure of Dhanvatari in 
    medical literature, see \cite[IA 358--361]{meul-hist}.} %     <!-- Notes Dec 8:
    %    Dom: note the omission of Dhanvantari, which is in the edition.
    %    On Dhanvantari, see Meulenbeld HIM, authorities associated with Suśruta. He's
    % an authority on
    %surgery or toxicology-->
    
    \item[2] Now, as is well-known, Aupadhenava, Vaitaraṇa, Aurabhra, Puṣkalāvata,
    Karavīra, Gopurarakṣita, Bhoja, Suśruta and others addressed Lord Divodāsa,
    king of Kāśi, the best of the immortals, who was in his ashram surrounded by
    an entourage of sages.\footnote{On these persons, see \cite[IA
    361--363, 369\,ff.]{meul-hist}. The authority Bhoja does not appear in the list as
    published in the vulgate edition \citep[1]{susr-trikamji2}, and was not
    included in \cite{meul-hist} amongst “authorities mentioned in the \SS.” 
    \citeauthor{meul-hist} gathered textual evidence about Bhoja at \cite[IA
    690--691]{meul-hist}. \citet{kleb-2021a} has discussed these authors in the
    context of an anonymous commentary on the \SS\ that cites them.}
    
    \nocite{emen-1969}
    
    %    Notes Dec 8:
    %    Dom: Check these names in Meulenbeld 
    %    Bhoja is an early lost authority on medicine. Not the same person as King Bhoja, 
    %commentator 
    %on the Yogasūtras.
    %    Ḍalhaṇa's comm. mentions Bhoja as also included in prabhṛtayaḥ: so the version of 
    %the text he 
    %was using did not mention Bhoja, but he was aware of him: His provenance makes it 
    %possible that 
    %he knew the Nepalese version of the SŚ
    
    \item[3]
    %O Lord, after seeing people who are assailed by the impingements of various pains 
    %caused by 
    %physical, mental and accidental diseases, who have the support of friends [but] 
    %feeling as if they 
    %were alone, and acting frantically, shouting out, we have been distressed. 
    
    “O Lord, distress arose in our minds after witnessing people thrashing about with
    cries, assailed by different kinds of \se{vedanābhighāta}{pain and injury}, 
    feeling helpless in spite 
    of having friends, because of diseases arising from the body, the mind and
    external sources.
    
    
    
    %    Notes Dec 8:
    %    āgantu - caused by something from outside the body
    %    abhighāta - threats, impingements 
    %    vedanābhighāta - tatpuruṣa 
    %    anātha - among a list of people who shouldn't be treated.
    %    Ḍalhaṇa- sanātha: samitra someone with a friend -->
    
    \let\uncertain\texttt
    
    \item[4]    
    “To quell the illnesses of those who seek happiness and for our own purpose of
    prolonging life, we desire \se{āyurveda}{the science of life} that is being
    taught.  Welfare\ssaneng{śreyas}{welfare}, both in this world and in
    the next, depends upon it. Therefore, we have come to the Lord in pupillage." %
    % reading bhagavan (voc.) and tam (it, ayurveda, masc. acc.)
    
    % we think upasannāḥ smaḥ is probably wrong, but we can't see how to improve it.
    % upapannā sma ?
    
    \item[5] The Lord said to them:
    
    “Welcome to you!  My children, all of you are beyond reproach and worthy 
    to be taught.
    
    
    \item[6] 
    %    "As is well-known in this world, before creating people, Brahmā composed 
    %    what is called Āyurveda.\footnote{The relative pronoun \emph{yad}, that has 
    %no  
    %    correlative \citep[\P 461]{spei-1886}, is omitted.} 
    %    It is taught as part of the \emph{Atharvaveda}, in hundreds of 
    %    thousands of verses and a thousand chapters and, after observing the short 
    %    lifespan and low intelligence of people, made it again in eight parts. 
    %    %infer tat as the object of kṛtavān
    %    
    
    “As is well known, Ayurveda is the name of what is said to be the subsidiary
    part of the Atharvaveda.   Before creating people, Svayambhū composed it in
    hundreds of thousands of verses and a thousand chapters and, after observing the
    short lifespan and low intelligence of people, he presented it again in eight
    parts.\footnote{Svayambhū is another name for Brahmā, the creator.}
    
    \item[7] “Surgery, treatment of body parts above the clavicle, general medicine, 
    knowledge of spirits, care of children, and the disciplines of antidotes, rejuvenation 
    and aphrodisiacs.
    % why do some of the auxiliaries end in tantra? Dom: Some were disciplines that had a 
    %separate life outside āyurveda. The others were more particular to vaidyas.
    
    \item[8] “Now,  a collection of the characteristics of each component of 
    Āyurveda.
    
    \item[9] “Among them, [the component] called surgery has the goal of 
    extracting 
    various grasses, wood, stone, dust, iron (?), soil, bone, hair, nails, discharge of 
    pus, malignant wounds and foreign bodies inside the womb, and of determining 
    the application of surgical instruments, knives, caustics and fire by means of 
    sixty definitions.
    %Ḍalhaṇa seems to read duṣṭavraṇāntar, and glosses antar as madhyāt. He then reads 
    %garbhaśalya (HIM - foetuses stuck in the womb). Ḍalhaṇa is aware of the reading ṣaṣṭyā 
    %vidhānaiḥ (following uddhraṇārtha), and says some explain it as apatarpaṇādyai 
    %rakṣāvidhānāntair dvivraṇīyoktair ity arthaḥ. However, mss. 699 and 533 read abhi° not 
    %vi°
    
    \item[10] “[The component] named the doctrine of treating body parts above 
    the clavicles has the aim of curing diseases situated above clavicles that is,  
    diseases located in ears, eyes, mouth, nose and so on.
    
    \item[11] “[The component] called general medicine has the goal of curing 
    illnesses established in the whole body and [diseases] such as fever, tumour, 
    swelling, hemorrhagic disorders, insanity, epilepsy, urinary diseases, diarrhoea 
    and the like.
    
    \item[12] “[The component] called knowledge of spirits is for appeasing
    demons by pacification rites and making food offerings for those whose
    minds have been possessed by gods, their enemies,\footnote{Dānavas.  The
    insertion marks (\emph{kākapada}s) below the text at this point appears to
    be by the original scribe.} Gandharvas, Yakṣas, demons, deceased
    ancestors, Piśācas, Vināyakas, \footnote{The vulgate doesn't have
    \emph{vināyaka}s but does add \emph{asura}s, probably under the influence
    of Ḍalhaṇa.  Cite Paul Courtright, Ganesha book.} Nāgas and evil spirits
    that possess children. % Notes: vināyaka is omitted from the vulgate. In
    % Mahābhārata, etc. It refers to
    %a class of demons.-->
    
    
    \item[13] “[The component] called care of children is for bearing children and 
    purifying defects in a wet-nurse's milk, and curing diseases that have arisen 
    from bad breast milk and demons.
    
    \item[14] “[The component] called the discipline of toxicology is for
    [knowing] the signs of poison from snake and insect bites and for
    neutralising various combinations of poisons.\footnote{The scribal
    insertion marks (crosses) above the line at this point in MS K appear to
    be in a later hand and their referent is lost in the damaged part of the
    folio.  Although MSS \MScite{Kathmandu NAK 1-1079} and \MScite{Kathmandu 
    NAK 5-333} include \se{lūtā}{spiders} and
    \se{sarīsṛpa}{creepy-crawlies} in the list, it does seem that MS K had
    a shorter list, and the vulgate edition adds \se{mūṣika}{rodents}.}
    
    \item[15] “[The component] called the discipline of rejuvenation is 
    maintaining 
    youth, bringing about a long life and mental vigour and for curing diseases.
    
    % Got to here 2021-01-13
    
    \item[16] “[The component] called the discipline of aphrodisiacs brings about 
    the 
    increase, purity, accumulation and  production of semen for those whose semen 
    is minimal, bad, depleted, and dry [respectively] and for inducing an erection.
    
    \item[17] “Thus, this Āyurveda is taught with eight components."
    
    "Among these [components], tell us which is for whom."
    
    \item[18] They said, "After you have conveyed the knowledge of surgery, 
    teach 
    us everything."
    
    \item[19] He said, "so be it."
    
    \item[20] They then said, "Having considered the view of all of us, when we 
    are 
    unanimous, Suśruta will question you. We too will learn what is being taught to 
    him."
    
    \item[21] He said, "so be it.
    
    \item[22] “Now, as is well-known, the aim of Āyurveda is eliminating the 
    disease of one who have been assailed by disease and protecting the healthy;  
    āyurveda is [that knowledge] in which they find a long life, or that by which 
    long life is known. Learn its best component (i.e., surgery), which is being 
    taught in accordance with tradition, perception, inference and analogy.
    
    \item[23] "For this component is first, the most important, because it is 
    referred to first; it cures wounds and joins together the most important thing, 
    Yajña's head. For, just as it has been said of old, 'the head that had been cut off 
    by Rudra was joined again by the two Aśvins.'
    
    \item[24] "And also, of the eight disciplines of Āyurveda, [surgery] alone is 
    the best because of the quick action of its \se{kriyā}{procedures}, its 
    application of blunt 
    instruments, knives, caustics and fire, and it is common to all disciplines.
    
    \item[25] "Therefore, [surgery] is eternal, meritorious, leads to heaven, 
    brings renown, bestows a long life, and affords a livelihood.
    
    \item[26] "Brahmā said this, 'Prajāpati learned it. From him, the Aśvins. From 
    the Aśvins, 
    Indra. From Indra, I. In this world, I will transmit to those who desire it for the benefit of 
    people.' 
    
    [There a verse about this.].\footnote{This is an expansion 
    of the scribe's abbreviation \emph{bha} for \emph{bhavati cātra ślokaḥ} 
    “There is a verse about this” (sometimes plural).\label{bha}}
    
    \item[27]           
    \begin{sloka}
        For, I (i.e., Brahmā) am Dhanvantari, the first god, the remover of old age, 
        pain and 
        death of mortals.\\ Having understood surgery, the best of the great 
        knowledge 
        systems, I arrived on earth again to teach it here.
    \end{sloka}    
    
    % draft tr.
    
    \item[28] In this context, as far as this discipline is concerned, a 
    \se{puruṣa}{human being} is called an amalgam of the five elements 
    and the embodied soul.  This is where \se{kriyā}{procedures} apply. This 
    is the locus. 
    
    Why?
    
    Because of the duality of the world, the world is twofold: the stationary
    and the moving. Its \se{ātmaka}{nature} is twofold, depending on the
    preponderance of Agni and Soma.\footcite[See][]{wuja-2004}  Alternatively,
    it can be considered as being fivefold.  The multitude of beings in it are
    fourfold: they are termed “sweat-born, stone-born, caul-born and
    egg-born”.\footnote{This fourfold classification of beings is paralleled
    with closely-related vocabulary in  \emph{Bhelasaṃhitā} 4.4.4	
    \parencites[206]{kris-2000}[81]{mook-1921}.}  Where they are concerned, the
    human being is the main thing; others are his support.  Therefore, the
    \se{puruṣa}{human being} is the locus.
    
    \item[29]  Diseases are said to be the conjunction of the person and 
    \se{duḥkha}{suffering}.
    There are four of them: invasive, bodily, mental and inherent.  The invasive ones 
    are caused by an injury.  The bodily ones are based on food, caused by 
    \se{vaiṣamya}{irregularities} in wind, bile, phlegm and blood.\footnote{Note 
    that four humoral substances are assumed here.} 
    
    The \se{mānasa}{mental} ones, caused by 
    \se{icchā}{desire} and 
    \se{dveṣa}{hatred}, 
    include:  
    \se{krodha}{anger}, 
    \se{āśoka}{grief}, 
    \se{dainya}{misery}, 
    \se{harṣa}{overexcitement}, 
    \se{kāma}{lust}, 
    \se{viṣāda}{depression},
    \se{īrṣyā}{envy},
    \se{asūyā}{jealousy},
    \se{mātsarya}{malice}, 
    and
    \se{lobha}{greed}.
    
    The \se{svābhāvika}{inherent} ones are hunger, thirst, old age, death, 
    sleep and  those of the \se{prakṛti}{temperament}.
    
    These too are \se{adhiṣṭhāna}{located} in the mind and body.
    
    \se{lekhana}{Scarification},
    \se{bṛṃhaṇa}{nourishment},
    \se{saṃśodhana}{purification},
    \se{saṃśamana}{pacification},
    \se{āhāra}{diet} and
    \se{ācāra}{regimen}, 
    properly employed, bring about their cure.
    
    
    \item [30] Furthermore, food is the  \se{mūla}{root} of living beings as well
    as of \se{bala}{strength}, \se{varṇa}{complexion} and 
    \se{ojas}{vital
        energy}. It \se{āyatta}{depends on} the six \se{rasa}{flavours}.
    Flavours, furthermore, have substances as their \se{āśrayin}{substrate}.  And
    substances are \se{oṣadhī-}{remedies}.\footnote{Pāṇini 6.3.132 provides that
    the final vowel of the noun \emph{oṣadhi} may be lengthened
    (\emph{$\rightarrow$oṣadhī}) under certain conditions.  These conditions require
    that the word be used in a Vedic mantra and not in the nominative.  Neither
    condition is met in this passage, yet the author uses the form \emph{oṣadhī}. 
    This form is in fact not uncommon in medical literature as well as in epics,
    purāṇas, smṛtis, and other parts of Sanskrit literature.} There are 
    two types:
    \se{sthāvara}{stationary} and \se{jaṅgama}{moving}.
    
    
    
    \item [31]  Of these, there are four types of stationary ones:
    \se{vanaspati}{fruit trees}, \se{vṛkṣa}{flowering trees},
    \se{oṣadhi}{herbs} and \se{vīrudh}{shrubs}.\footnote{Ca.sū.1.71--72 
    also
    describes these four types of medicinal plant in similar terms but with slightly
    differing names: \emph{oṣadhi} is a plant that ends after fruiting, \emph{vīrudh}
    is a plant that branches out, \emph{vanaspati} is a tree with fruit, and
    \emph{vānaspatya} is a tree with fruit and flowers.}
    Amongst these, the “fruit trees” have fruit but no flowers.\footnote{The MSS agree 
    in reading \emph{phalavantyaḥ} “having flowers” which is grammatically 
    non-standard. This form is also found in the  \emph{Viṣṇudharmottarapurāṇa} 
    (1.92.27, \cite[1.92.27][56r]{sarm-1912}).}  The “flowering trees” 
    have flowers and fruit.  The “herbs” die when the fruit is ripe. “Shrubs” put out 
    shoots.
    
    % \citep{ober-2003} didn't have anything on phalavantyo that I could find quickly.
    
    
    \item[32]  As is well known, moving remedies are also of four types: those
    \se{jarāyuja}{born in in a caul}, those \se{aṇḍaja}{born from eggs},
    those \se{svedaja}{born of sweat}, and \se{udbhid}{shoots}. Amongst
    these, those born in a caul include \se{paśu}{animals}, humans, and
    \se{vyāla}{wild animals}.  Birds, \se{sarīsṛpa}{creepy-crawlies} and
    snakes are “born of eggs.” \se{kṛmi}{Worms}, \se{kunta}{small insects}
    and \se{pipīlika}{ants} and others are born of sweat.\footnote{The word
    \emph{kunta}, though marked as “lexical” in most dictionaries, is in fact found
    in literature, commonly as a compound with \emph{pipīlika}; the compound
    sometimes seems to be understood a type of ant (\emph{tatpuruṣa} compound)
    rather than as a pair of insects (\emph{dvandva} compound).}  Shoots include
    \se{indragopa}{red velvet mites} and \se{maṇḍūka}{frogs}.\footnote{On
    \emph{indragopa}, see \cite{lien-1978}.}|
    
    \item[33] In this context, among the stationary remedies, 
    \se{tvak}{skin}, 
    \se{patra}{leaves}, 
    \se{puṣpa}{flowers}, 
    \se{phala}{fruits},
    \se{mūla}{roots},
    \se{kanda}{bulbs},
    \se{kṣīra}{sap},
    \se{niryāsa}{resin},
    \se{sāra}{essence},
    \se{sneha}{oil}, 
    and
    \se{svarasa}{juice extract}\footnote{On \se{svarasa}{juice extract} see
    CS 1.1.73, 1.4.7; Ḍalhaṇa on \Su{4.10.12}{450}.} 
    are useful; among the moving remedies 
    \se{carman}{pelt}, hair, nails, and 
    \se{rudhira}{blood} and so forth. 
    
    \item[34] And \se{pārthiva}{earthen products} include gold and 
    silver.\footnote{The flow of concepts in the treatise seems to be interrupted here.}
    
    \item[35] The \se{kālakṛta}{items created by time} are \se{samplava}{clusters} 
    as far as wind and \se{nivāta}{no wind}, heat and shade, darkness and light
    and the cold, hot and \se{varṣā}{rainy seasons} are concerned. 
    The divisions of time are the
    \se{nimeṣa}{blink of the eye}, a
    \se{kāṣṭhā}{trice}, 
    \se{kalā}{minutes}, 
    \se{muhūrta}{three-quarters of an hour}, a
    \se{ahorātra}{day and night}, a
    \se{pakṣa}{fortnight}, a
    \se{māsa}{month}, a
    \se{ṛtu}{season}, a
    \se{ayana}{half-year}, a
    \se{saṃvatsara}{year},
    and
    \se{yuga}{yuga}.\footnote{These units are presented at 
    \Su{1.6.5}{24} and discussed by \citet[\S\,59]{haya-2017}.}
    
    
    \item[36]  These naturally cause 
    \se{sañcaya}{accumulation}, 
    \se{prakopa}{irritation}, 
    \se{upaśama}{pacification} 
    and 
    \se{pratīkāra}{alleviation} of the \se{doṣa}{humours}. And they have
    \se{prayojanavat}{practical purposes}.
    
    \medskip[There are verses about this:]\footnote{See footnote \ref{bha}.}
    
    \item[37] 
    \begin{sloka}
        This fourfold category is taught by physicians as a cause for the agitation and 
        quelling of bodily diseases.%
        %
        \footnote{On the topic of the “group of four,” the commentator Ḍalhaṇa
        considers them to be “food, behaviour, earthen products and items created by
        time.”  He refers to the author of the lost commentary entitled \emph{Pañjikā},
        and to Jejjaṭa \citep[IA, 372--3, 192]{meul-hist}.  In his view, these early
        commentators  do not agree that the \se{caturvarga}{fourfold grouping} refers
        to the quartet of \se{sthāvara}{stationary}, \se{jaṅgama}{moving},
        \se{pārthiva}{earthen products} and \se{kālakṛta}{items created by 
        time} \citep[9a]{vulgate}.}
    \end{sloka}
    
    \item[38] \begin{sloka}
        There are two kinds of invasive diseases. Some certainly\footnote{The
        text uses an archaic interjection here, \emph{ha}.} 
        affect (\emph{ni\root pat}) \index{nipat@ni\root pat}
        the
        mind, others the body. Their \se{kriyā}{treatment} is of two kinds too.
    \end{sloka}
    
    \item[39]\begin{sloka}
        For those that affect the body there is \se{śārīravad}{physical} 
        therapy, whereas for those that affect the mind there is the 
        \se{varga}{collection} of desirable sensory experiences like sound that 
        bring \se{sukha }{comfort}.
        
    \end{sloka}
    %Those that affect the body have therapy that is  
    %\se{śārīravat}{physical}, whereas for those of the mind it is 
    
    \item [40] 
    
    \se{evam}{Along these lines}, this brief explanation of the 
    \se{catuṣtaya}{four factors}
    is given: \begin{itemize}
        \item    
        \se{puruṣa}{human being},
        \item
        \se{vyadhi}{disease},
        \item
        \se{oṣadhi}{remedies},
        \item
        \se{kriyākāla}{the time for therapies}.
    \end{itemize}
    In this context, 
    \begin{itemize}
        \item from the mention of the word “human,” the collection of 
        substances that arise from it, such as the elements, and the 
        \se{vikalpa}{particulars} of its major 
        and minor \se{aṅga}{parts} such as 
        \se{tvak}{skin}, 
        \se{māṃsa}{flesh}, 
        \se{sirā}{ducts}, 
        \se{snāyu}{sinews}, 
        \se{asthi}{bones} and 
        \se{sandhi}{joints}
        are meant.
        \item
        From the mention of “diseases,” all diseases 
        caused by
        wind, bile, phlegm,
        \se{sannipāta}{congested humours},
        \se{āgantu}{external factors} and 
        \se{svabhāva}{inherent factors} are \se{vyākhyāta}{intended}.
        \item
        From the mention of “remedies,”
        there is the teaching of 
        substances,
        tastes, 
        potencies,
        post-digestive tastes.
        \item
        From the mention of 
        “\se{kriyā}{procedures},”
        \se{karman}{therapies} such as oiling
        and 
        \se{chedya}{excision} are taught.
        \item
        From the mention of the word “time,” every single teaching about the times for
        procedures is meant. 
        
    \end{itemize}
    \medskip[There is a verse about this:]\footnote{See footnote \ref{bha}.}
    
    \item[41]
    
    \begin{sloka}
        This seed of medicine has been declared in brief.  Its explanation will be given in one 
        hundred and twenty chapters.\footnote{This is the number of chapters in the first 
        five sections of the work, namely the  \emph{Sūtra-, Nidāna-, Śārīra-, Cikitsā- 
        \emph{and} 
        Kalpa-sthāna}s. These have 46, 16, 10, 40 and 8 chapters respectively.  The 
        \emph{Uttaratantra} has 66 chapters.}
\end{sloka}

%[There are verses on this:]\q{is this bha
%really in the MSS?}

\item [42] There are one hundred and twenty chapters in five
\se{adhyāya}{sections}.\footnote{On \emph{viṃśa} in the sense of “greater by 20”
see P.5.2.46 \emph{śadantaviṃśateś ca}.}  In that regard, having divided them,
according to their subject matter, into the Ślokasthāna, the Nidāna, the Śārīra,
the Cikitsita and the Kalpa, we shall mention this in the
Uttaratantra.\footnote{The end of this sentence reads oddly.  The vulgate edition
adds an object: “[we shall mention] the remaining topics [in the Uttara]” which
smooths out the difficulty, but this is supported in none of the Nepalese MSS.  At
the start of the Uttaratantra \citep[1.3--4ab]{susr-trikamji3} there is indeed a
statement that picks up the point about there being 120 chapters.}

\medskip[There is a verse about this:]\footnote{See footnote \ref{bha}.}

\item[43]    
\begin{sloka}
    Someone who reads this eternal proclamation of the King of
    Kāśī, that was declared by Svayambhu, will have good karma on earth, will
    be respected by kings and upon death will achieve the world of Śakra.
\end{sloka}

\end{translation}    



