\documentclass[14pt]{extarticle}
\usepackage{polyglossia,fontspec,xunicode}
\usepackage[normalem]{ulem}
\usepackage[noend,noeledsec,noledgroup]{reledmac}
\usepackage[margin=1in]{geometry}

\arrangementX[A]{paragraph}
\arrangementX[B]{paragraph}
\renewcommand*{\thefootnoteB}{\Roman{footnoteB}}
\arrangementX[C]{paragraph}
\renewcommand*{\thefootnoteC}{\roman{footnoteC}}


\Xarrangement[A]{paragraph}
\Xarrangement[D]{paragraph}
\Xnotenumfont[A]{\bfseries}
\Xlemmafont[A]{\bfseries}
\Xnotenumfont[D]{\bfseries}
\Xlemmafont[D]{\bfseries}


\setdefaultlanguage{sanskrit}
\setotherlanguage{english}
\newfontfamily\devanagarifont{Brill}
%\newfontfamily{\devafont}{Pedantic Devanagari}

\usepackage[Devanagari,DevanagariExtended]{ucharclasses}

\makeatletter
\setTransitionsFor{Devanagari}%
 {\let\curfamily\f@family\let\curshape\f@shape\let\curseries\f@series\devafont}
 {\fontfamily{\curfamily}\fontshape{\curshape}\fontseries{\curseries}\selectfont}
\makeatother

\makeatletter
\setTransitionsFor{DevanagariExtended}%
 {\let\curfamily\f@family\let\curshape\f@shape\let\curseries\f@series\devafont}
 {\fontfamily{\curfamily}\fontshape{\curshape}\fontseries{\curseries}\selectfont}
\makeatother

% DW macros:
\renewcommand{\omit}{\emph{om.}}
\newcommand{\add}{\emph{add.}}
\newcommand{\kakapada}{$\mathcal{X}$}

\begin{document}
    \raggedright
% Manual hyphenation points for Sanskrit words and compounds.
% By Dominik Wujastyk.
% Copyright Dominik Wujastyk 2021.
% Released under a BY-SA Creative Commons license 
% (Attribution-ShareAlike 4.0 International http://creativecommons.org/licenses/by-sa/4.0/).
% This file is still actively growing, slowly but steadily (March 2021) .
%
% These special hyphenations have to be loaded after
% \begin{document}. See
% http://www.tug.org/pipermail/xetex/2008-July/010362.html
% Or use 
% \AtBeginDocument{% Manual hyphenation points for Sanskrit words and compounds.
% By Dominik Wujastyk.
% Copyright Dominik Wujastyk 2021.
% Released under a BY-SA Creative Commons license 
% (Attribution-ShareAlike 4.0 International http://creativecommons.org/licenses/by-sa/4.0/).
% This file is still actively growing, slowly but steadily (March 2021) .
%
% These special hyphenations have to be loaded after
% \begin{document}. See
% http://www.tug.org/pipermail/xetex/2008-July/010362.html
% Or use 
% \AtBeginDocument{% Manual hyphenation points for Sanskrit words and compounds.
% By Dominik Wujastyk.
% Copyright Dominik Wujastyk 2021.
% Released under a BY-SA Creative Commons license 
% (Attribution-ShareAlike 4.0 International http://creativecommons.org/licenses/by-sa/4.0/).
% This file is still actively growing, slowly but steadily (March 2021) .
%
% These special hyphenations have to be loaded after
% \begin{document}. See
% http://www.tug.org/pipermail/xetex/2008-July/010362.html
% Or use 
% \AtBeginDocument{\input{sanskrit-hyphenations}}% should work, but doesn't
% special hyphenations for Sanskrit words tagged in
% Polyglossia.
% *English,\textenglish{},text,and
% *Sanskrit,\textsanskrit{},text.
%
% English (see below for \textsanskrit)
%
\hyphenation{%
    dhanva-ntariṇopa-diṣ-ṭaḥ
    suśruta-nāma-dheyena
    tac-chiṣyeṇa
    kāśyapa-saṃ-hitā
    cikitsā-sthāna
    su-śruta-san-dīpana-bhāṣya
    dṛṣṭi-maṇḍala
    uc-chiṅga-na
    sarva-siddhānta-tattva-cūḍā-maṇi
    tulya-sau-vīrāñja-na
    indra-gopa
    śrī-mad-abhi-nava-guptā-cārya-vi-ra-cita-vi-vṛti-same-tam
    viśva-nātha
śrī-mad-devī-bhāga-vata-mahā-purāṇa
    siddhā-n-ta-sun-dara
    brāhma-sphuṭa-siddh-ānta
    bhū-ta-saṅ-khyā
    bhū-ta-saṃ-khyā
    kathi-ta-pada
    devī-bhā-ga-vata-purāṇa
    devī-bhā-ga-vata-mahā-purāṇa
    Siddhānta-saṃ-hitā-sāra-sam-uc-caya
    sau-ra-pau-rāṇi-ka-mata-sam-artha-na
    Pṛthū-da-ka-svā-min
    Brah-ma-gupta
    Brāh-ma-sphu-ṭa-siddhānta
    siddhānta-sun-dara
    vāsa-nā-bhāṣya
    catur-veda
    bhū-maṇḍala
    jñāna-rāja
    graha-gaṇi-ta-cintā-maṇi
    Śiṣya-dhī-vṛd-dhi-da-tan-tra
    brah-māṇḍa-pu-rā-ṇa
    kūr-ma-pu-rā-ṇa
    jam-bū-dvī-pa
    bhā-ga-vata-pu-rā-ṇa
    kupya-ka
    nandi-suttam
    nandi-sutta
    su-bodhiā-bāī
    asaṅ-khyāta
    saṅ-khyāta
    saṅ-khyā-pra-māṇa
    saṃ-khā-pamāṇa
    nemi-chandra
    anu-yoga-dvāra
    tattvārtha-vārtika
    aka-laṅka
    tri-loka-sāra
    gaṇi-ma-pra-māṇa
    gaṇi-ma-ppa-māṇa
    eka-pra-bhṛti
gaṇaṇā-saṃ-khā
gaṇaṇā-saṅ-khyā
dvi-pra-bhṛti
duppa-bhi-ti-saṃ-khā
vedanābhi-ghāta
Viṣṇu-dharmottara-pu-rāṇa
abhaya-deva-sūri-vi-racita-vṛtti-vi-bhūṣi-tam
abhi-dhar-ma
abhi-dhar-ma-ko-śa
abhi-dhar-ma-ko-śa-bhā-ṣya
abhi-dharma-kośa-bhāṣya
abhi-dharma-kośa-bhāṣyam
abhi-nava
abhyaṃ-karopāhva-vāsu-deva-śāstri-vi-ra-ci-ta-yā
ācārya-śrī-jina-vijayālekhitāgra-vacanālaṃ-kṛtaś-ca
ācāry-opā-hvena
ādhāra
adhi-kāra
adhi-kāras
ādi-nātha
agni-besha
agni-veśa
ahir-budhnya
ahir-budhnya-saṃ-hitā
aita-reya-brāhma-ṇa
akusī-dasya
amara-bharati
Amar-augha-pra-bo-dha
amṛ-ta-siddhi
ānanda-kanda
ānan-da-rā-ya
ānand-āśra-ma-mudraṇā-la-ya
ānand-āśra-ma-saṃ-skṛta-granth-āva-liḥ
anna-pāna-mūlā
anu-ban-dhya-lakṣaṇa-sam-anv-itās
anu-bhav-ād
anu-bhū-ta-viṣayā-sam-pra-moṣa
anu-bhū-ta-viṣayā-sam-pra-moṣaḥ
aparo-kṣā-nu-bhū-ti
app-proxi-mate-ly
ardha-rātrika-karaṇa
ārdha-rātrika-karaṇa
ariya-pary-esana-sutta
arun-dhatī
ārya-bhaṭa
ārya-bhaṭā-cārya-vi-racitam
ārya-bhaṭīya
ārya-bhaṭīyaṃ
ārya-lalita-vistara-nāma-mahā-yāna-sūtra
ārya-mañju-śrī-mūla-kalpa
ārya-mañju-śrī-mūla-kalpaḥ
asaṃ-pra-moṣa
aṣṭāṅga-hṛdaya-saṃ-hitā
aṣṭāṅga-saṃ-graha
asura-bhavana
aśva-ghoṣa
ātaṅka-darpaṇa-vyā-khyā-yā
atha-vā
ava-sāda-na
āyār-aṅga-suttaṃ
ayur-ved
ayur-veda
āyur-veda
āyur-veda-dīpikā
āyur-veda-dīpikā-vyā-khyayā
āyur-ve-da-ra-sā-yana
āyur-veda-sū-tra
ayur-vedic
āyur-vedic
ayur-yog
bādhirya
bahir-deśa-ka
bala-bhadra
bala-kot
bala-krishnan
bāla-kṛṣṇa
bau-dhā-yana-dhar-ma-sūtra
bel-valkar
bhadra-kālī-man-tra-vi-dhi-pra-karaṇa
bhadrā-sana
bhadrā-sanam
bha-ga-vat-pāda
bhaiṣajya-ratnāvalī
bhan-d-ar-kar
bhartṛhari-viracitaḥ
bhaṭṭā-cārya
bhaṭṭot-pala-vi-vṛti-sahitā
Bhiṣag-varāḍha-malla-vi-racita-dīpikā-Kāśī-rāma-vaidya-vi-raci-ta-gūḍhā-rtha-dīpikā-bhyāṃ
bhiṣag-varāḍha-malla-vi-racita-dīpikā-Kāśī-rāma-vaidya-vi-racita-gūḍhārtha-dīpikā-bhyāṃ
bhoja-deva-vi-raci-ta-rāja-mārtaṇḍā-bhi-dha-vṛtti-sam-e-tāni
bhu--va-na-dī-pa-ka
bīja-pallava
bi-kaner
bodhi-sat-tva-bhūmi
brahma-gupta
brahmā-nanda
brahmāṇḍa-mahā-purā-ṇa
brahmāṇḍa-mahā-purā-ṇam
brahma-randhra
brahma-siddh-ānta
brāhma-sphuṭa-siddh-ānta
brāhma-sphu-ṭa-siddhānta
brahma-vi-hāra
brahma-vi-hāras
brahma-yā-mala-tan-tra
Bra-ja-bhāṣā
bṛhad-āraṇya-ka
bṛhad-yā-trā
bṛhad-yogi-yājña-valkya-smṛti
bṛhad-yogī-yājña-valkya-smṛti
bṛhaj-jāta-kam
bṛhat-khe-carī-pra-kāśa
buddhi-tattva-pra-karaṇa
cak-ra-dat-ta
cakra-datta
cakra-pāṇi-datta
cā-luk-ya
caraka-prati-saṃ-s-kṛta
caraka-prati-saṃ-s-kṛte
caraka-saṃ-hitā
casam-ul-lasi-tāmaharṣiṇāsu-śrutenavi-raci-tāsu-śruta-saṃ-hitā
cau-kham-ba
cau-luk-yas
chandi-garh
chara-ka
cha-rīre
chatt-opa-dh-ya-ya
chau-kham-bha
chi-ki-tsi-ta
cid-ghanā-nanda-nātha
ci-ka-ner
com-men-taries
com-men-tary
com-pre-hen-sive-ly
daiva-jñālaṃ-kṛti
daiva-jñālaṅ-kṛti
dāmo-dara-sūnu-Śārṅga-dharācārya-vi-racitā
Dāmodara-sūnu-Śārṅga-dharācārya-vi-racitā
darśanā-ṅkur-ābhi-dhayā
das-gupta
deha-madhya
deha-saṃ-bhava-hetavaḥ
deva-datta
deva-nagari
deva-nāgarī
devā-sura-siddha-gaṇaiḥ
dha-ra-ni-dhar
dharma-megha
dharma-meghaḥ
dhru-vam
dhru-va-sya
dhru-va-yonir
dhyā-na-grahopa-deśā-dhyā-yaś
dṛḍha-śūla-yukta-rakta
dvy-ulbaṇaikolba-ṇ-aiḥ
four-fold
gan-dh-ā-ra
gārgīya-jyoti-ṣa
gārgya-ke-rala-nīla-kaṇṭha-so-ma-sutva-vi-racita-bhāṣyo-pe-tam
garuḍa-mahā-purāṇa
gaurī-kāñcali-kā-tan-tra
gau-tama
gauta-mādi-tra-yo-da-śa-smṛty-ātma-kaḥ
gheraṇḍa-saṃ-hitā
gorakṣa-śata-ka
go-tama
granth-ā-laya
grantha-mālā
gran-tha-śreṇiḥ
grāsa-pramāṇa
guru-maṇḍala-grantha-mālā
gyatso
hari-śāstrī
haṭhābhyāsa-paddhati
haṭha-ratnā-valī
Haṭha-saṅ-keta-candri-kā
haṭha-tattva-kau-mudī
haṭha-yoga
hāyana-rat-na
haya-ta-gran-tha
hema-pra-bha-sūri
hetu-lakṣaṇa-saṃ-sargād
hīna-madhyādhi-kaiś
hindī-vyā-khyā-vi-marśope-taḥ
hoern-le
ijya-rkṣa
ikka-vālaga
indra-dhvaja
indrāṇī-kalpa
indria
Īśāna-śiva-guru-deva-pad-dhati
jābāla-darśanopa-ni-ṣad
jadav-ji
jagan-nā-tha
jala-basti
jal-pa-kal-pa-tāru
jam-bū-dvī-pa-pra-jña-pti
jam-bū-dvī-pa-pra-jña-pti-sūtra
jana-pad-a-sya
jāta-ka-kar-ma-pad-dhati
jaya-siṃha
jinā-agama-grantha-mālā
jin-en-dra-bud-dhi
jīvan-muk-ti-vi-veka
jñā-na-nir-mala
jñā-na-nir-malaṃ
joga-pra-dīpya-kā
jya-rkṣe
Jyo-tiḥ-śās-tra
jyo-ti-ṣa-rāya
jyoti-ṣa-rāya
jyotiṣa-siddhānta-saṃ-graha
jyotiṣa-siddhānta-saṅ-graha
kāka-caṇḍīśvara-kal-pa-tan-tra
kakṣa-puṭa
kali-kāla-sarva-jña
kali-kāla-sarva-jña-śrī-hema-candrācārya-vi-raci-ta
kali-kāla-sarva-jña-śrī-hema-candrācārya-vi-raci-taḥ
kali-yuga
kal-pa
kal-pa-sthāna
kalyāṇa-kāraka
Kāmeśva-ra-siṃha-dara-bhaṅgā-saṃ-skṛta-viśva-vidyā-layaḥ
kapāla-bhāti
karaṇa-tilaka
kar-ma
kar-man
kāṭhaka-saṃ-hitā
kavia-rasu
kavi-raj
keśa-va-śāstrī
ke-vala--rāma
keva-la-rāma
khaṇḍa-khādyaka-tappā
khe-carī-vidyā
knowl-edge
kol-ka-ta
kriyā-krama-karī
kṛṣṇa-pakṣa
kṛtti-kā
kṛtti-kās
kubji-kā-mata-tantra
kula-pañji-kā
kul-karni
ku-māra-saṃ-bhava
kuṭi-pra-veśa
kuṭi-pra-veśika
lakṣ-mī-veṅ-kaṭ-e-ś-va-ra
lit-era-ture
lit-era-tures
locana-roga
mādha-va
mādhava-kara
mādhava-ni-dāna
mādhava-ni-dā-nam
madh-ūni
madhya
mādhyan-dina
madhye
mahā-bhāra-ta
mahā-deva
mahā-kavi-bhartṛ-hari-praṇīta-tvena
maha-mahopa-dhyaya
mahā-maho-pā-dhyā-ya-śrī-vi-jñā-na-bhikṣu-vi-raci-taṃ
mahā-mati-śrī-mādhava-kara-pra-ṇī-taṃ
mahā-mudrā
mahā-muni-śrī-mad-vyāsa-pra-ṇī-ta
mahā-muni-śrī-mad-vyāsa-pra-ṇī-taṃ
maharṣiṇā
maha-rṣi-pra-ṇīta-dharma-śāstra-saṃ-grahaḥ
Maha-rṣi-varya-śrī-yogi-yā-jña-valkya-śiṣya-vi-racitā
mahā-sacca-ka-sutta
mahā-sati-paṭṭhā-na-sutta
mahā-vra-ta
mahā-yāna-sūtrālaṅ-kāra
maitrāya-ṇī-saṃ-hitā
maktab-khānas
māla-jit
māli-nī-vijayot-tara-tan-tra
manaḥ-sam-ā-dhi
mānasol-lāsa
mānava-dharma-śāstra
mandāgni-doṣa
mannar-guḍi
mano-har-lal
mano-ratha-nandin
man-u-script
man-u-scripts
mataṅga-pārame-śvara
mater-ials
matsya-purāṇam
medh-ā-ti-thi
medhā-tithi
mithilā-stha
mithilā-stham
mithilā-sthaṃ
mṛgendra-tantra-vṛtti
mud-rā-yantr-ā-laye
muktā-pīḍa
mūla-pāṭha
muṇḍī-kalpa
mun-sh-ram
Nāda-bindū-pa-ni-ṣat
nāga-bodhi
nāga-buddhi
nakṣa-tra
nara-siṃha
nārā-yaṇa-dāsa
nārā-yaṇa-dāsa
nārā-yaṇa-kaṇṭha
nārā-yaṇa-paṇḍi-ta-kṛtā
nar-ra-tive
nata-rajan
nava-pañca-mayor
nava-re
naya-na-sukho-pā--dhyāya
ni-ban-dha-saṃ-grahā-khya-vyākhya-yā
niban-dha-san-graha
ni-dā-na
nidā-na-sthā-na-sya
ni-dāna-sthānasyaśrī-gaya-dāsācārya-vi-racitayānyāya-candri-kā-khya-pañjikā-vyā-khyayā
nir-anta-ra-pa-da-vyā-khyā
nir-guṇḍī-kalpa
nir-ṇaya-sā-gara
Nir-ṇaya-sāgara
nir-ṇa-ya-sā-gara-mudrā-yantrā-laye
nir-ṇa-ya-sā-ga-ra-yantr-āla-ya
nir-ṇaya-sā-gara-yantr-ā-laye
niśvāsa-kārikā
nīti-śṛṅgāra-vai-rāgyādi-nāmnāsamākhyā-tānāṃ
nityā-nanda
nya-grodha
nya-grodho
nyā-ya-candri-kā-khya-pañji-kā-vyā-khya-yā
nyāya-śās-tra
okaḥ-sātmya
okaḥ-sātmyam
okaḥ-sātmyaṃ
oka-sātmya
oka-sātmyam
oka-sātmyaṃ
oris-sa
oṣṭha-saṃ-puṭa
ousha-da-sala
padma-pra-bha-sūri
Padma-prā-bhṛ-ta-ka
padma-sva-sti-kārdha-candrādike
paitā-maha-siddhā-nta
pañca-karma
pañca-karman
pāñca-rātrā-gama
pañca-siddh-āntikā
paṅkti-śūla
Paraśu-rāma
paraśu-rāma
pari-likh-ya
pāśu-pata-sū-tra-bhāṣya
pātañ-jala-yoga-śās-tra
pātañ-jala-yoga-śās-tra-vi-varaṇa
pat-añ-jali
pat-na
pāva-suya
phiraṅgi-can-dra-cchedyo-pa-yogi-ka
pim-pal-gaon
pipal-gaon
pitta-śleṣ-man
pit-ta-śleṣ-ma-śoṇi-ta
pitta-śoṇi-ta
prā-cīna-rasa-granthaḥ
prā-cya
prā-cya-hindu-gran-tha-śreṇiḥ
prācya-vidyā-saṃ-śodhana-mandira
pra-dhān-in
pra-ka-shan
pra-kaṭa-mūṣā
pra-kṛ-ti-bhū-tāḥ
pra-mā-ṇa-vārt-tika
pra-ṇītā
pra-saṅ-khyāne
pra-śas-ta-pāda-bhāṣya
pra-śna-pra-dīpa
pra-śnārṇa-va-plava
praśnārṇava-plava
pra-śna-vai-ṣṇava
pra-śna-vaiṣṇava
prati-padyate
pra-yatna-śaithilyānan-ta-sam-āpatti-bhyām
prei-sen-danz
punar-vashu
puṇya-pattana
pūrṇi-mā-nta
raghu-nātha
rāja-kīya
rāja-kīya-mudraṇa-yantrā-laya
rāja-śe-khara
rajjv-ābhyas-ya
raj-put
rāj-put
rakta-mokṣa-na
rāma-candra-śāstrī
rāma-kṛṣṇa
rāma-kṛṣṇa-śāstri-ṇā
rama-su-bra-manian
rāmā-yaṇa
rasa-ratnā-kara
rasa-ratnākarāntar-ga-taś
rasa-ratna-sam-uc-caya
rasa-ratna-sam-uc-ca-yaḥ
rasa-vīry-auṣa-dha-pra-bhāvena
rasā-yana
rasendra-maṅgala
rasendra-maṅgalam
rāṣṭra-kūṭa
rāṣṭra-kūṭas
sādhana
śākalya-saṃ-hitā
śāla-grāma-kṛta
śāla-grāma-kṛta
sāmañña-pha-la-sutta
sāmañña-phala-sutta
sama-ran-gana-su-tra-dhara
samā-raṅga-ṇa-sū-tra-dhāra
sama-ra-siṃ-ha
sama-ra-siṃ-haḥ
sāmba-śiva-śāstri
same-taḥ
saṃ-hitā
śāṃ-ka-ra-bhāṣ-ya-sam-etā
sam-rāṭ
saṃ-rāṭ
Sam-rāṭ-siddhānta
Sam-rāṭ-siddhānta-kau-stu-bha
sam-rāṭ-siddhānta-kau-stu-bha
saṃ-sargam
saṃ-sargaṃ
saṃ-s-kṛta
saṃ-s-kṛta-pārasī-ka-pra-da-pra-kāśa
saṃ-śo-dhana
saṃ-śodhitā
saṃ-sthāna
sam-ullasitā
sam-ul-lasi-tam
saṃ-valitā
saṃ-valitā
śāndilyopa-ni-ṣad
śaṅ-kara
śaṅ-kara-bha-ga-vat-pāda
śaṅ-karā-cārya
san-kara-charya
Śaṅ-kara-nārā-yaṇa
sāṅ-kṛt-yā-yana
san-s-krit
śāra-dā-tila-ka-tan-tra
śa-raṅ-ga-deva
śār-dūla-karṇā-va-dāna
śār-dūla-karṇā-va-dāna
śā-rī-ra-sthāna
śārṅga-dhara-saṃ-hitā
Śārṅga-dhara-saṃ-hitā
sar-va-dar-śana-saṅ-gra-ha
sarva-kapha-ja
sarv-arthāvi-veka-khyā-ter
sar-va-śa-rīra-carās
sarva-siddhānta-rāja
Sarva-siddhā-nt-rāja
sarva-vyā-dhi-viṣāpa-ha
sarva-yoga-sam-uc-caya
sar-va-yogeśvareśva-ram
śāstrā-rambha-sam-artha-na
śatakatrayādi-subhāṣitasaṃgrahaḥ
sati-paṭṭhā-na-sutta
ṣaṭ-karma
ṣaṭ-karman
sat-karma-saṅ-graha
sat-karma-saṅ-grahaḥ
ṣaṭ-pañcā-śi-kā
saun-da-ra-nanda
sa-v-āī
schef-tel-o-witz
scholars
sharī-ra
sheth
sid-dha-man-tra
siddha-nanda-na-miśra
siddha-nanda-na-miśraḥ
siddha-nitya-nātha-pra-ṇītaḥ
Siddhānta-saṃ-hitā-sāra-sam-uc-caya
Siddhā-nta-sār-va-bhauma
siddhānta-sindhu
siddhānta-śiro-maṇ
Siddhānta-śiro-maṇi
Siddhā-nta-tat-tva-vi-veka
sid-dha-yoga
siddha-yoga
sid-dhi
sid-dhi-sthā-na
sid-dhi-sthāna
śikhi-sthāna
śiraḥ-karṇā-kṣi-vedana
śiro-bhūṣaṇam
Śivā-nanda-saras-vatī
śiva-saṃ-hitā
śiva-yo-ga-dī-pi-kā
ska-nda-pu-rā-ṇa
śleṣ-man
śleṣ-ma-śoni-ta
sodā-haraṇa-saṃ-s-kṛta-vyā-khyayā
śodha-ka-pusta-kaa
śoṇi-ta
spaṣ-ṭa-krānty-ādhi-kāra
śrī-cakra-pāṇi-datta
śrī-cakra-pāṇi-datta-viracitayā
śrī-ḍalhaṇācārya-vi-raci-tayāni-bandha-saṃ-grahākhya-vyā-khyayā
śrī-dayā-nanda
śrī-hari-kṛṣṇa-ni-bandha-bhava-nam
śrī-hema-candrā-cārya-vi-raci-taḥ
śrī-kaṇtha-dattā-bhyāṃ
śrī-kṛṣṇa-dāsa
śrī-kṛṣṇa-dāsa-śreṣṭhinā
śrīmac-chaṅ-kara-bhaga-vat-pāda-vi-raci-tā
śrī-mad-amara-siṃha-vi-racitam
śrī-mad-bha-ga-vad-gī-tā
śrī-mad-bhaṭṭot-pala-kṛta-saṃ-s-kṛta-ṭīkā-sahitam
śrī-mad-dvai-pā-yana-muni-pra-ṇītaṃ
śrī-mad-vāg-bhaṭa-vi-raci-tam
śrī-maṃ-trī-vi-jaya-siṃha-suta-maṃ-trī-teja-siṃhena
śrī-mat-kalyāṇa-varma-vi-racitā
śrī-mat-sāyaṇa-mādhavācārya-pra-ṇītaḥsarva-darśana-saṃ-grahaḥ
śrī-nitya-nātha-siddha-vi-raci-taḥ
śrī-rāja-śe-khara
śrī-śaṃ-karā-cārya-vi-raci-tam
śrī-vā-cas-pati-vaidya-vi-racita-yā
śrī-vatsa
śrī-veda-vyāsa-pra-ṇīta-mahā-bhā-ratāntar-ga-tā
śrī-veṅkaṭeś-vara
śrī-vi-jaya-rakṣi-ta
sruta-rakta
sruta-raktasya
stambha-karam
sthānāṅga-sūtra
sthira-sukha
sthira-sukham
stra-sthā-na
subhāṣitānāṃ
su-brah-man-ya
su-bra-man-ya
śukla-pakṣa
śukrā-srava
suk-than-kar
su-pariṣkṛta-saṃgrahaḥ
sura-bhi-pra-kash-an
sūrya-dāsa
sūrya-siddhānta
su-shru-ta
su-śru-ta
su-shru-ta-saṃ-hitā
su-śru-ta-saṃ-hitā
su-śru-tena
sutra
sūtra
sūtra-neti
sūtra-ni-dāna-śā-rīra-ci-ki-tsā-kal-pa-sthānot-tara-tan-trātma-kaḥ
sūtra-sthāna
su-varṇa-pra-bhāsot-tama-sū-tra
Su-var-ṇa-pra-bhās-ot-tama-sū-tra
su-varṇa-pra-bhāsotta-ma-sūtra
su-vistṛta-pari-cayātmikyāṅla-prastāvanā-vividha-pāṭhān-tara-pari-śiṣṭādi-sam-anvitaḥ
sva-bhāva-vyādhi-ni-vāraṇa-vi-śiṣṭ-auṣa-dha-cintakās
svā-bhāvika
svā-bhāvikās
sva-cchanda-tantra
śvetāśva-taropa-ni-ṣad
taila-sarpir-ma-dhūni
tait-tirīya-brāhma-ṇa
tājaka-muktā-valeḥ
tājika-kau-stu-bha
tājika-nīla-kaṇṭhī
tājika-yoga-sudhā-ni-dhi
tapo-dhana
tapo-dhanā
tārā-bhakti-su-dhārṇava
tārtīya-yoga-su-sudhā-ni-dhi
tegi-ccha
te-jaḥ-siṃ-ha
ṭhāṇ-āṅga-sutta
ṭīkā-bhyāṃ
ṭīkā-bhyāṃ
tiru-mantiram
tiru-ttoṇṭar-purāṇam
tiru-va-nanta-puram
trai-lok-ya
trai-lokya-pra-kāśa
tri-bhāga
tri-kam-ji
tri-pita-ka
tri-piṭa-ka
tri-vik-ra-mātma-jena
ud-ā-haraṇa
un-mārga-gama-na
upa-ca-ya-bala-varṇa-pra-sādādī-ni
upa-laghana
upa-ni-ṣads
upa-patt-ti
ut-sneha-na
utta-rā-dhya-ya-na
utta-rā-dhya-ya-na-sūtra
uttara-khaṇḍa-khādyaka
uttara-sthāna
uttara-tantra
vācas-pati-miśra-vi-racita-ṭīkā-saṃ-valita
vācas-pati-miśra-vi-racita-ṭīkā-saṃ-valita-vyā-sa-bhā-ṣya-sam-e-tāni
vag-bhata-rasa-ratna-sam-uc-caya
vāg-bhaṭa-rasa-ratna-sam-uc-caya
vaidya-vara-śrī-ḍalhaṇā-cārya-vi-racitayā
vai-śā-kha
vai-śeṣ-ika-sūtra
vāja-sa-neyi-saṃ-hitā
vājī-kara-ṇam
vākya-śeṣa
vākya-śeṣaḥ
vaṅga-sena
vaṅga-sena-saṃ-hitā
varā-ha-mihi-ra
vārāhī-kalpa
vā-rāṇa-seya
va-ra-na-si
var-mam
var-man
var-ṇa-saṃ-khyā
var-ṇa-saṅ-khyā
vā-si-ṣṭha
vasiṣṭha-saṃ-hitā
vā-siṣṭha-saṃ-hitā
Va-sistha-Sam-hita-Yoga-Kanda-With-Comm-ent-ary-Kai-valya-Dham
vastra-dhauti
vasu-bandhu
vāta-pit-ta
vāta-pit-ta-kapha
vāta-pit-ta-kapha-śoṇi-ta
vāta-pitta-kapha-śoṇita-san-nipāta-vai-ṣamya-ni-mittāḥ
vāta-pit-ta-śoṇi-ta
vāta-śleṣ-man
vāta-śleṣ-ma-śoṇi-ta
vāta-śoṇi-ta
vātā-tapika
vātsyā-ya-na
vāya-vīya-saṃ-hitā
vedāṅga-rāya
veezhi-nathan
venkat-raman
vid-vad-vara-śrī-gaṇeśa-daiva-jña-vi-racita
vidya-bhu-sana
vi-jaya-siṃ-ha
vi-jñāna-bhikṣu
Vijñāneśvara-vi-racita-mitākṣarā-vyā-khyā-sam-alaṅ-kṛtā
vi-mā-na
vi-mā-na-sthāna
vimāna-sthā-na
vi-racitā
vi-racita-yāmadhu-kośākhya-vyā-khya-yā
vi-recana
vishveshvar-anand
vi-śiṣṭ-āṃśena
vi-suddhi-magga
vi-vi-dha-tṛṇa-kāṣṭha-pāṣāṇa-pāṃ-su-loha-loṣṭāsthi-bāla-nakha-pūyā-srāva-duṣṭa-vraṇāntar-garbha-śalyo-ddharaṇārthaṃ
vṛd-dha-vṛd-dha-tara-vṛd-dha-tamaiḥ
vṛddha-vṛddha-tara-vṛddha-tamaiḥ
vṛnda-mādhava
vyāḍī-ya-pa-ri-bhā-ṣā-vṛtti
vyā-khya-yā
vy-akta-liṅgādi-dharma-yuk-te
vyā-sa-bhā-ṣya-sam-e-tāni
vyati-krāmati
Xiuyao
yādava-bhaṭṭa
yāda-va-śarma-ṇā
yādava-sūri
yājña-valkya-smṛti
yājña-valkya-smṛtiḥ
yantrā-dhyāya
Yantra-rāja-vicāra-viṃśā-dhyāyī
yavanā-cā-rya
yoga-bhā-ṣya-vyā-khyā-rūpaṃ
yoga-cintā-maṇi
yoga-cintā-maṇiḥ
yoga-ratnā-kara
yoga-sāra-mañjarī
yoga-sāra-sam-uc-caya
yoga-sāra-saṅ-graha
yoga-śikh-opa-ni-ṣat
yoga-tārā-valī
yoga-yājña-val-kya
yoga-yājña-valkya-gītāsūpa-ni-ṣatsu
yogi-yājña-valkya-smṛti
yoshi-mizu
yukta-bhava-deva
}
%%%%%%%%%%%%%%%%%%%%
%Sanskrit:
%%%%%%%%%%%%%%%%%%%%
\textsanskrit{\hyphenation{%
    dhanva-ntariṇopa-diṣ-ṭaḥ
suśruta-nāma-dheyena
tac-chiṣyeṇa
    su-śruta-san-dīpana-bhāṣya
    cikitsā-sthāna
tulya-sau-vīrāñjana
indra-gopa
dṛṣṭi-maṇḍala
uc-chiṅga-na
vi-vi-dha-tṛṇa-kāṣṭha-pāṣāṇa-pāṃ-su-loha-loṣṭāsthi-bāla-nakha-pūyā-srāva-duṣṭa-vraṇāntar-garbha-śalyo-ddharaṇārthaṃ
śrī-ḍalhaṇācārya-vi-raci-tayāni-bandha-saṃ-grahākhya-vyā-khyayā
ni-dāna-sthānasyaśrī-gaya-dāsācārya-vi-racitayānyāya-candri-kā-khya-pañjikā-vyā-khyayā
casam-ul-lasi-tāmaharṣiṇāsu-śrutenavi-raci-tāsu-śruta-saṃ-hitā
bhartṛhari-viracitaḥ
śatakatrayādi-subhāṣitasaṃgrahaḥ
mahā-kavi-bhartṛ-hari-praṇīta-tvena
nīti-śṛṅgāra-vai-rāgyādi-nāmnāsamākhyā-tānāṃ
subhāṣitānāṃ
su-pariṣkṛta-saṃgrahaḥ
su-vistṛta-pari-cayātmikyāṅla-prastāvanā-vividha-pāṭhān-tara-pari-śiṣṭādi-sam-anvitaḥ
ācārya-śrī-jina-vijayālekhitāgra-vacanālaṃ-kṛtaś-ca
abhaya-deva-sūri-vi-racita-vṛtti-vi-bhūṣi-tam
abhi-dhar-ma
abhi-dhar-ma-ko-śa
abhi-dhar-ma-ko-śa-bhā-ṣya
abhi-dharma-kośa-bhāṣyam
abhyaṃ-karopāhva-vāsu-deva-śāstri-vi-racita-yā
agni-veśa
āhā-ra-vi-hā-ra-pra-kṛ-tiṃ
ahir-budhnya
ahir-budhnya-saṃ-hitā
akusī-dasya
alter-na-tively
amara-bharati
amara-bhāratī
āmla
amlīkā
ānan-da-rā-ya
anna-mardanādi-bhiś
anu-bhav-ād
anu-bhū-ta-viṣayā-sam-pra-moṣa
anu-bhū-ta-viṣayā-sam-pra-moṣaḥ
anu-māna
anu-miti-mānasa-vāda
ariya-pary-esana-sutta
ārogya-śālā-karaṇā-sam-arthas
ārogya-śālām
ārogyāyopa-kal-pya
arś-āṃ-si
ar-tha
ar-thaḥ
ārya-bhaṭa
ārya-lalita-vistara-nāma-mahā-yāna-sūtra
ārya-mañju-śrī-mūla-kalpa
ārya-mañju-śrī-mūla-kalpaḥ
asaṃ-pra-moṣa
āsana
āsanam
āsanaṃ
asid-dhe
aṣṭāṅga-hṛdaya
aṣṭāṅga-hṛdaya-saṃ-hitā
aṣṭ-āṅga-saṅ-graha
aṣṭ-āṅgā-yur-veda
aśva-gan-dha-kalpa
aśva-ghoṣa
ātaṅka-darpaṇa
ātaṅka-darpaṇa-vyā-khyā-yā
atha-vā
ātu-r-ā-hā-ra-vi-hā-ra-pra-kṛ-tiṃ
aty-al-pam
auṣa-dha-pāvanādi-śālāś
ava-sāda-na
avic-chin-na-sam-pra-dāya-tvād
āyur-veda
āyur-veda-sāra
āyur-vedod-dhāra-ka-vaid-ya-pañc-ānana-vaid-ya-rat-na-rāja-vaid-ya-paṇḍi-ta-rā-ma-pra-sāda-vaid-yo-pādhyā-ya-vi-ra-ci-tā
bahir-deśa-ka
bala-bhadra
bāla-kṛṣṇa
bau-dhā-yana-dhar-ma-sūtra
bhadrā-sana
bhadrā-sanam
bha-ga-vad-gī-tā
bha-ga-vat-pāda
bhaṭṭot-pala-vi-vṛti-sahitā
bhṛtyāva-satha-saṃ-yuktām
bhū-miṃ
bhu--va-na-dī-pa-ka
bīja-pallava
bodhi-sat-tva-bhūmi
brāhmaṇa-pra-mukha-nānā-sat-tva-vyā-dhi-śānty-ar-tham
brāhmaṇa-pra-mukha-nānā-sat-tve-bhyo
brahmāṇḍa-mahā-purā-ṇa
brahmāṇḍa-mahā-purā-ṇam
brāhma-sphu-ṭa-siddhānta
brahma-vi-hāra
brahma-vi-hāras
bṛhad-āraṇya-ka
bṛhad-yā-trā
bṛhad-yogi-yājña-valkya-smṛti
bṛhad-yogī-yājña-valkya-smṛti
bṛhaj-jāta-kam
cak-ra-dat-ta
cak-ra-pā-ṇi-datta
cā-luk-ya
caraka-prati-saṃ-s-kṛta
caraka-prati-saṃ-s-kṛte
cara-ka-saṃ-hitā
ca-tur-thī-vi-bhak-ti
cau-kham-ba
cau-luk-yas
chau-kham-bha
chun-nam
cikit-sā-saṅ-gra-ha
daiva-jñālaṃ-kṛti
daiva-jñālaṅ-kṛti
darśa-nāṅkur-ābhi-dhayāvyā-khya-yā
deva-nagari
deva-nāgarī
dhar-ma-megha
dhar-ma-meghaḥ
dhyā-na-grahopa-deśā-dhyā-yaś
dṛṣṭ-ān-ta
dṛṣṭ-ār-tha
dvāra-tvam
evaṃ-gṛ-hī-tam
evaṃ-vi-dh-a-sya
gala-gaṇḍa
gala-gaṇḍādi-kar-tṛ-tvaṃ
gan-dh-ā-ra
gar-bha-śa-rī-ram
gaurī-kāñcali-kā-tan-tra
gauta-mādi-tra-yo-da-śa-smṛty-ātma-kaḥ
gheraṇḍa-saṃ-hitā
gran-tha-śreṇi
gran-tha-śreṇiḥ
guru-maṇḍala-grantha-mālā
hari-śāstrī
hari-śās-trī
haṭha-yoga
hāyana-rat-na
hema-pra-bha-sūri
hetv-ābhā-sa
hīna-mithy-āti-yoga
hīna-mithy-āti-yogena
hindī-vyā-khyā-vi-marśope-taḥ
hoern-le
idam
ijya-rkṣe
ikka-vālaga
ity-arthaḥ
jābāla-darśanopa-ni-ṣad
jal-pa-kal-pa-tāru
jam-bū-dvī-pa
jam-bū-dvī-pa-pra-jña-pti
jam-bū-dvī-pa-pra-jña-pti-sūtra
jāta-ka-kar-ma-pad-dhati
jinā-agama-grantha-mālā
jī-vā-nan-da-nam
jñā-na-nir-mala
jñā-na-nir-malaṃ
jya-rkṣe
kāka-caṇḍīśvara-kal-pa-tan-tra
kā-la-gar-bhā-śa-ya-pra-kṛ-tim
kā-la-gar-bhā-śa-ya-pra-kṛ-tiṃ
kali-kāla-sarva-jña
kali-kāla-sarva-jña-śrī-hema-candrācārya-vi-raci-ta
kali-kāla-sarva-jña-śrī-hema-candrācārya-vi-raci-taḥ
kali-yuga
kal-pa-sthāna
kar-ma
kar-man
kārt-snyena
katham
kāvya-mālā
keśa-va-śāstrī
kol-ka-ta
kṛṣṇa-pakṣa
kṛtti-kā
kṛtti-kās
kula-pañji-kā
ku-māra-saṃ-bhava
lab-dhāni
mada-na-phalam
mādha-va
Mādhava-karaaita-reya-brāhma-ṇa
Mādhava-ni-dāna
mādhava-ni-dā-nam
madhu-kośa
madhu-kośākhya-vyā-khya-yā
madhya
madhye
ma-hā-bhū-ta-vi-kā-ra-pra-kṛ-tiṃ
mahā-deva
mahā-mati-śrī-mādhava-kara-pra-ṇī-taṃ
mahā-muni-śrī-mad-vyāsa-pra-ṇī-ta
mahā-muni-śrī-mad-vyāsa-pra-ṇī-taṃ
maha-rṣi-pra-ṇīta-dharma-śāstra-saṃ-grahaḥ
mahā-sacca-ka-sutta
mahau-ṣadhi-pari-cchadāṃ
mahā-vra-ta
mahā-yāna-sūtrālaṅ-kāra
mano-ratha-nandin
matsya-purāṇam
me-dhā-ti-thi
medhā-tithi
mithilā-stha
mithilā-stham
mithilā-sthaṃ
mud-rā-yantr-ā-laye
muktā-pīḍa
mūla-pāṭha
nakṣa-tra
nandi-purāṇoktārogya-śālā-dāna-phala-prāpti-kāmo
nara-siṃha
nara-siṃha-bhāṣya
nārā-ya-ṇa-dāsa
nārā-yaṇa-kaṇṭha
nārā-yaṇa-paṇḍi-ta-kṛtā
nava-pañca-mayor
nidā-na-sthā-na-sya
ni-ghaṇ-ṭu
nir-anta-ra-pa-da-vyā-khyā
nir-ṇaya-sā-gara
nir-ṇaya-sā-gara-yantr-ā-laye
nirūha-vasti
niś-cala-kara
ni-yukta-vaidyāṃ
nya-grodha
nya-grodho
nyāya-śās-tra
nyāya-sū-tra-śaṃ-kar
okaḥ-sātmya
okaḥ-sātmyam
okaḥ-sātmyaṃ
oka-sātmya
oka-sātmyam
oka-sātmyaṃ
oṣṭha-saṃ-puṭa
ousha-da-sala
padma-pra-bha-sūri
padma-sva-sti-kārdha-candrādike
paitā-maha-siddhā-nta
pañca-karma
pañca-karma-bhava-rogāḥ
pañca-karmādhi-kāra
pañca-karma-vi-cāra
pāñca-rātrā-gama
pañca-siddh-āntikā
pari-bhāṣā
pari-likh-ya
pātañ-jala-yoga-śās-tra
pātañ-jala-yoga-śās-tra-vi-varaṇa
pat-añ-jali
pāṭī-gaṇita
pāva-suya
pim-pal-gaon
pipal-gaon
pit-ta-kṛt
pit-ta-śleṣma-ghna
pit-ta-śleṣma-medo-meha-hik-kā-śvā-sa-kā-sāti-sā-ra-cchardi-tṛṣṇā-kṛmi-vi-ṣa-pra-śa-ma-naṃ
prā-cya
prā-cya-hindu-gran-tha-śreṇiḥ
prācya-vidyā-saṃ-śodhana-mandira
pra-dhān-āṅ-gaṃ
pra-dhān-in
pra-ka-shan
pra-kṛ-ti
pra-kṛ-tiṃ
pra-mā-ṇa-vārt-tika
pra-saṅ-khyāne
pra-śas-ta-pāda-bhāṣya
pra-śna-pra-dīpa
pra-śnārṇa-va-plava
praśnārṇava-plava
pra-śna-vaiṣṇava
pra-śna-vai-ṣṇava
prati-padyate
pra-ty-akṣa
pra-yat-na-śai-thilyā-nan-ta-sam-ā-pat-ti-bhyām
pra-yat-na-śai-thilyā-nān-tya-sam-ā-pat-ti-bhyāṃ
pra-yatna-śai-thilya-sya
puṇya-pattana
pūrṇi-mā-nta
rāja-kīya
rajjv-ābhyas-ya
rāma-kṛṣṇa
rasa-ratnā-kara
rasa-vai-śeṣika-sūtra
rogi-svasthī-karaṇānu-ṣṭhāna-mātraṃ
rūkṣa-vasti
sād-guṇya
śākalya-saṃ-hitā
sam-ā-mnāya
sāmañña-pha-la-sutta
sama-ran-gana-su-tra-dhara
samā-raṅga-ṇa-sū-tra-dhāra
sama-ra-siṃ-ha
sama-ra-siṃ-haḥ
saṃ-hitā
sāṃ-sid-dhi-ka
saṃ-śo-dhana
sam-ul-lasi-tam
śāndilyopa-ni-ṣad
śaṅ-kara
śaṅ-kara-bha-ga-vat-pāda
Śaṅ-kara-nārā-yaṇa
saṅ-khyā
sāṅ-kṛt-yā-yana
san-s-krit
sap-tame
śāra-dā-tila-ka-tan-tra
śa-raṅ-ga-deva
śār-dūla-karṇā-va-dāna
śā-rī-ra
śā-rī-ra-sthāna
śārṅga-dhara
śārṅga-dhara-saṃ-hitā
sar-va
sarva-darśana-saṃ-grahaḥ
sar-va-dar-śāna-saṅ-gra-ha
sar-va-dar-śāna-saṅ-gra-haḥ
sarv-arthāvi-veka-khyā-ter
sar-va-tan-tra-sid-dhān-ta
sar-va-tan-tra-sid-dhān-taḥ
sarva-yoga-sam-uc-caya
sar-va-yogeśvareśva-ram
śāstrā-rambha-sam-artha-na
śāstrāram-bha-sam-arthana
ṣaṭ-pañcā-śi-kā
sat-tva
saunda-ra-na-nda
sid-dha
sid-dha-man-tra
sid-dha-man-trā-hvayo
sid-dha-man-tra-pra-kāśa
sid-dha-man-tra-pra-kāśaḥ
sid-dha-man-tra-pra-kāśaś
sid-dh-ān-ta
siddhānta-śiro-maṇ
sid-dha-yoga
sid-dhi-sthāna
śi-va-śar-ma-ṇā
ska-nda-pu-rā-ṇa
sneha-basty-upa-deśāt
sodā-haraṇa-saṃ-s-kṛta-vyā-khyayā
śodha-ka-pusta-kaṃ
śo-dha-na-ci-kitsā
so-ma-val-ka
śrī-mad-devī-bhāga-vata-mahā-purāṇa
srag-dharā-tārā-sto-tra
śrī-hari-kṛṣṇa-ni-bandha-bhava-nam
śrī-hema-candrā-cārya-vi-raci-taḥ
śrī-kaṇtha-datta
śrī-kaṇtha-dattā-bhyāṃ
śrī-kṛṣṇa-dāsa
śrī-mad-amara-siṃha-vi-racitam
śrī-mad-aruṇa-dat-ta-vi-ra-ci-tayā
śrī-mad-bhaṭṭot-pala-kṛta-saṃ-s-kṛta-ṭīkā-sahitam
śrī-mad-dvai-pā-yana-muni-pra-ṇītaṃ
śrī-mad-vāg-bha-ṭa-vi-ra-ci-tam
śrī-maṃ-trī-vi-jaya-siṃha-suta-maṃ-trī-teja-siṃhena
śrī-mat-kalyāṇa-varma-vi-racitā
śrīmat-sāyaṇa-mādhavācārya-pra-ṇītaḥ
śrī-vā-cas-pati-vaidya-vi-racita-yā
śrī-vatsa
śrī-vi-jaya-rakṣi-ta
sthānāṅga-sūtra
sthira-sukha
sthira-sukham
strī-niṣevaṇa
śukla-pakṣa
su-śru-ta-saṃ-hitā
sū-tra
sūtrārthānān-upa-patti-sūca-nāt
sūtra-sthāna
su-varṇa-pra-bhāsot-tama-sū-tra
svalpauṣadha-dāna-mā-tram
śvetāśva-taropa-ni-ṣad
tad-upa-karaṇa-tāmra-kaṭāha-kalasādi-pātra-pari-cchada-nānā-vidha-vyādhi-śānty-ucitauṣadha-gaṇa-yathokta-lakṣaṇa-vaidya-nānā-vidha-pari-cāraka-yutāṃ
tājaka-muktā-valeḥ
tājika-kau-stu-bha
tājika-nīla-kaṇṭhī
tājika-yoga-sudhā-ni-dhi
tāmra-paṭṭādi-li-khi-tāṃ
tan-nir-vāhāya
tapo-dhana
tapo-dhanā
tārā-bhakti-su-dhārṇava
tārtīya-yoga-su-sudhā-ni-dhi
tegi-ccha
te-jaḥ-siṃ-ha
trai-lok-ya
trai-lokya-pra-kāśa
tri-piṭa-ka
tri-var-gaḥ
un-mār-ga-gama-na
upa-de-śa
upa-patt-ti
ut-sneha-na
utta-rā-dhyā-ya-na
uttara-sthāna
uttara-tantra
vāchas-pati
vād-ā-valī
vai-śā-kha
vai-ta-raṇa-vasti
vai-ta-raṇok-ta-guṇa-gaṇa-yu-k-taṃ
vājī-kara-ṇam
vāk-patis
vākya-śeṣa
vākya-śeṣaḥ
varā-ha-mihi-ra
va-ra-na-si
vā-rā-ṇa-sī
var-mam
var-man
varṇa-sam-ā-mnāya
va-siṣṭha-saṃ-hitā
vā-siṣṭha-saṃ-hitā
vasu-bandhu
vasu-bandhu
vāta-ghna-pit-talāl-pa-ka-pha
vātsyā-ya-na
vidya-bhu-sana
vidyā-bhū-ṣaṇa
vi-jaya-siṃ-ha
vi-jñāna-bhikṣu
vi-kal-pa
vi-kamp-i-tum
vi-mā-na-sthāna
vi-racita-yā
vishveshvar-anand
vi-śiṣṭ-āṃśena
viṣṇu-dharmot-tara-purāṇa
viśrāma-gṛha-sahitā
vi-suddhi-magga
vopa-de-vīya-sid-dha-man-tra-pra-kāśe
vyādhi-pratī-kārār-tham
vyāḍī-ya-pa-ri-bhā-ṣā-vṛtti
vyati-krāmati
vy-ava-haranti
yādava-bhaṭṭa
yāda-va-śarma-ṇā
yādava-sūri
yājña-valkya-smṛti
yavanā-cā-rya
yoga-ratnā-kara
yoga-sāra-sam-uc-caya
yoga-sāra-sam-uc-cayaḥ
yoga-sūtra-vi-vara-ṇa
yoga-yājña-valkya
yoga-yājña-valkya-gītāsūpa-ni-ṣatsu
yoga-yājña-valkyaḥ
yogi-yājña-valkya-smṛti
yuk-tiḥ
yuk-tis
}}
\normalfontlatin
\endinput
}% should work, but doesn't
% special hyphenations for Sanskrit words tagged in
% Polyglossia.
% *English,\textenglish{},text,and
% *Sanskrit,\textsanskrit{},text.
%
% English (see below for \textsanskrit)
%
\hyphenation{%
    dhanva-ntariṇopa-diṣ-ṭaḥ
    suśruta-nāma-dheyena
    tac-chiṣyeṇa
    kāśyapa-saṃ-hitā
    cikitsā-sthāna
    su-śruta-san-dīpana-bhāṣya
    dṛṣṭi-maṇḍala
    uc-chiṅga-na
    sarva-siddhānta-tattva-cūḍā-maṇi
    tulya-sau-vīrāñja-na
    indra-gopa
    śrī-mad-abhi-nava-guptā-cārya-vi-ra-cita-vi-vṛti-same-tam
    viśva-nātha
śrī-mad-devī-bhāga-vata-mahā-purāṇa
    siddhā-n-ta-sun-dara
    brāhma-sphuṭa-siddh-ānta
    bhū-ta-saṅ-khyā
    bhū-ta-saṃ-khyā
    kathi-ta-pada
    devī-bhā-ga-vata-purāṇa
    devī-bhā-ga-vata-mahā-purāṇa
    Siddhānta-saṃ-hitā-sāra-sam-uc-caya
    sau-ra-pau-rāṇi-ka-mata-sam-artha-na
    Pṛthū-da-ka-svā-min
    Brah-ma-gupta
    Brāh-ma-sphu-ṭa-siddhānta
    siddhānta-sun-dara
    vāsa-nā-bhāṣya
    catur-veda
    bhū-maṇḍala
    jñāna-rāja
    graha-gaṇi-ta-cintā-maṇi
    Śiṣya-dhī-vṛd-dhi-da-tan-tra
    brah-māṇḍa-pu-rā-ṇa
    kūr-ma-pu-rā-ṇa
    jam-bū-dvī-pa
    bhā-ga-vata-pu-rā-ṇa
    kupya-ka
    nandi-suttam
    nandi-sutta
    su-bodhiā-bāī
    asaṅ-khyāta
    saṅ-khyāta
    saṅ-khyā-pra-māṇa
    saṃ-khā-pamāṇa
    nemi-chandra
    anu-yoga-dvāra
    tattvārtha-vārtika
    aka-laṅka
    tri-loka-sāra
    gaṇi-ma-pra-māṇa
    gaṇi-ma-ppa-māṇa
    eka-pra-bhṛti
gaṇaṇā-saṃ-khā
gaṇaṇā-saṅ-khyā
dvi-pra-bhṛti
duppa-bhi-ti-saṃ-khā
vedanābhi-ghāta
Viṣṇu-dharmottara-pu-rāṇa
abhaya-deva-sūri-vi-racita-vṛtti-vi-bhūṣi-tam
abhi-dhar-ma
abhi-dhar-ma-ko-śa
abhi-dhar-ma-ko-śa-bhā-ṣya
abhi-dharma-kośa-bhāṣya
abhi-dharma-kośa-bhāṣyam
abhi-nava
abhyaṃ-karopāhva-vāsu-deva-śāstri-vi-ra-ci-ta-yā
ācārya-śrī-jina-vijayālekhitāgra-vacanālaṃ-kṛtaś-ca
ācāry-opā-hvena
ādhāra
adhi-kāra
adhi-kāras
ādi-nātha
agni-besha
agni-veśa
ahir-budhnya
ahir-budhnya-saṃ-hitā
aita-reya-brāhma-ṇa
akusī-dasya
amara-bharati
Amar-augha-pra-bo-dha
amṛ-ta-siddhi
ānanda-kanda
ānan-da-rā-ya
ānand-āśra-ma-mudraṇā-la-ya
ānand-āśra-ma-saṃ-skṛta-granth-āva-liḥ
anna-pāna-mūlā
anu-ban-dhya-lakṣaṇa-sam-anv-itās
anu-bhav-ād
anu-bhū-ta-viṣayā-sam-pra-moṣa
anu-bhū-ta-viṣayā-sam-pra-moṣaḥ
aparo-kṣā-nu-bhū-ti
app-proxi-mate-ly
ardha-rātrika-karaṇa
ārdha-rātrika-karaṇa
ariya-pary-esana-sutta
arun-dhatī
ārya-bhaṭa
ārya-bhaṭā-cārya-vi-racitam
ārya-bhaṭīya
ārya-bhaṭīyaṃ
ārya-lalita-vistara-nāma-mahā-yāna-sūtra
ārya-mañju-śrī-mūla-kalpa
ārya-mañju-śrī-mūla-kalpaḥ
asaṃ-pra-moṣa
aṣṭāṅga-hṛdaya-saṃ-hitā
aṣṭāṅga-saṃ-graha
asura-bhavana
aśva-ghoṣa
ātaṅka-darpaṇa-vyā-khyā-yā
atha-vā
ava-sāda-na
āyār-aṅga-suttaṃ
ayur-ved
ayur-veda
āyur-veda
āyur-veda-dīpikā
āyur-veda-dīpikā-vyā-khyayā
āyur-ve-da-ra-sā-yana
āyur-veda-sū-tra
ayur-vedic
āyur-vedic
ayur-yog
bādhirya
bahir-deśa-ka
bala-bhadra
bala-kot
bala-krishnan
bāla-kṛṣṇa
bau-dhā-yana-dhar-ma-sūtra
bel-valkar
bhadra-kālī-man-tra-vi-dhi-pra-karaṇa
bhadrā-sana
bhadrā-sanam
bha-ga-vat-pāda
bhaiṣajya-ratnāvalī
bhan-d-ar-kar
bhartṛhari-viracitaḥ
bhaṭṭā-cārya
bhaṭṭot-pala-vi-vṛti-sahitā
Bhiṣag-varāḍha-malla-vi-racita-dīpikā-Kāśī-rāma-vaidya-vi-raci-ta-gūḍhā-rtha-dīpikā-bhyāṃ
bhiṣag-varāḍha-malla-vi-racita-dīpikā-Kāśī-rāma-vaidya-vi-racita-gūḍhārtha-dīpikā-bhyāṃ
bhoja-deva-vi-raci-ta-rāja-mārtaṇḍā-bhi-dha-vṛtti-sam-e-tāni
bhu--va-na-dī-pa-ka
bīja-pallava
bi-kaner
bodhi-sat-tva-bhūmi
brahma-gupta
brahmā-nanda
brahmāṇḍa-mahā-purā-ṇa
brahmāṇḍa-mahā-purā-ṇam
brahma-randhra
brahma-siddh-ānta
brāhma-sphuṭa-siddh-ānta
brāhma-sphu-ṭa-siddhānta
brahma-vi-hāra
brahma-vi-hāras
brahma-yā-mala-tan-tra
Bra-ja-bhāṣā
bṛhad-āraṇya-ka
bṛhad-yā-trā
bṛhad-yogi-yājña-valkya-smṛti
bṛhad-yogī-yājña-valkya-smṛti
bṛhaj-jāta-kam
bṛhat-khe-carī-pra-kāśa
buddhi-tattva-pra-karaṇa
cak-ra-dat-ta
cakra-datta
cakra-pāṇi-datta
cā-luk-ya
caraka-prati-saṃ-s-kṛta
caraka-prati-saṃ-s-kṛte
caraka-saṃ-hitā
casam-ul-lasi-tāmaharṣiṇāsu-śrutenavi-raci-tāsu-śruta-saṃ-hitā
cau-kham-ba
cau-luk-yas
chandi-garh
chara-ka
cha-rīre
chatt-opa-dh-ya-ya
chau-kham-bha
chi-ki-tsi-ta
cid-ghanā-nanda-nātha
ci-ka-ner
com-men-taries
com-men-tary
com-pre-hen-sive-ly
daiva-jñālaṃ-kṛti
daiva-jñālaṅ-kṛti
dāmo-dara-sūnu-Śārṅga-dharācārya-vi-racitā
Dāmodara-sūnu-Śārṅga-dharācārya-vi-racitā
darśanā-ṅkur-ābhi-dhayā
das-gupta
deha-madhya
deha-saṃ-bhava-hetavaḥ
deva-datta
deva-nagari
deva-nāgarī
devā-sura-siddha-gaṇaiḥ
dha-ra-ni-dhar
dharma-megha
dharma-meghaḥ
dhru-vam
dhru-va-sya
dhru-va-yonir
dhyā-na-grahopa-deśā-dhyā-yaś
dṛḍha-śūla-yukta-rakta
dvy-ulbaṇaikolba-ṇ-aiḥ
four-fold
gan-dh-ā-ra
gārgīya-jyoti-ṣa
gārgya-ke-rala-nīla-kaṇṭha-so-ma-sutva-vi-racita-bhāṣyo-pe-tam
garuḍa-mahā-purāṇa
gaurī-kāñcali-kā-tan-tra
gau-tama
gauta-mādi-tra-yo-da-śa-smṛty-ātma-kaḥ
gheraṇḍa-saṃ-hitā
gorakṣa-śata-ka
go-tama
granth-ā-laya
grantha-mālā
gran-tha-śreṇiḥ
grāsa-pramāṇa
guru-maṇḍala-grantha-mālā
gyatso
hari-śāstrī
haṭhābhyāsa-paddhati
haṭha-ratnā-valī
Haṭha-saṅ-keta-candri-kā
haṭha-tattva-kau-mudī
haṭha-yoga
hāyana-rat-na
haya-ta-gran-tha
hema-pra-bha-sūri
hetu-lakṣaṇa-saṃ-sargād
hīna-madhyādhi-kaiś
hindī-vyā-khyā-vi-marśope-taḥ
hoern-le
ijya-rkṣa
ikka-vālaga
indra-dhvaja
indrāṇī-kalpa
indria
Īśāna-śiva-guru-deva-pad-dhati
jābāla-darśanopa-ni-ṣad
jadav-ji
jagan-nā-tha
jala-basti
jal-pa-kal-pa-tāru
jam-bū-dvī-pa-pra-jña-pti
jam-bū-dvī-pa-pra-jña-pti-sūtra
jana-pad-a-sya
jāta-ka-kar-ma-pad-dhati
jaya-siṃha
jinā-agama-grantha-mālā
jin-en-dra-bud-dhi
jīvan-muk-ti-vi-veka
jñā-na-nir-mala
jñā-na-nir-malaṃ
joga-pra-dīpya-kā
jya-rkṣe
Jyo-tiḥ-śās-tra
jyo-ti-ṣa-rāya
jyoti-ṣa-rāya
jyotiṣa-siddhānta-saṃ-graha
jyotiṣa-siddhānta-saṅ-graha
kāka-caṇḍīśvara-kal-pa-tan-tra
kakṣa-puṭa
kali-kāla-sarva-jña
kali-kāla-sarva-jña-śrī-hema-candrācārya-vi-raci-ta
kali-kāla-sarva-jña-śrī-hema-candrācārya-vi-raci-taḥ
kali-yuga
kal-pa
kal-pa-sthāna
kalyāṇa-kāraka
Kāmeśva-ra-siṃha-dara-bhaṅgā-saṃ-skṛta-viśva-vidyā-layaḥ
kapāla-bhāti
karaṇa-tilaka
kar-ma
kar-man
kāṭhaka-saṃ-hitā
kavia-rasu
kavi-raj
keśa-va-śāstrī
ke-vala--rāma
keva-la-rāma
khaṇḍa-khādyaka-tappā
khe-carī-vidyā
knowl-edge
kol-ka-ta
kriyā-krama-karī
kṛṣṇa-pakṣa
kṛtti-kā
kṛtti-kās
kubji-kā-mata-tantra
kula-pañji-kā
kul-karni
ku-māra-saṃ-bhava
kuṭi-pra-veśa
kuṭi-pra-veśika
lakṣ-mī-veṅ-kaṭ-e-ś-va-ra
lit-era-ture
lit-era-tures
locana-roga
mādha-va
mādhava-kara
mādhava-ni-dāna
mādhava-ni-dā-nam
madh-ūni
madhya
mādhyan-dina
madhye
mahā-bhāra-ta
mahā-deva
mahā-kavi-bhartṛ-hari-praṇīta-tvena
maha-mahopa-dhyaya
mahā-maho-pā-dhyā-ya-śrī-vi-jñā-na-bhikṣu-vi-raci-taṃ
mahā-mati-śrī-mādhava-kara-pra-ṇī-taṃ
mahā-mudrā
mahā-muni-śrī-mad-vyāsa-pra-ṇī-ta
mahā-muni-śrī-mad-vyāsa-pra-ṇī-taṃ
maharṣiṇā
maha-rṣi-pra-ṇīta-dharma-śāstra-saṃ-grahaḥ
Maha-rṣi-varya-śrī-yogi-yā-jña-valkya-śiṣya-vi-racitā
mahā-sacca-ka-sutta
mahā-sati-paṭṭhā-na-sutta
mahā-vra-ta
mahā-yāna-sūtrālaṅ-kāra
maitrāya-ṇī-saṃ-hitā
maktab-khānas
māla-jit
māli-nī-vijayot-tara-tan-tra
manaḥ-sam-ā-dhi
mānasol-lāsa
mānava-dharma-śāstra
mandāgni-doṣa
mannar-guḍi
mano-har-lal
mano-ratha-nandin
man-u-script
man-u-scripts
mataṅga-pārame-śvara
mater-ials
matsya-purāṇam
medh-ā-ti-thi
medhā-tithi
mithilā-stha
mithilā-stham
mithilā-sthaṃ
mṛgendra-tantra-vṛtti
mud-rā-yantr-ā-laye
muktā-pīḍa
mūla-pāṭha
muṇḍī-kalpa
mun-sh-ram
Nāda-bindū-pa-ni-ṣat
nāga-bodhi
nāga-buddhi
nakṣa-tra
nara-siṃha
nārā-yaṇa-dāsa
nārā-yaṇa-dāsa
nārā-yaṇa-kaṇṭha
nārā-yaṇa-paṇḍi-ta-kṛtā
nar-ra-tive
nata-rajan
nava-pañca-mayor
nava-re
naya-na-sukho-pā--dhyāya
ni-ban-dha-saṃ-grahā-khya-vyākhya-yā
niban-dha-san-graha
ni-dā-na
nidā-na-sthā-na-sya
ni-dāna-sthānasyaśrī-gaya-dāsācārya-vi-racitayānyāya-candri-kā-khya-pañjikā-vyā-khyayā
nir-anta-ra-pa-da-vyā-khyā
nir-guṇḍī-kalpa
nir-ṇaya-sā-gara
Nir-ṇaya-sāgara
nir-ṇa-ya-sā-gara-mudrā-yantrā-laye
nir-ṇa-ya-sā-ga-ra-yantr-āla-ya
nir-ṇaya-sā-gara-yantr-ā-laye
niśvāsa-kārikā
nīti-śṛṅgāra-vai-rāgyādi-nāmnāsamākhyā-tānāṃ
nityā-nanda
nya-grodha
nya-grodho
nyā-ya-candri-kā-khya-pañji-kā-vyā-khya-yā
nyāya-śās-tra
okaḥ-sātmya
okaḥ-sātmyam
okaḥ-sātmyaṃ
oka-sātmya
oka-sātmyam
oka-sātmyaṃ
oris-sa
oṣṭha-saṃ-puṭa
ousha-da-sala
padma-pra-bha-sūri
Padma-prā-bhṛ-ta-ka
padma-sva-sti-kārdha-candrādike
paitā-maha-siddhā-nta
pañca-karma
pañca-karman
pāñca-rātrā-gama
pañca-siddh-āntikā
paṅkti-śūla
Paraśu-rāma
paraśu-rāma
pari-likh-ya
pāśu-pata-sū-tra-bhāṣya
pātañ-jala-yoga-śās-tra
pātañ-jala-yoga-śās-tra-vi-varaṇa
pat-añ-jali
pat-na
pāva-suya
phiraṅgi-can-dra-cchedyo-pa-yogi-ka
pim-pal-gaon
pipal-gaon
pitta-śleṣ-man
pit-ta-śleṣ-ma-śoṇi-ta
pitta-śoṇi-ta
prā-cīna-rasa-granthaḥ
prā-cya
prā-cya-hindu-gran-tha-śreṇiḥ
prācya-vidyā-saṃ-śodhana-mandira
pra-dhān-in
pra-ka-shan
pra-kaṭa-mūṣā
pra-kṛ-ti-bhū-tāḥ
pra-mā-ṇa-vārt-tika
pra-ṇītā
pra-saṅ-khyāne
pra-śas-ta-pāda-bhāṣya
pra-śna-pra-dīpa
pra-śnārṇa-va-plava
praśnārṇava-plava
pra-śna-vai-ṣṇava
pra-śna-vaiṣṇava
prati-padyate
pra-yatna-śaithilyānan-ta-sam-āpatti-bhyām
prei-sen-danz
punar-vashu
puṇya-pattana
pūrṇi-mā-nta
raghu-nātha
rāja-kīya
rāja-kīya-mudraṇa-yantrā-laya
rāja-śe-khara
rajjv-ābhyas-ya
raj-put
rāj-put
rakta-mokṣa-na
rāma-candra-śāstrī
rāma-kṛṣṇa
rāma-kṛṣṇa-śāstri-ṇā
rama-su-bra-manian
rāmā-yaṇa
rasa-ratnā-kara
rasa-ratnākarāntar-ga-taś
rasa-ratna-sam-uc-caya
rasa-ratna-sam-uc-ca-yaḥ
rasa-vīry-auṣa-dha-pra-bhāvena
rasā-yana
rasendra-maṅgala
rasendra-maṅgalam
rāṣṭra-kūṭa
rāṣṭra-kūṭas
sādhana
śākalya-saṃ-hitā
śāla-grāma-kṛta
śāla-grāma-kṛta
sāmañña-pha-la-sutta
sāmañña-phala-sutta
sama-ran-gana-su-tra-dhara
samā-raṅga-ṇa-sū-tra-dhāra
sama-ra-siṃ-ha
sama-ra-siṃ-haḥ
sāmba-śiva-śāstri
same-taḥ
saṃ-hitā
śāṃ-ka-ra-bhāṣ-ya-sam-etā
sam-rāṭ
saṃ-rāṭ
Sam-rāṭ-siddhānta
Sam-rāṭ-siddhānta-kau-stu-bha
sam-rāṭ-siddhānta-kau-stu-bha
saṃ-sargam
saṃ-sargaṃ
saṃ-s-kṛta
saṃ-s-kṛta-pārasī-ka-pra-da-pra-kāśa
saṃ-śo-dhana
saṃ-śodhitā
saṃ-sthāna
sam-ullasitā
sam-ul-lasi-tam
saṃ-valitā
saṃ-valitā
śāndilyopa-ni-ṣad
śaṅ-kara
śaṅ-kara-bha-ga-vat-pāda
śaṅ-karā-cārya
san-kara-charya
Śaṅ-kara-nārā-yaṇa
sāṅ-kṛt-yā-yana
san-s-krit
śāra-dā-tila-ka-tan-tra
śa-raṅ-ga-deva
śār-dūla-karṇā-va-dāna
śār-dūla-karṇā-va-dāna
śā-rī-ra-sthāna
śārṅga-dhara-saṃ-hitā
Śārṅga-dhara-saṃ-hitā
sar-va-dar-śana-saṅ-gra-ha
sarva-kapha-ja
sarv-arthāvi-veka-khyā-ter
sar-va-śa-rīra-carās
sarva-siddhānta-rāja
Sarva-siddhā-nt-rāja
sarva-vyā-dhi-viṣāpa-ha
sarva-yoga-sam-uc-caya
sar-va-yogeśvareśva-ram
śāstrā-rambha-sam-artha-na
śatakatrayādi-subhāṣitasaṃgrahaḥ
sati-paṭṭhā-na-sutta
ṣaṭ-karma
ṣaṭ-karman
sat-karma-saṅ-graha
sat-karma-saṅ-grahaḥ
ṣaṭ-pañcā-śi-kā
saun-da-ra-nanda
sa-v-āī
schef-tel-o-witz
scholars
sharī-ra
sheth
sid-dha-man-tra
siddha-nanda-na-miśra
siddha-nanda-na-miśraḥ
siddha-nitya-nātha-pra-ṇītaḥ
Siddhānta-saṃ-hitā-sāra-sam-uc-caya
Siddhā-nta-sār-va-bhauma
siddhānta-sindhu
siddhānta-śiro-maṇ
Siddhānta-śiro-maṇi
Siddhā-nta-tat-tva-vi-veka
sid-dha-yoga
siddha-yoga
sid-dhi
sid-dhi-sthā-na
sid-dhi-sthāna
śikhi-sthāna
śiraḥ-karṇā-kṣi-vedana
śiro-bhūṣaṇam
Śivā-nanda-saras-vatī
śiva-saṃ-hitā
śiva-yo-ga-dī-pi-kā
ska-nda-pu-rā-ṇa
śleṣ-man
śleṣ-ma-śoni-ta
sodā-haraṇa-saṃ-s-kṛta-vyā-khyayā
śodha-ka-pusta-kaa
śoṇi-ta
spaṣ-ṭa-krānty-ādhi-kāra
śrī-cakra-pāṇi-datta
śrī-cakra-pāṇi-datta-viracitayā
śrī-ḍalhaṇācārya-vi-raci-tayāni-bandha-saṃ-grahākhya-vyā-khyayā
śrī-dayā-nanda
śrī-hari-kṛṣṇa-ni-bandha-bhava-nam
śrī-hema-candrā-cārya-vi-raci-taḥ
śrī-kaṇtha-dattā-bhyāṃ
śrī-kṛṣṇa-dāsa
śrī-kṛṣṇa-dāsa-śreṣṭhinā
śrīmac-chaṅ-kara-bhaga-vat-pāda-vi-raci-tā
śrī-mad-amara-siṃha-vi-racitam
śrī-mad-bha-ga-vad-gī-tā
śrī-mad-bhaṭṭot-pala-kṛta-saṃ-s-kṛta-ṭīkā-sahitam
śrī-mad-dvai-pā-yana-muni-pra-ṇītaṃ
śrī-mad-vāg-bhaṭa-vi-raci-tam
śrī-maṃ-trī-vi-jaya-siṃha-suta-maṃ-trī-teja-siṃhena
śrī-mat-kalyāṇa-varma-vi-racitā
śrī-mat-sāyaṇa-mādhavācārya-pra-ṇītaḥsarva-darśana-saṃ-grahaḥ
śrī-nitya-nātha-siddha-vi-raci-taḥ
śrī-rāja-śe-khara
śrī-śaṃ-karā-cārya-vi-raci-tam
śrī-vā-cas-pati-vaidya-vi-racita-yā
śrī-vatsa
śrī-veda-vyāsa-pra-ṇīta-mahā-bhā-ratāntar-ga-tā
śrī-veṅkaṭeś-vara
śrī-vi-jaya-rakṣi-ta
sruta-rakta
sruta-raktasya
stambha-karam
sthānāṅga-sūtra
sthira-sukha
sthira-sukham
stra-sthā-na
subhāṣitānāṃ
su-brah-man-ya
su-bra-man-ya
śukla-pakṣa
śukrā-srava
suk-than-kar
su-pariṣkṛta-saṃgrahaḥ
sura-bhi-pra-kash-an
sūrya-dāsa
sūrya-siddhānta
su-shru-ta
su-śru-ta
su-shru-ta-saṃ-hitā
su-śru-ta-saṃ-hitā
su-śru-tena
sutra
sūtra
sūtra-neti
sūtra-ni-dāna-śā-rīra-ci-ki-tsā-kal-pa-sthānot-tara-tan-trātma-kaḥ
sūtra-sthāna
su-varṇa-pra-bhāsot-tama-sū-tra
Su-var-ṇa-pra-bhās-ot-tama-sū-tra
su-varṇa-pra-bhāsotta-ma-sūtra
su-vistṛta-pari-cayātmikyāṅla-prastāvanā-vividha-pāṭhān-tara-pari-śiṣṭādi-sam-anvitaḥ
sva-bhāva-vyādhi-ni-vāraṇa-vi-śiṣṭ-auṣa-dha-cintakās
svā-bhāvika
svā-bhāvikās
sva-cchanda-tantra
śvetāśva-taropa-ni-ṣad
taila-sarpir-ma-dhūni
tait-tirīya-brāhma-ṇa
tājaka-muktā-valeḥ
tājika-kau-stu-bha
tājika-nīla-kaṇṭhī
tājika-yoga-sudhā-ni-dhi
tapo-dhana
tapo-dhanā
tārā-bhakti-su-dhārṇava
tārtīya-yoga-su-sudhā-ni-dhi
tegi-ccha
te-jaḥ-siṃ-ha
ṭhāṇ-āṅga-sutta
ṭīkā-bhyāṃ
ṭīkā-bhyāṃ
tiru-mantiram
tiru-ttoṇṭar-purāṇam
tiru-va-nanta-puram
trai-lok-ya
trai-lokya-pra-kāśa
tri-bhāga
tri-kam-ji
tri-pita-ka
tri-piṭa-ka
tri-vik-ra-mātma-jena
ud-ā-haraṇa
un-mārga-gama-na
upa-ca-ya-bala-varṇa-pra-sādādī-ni
upa-laghana
upa-ni-ṣads
upa-patt-ti
ut-sneha-na
utta-rā-dhya-ya-na
utta-rā-dhya-ya-na-sūtra
uttara-khaṇḍa-khādyaka
uttara-sthāna
uttara-tantra
vācas-pati-miśra-vi-racita-ṭīkā-saṃ-valita
vācas-pati-miśra-vi-racita-ṭīkā-saṃ-valita-vyā-sa-bhā-ṣya-sam-e-tāni
vag-bhata-rasa-ratna-sam-uc-caya
vāg-bhaṭa-rasa-ratna-sam-uc-caya
vaidya-vara-śrī-ḍalhaṇā-cārya-vi-racitayā
vai-śā-kha
vai-śeṣ-ika-sūtra
vāja-sa-neyi-saṃ-hitā
vājī-kara-ṇam
vākya-śeṣa
vākya-śeṣaḥ
vaṅga-sena
vaṅga-sena-saṃ-hitā
varā-ha-mihi-ra
vārāhī-kalpa
vā-rāṇa-seya
va-ra-na-si
var-mam
var-man
var-ṇa-saṃ-khyā
var-ṇa-saṅ-khyā
vā-si-ṣṭha
vasiṣṭha-saṃ-hitā
vā-siṣṭha-saṃ-hitā
Va-sistha-Sam-hita-Yoga-Kanda-With-Comm-ent-ary-Kai-valya-Dham
vastra-dhauti
vasu-bandhu
vāta-pit-ta
vāta-pit-ta-kapha
vāta-pit-ta-kapha-śoṇi-ta
vāta-pitta-kapha-śoṇita-san-nipāta-vai-ṣamya-ni-mittāḥ
vāta-pit-ta-śoṇi-ta
vāta-śleṣ-man
vāta-śleṣ-ma-śoṇi-ta
vāta-śoṇi-ta
vātā-tapika
vātsyā-ya-na
vāya-vīya-saṃ-hitā
vedāṅga-rāya
veezhi-nathan
venkat-raman
vid-vad-vara-śrī-gaṇeśa-daiva-jña-vi-racita
vidya-bhu-sana
vi-jaya-siṃ-ha
vi-jñāna-bhikṣu
Vijñāneśvara-vi-racita-mitākṣarā-vyā-khyā-sam-alaṅ-kṛtā
vi-mā-na
vi-mā-na-sthāna
vimāna-sthā-na
vi-racitā
vi-racita-yāmadhu-kośākhya-vyā-khya-yā
vi-recana
vishveshvar-anand
vi-śiṣṭ-āṃśena
vi-suddhi-magga
vi-vi-dha-tṛṇa-kāṣṭha-pāṣāṇa-pāṃ-su-loha-loṣṭāsthi-bāla-nakha-pūyā-srāva-duṣṭa-vraṇāntar-garbha-śalyo-ddharaṇārthaṃ
vṛd-dha-vṛd-dha-tara-vṛd-dha-tamaiḥ
vṛddha-vṛddha-tara-vṛddha-tamaiḥ
vṛnda-mādhava
vyāḍī-ya-pa-ri-bhā-ṣā-vṛtti
vyā-khya-yā
vy-akta-liṅgādi-dharma-yuk-te
vyā-sa-bhā-ṣya-sam-e-tāni
vyati-krāmati
Xiuyao
yādava-bhaṭṭa
yāda-va-śarma-ṇā
yādava-sūri
yājña-valkya-smṛti
yājña-valkya-smṛtiḥ
yantrā-dhyāya
Yantra-rāja-vicāra-viṃśā-dhyāyī
yavanā-cā-rya
yoga-bhā-ṣya-vyā-khyā-rūpaṃ
yoga-cintā-maṇi
yoga-cintā-maṇiḥ
yoga-ratnā-kara
yoga-sāra-mañjarī
yoga-sāra-sam-uc-caya
yoga-sāra-saṅ-graha
yoga-śikh-opa-ni-ṣat
yoga-tārā-valī
yoga-yājña-val-kya
yoga-yājña-valkya-gītāsūpa-ni-ṣatsu
yogi-yājña-valkya-smṛti
yoshi-mizu
yukta-bhava-deva
}
%%%%%%%%%%%%%%%%%%%%
%Sanskrit:
%%%%%%%%%%%%%%%%%%%%
\textsanskrit{\hyphenation{%
    dhanva-ntariṇopa-diṣ-ṭaḥ
suśruta-nāma-dheyena
tac-chiṣyeṇa
    su-śruta-san-dīpana-bhāṣya
    cikitsā-sthāna
tulya-sau-vīrāñjana
indra-gopa
dṛṣṭi-maṇḍala
uc-chiṅga-na
vi-vi-dha-tṛṇa-kāṣṭha-pāṣāṇa-pāṃ-su-loha-loṣṭāsthi-bāla-nakha-pūyā-srāva-duṣṭa-vraṇāntar-garbha-śalyo-ddharaṇārthaṃ
śrī-ḍalhaṇācārya-vi-raci-tayāni-bandha-saṃ-grahākhya-vyā-khyayā
ni-dāna-sthānasyaśrī-gaya-dāsācārya-vi-racitayānyāya-candri-kā-khya-pañjikā-vyā-khyayā
casam-ul-lasi-tāmaharṣiṇāsu-śrutenavi-raci-tāsu-śruta-saṃ-hitā
bhartṛhari-viracitaḥ
śatakatrayādi-subhāṣitasaṃgrahaḥ
mahā-kavi-bhartṛ-hari-praṇīta-tvena
nīti-śṛṅgāra-vai-rāgyādi-nāmnāsamākhyā-tānāṃ
subhāṣitānāṃ
su-pariṣkṛta-saṃgrahaḥ
su-vistṛta-pari-cayātmikyāṅla-prastāvanā-vividha-pāṭhān-tara-pari-śiṣṭādi-sam-anvitaḥ
ācārya-śrī-jina-vijayālekhitāgra-vacanālaṃ-kṛtaś-ca
abhaya-deva-sūri-vi-racita-vṛtti-vi-bhūṣi-tam
abhi-dhar-ma
abhi-dhar-ma-ko-śa
abhi-dhar-ma-ko-śa-bhā-ṣya
abhi-dharma-kośa-bhāṣyam
abhyaṃ-karopāhva-vāsu-deva-śāstri-vi-racita-yā
agni-veśa
āhā-ra-vi-hā-ra-pra-kṛ-tiṃ
ahir-budhnya
ahir-budhnya-saṃ-hitā
akusī-dasya
alter-na-tively
amara-bharati
amara-bhāratī
āmla
amlīkā
ānan-da-rā-ya
anna-mardanādi-bhiś
anu-bhav-ād
anu-bhū-ta-viṣayā-sam-pra-moṣa
anu-bhū-ta-viṣayā-sam-pra-moṣaḥ
anu-māna
anu-miti-mānasa-vāda
ariya-pary-esana-sutta
ārogya-śālā-karaṇā-sam-arthas
ārogya-śālām
ārogyāyopa-kal-pya
arś-āṃ-si
ar-tha
ar-thaḥ
ārya-bhaṭa
ārya-lalita-vistara-nāma-mahā-yāna-sūtra
ārya-mañju-śrī-mūla-kalpa
ārya-mañju-śrī-mūla-kalpaḥ
asaṃ-pra-moṣa
āsana
āsanam
āsanaṃ
asid-dhe
aṣṭāṅga-hṛdaya
aṣṭāṅga-hṛdaya-saṃ-hitā
aṣṭ-āṅga-saṅ-graha
aṣṭ-āṅgā-yur-veda
aśva-gan-dha-kalpa
aśva-ghoṣa
ātaṅka-darpaṇa
ātaṅka-darpaṇa-vyā-khyā-yā
atha-vā
ātu-r-ā-hā-ra-vi-hā-ra-pra-kṛ-tiṃ
aty-al-pam
auṣa-dha-pāvanādi-śālāś
ava-sāda-na
avic-chin-na-sam-pra-dāya-tvād
āyur-veda
āyur-veda-sāra
āyur-vedod-dhāra-ka-vaid-ya-pañc-ānana-vaid-ya-rat-na-rāja-vaid-ya-paṇḍi-ta-rā-ma-pra-sāda-vaid-yo-pādhyā-ya-vi-ra-ci-tā
bahir-deśa-ka
bala-bhadra
bāla-kṛṣṇa
bau-dhā-yana-dhar-ma-sūtra
bhadrā-sana
bhadrā-sanam
bha-ga-vad-gī-tā
bha-ga-vat-pāda
bhaṭṭot-pala-vi-vṛti-sahitā
bhṛtyāva-satha-saṃ-yuktām
bhū-miṃ
bhu--va-na-dī-pa-ka
bīja-pallava
bodhi-sat-tva-bhūmi
brāhmaṇa-pra-mukha-nānā-sat-tva-vyā-dhi-śānty-ar-tham
brāhmaṇa-pra-mukha-nānā-sat-tve-bhyo
brahmāṇḍa-mahā-purā-ṇa
brahmāṇḍa-mahā-purā-ṇam
brāhma-sphu-ṭa-siddhānta
brahma-vi-hāra
brahma-vi-hāras
bṛhad-āraṇya-ka
bṛhad-yā-trā
bṛhad-yogi-yājña-valkya-smṛti
bṛhad-yogī-yājña-valkya-smṛti
bṛhaj-jāta-kam
cak-ra-dat-ta
cak-ra-pā-ṇi-datta
cā-luk-ya
caraka-prati-saṃ-s-kṛta
caraka-prati-saṃ-s-kṛte
cara-ka-saṃ-hitā
ca-tur-thī-vi-bhak-ti
cau-kham-ba
cau-luk-yas
chau-kham-bha
chun-nam
cikit-sā-saṅ-gra-ha
daiva-jñālaṃ-kṛti
daiva-jñālaṅ-kṛti
darśa-nāṅkur-ābhi-dhayāvyā-khya-yā
deva-nagari
deva-nāgarī
dhar-ma-megha
dhar-ma-meghaḥ
dhyā-na-grahopa-deśā-dhyā-yaś
dṛṣṭ-ān-ta
dṛṣṭ-ār-tha
dvāra-tvam
evaṃ-gṛ-hī-tam
evaṃ-vi-dh-a-sya
gala-gaṇḍa
gala-gaṇḍādi-kar-tṛ-tvaṃ
gan-dh-ā-ra
gar-bha-śa-rī-ram
gaurī-kāñcali-kā-tan-tra
gauta-mādi-tra-yo-da-śa-smṛty-ātma-kaḥ
gheraṇḍa-saṃ-hitā
gran-tha-śreṇi
gran-tha-śreṇiḥ
guru-maṇḍala-grantha-mālā
hari-śāstrī
hari-śās-trī
haṭha-yoga
hāyana-rat-na
hema-pra-bha-sūri
hetv-ābhā-sa
hīna-mithy-āti-yoga
hīna-mithy-āti-yogena
hindī-vyā-khyā-vi-marśope-taḥ
hoern-le
idam
ijya-rkṣe
ikka-vālaga
ity-arthaḥ
jābāla-darśanopa-ni-ṣad
jal-pa-kal-pa-tāru
jam-bū-dvī-pa
jam-bū-dvī-pa-pra-jña-pti
jam-bū-dvī-pa-pra-jña-pti-sūtra
jāta-ka-kar-ma-pad-dhati
jinā-agama-grantha-mālā
jī-vā-nan-da-nam
jñā-na-nir-mala
jñā-na-nir-malaṃ
jya-rkṣe
kāka-caṇḍīśvara-kal-pa-tan-tra
kā-la-gar-bhā-śa-ya-pra-kṛ-tim
kā-la-gar-bhā-śa-ya-pra-kṛ-tiṃ
kali-kāla-sarva-jña
kali-kāla-sarva-jña-śrī-hema-candrācārya-vi-raci-ta
kali-kāla-sarva-jña-śrī-hema-candrācārya-vi-raci-taḥ
kali-yuga
kal-pa-sthāna
kar-ma
kar-man
kārt-snyena
katham
kāvya-mālā
keśa-va-śāstrī
kol-ka-ta
kṛṣṇa-pakṣa
kṛtti-kā
kṛtti-kās
kula-pañji-kā
ku-māra-saṃ-bhava
lab-dhāni
mada-na-phalam
mādha-va
Mādhava-karaaita-reya-brāhma-ṇa
Mādhava-ni-dāna
mādhava-ni-dā-nam
madhu-kośa
madhu-kośākhya-vyā-khya-yā
madhya
madhye
ma-hā-bhū-ta-vi-kā-ra-pra-kṛ-tiṃ
mahā-deva
mahā-mati-śrī-mādhava-kara-pra-ṇī-taṃ
mahā-muni-śrī-mad-vyāsa-pra-ṇī-ta
mahā-muni-śrī-mad-vyāsa-pra-ṇī-taṃ
maha-rṣi-pra-ṇīta-dharma-śāstra-saṃ-grahaḥ
mahā-sacca-ka-sutta
mahau-ṣadhi-pari-cchadāṃ
mahā-vra-ta
mahā-yāna-sūtrālaṅ-kāra
mano-ratha-nandin
matsya-purāṇam
me-dhā-ti-thi
medhā-tithi
mithilā-stha
mithilā-stham
mithilā-sthaṃ
mud-rā-yantr-ā-laye
muktā-pīḍa
mūla-pāṭha
nakṣa-tra
nandi-purāṇoktārogya-śālā-dāna-phala-prāpti-kāmo
nara-siṃha
nara-siṃha-bhāṣya
nārā-ya-ṇa-dāsa
nārā-yaṇa-kaṇṭha
nārā-yaṇa-paṇḍi-ta-kṛtā
nava-pañca-mayor
nidā-na-sthā-na-sya
ni-ghaṇ-ṭu
nir-anta-ra-pa-da-vyā-khyā
nir-ṇaya-sā-gara
nir-ṇaya-sā-gara-yantr-ā-laye
nirūha-vasti
niś-cala-kara
ni-yukta-vaidyāṃ
nya-grodha
nya-grodho
nyāya-śās-tra
nyāya-sū-tra-śaṃ-kar
okaḥ-sātmya
okaḥ-sātmyam
okaḥ-sātmyaṃ
oka-sātmya
oka-sātmyam
oka-sātmyaṃ
oṣṭha-saṃ-puṭa
ousha-da-sala
padma-pra-bha-sūri
padma-sva-sti-kārdha-candrādike
paitā-maha-siddhā-nta
pañca-karma
pañca-karma-bhava-rogāḥ
pañca-karmādhi-kāra
pañca-karma-vi-cāra
pāñca-rātrā-gama
pañca-siddh-āntikā
pari-bhāṣā
pari-likh-ya
pātañ-jala-yoga-śās-tra
pātañ-jala-yoga-śās-tra-vi-varaṇa
pat-añ-jali
pāṭī-gaṇita
pāva-suya
pim-pal-gaon
pipal-gaon
pit-ta-kṛt
pit-ta-śleṣma-ghna
pit-ta-śleṣma-medo-meha-hik-kā-śvā-sa-kā-sāti-sā-ra-cchardi-tṛṣṇā-kṛmi-vi-ṣa-pra-śa-ma-naṃ
prā-cya
prā-cya-hindu-gran-tha-śreṇiḥ
prācya-vidyā-saṃ-śodhana-mandira
pra-dhān-āṅ-gaṃ
pra-dhān-in
pra-ka-shan
pra-kṛ-ti
pra-kṛ-tiṃ
pra-mā-ṇa-vārt-tika
pra-saṅ-khyāne
pra-śas-ta-pāda-bhāṣya
pra-śna-pra-dīpa
pra-śnārṇa-va-plava
praśnārṇava-plava
pra-śna-vaiṣṇava
pra-śna-vai-ṣṇava
prati-padyate
pra-ty-akṣa
pra-yat-na-śai-thilyā-nan-ta-sam-ā-pat-ti-bhyām
pra-yat-na-śai-thilyā-nān-tya-sam-ā-pat-ti-bhyāṃ
pra-yatna-śai-thilya-sya
puṇya-pattana
pūrṇi-mā-nta
rāja-kīya
rajjv-ābhyas-ya
rāma-kṛṣṇa
rasa-ratnā-kara
rasa-vai-śeṣika-sūtra
rogi-svasthī-karaṇānu-ṣṭhāna-mātraṃ
rūkṣa-vasti
sād-guṇya
śākalya-saṃ-hitā
sam-ā-mnāya
sāmañña-pha-la-sutta
sama-ran-gana-su-tra-dhara
samā-raṅga-ṇa-sū-tra-dhāra
sama-ra-siṃ-ha
sama-ra-siṃ-haḥ
saṃ-hitā
sāṃ-sid-dhi-ka
saṃ-śo-dhana
sam-ul-lasi-tam
śāndilyopa-ni-ṣad
śaṅ-kara
śaṅ-kara-bha-ga-vat-pāda
Śaṅ-kara-nārā-yaṇa
saṅ-khyā
sāṅ-kṛt-yā-yana
san-s-krit
sap-tame
śāra-dā-tila-ka-tan-tra
śa-raṅ-ga-deva
śār-dūla-karṇā-va-dāna
śā-rī-ra
śā-rī-ra-sthāna
śārṅga-dhara
śārṅga-dhara-saṃ-hitā
sar-va
sarva-darśana-saṃ-grahaḥ
sar-va-dar-śāna-saṅ-gra-ha
sar-va-dar-śāna-saṅ-gra-haḥ
sarv-arthāvi-veka-khyā-ter
sar-va-tan-tra-sid-dhān-ta
sar-va-tan-tra-sid-dhān-taḥ
sarva-yoga-sam-uc-caya
sar-va-yogeśvareśva-ram
śāstrā-rambha-sam-artha-na
śāstrāram-bha-sam-arthana
ṣaṭ-pañcā-śi-kā
sat-tva
saunda-ra-na-nda
sid-dha
sid-dha-man-tra
sid-dha-man-trā-hvayo
sid-dha-man-tra-pra-kāśa
sid-dha-man-tra-pra-kāśaḥ
sid-dha-man-tra-pra-kāśaś
sid-dh-ān-ta
siddhānta-śiro-maṇ
sid-dha-yoga
sid-dhi-sthāna
śi-va-śar-ma-ṇā
ska-nda-pu-rā-ṇa
sneha-basty-upa-deśāt
sodā-haraṇa-saṃ-s-kṛta-vyā-khyayā
śodha-ka-pusta-kaṃ
śo-dha-na-ci-kitsā
so-ma-val-ka
śrī-mad-devī-bhāga-vata-mahā-purāṇa
srag-dharā-tārā-sto-tra
śrī-hari-kṛṣṇa-ni-bandha-bhava-nam
śrī-hema-candrā-cārya-vi-raci-taḥ
śrī-kaṇtha-datta
śrī-kaṇtha-dattā-bhyāṃ
śrī-kṛṣṇa-dāsa
śrī-mad-amara-siṃha-vi-racitam
śrī-mad-aruṇa-dat-ta-vi-ra-ci-tayā
śrī-mad-bhaṭṭot-pala-kṛta-saṃ-s-kṛta-ṭīkā-sahitam
śrī-mad-dvai-pā-yana-muni-pra-ṇītaṃ
śrī-mad-vāg-bha-ṭa-vi-ra-ci-tam
śrī-maṃ-trī-vi-jaya-siṃha-suta-maṃ-trī-teja-siṃhena
śrī-mat-kalyāṇa-varma-vi-racitā
śrīmat-sāyaṇa-mādhavācārya-pra-ṇītaḥ
śrī-vā-cas-pati-vaidya-vi-racita-yā
śrī-vatsa
śrī-vi-jaya-rakṣi-ta
sthānāṅga-sūtra
sthira-sukha
sthira-sukham
strī-niṣevaṇa
śukla-pakṣa
su-śru-ta-saṃ-hitā
sū-tra
sūtrārthānān-upa-patti-sūca-nāt
sūtra-sthāna
su-varṇa-pra-bhāsot-tama-sū-tra
svalpauṣadha-dāna-mā-tram
śvetāśva-taropa-ni-ṣad
tad-upa-karaṇa-tāmra-kaṭāha-kalasādi-pātra-pari-cchada-nānā-vidha-vyādhi-śānty-ucitauṣadha-gaṇa-yathokta-lakṣaṇa-vaidya-nānā-vidha-pari-cāraka-yutāṃ
tājaka-muktā-valeḥ
tājika-kau-stu-bha
tājika-nīla-kaṇṭhī
tājika-yoga-sudhā-ni-dhi
tāmra-paṭṭādi-li-khi-tāṃ
tan-nir-vāhāya
tapo-dhana
tapo-dhanā
tārā-bhakti-su-dhārṇava
tārtīya-yoga-su-sudhā-ni-dhi
tegi-ccha
te-jaḥ-siṃ-ha
trai-lok-ya
trai-lokya-pra-kāśa
tri-piṭa-ka
tri-var-gaḥ
un-mār-ga-gama-na
upa-de-śa
upa-patt-ti
ut-sneha-na
utta-rā-dhyā-ya-na
uttara-sthāna
uttara-tantra
vāchas-pati
vād-ā-valī
vai-śā-kha
vai-ta-raṇa-vasti
vai-ta-raṇok-ta-guṇa-gaṇa-yu-k-taṃ
vājī-kara-ṇam
vāk-patis
vākya-śeṣa
vākya-śeṣaḥ
varā-ha-mihi-ra
va-ra-na-si
vā-rā-ṇa-sī
var-mam
var-man
varṇa-sam-ā-mnāya
va-siṣṭha-saṃ-hitā
vā-siṣṭha-saṃ-hitā
vasu-bandhu
vasu-bandhu
vāta-ghna-pit-talāl-pa-ka-pha
vātsyā-ya-na
vidya-bhu-sana
vidyā-bhū-ṣaṇa
vi-jaya-siṃ-ha
vi-jñāna-bhikṣu
vi-kal-pa
vi-kamp-i-tum
vi-mā-na-sthāna
vi-racita-yā
vishveshvar-anand
vi-śiṣṭ-āṃśena
viṣṇu-dharmot-tara-purāṇa
viśrāma-gṛha-sahitā
vi-suddhi-magga
vopa-de-vīya-sid-dha-man-tra-pra-kāśe
vyādhi-pratī-kārār-tham
vyāḍī-ya-pa-ri-bhā-ṣā-vṛtti
vyati-krāmati
vy-ava-haranti
yādava-bhaṭṭa
yāda-va-śarma-ṇā
yādava-sūri
yājña-valkya-smṛti
yavanā-cā-rya
yoga-ratnā-kara
yoga-sāra-sam-uc-caya
yoga-sāra-sam-uc-cayaḥ
yoga-sūtra-vi-vara-ṇa
yoga-yājña-valkya
yoga-yājña-valkya-gītāsūpa-ni-ṣatsu
yoga-yājña-valkyaḥ
yogi-yājña-valkya-smṛti
yuk-tiḥ
yuk-tis
}}
\normalfontlatin
\endinput
}% should work, but doesn't
% special hyphenations for Sanskrit words tagged in
% Polyglossia.
% *English,\textenglish{},text,and
% *Sanskrit,\textsanskrit{},text.
%
% English (see below for \textsanskrit)
%
\hyphenation{%
    dhanva-ntariṇopa-diṣ-ṭaḥ
    suśruta-nāma-dheyena
    tac-chiṣyeṇa
    kāśyapa-saṃ-hitā
    cikitsā-sthāna
    su-śruta-san-dīpana-bhāṣya
    dṛṣṭi-maṇḍala
    uc-chiṅga-na
    sarva-siddhānta-tattva-cūḍā-maṇi
    tulya-sau-vīrāñja-na
    indra-gopa
    śrī-mad-abhi-nava-guptā-cārya-vi-ra-cita-vi-vṛti-same-tam
    viśva-nātha
śrī-mad-devī-bhāga-vata-mahā-purāṇa
    siddhā-n-ta-sun-dara
    brāhma-sphuṭa-siddh-ānta
    bhū-ta-saṅ-khyā
    bhū-ta-saṃ-khyā
    kathi-ta-pada
    devī-bhā-ga-vata-purāṇa
    devī-bhā-ga-vata-mahā-purāṇa
    Siddhānta-saṃ-hitā-sāra-sam-uc-caya
    sau-ra-pau-rāṇi-ka-mata-sam-artha-na
    Pṛthū-da-ka-svā-min
    Brah-ma-gupta
    Brāh-ma-sphu-ṭa-siddhānta
    siddhānta-sun-dara
    vāsa-nā-bhāṣya
    catur-veda
    bhū-maṇḍala
    jñāna-rāja
    graha-gaṇi-ta-cintā-maṇi
    Śiṣya-dhī-vṛd-dhi-da-tan-tra
    brah-māṇḍa-pu-rā-ṇa
    kūr-ma-pu-rā-ṇa
    jam-bū-dvī-pa
    bhā-ga-vata-pu-rā-ṇa
    kupya-ka
    nandi-suttam
    nandi-sutta
    su-bodhiā-bāī
    asaṅ-khyāta
    saṅ-khyāta
    saṅ-khyā-pra-māṇa
    saṃ-khā-pamāṇa
    nemi-chandra
    anu-yoga-dvāra
    tattvārtha-vārtika
    aka-laṅka
    tri-loka-sāra
    gaṇi-ma-pra-māṇa
    gaṇi-ma-ppa-māṇa
    eka-pra-bhṛti
gaṇaṇā-saṃ-khā
gaṇaṇā-saṅ-khyā
dvi-pra-bhṛti
duppa-bhi-ti-saṃ-khā
vedanābhi-ghāta
Viṣṇu-dharmottara-pu-rāṇa
abhaya-deva-sūri-vi-racita-vṛtti-vi-bhūṣi-tam
abhi-dhar-ma
abhi-dhar-ma-ko-śa
abhi-dhar-ma-ko-śa-bhā-ṣya
abhi-dharma-kośa-bhāṣya
abhi-dharma-kośa-bhāṣyam
abhi-nava
abhyaṃ-karopāhva-vāsu-deva-śāstri-vi-ra-ci-ta-yā
ācārya-śrī-jina-vijayālekhitāgra-vacanālaṃ-kṛtaś-ca
ācāry-opā-hvena
ādhāra
adhi-kāra
adhi-kāras
ādi-nātha
agni-besha
agni-veśa
ahir-budhnya
ahir-budhnya-saṃ-hitā
aita-reya-brāhma-ṇa
akusī-dasya
amara-bharati
Amar-augha-pra-bo-dha
amṛ-ta-siddhi
ānanda-kanda
ānan-da-rā-ya
ānand-āśra-ma-mudraṇā-la-ya
ānand-āśra-ma-saṃ-skṛta-granth-āva-liḥ
anna-pāna-mūlā
anu-ban-dhya-lakṣaṇa-sam-anv-itās
anu-bhav-ād
anu-bhū-ta-viṣayā-sam-pra-moṣa
anu-bhū-ta-viṣayā-sam-pra-moṣaḥ
aparo-kṣā-nu-bhū-ti
app-proxi-mate-ly
ardha-rātrika-karaṇa
ārdha-rātrika-karaṇa
ariya-pary-esana-sutta
arun-dhatī
ārya-bhaṭa
ārya-bhaṭā-cārya-vi-racitam
ārya-bhaṭīya
ārya-bhaṭīyaṃ
ārya-lalita-vistara-nāma-mahā-yāna-sūtra
ārya-mañju-śrī-mūla-kalpa
ārya-mañju-śrī-mūla-kalpaḥ
asaṃ-pra-moṣa
aṣṭāṅga-hṛdaya-saṃ-hitā
aṣṭāṅga-saṃ-graha
asura-bhavana
aśva-ghoṣa
ātaṅka-darpaṇa-vyā-khyā-yā
atha-vā
ava-sāda-na
āyār-aṅga-suttaṃ
ayur-ved
ayur-veda
āyur-veda
āyur-veda-dīpikā
āyur-veda-dīpikā-vyā-khyayā
āyur-ve-da-ra-sā-yana
āyur-veda-sū-tra
ayur-vedic
āyur-vedic
ayur-yog
bādhirya
bahir-deśa-ka
bala-bhadra
bala-kot
bala-krishnan
bāla-kṛṣṇa
bau-dhā-yana-dhar-ma-sūtra
bel-valkar
bhadra-kālī-man-tra-vi-dhi-pra-karaṇa
bhadrā-sana
bhadrā-sanam
bha-ga-vat-pāda
bhaiṣajya-ratnāvalī
bhan-d-ar-kar
bhartṛhari-viracitaḥ
bhaṭṭā-cārya
bhaṭṭot-pala-vi-vṛti-sahitā
Bhiṣag-varāḍha-malla-vi-racita-dīpikā-Kāśī-rāma-vaidya-vi-raci-ta-gūḍhā-rtha-dīpikā-bhyāṃ
bhiṣag-varāḍha-malla-vi-racita-dīpikā-Kāśī-rāma-vaidya-vi-racita-gūḍhārtha-dīpikā-bhyāṃ
bhoja-deva-vi-raci-ta-rāja-mārtaṇḍā-bhi-dha-vṛtti-sam-e-tāni
bhu--va-na-dī-pa-ka
bīja-pallava
bi-kaner
bodhi-sat-tva-bhūmi
brahma-gupta
brahmā-nanda
brahmāṇḍa-mahā-purā-ṇa
brahmāṇḍa-mahā-purā-ṇam
brahma-randhra
brahma-siddh-ānta
brāhma-sphuṭa-siddh-ānta
brāhma-sphu-ṭa-siddhānta
brahma-vi-hāra
brahma-vi-hāras
brahma-yā-mala-tan-tra
Bra-ja-bhāṣā
bṛhad-āraṇya-ka
bṛhad-yā-trā
bṛhad-yogi-yājña-valkya-smṛti
bṛhad-yogī-yājña-valkya-smṛti
bṛhaj-jāta-kam
bṛhat-khe-carī-pra-kāśa
buddhi-tattva-pra-karaṇa
cak-ra-dat-ta
cakra-datta
cakra-pāṇi-datta
cā-luk-ya
caraka-prati-saṃ-s-kṛta
caraka-prati-saṃ-s-kṛte
caraka-saṃ-hitā
casam-ul-lasi-tāmaharṣiṇāsu-śrutenavi-raci-tāsu-śruta-saṃ-hitā
cau-kham-ba
cau-luk-yas
chandi-garh
chara-ka
cha-rīre
chatt-opa-dh-ya-ya
chau-kham-bha
chi-ki-tsi-ta
cid-ghanā-nanda-nātha
ci-ka-ner
com-men-taries
com-men-tary
com-pre-hen-sive-ly
daiva-jñālaṃ-kṛti
daiva-jñālaṅ-kṛti
dāmo-dara-sūnu-Śārṅga-dharācārya-vi-racitā
Dāmodara-sūnu-Śārṅga-dharācārya-vi-racitā
darśanā-ṅkur-ābhi-dhayā
das-gupta
deha-madhya
deha-saṃ-bhava-hetavaḥ
deva-datta
deva-nagari
deva-nāgarī
devā-sura-siddha-gaṇaiḥ
dha-ra-ni-dhar
dharma-megha
dharma-meghaḥ
dhru-vam
dhru-va-sya
dhru-va-yonir
dhyā-na-grahopa-deśā-dhyā-yaś
dṛḍha-śūla-yukta-rakta
dvy-ulbaṇaikolba-ṇ-aiḥ
four-fold
gan-dh-ā-ra
gārgīya-jyoti-ṣa
gārgya-ke-rala-nīla-kaṇṭha-so-ma-sutva-vi-racita-bhāṣyo-pe-tam
garuḍa-mahā-purāṇa
gaurī-kāñcali-kā-tan-tra
gau-tama
gauta-mādi-tra-yo-da-śa-smṛty-ātma-kaḥ
gheraṇḍa-saṃ-hitā
gorakṣa-śata-ka
go-tama
granth-ā-laya
grantha-mālā
gran-tha-śreṇiḥ
grāsa-pramāṇa
guru-maṇḍala-grantha-mālā
gyatso
hari-śāstrī
haṭhābhyāsa-paddhati
haṭha-ratnā-valī
Haṭha-saṅ-keta-candri-kā
haṭha-tattva-kau-mudī
haṭha-yoga
hāyana-rat-na
haya-ta-gran-tha
hema-pra-bha-sūri
hetu-lakṣaṇa-saṃ-sargād
hīna-madhyādhi-kaiś
hindī-vyā-khyā-vi-marśope-taḥ
hoern-le
ijya-rkṣa
ikka-vālaga
indra-dhvaja
indrāṇī-kalpa
indria
Īśāna-śiva-guru-deva-pad-dhati
jābāla-darśanopa-ni-ṣad
jadav-ji
jagan-nā-tha
jala-basti
jal-pa-kal-pa-tāru
jam-bū-dvī-pa-pra-jña-pti
jam-bū-dvī-pa-pra-jña-pti-sūtra
jana-pad-a-sya
jāta-ka-kar-ma-pad-dhati
jaya-siṃha
jinā-agama-grantha-mālā
jin-en-dra-bud-dhi
jīvan-muk-ti-vi-veka
jñā-na-nir-mala
jñā-na-nir-malaṃ
joga-pra-dīpya-kā
jya-rkṣe
Jyo-tiḥ-śās-tra
jyo-ti-ṣa-rāya
jyoti-ṣa-rāya
jyotiṣa-siddhānta-saṃ-graha
jyotiṣa-siddhānta-saṅ-graha
kāka-caṇḍīśvara-kal-pa-tan-tra
kakṣa-puṭa
kali-kāla-sarva-jña
kali-kāla-sarva-jña-śrī-hema-candrācārya-vi-raci-ta
kali-kāla-sarva-jña-śrī-hema-candrācārya-vi-raci-taḥ
kali-yuga
kal-pa
kal-pa-sthāna
kalyāṇa-kāraka
Kāmeśva-ra-siṃha-dara-bhaṅgā-saṃ-skṛta-viśva-vidyā-layaḥ
kapāla-bhāti
karaṇa-tilaka
kar-ma
kar-man
kāṭhaka-saṃ-hitā
kavia-rasu
kavi-raj
keśa-va-śāstrī
ke-vala--rāma
keva-la-rāma
khaṇḍa-khādyaka-tappā
khe-carī-vidyā
knowl-edge
kol-ka-ta
kriyā-krama-karī
kṛṣṇa-pakṣa
kṛtti-kā
kṛtti-kās
kubji-kā-mata-tantra
kula-pañji-kā
kul-karni
ku-māra-saṃ-bhava
kuṭi-pra-veśa
kuṭi-pra-veśika
lakṣ-mī-veṅ-kaṭ-e-ś-va-ra
lit-era-ture
lit-era-tures
locana-roga
mādha-va
mādhava-kara
mādhava-ni-dāna
mādhava-ni-dā-nam
madh-ūni
madhya
mādhyan-dina
madhye
mahā-bhāra-ta
mahā-deva
mahā-kavi-bhartṛ-hari-praṇīta-tvena
maha-mahopa-dhyaya
mahā-maho-pā-dhyā-ya-śrī-vi-jñā-na-bhikṣu-vi-raci-taṃ
mahā-mati-śrī-mādhava-kara-pra-ṇī-taṃ
mahā-mudrā
mahā-muni-śrī-mad-vyāsa-pra-ṇī-ta
mahā-muni-śrī-mad-vyāsa-pra-ṇī-taṃ
maharṣiṇā
maha-rṣi-pra-ṇīta-dharma-śāstra-saṃ-grahaḥ
Maha-rṣi-varya-śrī-yogi-yā-jña-valkya-śiṣya-vi-racitā
mahā-sacca-ka-sutta
mahā-sati-paṭṭhā-na-sutta
mahā-vra-ta
mahā-yāna-sūtrālaṅ-kāra
maitrāya-ṇī-saṃ-hitā
maktab-khānas
māla-jit
māli-nī-vijayot-tara-tan-tra
manaḥ-sam-ā-dhi
mānasol-lāsa
mānava-dharma-śāstra
mandāgni-doṣa
mannar-guḍi
mano-har-lal
mano-ratha-nandin
man-u-script
man-u-scripts
mataṅga-pārame-śvara
mater-ials
matsya-purāṇam
medh-ā-ti-thi
medhā-tithi
mithilā-stha
mithilā-stham
mithilā-sthaṃ
mṛgendra-tantra-vṛtti
mud-rā-yantr-ā-laye
muktā-pīḍa
mūla-pāṭha
muṇḍī-kalpa
mun-sh-ram
Nāda-bindū-pa-ni-ṣat
nāga-bodhi
nāga-buddhi
nakṣa-tra
nara-siṃha
nārā-yaṇa-dāsa
nārā-yaṇa-dāsa
nārā-yaṇa-kaṇṭha
nārā-yaṇa-paṇḍi-ta-kṛtā
nar-ra-tive
nata-rajan
nava-pañca-mayor
nava-re
naya-na-sukho-pā--dhyāya
ni-ban-dha-saṃ-grahā-khya-vyākhya-yā
niban-dha-san-graha
ni-dā-na
nidā-na-sthā-na-sya
ni-dāna-sthānasyaśrī-gaya-dāsācārya-vi-racitayānyāya-candri-kā-khya-pañjikā-vyā-khyayā
nir-anta-ra-pa-da-vyā-khyā
nir-guṇḍī-kalpa
nir-ṇaya-sā-gara
Nir-ṇaya-sāgara
nir-ṇa-ya-sā-gara-mudrā-yantrā-laye
nir-ṇa-ya-sā-ga-ra-yantr-āla-ya
nir-ṇaya-sā-gara-yantr-ā-laye
niśvāsa-kārikā
nīti-śṛṅgāra-vai-rāgyādi-nāmnāsamākhyā-tānāṃ
nityā-nanda
nya-grodha
nya-grodho
nyā-ya-candri-kā-khya-pañji-kā-vyā-khya-yā
nyāya-śās-tra
okaḥ-sātmya
okaḥ-sātmyam
okaḥ-sātmyaṃ
oka-sātmya
oka-sātmyam
oka-sātmyaṃ
oris-sa
oṣṭha-saṃ-puṭa
ousha-da-sala
padma-pra-bha-sūri
Padma-prā-bhṛ-ta-ka
padma-sva-sti-kārdha-candrādike
paitā-maha-siddhā-nta
pañca-karma
pañca-karman
pāñca-rātrā-gama
pañca-siddh-āntikā
paṅkti-śūla
Paraśu-rāma
paraśu-rāma
pari-likh-ya
pāśu-pata-sū-tra-bhāṣya
pātañ-jala-yoga-śās-tra
pātañ-jala-yoga-śās-tra-vi-varaṇa
pat-añ-jali
pat-na
pāva-suya
phiraṅgi-can-dra-cchedyo-pa-yogi-ka
pim-pal-gaon
pipal-gaon
pitta-śleṣ-man
pit-ta-śleṣ-ma-śoṇi-ta
pitta-śoṇi-ta
prā-cīna-rasa-granthaḥ
prā-cya
prā-cya-hindu-gran-tha-śreṇiḥ
prācya-vidyā-saṃ-śodhana-mandira
pra-dhān-in
pra-ka-shan
pra-kaṭa-mūṣā
pra-kṛ-ti-bhū-tāḥ
pra-mā-ṇa-vārt-tika
pra-ṇītā
pra-saṅ-khyāne
pra-śas-ta-pāda-bhāṣya
pra-śna-pra-dīpa
pra-śnārṇa-va-plava
praśnārṇava-plava
pra-śna-vai-ṣṇava
pra-śna-vaiṣṇava
prati-padyate
pra-yatna-śaithilyānan-ta-sam-āpatti-bhyām
prei-sen-danz
punar-vashu
puṇya-pattana
pūrṇi-mā-nta
raghu-nātha
rāja-kīya
rāja-kīya-mudraṇa-yantrā-laya
rāja-śe-khara
rajjv-ābhyas-ya
raj-put
rāj-put
rakta-mokṣa-na
rāma-candra-śāstrī
rāma-kṛṣṇa
rāma-kṛṣṇa-śāstri-ṇā
rama-su-bra-manian
rāmā-yaṇa
rasa-ratnā-kara
rasa-ratnākarāntar-ga-taś
rasa-ratna-sam-uc-caya
rasa-ratna-sam-uc-ca-yaḥ
rasa-vīry-auṣa-dha-pra-bhāvena
rasā-yana
rasendra-maṅgala
rasendra-maṅgalam
rāṣṭra-kūṭa
rāṣṭra-kūṭas
sādhana
śākalya-saṃ-hitā
śāla-grāma-kṛta
śāla-grāma-kṛta
sāmañña-pha-la-sutta
sāmañña-phala-sutta
sama-ran-gana-su-tra-dhara
samā-raṅga-ṇa-sū-tra-dhāra
sama-ra-siṃ-ha
sama-ra-siṃ-haḥ
sāmba-śiva-śāstri
same-taḥ
saṃ-hitā
śāṃ-ka-ra-bhāṣ-ya-sam-etā
sam-rāṭ
saṃ-rāṭ
Sam-rāṭ-siddhānta
Sam-rāṭ-siddhānta-kau-stu-bha
sam-rāṭ-siddhānta-kau-stu-bha
saṃ-sargam
saṃ-sargaṃ
saṃ-s-kṛta
saṃ-s-kṛta-pārasī-ka-pra-da-pra-kāśa
saṃ-śo-dhana
saṃ-śodhitā
saṃ-sthāna
sam-ullasitā
sam-ul-lasi-tam
saṃ-valitā
saṃ-valitā
śāndilyopa-ni-ṣad
śaṅ-kara
śaṅ-kara-bha-ga-vat-pāda
śaṅ-karā-cārya
san-kara-charya
Śaṅ-kara-nārā-yaṇa
sāṅ-kṛt-yā-yana
san-s-krit
śāra-dā-tila-ka-tan-tra
śa-raṅ-ga-deva
śār-dūla-karṇā-va-dāna
śār-dūla-karṇā-va-dāna
śā-rī-ra-sthāna
śārṅga-dhara-saṃ-hitā
Śārṅga-dhara-saṃ-hitā
sar-va-dar-śana-saṅ-gra-ha
sarva-kapha-ja
sarv-arthāvi-veka-khyā-ter
sar-va-śa-rīra-carās
sarva-siddhānta-rāja
Sarva-siddhā-nt-rāja
sarva-vyā-dhi-viṣāpa-ha
sarva-yoga-sam-uc-caya
sar-va-yogeśvareśva-ram
śāstrā-rambha-sam-artha-na
śatakatrayādi-subhāṣitasaṃgrahaḥ
sati-paṭṭhā-na-sutta
ṣaṭ-karma
ṣaṭ-karman
sat-karma-saṅ-graha
sat-karma-saṅ-grahaḥ
ṣaṭ-pañcā-śi-kā
saun-da-ra-nanda
sa-v-āī
schef-tel-o-witz
scholars
sharī-ra
sheth
sid-dha-man-tra
siddha-nanda-na-miśra
siddha-nanda-na-miśraḥ
siddha-nitya-nātha-pra-ṇītaḥ
Siddhānta-saṃ-hitā-sāra-sam-uc-caya
Siddhā-nta-sār-va-bhauma
siddhānta-sindhu
siddhānta-śiro-maṇ
Siddhānta-śiro-maṇi
Siddhā-nta-tat-tva-vi-veka
sid-dha-yoga
siddha-yoga
sid-dhi
sid-dhi-sthā-na
sid-dhi-sthāna
śikhi-sthāna
śiraḥ-karṇā-kṣi-vedana
śiro-bhūṣaṇam
Śivā-nanda-saras-vatī
śiva-saṃ-hitā
śiva-yo-ga-dī-pi-kā
ska-nda-pu-rā-ṇa
śleṣ-man
śleṣ-ma-śoni-ta
sodā-haraṇa-saṃ-s-kṛta-vyā-khyayā
śodha-ka-pusta-kaa
śoṇi-ta
spaṣ-ṭa-krānty-ādhi-kāra
śrī-cakra-pāṇi-datta
śrī-cakra-pāṇi-datta-viracitayā
śrī-ḍalhaṇācārya-vi-raci-tayāni-bandha-saṃ-grahākhya-vyā-khyayā
śrī-dayā-nanda
śrī-hari-kṛṣṇa-ni-bandha-bhava-nam
śrī-hema-candrā-cārya-vi-raci-taḥ
śrī-kaṇtha-dattā-bhyāṃ
śrī-kṛṣṇa-dāsa
śrī-kṛṣṇa-dāsa-śreṣṭhinā
śrīmac-chaṅ-kara-bhaga-vat-pāda-vi-raci-tā
śrī-mad-amara-siṃha-vi-racitam
śrī-mad-bha-ga-vad-gī-tā
śrī-mad-bhaṭṭot-pala-kṛta-saṃ-s-kṛta-ṭīkā-sahitam
śrī-mad-dvai-pā-yana-muni-pra-ṇītaṃ
śrī-mad-vāg-bhaṭa-vi-raci-tam
śrī-maṃ-trī-vi-jaya-siṃha-suta-maṃ-trī-teja-siṃhena
śrī-mat-kalyāṇa-varma-vi-racitā
śrī-mat-sāyaṇa-mādhavācārya-pra-ṇītaḥsarva-darśana-saṃ-grahaḥ
śrī-nitya-nātha-siddha-vi-raci-taḥ
śrī-rāja-śe-khara
śrī-śaṃ-karā-cārya-vi-raci-tam
śrī-vā-cas-pati-vaidya-vi-racita-yā
śrī-vatsa
śrī-veda-vyāsa-pra-ṇīta-mahā-bhā-ratāntar-ga-tā
śrī-veṅkaṭeś-vara
śrī-vi-jaya-rakṣi-ta
sruta-rakta
sruta-raktasya
stambha-karam
sthānāṅga-sūtra
sthira-sukha
sthira-sukham
stra-sthā-na
subhāṣitānāṃ
su-brah-man-ya
su-bra-man-ya
śukla-pakṣa
śukrā-srava
suk-than-kar
su-pariṣkṛta-saṃgrahaḥ
sura-bhi-pra-kash-an
sūrya-dāsa
sūrya-siddhānta
su-shru-ta
su-śru-ta
su-shru-ta-saṃ-hitā
su-śru-ta-saṃ-hitā
su-śru-tena
sutra
sūtra
sūtra-neti
sūtra-ni-dāna-śā-rīra-ci-ki-tsā-kal-pa-sthānot-tara-tan-trātma-kaḥ
sūtra-sthāna
su-varṇa-pra-bhāsot-tama-sū-tra
Su-var-ṇa-pra-bhās-ot-tama-sū-tra
su-varṇa-pra-bhāsotta-ma-sūtra
su-vistṛta-pari-cayātmikyāṅla-prastāvanā-vividha-pāṭhān-tara-pari-śiṣṭādi-sam-anvitaḥ
sva-bhāva-vyādhi-ni-vāraṇa-vi-śiṣṭ-auṣa-dha-cintakās
svā-bhāvika
svā-bhāvikās
sva-cchanda-tantra
śvetāśva-taropa-ni-ṣad
taila-sarpir-ma-dhūni
tait-tirīya-brāhma-ṇa
tājaka-muktā-valeḥ
tājika-kau-stu-bha
tājika-nīla-kaṇṭhī
tājika-yoga-sudhā-ni-dhi
tapo-dhana
tapo-dhanā
tārā-bhakti-su-dhārṇava
tārtīya-yoga-su-sudhā-ni-dhi
tegi-ccha
te-jaḥ-siṃ-ha
ṭhāṇ-āṅga-sutta
ṭīkā-bhyāṃ
ṭīkā-bhyāṃ
tiru-mantiram
tiru-ttoṇṭar-purāṇam
tiru-va-nanta-puram
trai-lok-ya
trai-lokya-pra-kāśa
tri-bhāga
tri-kam-ji
tri-pita-ka
tri-piṭa-ka
tri-vik-ra-mātma-jena
ud-ā-haraṇa
un-mārga-gama-na
upa-ca-ya-bala-varṇa-pra-sādādī-ni
upa-laghana
upa-ni-ṣads
upa-patt-ti
ut-sneha-na
utta-rā-dhya-ya-na
utta-rā-dhya-ya-na-sūtra
uttara-khaṇḍa-khādyaka
uttara-sthāna
uttara-tantra
vācas-pati-miśra-vi-racita-ṭīkā-saṃ-valita
vācas-pati-miśra-vi-racita-ṭīkā-saṃ-valita-vyā-sa-bhā-ṣya-sam-e-tāni
vag-bhata-rasa-ratna-sam-uc-caya
vāg-bhaṭa-rasa-ratna-sam-uc-caya
vaidya-vara-śrī-ḍalhaṇā-cārya-vi-racitayā
vai-śā-kha
vai-śeṣ-ika-sūtra
vāja-sa-neyi-saṃ-hitā
vājī-kara-ṇam
vākya-śeṣa
vākya-śeṣaḥ
vaṅga-sena
vaṅga-sena-saṃ-hitā
varā-ha-mihi-ra
vārāhī-kalpa
vā-rāṇa-seya
va-ra-na-si
var-mam
var-man
var-ṇa-saṃ-khyā
var-ṇa-saṅ-khyā
vā-si-ṣṭha
vasiṣṭha-saṃ-hitā
vā-siṣṭha-saṃ-hitā
Va-sistha-Sam-hita-Yoga-Kanda-With-Comm-ent-ary-Kai-valya-Dham
vastra-dhauti
vasu-bandhu
vāta-pit-ta
vāta-pit-ta-kapha
vāta-pit-ta-kapha-śoṇi-ta
vāta-pitta-kapha-śoṇita-san-nipāta-vai-ṣamya-ni-mittāḥ
vāta-pit-ta-śoṇi-ta
vāta-śleṣ-man
vāta-śleṣ-ma-śoṇi-ta
vāta-śoṇi-ta
vātā-tapika
vātsyā-ya-na
vāya-vīya-saṃ-hitā
vedāṅga-rāya
veezhi-nathan
venkat-raman
vid-vad-vara-śrī-gaṇeśa-daiva-jña-vi-racita
vidya-bhu-sana
vi-jaya-siṃ-ha
vi-jñāna-bhikṣu
Vijñāneśvara-vi-racita-mitākṣarā-vyā-khyā-sam-alaṅ-kṛtā
vi-mā-na
vi-mā-na-sthāna
vimāna-sthā-na
vi-racitā
vi-racita-yāmadhu-kośākhya-vyā-khya-yā
vi-recana
vishveshvar-anand
vi-śiṣṭ-āṃśena
vi-suddhi-magga
vi-vi-dha-tṛṇa-kāṣṭha-pāṣāṇa-pāṃ-su-loha-loṣṭāsthi-bāla-nakha-pūyā-srāva-duṣṭa-vraṇāntar-garbha-śalyo-ddharaṇārthaṃ
vṛd-dha-vṛd-dha-tara-vṛd-dha-tamaiḥ
vṛddha-vṛddha-tara-vṛddha-tamaiḥ
vṛnda-mādhava
vyāḍī-ya-pa-ri-bhā-ṣā-vṛtti
vyā-khya-yā
vy-akta-liṅgādi-dharma-yuk-te
vyā-sa-bhā-ṣya-sam-e-tāni
vyati-krāmati
Xiuyao
yādava-bhaṭṭa
yāda-va-śarma-ṇā
yādava-sūri
yājña-valkya-smṛti
yājña-valkya-smṛtiḥ
yantrā-dhyāya
Yantra-rāja-vicāra-viṃśā-dhyāyī
yavanā-cā-rya
yoga-bhā-ṣya-vyā-khyā-rūpaṃ
yoga-cintā-maṇi
yoga-cintā-maṇiḥ
yoga-ratnā-kara
yoga-sāra-mañjarī
yoga-sāra-sam-uc-caya
yoga-sāra-saṅ-graha
yoga-śikh-opa-ni-ṣat
yoga-tārā-valī
yoga-yājña-val-kya
yoga-yājña-valkya-gītāsūpa-ni-ṣatsu
yogi-yājña-valkya-smṛti
yoshi-mizu
yukta-bhava-deva
}
%%%%%%%%%%%%%%%%%%%%
%Sanskrit:
%%%%%%%%%%%%%%%%%%%%
\textsanskrit{\hyphenation{%
    dhanva-ntariṇopa-diṣ-ṭaḥ
suśruta-nāma-dheyena
tac-chiṣyeṇa
    su-śruta-san-dīpana-bhāṣya
    cikitsā-sthāna
tulya-sau-vīrāñjana
indra-gopa
dṛṣṭi-maṇḍala
uc-chiṅga-na
vi-vi-dha-tṛṇa-kāṣṭha-pāṣāṇa-pāṃ-su-loha-loṣṭāsthi-bāla-nakha-pūyā-srāva-duṣṭa-vraṇāntar-garbha-śalyo-ddharaṇārthaṃ
śrī-ḍalhaṇācārya-vi-raci-tayāni-bandha-saṃ-grahākhya-vyā-khyayā
ni-dāna-sthānasyaśrī-gaya-dāsācārya-vi-racitayānyāya-candri-kā-khya-pañjikā-vyā-khyayā
casam-ul-lasi-tāmaharṣiṇāsu-śrutenavi-raci-tāsu-śruta-saṃ-hitā
bhartṛhari-viracitaḥ
śatakatrayādi-subhāṣitasaṃgrahaḥ
mahā-kavi-bhartṛ-hari-praṇīta-tvena
nīti-śṛṅgāra-vai-rāgyādi-nāmnāsamākhyā-tānāṃ
subhāṣitānāṃ
su-pariṣkṛta-saṃgrahaḥ
su-vistṛta-pari-cayātmikyāṅla-prastāvanā-vividha-pāṭhān-tara-pari-śiṣṭādi-sam-anvitaḥ
ācārya-śrī-jina-vijayālekhitāgra-vacanālaṃ-kṛtaś-ca
abhaya-deva-sūri-vi-racita-vṛtti-vi-bhūṣi-tam
abhi-dhar-ma
abhi-dhar-ma-ko-śa
abhi-dhar-ma-ko-śa-bhā-ṣya
abhi-dharma-kośa-bhāṣyam
abhyaṃ-karopāhva-vāsu-deva-śāstri-vi-racita-yā
agni-veśa
āhā-ra-vi-hā-ra-pra-kṛ-tiṃ
ahir-budhnya
ahir-budhnya-saṃ-hitā
akusī-dasya
alter-na-tively
amara-bharati
amara-bhāratī
āmla
amlīkā
ānan-da-rā-ya
anna-mardanādi-bhiś
anu-bhav-ād
anu-bhū-ta-viṣayā-sam-pra-moṣa
anu-bhū-ta-viṣayā-sam-pra-moṣaḥ
anu-māna
anu-miti-mānasa-vāda
ariya-pary-esana-sutta
ārogya-śālā-karaṇā-sam-arthas
ārogya-śālām
ārogyāyopa-kal-pya
arś-āṃ-si
ar-tha
ar-thaḥ
ārya-bhaṭa
ārya-lalita-vistara-nāma-mahā-yāna-sūtra
ārya-mañju-śrī-mūla-kalpa
ārya-mañju-śrī-mūla-kalpaḥ
asaṃ-pra-moṣa
āsana
āsanam
āsanaṃ
asid-dhe
aṣṭāṅga-hṛdaya
aṣṭāṅga-hṛdaya-saṃ-hitā
aṣṭ-āṅga-saṅ-graha
aṣṭ-āṅgā-yur-veda
aśva-gan-dha-kalpa
aśva-ghoṣa
ātaṅka-darpaṇa
ātaṅka-darpaṇa-vyā-khyā-yā
atha-vā
ātu-r-ā-hā-ra-vi-hā-ra-pra-kṛ-tiṃ
aty-al-pam
auṣa-dha-pāvanādi-śālāś
ava-sāda-na
avic-chin-na-sam-pra-dāya-tvād
āyur-veda
āyur-veda-sāra
āyur-vedod-dhāra-ka-vaid-ya-pañc-ānana-vaid-ya-rat-na-rāja-vaid-ya-paṇḍi-ta-rā-ma-pra-sāda-vaid-yo-pādhyā-ya-vi-ra-ci-tā
bahir-deśa-ka
bala-bhadra
bāla-kṛṣṇa
bau-dhā-yana-dhar-ma-sūtra
bhadrā-sana
bhadrā-sanam
bha-ga-vad-gī-tā
bha-ga-vat-pāda
bhaṭṭot-pala-vi-vṛti-sahitā
bhṛtyāva-satha-saṃ-yuktām
bhū-miṃ
bhu--va-na-dī-pa-ka
bīja-pallava
bodhi-sat-tva-bhūmi
brāhmaṇa-pra-mukha-nānā-sat-tva-vyā-dhi-śānty-ar-tham
brāhmaṇa-pra-mukha-nānā-sat-tve-bhyo
brahmāṇḍa-mahā-purā-ṇa
brahmāṇḍa-mahā-purā-ṇam
brāhma-sphu-ṭa-siddhānta
brahma-vi-hāra
brahma-vi-hāras
bṛhad-āraṇya-ka
bṛhad-yā-trā
bṛhad-yogi-yājña-valkya-smṛti
bṛhad-yogī-yājña-valkya-smṛti
bṛhaj-jāta-kam
cak-ra-dat-ta
cak-ra-pā-ṇi-datta
cā-luk-ya
caraka-prati-saṃ-s-kṛta
caraka-prati-saṃ-s-kṛte
cara-ka-saṃ-hitā
ca-tur-thī-vi-bhak-ti
cau-kham-ba
cau-luk-yas
chau-kham-bha
chun-nam
cikit-sā-saṅ-gra-ha
daiva-jñālaṃ-kṛti
daiva-jñālaṅ-kṛti
darśa-nāṅkur-ābhi-dhayāvyā-khya-yā
deva-nagari
deva-nāgarī
dhar-ma-megha
dhar-ma-meghaḥ
dhyā-na-grahopa-deśā-dhyā-yaś
dṛṣṭ-ān-ta
dṛṣṭ-ār-tha
dvāra-tvam
evaṃ-gṛ-hī-tam
evaṃ-vi-dh-a-sya
gala-gaṇḍa
gala-gaṇḍādi-kar-tṛ-tvaṃ
gan-dh-ā-ra
gar-bha-śa-rī-ram
gaurī-kāñcali-kā-tan-tra
gauta-mādi-tra-yo-da-śa-smṛty-ātma-kaḥ
gheraṇḍa-saṃ-hitā
gran-tha-śreṇi
gran-tha-śreṇiḥ
guru-maṇḍala-grantha-mālā
hari-śāstrī
hari-śās-trī
haṭha-yoga
hāyana-rat-na
hema-pra-bha-sūri
hetv-ābhā-sa
hīna-mithy-āti-yoga
hīna-mithy-āti-yogena
hindī-vyā-khyā-vi-marśope-taḥ
hoern-le
idam
ijya-rkṣe
ikka-vālaga
ity-arthaḥ
jābāla-darśanopa-ni-ṣad
jal-pa-kal-pa-tāru
jam-bū-dvī-pa
jam-bū-dvī-pa-pra-jña-pti
jam-bū-dvī-pa-pra-jña-pti-sūtra
jāta-ka-kar-ma-pad-dhati
jinā-agama-grantha-mālā
jī-vā-nan-da-nam
jñā-na-nir-mala
jñā-na-nir-malaṃ
jya-rkṣe
kāka-caṇḍīśvara-kal-pa-tan-tra
kā-la-gar-bhā-śa-ya-pra-kṛ-tim
kā-la-gar-bhā-śa-ya-pra-kṛ-tiṃ
kali-kāla-sarva-jña
kali-kāla-sarva-jña-śrī-hema-candrācārya-vi-raci-ta
kali-kāla-sarva-jña-śrī-hema-candrācārya-vi-raci-taḥ
kali-yuga
kal-pa-sthāna
kar-ma
kar-man
kārt-snyena
katham
kāvya-mālā
keśa-va-śāstrī
kol-ka-ta
kṛṣṇa-pakṣa
kṛtti-kā
kṛtti-kās
kula-pañji-kā
ku-māra-saṃ-bhava
lab-dhāni
mada-na-phalam
mādha-va
Mādhava-karaaita-reya-brāhma-ṇa
Mādhava-ni-dāna
mādhava-ni-dā-nam
madhu-kośa
madhu-kośākhya-vyā-khya-yā
madhya
madhye
ma-hā-bhū-ta-vi-kā-ra-pra-kṛ-tiṃ
mahā-deva
mahā-mati-śrī-mādhava-kara-pra-ṇī-taṃ
mahā-muni-śrī-mad-vyāsa-pra-ṇī-ta
mahā-muni-śrī-mad-vyāsa-pra-ṇī-taṃ
maha-rṣi-pra-ṇīta-dharma-śāstra-saṃ-grahaḥ
mahā-sacca-ka-sutta
mahau-ṣadhi-pari-cchadāṃ
mahā-vra-ta
mahā-yāna-sūtrālaṅ-kāra
mano-ratha-nandin
matsya-purāṇam
me-dhā-ti-thi
medhā-tithi
mithilā-stha
mithilā-stham
mithilā-sthaṃ
mud-rā-yantr-ā-laye
muktā-pīḍa
mūla-pāṭha
nakṣa-tra
nandi-purāṇoktārogya-śālā-dāna-phala-prāpti-kāmo
nara-siṃha
nara-siṃha-bhāṣya
nārā-ya-ṇa-dāsa
nārā-yaṇa-kaṇṭha
nārā-yaṇa-paṇḍi-ta-kṛtā
nava-pañca-mayor
nidā-na-sthā-na-sya
ni-ghaṇ-ṭu
nir-anta-ra-pa-da-vyā-khyā
nir-ṇaya-sā-gara
nir-ṇaya-sā-gara-yantr-ā-laye
nirūha-vasti
niś-cala-kara
ni-yukta-vaidyāṃ
nya-grodha
nya-grodho
nyāya-śās-tra
nyāya-sū-tra-śaṃ-kar
okaḥ-sātmya
okaḥ-sātmyam
okaḥ-sātmyaṃ
oka-sātmya
oka-sātmyam
oka-sātmyaṃ
oṣṭha-saṃ-puṭa
ousha-da-sala
padma-pra-bha-sūri
padma-sva-sti-kārdha-candrādike
paitā-maha-siddhā-nta
pañca-karma
pañca-karma-bhava-rogāḥ
pañca-karmādhi-kāra
pañca-karma-vi-cāra
pāñca-rātrā-gama
pañca-siddh-āntikā
pari-bhāṣā
pari-likh-ya
pātañ-jala-yoga-śās-tra
pātañ-jala-yoga-śās-tra-vi-varaṇa
pat-añ-jali
pāṭī-gaṇita
pāva-suya
pim-pal-gaon
pipal-gaon
pit-ta-kṛt
pit-ta-śleṣma-ghna
pit-ta-śleṣma-medo-meha-hik-kā-śvā-sa-kā-sāti-sā-ra-cchardi-tṛṣṇā-kṛmi-vi-ṣa-pra-śa-ma-naṃ
prā-cya
prā-cya-hindu-gran-tha-śreṇiḥ
prācya-vidyā-saṃ-śodhana-mandira
pra-dhān-āṅ-gaṃ
pra-dhān-in
pra-ka-shan
pra-kṛ-ti
pra-kṛ-tiṃ
pra-mā-ṇa-vārt-tika
pra-saṅ-khyāne
pra-śas-ta-pāda-bhāṣya
pra-śna-pra-dīpa
pra-śnārṇa-va-plava
praśnārṇava-plava
pra-śna-vaiṣṇava
pra-śna-vai-ṣṇava
prati-padyate
pra-ty-akṣa
pra-yat-na-śai-thilyā-nan-ta-sam-ā-pat-ti-bhyām
pra-yat-na-śai-thilyā-nān-tya-sam-ā-pat-ti-bhyāṃ
pra-yatna-śai-thilya-sya
puṇya-pattana
pūrṇi-mā-nta
rāja-kīya
rajjv-ābhyas-ya
rāma-kṛṣṇa
rasa-ratnā-kara
rasa-vai-śeṣika-sūtra
rogi-svasthī-karaṇānu-ṣṭhāna-mātraṃ
rūkṣa-vasti
sād-guṇya
śākalya-saṃ-hitā
sam-ā-mnāya
sāmañña-pha-la-sutta
sama-ran-gana-su-tra-dhara
samā-raṅga-ṇa-sū-tra-dhāra
sama-ra-siṃ-ha
sama-ra-siṃ-haḥ
saṃ-hitā
sāṃ-sid-dhi-ka
saṃ-śo-dhana
sam-ul-lasi-tam
śāndilyopa-ni-ṣad
śaṅ-kara
śaṅ-kara-bha-ga-vat-pāda
Śaṅ-kara-nārā-yaṇa
saṅ-khyā
sāṅ-kṛt-yā-yana
san-s-krit
sap-tame
śāra-dā-tila-ka-tan-tra
śa-raṅ-ga-deva
śār-dūla-karṇā-va-dāna
śā-rī-ra
śā-rī-ra-sthāna
śārṅga-dhara
śārṅga-dhara-saṃ-hitā
sar-va
sarva-darśana-saṃ-grahaḥ
sar-va-dar-śāna-saṅ-gra-ha
sar-va-dar-śāna-saṅ-gra-haḥ
sarv-arthāvi-veka-khyā-ter
sar-va-tan-tra-sid-dhān-ta
sar-va-tan-tra-sid-dhān-taḥ
sarva-yoga-sam-uc-caya
sar-va-yogeśvareśva-ram
śāstrā-rambha-sam-artha-na
śāstrāram-bha-sam-arthana
ṣaṭ-pañcā-śi-kā
sat-tva
saunda-ra-na-nda
sid-dha
sid-dha-man-tra
sid-dha-man-trā-hvayo
sid-dha-man-tra-pra-kāśa
sid-dha-man-tra-pra-kāśaḥ
sid-dha-man-tra-pra-kāśaś
sid-dh-ān-ta
siddhānta-śiro-maṇ
sid-dha-yoga
sid-dhi-sthāna
śi-va-śar-ma-ṇā
ska-nda-pu-rā-ṇa
sneha-basty-upa-deśāt
sodā-haraṇa-saṃ-s-kṛta-vyā-khyayā
śodha-ka-pusta-kaṃ
śo-dha-na-ci-kitsā
so-ma-val-ka
śrī-mad-devī-bhāga-vata-mahā-purāṇa
srag-dharā-tārā-sto-tra
śrī-hari-kṛṣṇa-ni-bandha-bhava-nam
śrī-hema-candrā-cārya-vi-raci-taḥ
śrī-kaṇtha-datta
śrī-kaṇtha-dattā-bhyāṃ
śrī-kṛṣṇa-dāsa
śrī-mad-amara-siṃha-vi-racitam
śrī-mad-aruṇa-dat-ta-vi-ra-ci-tayā
śrī-mad-bhaṭṭot-pala-kṛta-saṃ-s-kṛta-ṭīkā-sahitam
śrī-mad-dvai-pā-yana-muni-pra-ṇītaṃ
śrī-mad-vāg-bha-ṭa-vi-ra-ci-tam
śrī-maṃ-trī-vi-jaya-siṃha-suta-maṃ-trī-teja-siṃhena
śrī-mat-kalyāṇa-varma-vi-racitā
śrīmat-sāyaṇa-mādhavācārya-pra-ṇītaḥ
śrī-vā-cas-pati-vaidya-vi-racita-yā
śrī-vatsa
śrī-vi-jaya-rakṣi-ta
sthānāṅga-sūtra
sthira-sukha
sthira-sukham
strī-niṣevaṇa
śukla-pakṣa
su-śru-ta-saṃ-hitā
sū-tra
sūtrārthānān-upa-patti-sūca-nāt
sūtra-sthāna
su-varṇa-pra-bhāsot-tama-sū-tra
svalpauṣadha-dāna-mā-tram
śvetāśva-taropa-ni-ṣad
tad-upa-karaṇa-tāmra-kaṭāha-kalasādi-pātra-pari-cchada-nānā-vidha-vyādhi-śānty-ucitauṣadha-gaṇa-yathokta-lakṣaṇa-vaidya-nānā-vidha-pari-cāraka-yutāṃ
tājaka-muktā-valeḥ
tājika-kau-stu-bha
tājika-nīla-kaṇṭhī
tājika-yoga-sudhā-ni-dhi
tāmra-paṭṭādi-li-khi-tāṃ
tan-nir-vāhāya
tapo-dhana
tapo-dhanā
tārā-bhakti-su-dhārṇava
tārtīya-yoga-su-sudhā-ni-dhi
tegi-ccha
te-jaḥ-siṃ-ha
trai-lok-ya
trai-lokya-pra-kāśa
tri-piṭa-ka
tri-var-gaḥ
un-mār-ga-gama-na
upa-de-śa
upa-patt-ti
ut-sneha-na
utta-rā-dhyā-ya-na
uttara-sthāna
uttara-tantra
vāchas-pati
vād-ā-valī
vai-śā-kha
vai-ta-raṇa-vasti
vai-ta-raṇok-ta-guṇa-gaṇa-yu-k-taṃ
vājī-kara-ṇam
vāk-patis
vākya-śeṣa
vākya-śeṣaḥ
varā-ha-mihi-ra
va-ra-na-si
vā-rā-ṇa-sī
var-mam
var-man
varṇa-sam-ā-mnāya
va-siṣṭha-saṃ-hitā
vā-siṣṭha-saṃ-hitā
vasu-bandhu
vasu-bandhu
vāta-ghna-pit-talāl-pa-ka-pha
vātsyā-ya-na
vidya-bhu-sana
vidyā-bhū-ṣaṇa
vi-jaya-siṃ-ha
vi-jñāna-bhikṣu
vi-kal-pa
vi-kamp-i-tum
vi-mā-na-sthāna
vi-racita-yā
vishveshvar-anand
vi-śiṣṭ-āṃśena
viṣṇu-dharmot-tara-purāṇa
viśrāma-gṛha-sahitā
vi-suddhi-magga
vopa-de-vīya-sid-dha-man-tra-pra-kāśe
vyādhi-pratī-kārār-tham
vyāḍī-ya-pa-ri-bhā-ṣā-vṛtti
vyati-krāmati
vy-ava-haranti
yādava-bhaṭṭa
yāda-va-śarma-ṇā
yādava-sūri
yājña-valkya-smṛti
yavanā-cā-rya
yoga-ratnā-kara
yoga-sāra-sam-uc-caya
yoga-sāra-sam-uc-cayaḥ
yoga-sūtra-vi-vara-ṇa
yoga-yājña-valkya
yoga-yājña-valkya-gītāsūpa-ni-ṣatsu
yoga-yājña-valkyaḥ
yogi-yājña-valkya-smṛti
yuk-tiḥ
yuk-tis
}}
\normalfontlatin
\endinput


\lineation{page}
\begingroup
\beginnumbering

\pstart
athātaḥ \edtext{karṇṇavyadhavidhim}{
  \Dfootnote{karṇavyadhabandhavidhim adhyāyaṃ  A.}
} vyākhyāsyāmaḥ\edlabel{SS.1.16.1--2} \edtext{||1||}{
  \linenum{|\xlineref{SS.1.16.1--2}}\lemma{vyākhyāsyāmaḥ}\Afootnote{vyā\ K.}
}
\pend


\pstart
\edtext{}{
  \lemma{\emph{inserted passage}}\Dfootnote{yathovāca bhagavān 
  dhanavatariḥ || A.}
}%
%\pend
%
%
%\pstart
rakṣābhūṣaṇanimittam bālasya karṇṇau vyadhayet\edlabel{SS.1.16.3--4} 
 \edtext{|}{
  \linenum{|\xlineref{SS.1.16.3--4}}\lemma{vyadhayet }\Dfootnote{vidhyete  
  A.}
} tau ṣaṣṭhe \edtext{māse}{
  \Dfootnote{māsi  A.}
} \edtext{saptame}{
  \Afootnote{\omit\ N.}
} vā śuklapakṣe praśasteṣu tithikaraṇamuhūrttanakṣatreṣu 
kṛtamaṅgalaṃ\edlabel{SS.1.16.3--14} \edtext{svastivācanan}{
  \linenum{|\xlineref{SS.1.16.3--14}}\lemma{°maṅgalaṃ 
  svastivācanan}\Dfootnote{°maṅgalasvastivācanaṃ  A.}
} \edtext{dhātryaṅke}{\Afootnote{dhātryaṅkā K.}\Dfootnote{\add\  kumāradharāṅke vā  A.}
} \edtext{kumāram}{
  \Afootnote{kumārakam N.}
} \edtext{upaveśyābhisāntvayamāno}{
  \Dfootnote{upaveśya bālakrīḍanakaiḥ pralobhyābhisāntvayan  A.}
} bhiṣag vāmahastenākṛṣya \edtext{karṇṇan}{
  \Dfootnote{karṇaṃ  A.}
} daivakṛte \edtext{chidre}{
  \Dfootnote{chidra ādityakarāvabhāsite śanaiḥ śanair  A.}
} dakṣiṇahastena \edtext{ṛju}{
  \Afootnote{rjum N H.}
  \Dfootnote{rju A}
} vidhyet | \edtext{pūrvvan}{
  \Dfootnote{pratanukaṃ sūcyā bahalam ārayā pūrvaṃ  A.}
} dakṣiṇaṃ \edtext{kumārasya}{
  \Afootnote{kumārarasya N.}
} vāmaṅ kanyāyāḥ\edlabel{SS.1.16.3--32} \edtext{|}{
  \linenum{|\xlineref{SS.1.16.3--32}}\lemma{kanyāyāḥ}\Dfootnote{kumāryāḥ  
  A.}
} \edtext{pratanuṃ}{
  \Afootnote{pratanū N H.}\Dfootnote{tataḥ A.}
} sūcyā\edlabel{SS.1.16.3--35} \edtext{bahalam}{
  \linenum{|\xlineref{SS.1.16.3--35}}\lemma{sūcyā 
  bahalam}\Dfootnote{picuvartiṃ  A.}
} ārayā\edlabel{SS.1.16.3--37} \edtext{||2||}{
  \linenum{|\xlineref{SS.1.16.3--37}}\lemma{ārayā}\Dfootnote{praveśayet A.}
} 
\footnoteD{Ḍalhaṇa records the alternative reading \emph{bhakṣyaviśeṣair vā} before \emph{bālakrīḍanakaiḥ pralobhya} in the vulgate.}
 \pend


\pstart
\edtext{śoṇitabahutvanivedanāyāñ}{\lemma{°bahutvanivedanāyāñ}\Afootnote{°bahutveti vedanāś N; °bahutvanivedanāyā\uwave{c} H.} \Dfootnote{°bahutvena vedanayā A.}
} cānyadeśaviddham iti jānīyāt | \edtext{nirupadravatā}{
  \Dfootnote{nirupadravatayā  A.}
} taddeśaviddhaliṅgam\edlabel{SS.1.16.4--6} \edtext{||3||
\footnoteD{At this point, witness K is missing a folio, so the rest of this chapter is constructed on the basis of witnesses N and H.}
}{
  \linenum{|\xlineref{SS.1.16.4--6}}\lemma{°viddhaliṅgam}\Dfootnote{°viddham iti  A.}
}
\pend


\pstart
\edtext{tatra\emph{\edlabel{SS.1.16.5--0}}}{
  \Dfootnote{tatrājñena  A.}
} \edtext{yadṛcchāviddhāyāṃ}{
  \Dfootnote{yadṛcchayā viddhāsu  A.}
} \edtext{sirāyām}{
  \Dfootnote{sirāsu kālikāmarmarikālohitikāsūpadravā bhavanti |  A.}
} \edtext{ajñena}{
  \Dfootnote{tatra kālikāyāṃ  A.}
} \edtext{jvara}{\lemma{jvara°}\Dfootnote{jvaro}
}\edtext{dāha}{\lemma{°dāha°}\Dfootnote{dāhaḥ}
}\edtext{śvayathu}{\lemma{°śvayathu° \emph{em.}}
  \Afootnote{°śvaya\uline{thur} N; °śvayathur H.} \Dfootnote{śvayathur A.}
}\edtext{vedanā}{
  \Dfootnote{\add\  ca bhavati marmarikāyāṃ vedanā jvaro  A.}
} \edtext{granthi}{
  \Dfootnote{granthayaś ca lohitikāyāṃ  A.}
} manyāstambhāpatānakaśirograhakarṇṇaśūlāni bhavanti \edtext{||4||}{
  \linenum{|\xlineref{SS.1.16.5--0}}\lemma{tatra\ldots ||4||}\Afootnote{\omit\ K.}
\lemma{||4||}  \Dfootnote{\add\  teṣu yathāsvaṃ pratikurvīt ||  A.}
}
\pend


\pstart
\edtext{doṣasamudayād\emph{\edlabel{SS.1.16.6--0}}}{
  \Dfootnote{doṣamudāyād kliṣṭajihmāpraśastasūcīvyadhād gāḍhataravartitvād  A.}
} apraśastavyadhād \edtext{vā}{
  \Dfootnote{\add\  yatra saṃrambho vedanā vā bhavati  A.}
} tatra varttim \edtext{apahṛtya}{
  \Dfootnote{upahṛtyāśu  A.}
} \edtext{yavamadhukamañjiṣṭhāgandharvvahastamūlair}{\lemma{°gandharvvahastamūlair}
  \Afootnote{°gandarvahastamūlai N.}\lemma{yavamadhukamañjiṣṭhāgandharvvahastamūlair} \Dfootnote{madhukairaṇḍamūlamañjiṣṭhāyavatilakalkair A.}
} mmadhughṛtapragāḍhair ālepayet \edtext{|}{
  \Dfootnote{\add\  tāvad yāvat surūḍha iti ||  A.}
} surūḍhañ cainam punar \edtext{vvidhyet}{\Dfootnote{\add\  vidhānaṃ tu pūrvoktam eva ||  A.}} \edtext{||5||}{
  \linenum{|\xlineref{SS.1.16.6--0}}\lemma{doṣasamudayād\ldots 
  ||5||}\Afootnote{\omit\ K.}
} 
\footnoteD{Ḍalhaṇa (1.16.6) states that some do not read \emph{surūḍhañ cainam punar vidhyet}.}
\pend


\pstart
\edtext{samyag}{\lemma{samyag°}\Dfootnote{tatra samyag° A.}}viddham\edlabel{SS.1.16.7--1} 
āmatailapariṣekeṇopacaret\edlabel{SS.1.16.7--2} \edtext{|}{
  \linenum{|\xlineref{SS.1.16.7--2}}\lemma{°pariṣekeṇopacaret}\Afootnote{°pariṣekaṇopacaret H.}\lemma{āmatailapariṣekeṇopacaret}\Dfootnote{āmatailena pariṣecayet A}
} tryahāt \edtext{tryahād}{
  \Dfootnote{\add\  ca  A.}
} varttiṃ \edtext{sthūlatarāṅ}{
  \Afootnote{sthūlatarīṃ N; sthūlatarīṅ H.}
} \edtext{kurvvīta}{
  \Dfootnote{dadyāt  A.}
} pariṣekañ ca tam eva \edtext{||6||}{
  \linenum{|\xlineref{SS.1.16.7--1}}\lemma{samyagviddham\ldots 
  ||6||}\Afootnote{\omit\ K.}
}
\pend


\pstart
 atha\edlabel{SS.1.16.8--1} vyapagatadoṣopadrave \edtext{karṇṇe}{
  \Afootnote{\add\  la \kakapada\  N; \add\  lam H.}
} \edtext{pravarddhanārthaṃ}{
  \Dfootnote{\omit\ pra°  A.}
} \edtext{laghupravarddhanakena}{\lemma{laghupravarddhanakena \emph{em.}}
  \Afootnote{\uwave{la}pravardhanakāmo N; laghupravarddhanakāmā H.} \Dfootnote{laghuvardhanakaṃ A.}
} muñcet\edlabel{SS.1.16.8--6} \edtext{||7||}{
  \linenum{|\xlineref{SS.1.16.8--1}}\lemma{atha\ldots ||7||}\Afootnote{\omit\ K.}
  \linenum{|\xlineref{SS.1.16.8--6}}\lemma{muñcet}\Dfootnote{kuryāt  A.}
}
\pend

\newpage

\pstart
\begin{verse}
 evaṃ\edlabel{SS.1.16.9--1} \edtext{samvarddhitaḥ}{
  \Dfootnote{vivardhitaḥ  A.}
} karṇṇaś chidyate tu dvidhā nṛṇām\edlabel{SS.1.16.9--7} \edtext{|}{
  \linenum{|\xlineref{SS.1.16.9--7}}\lemma{nṛṇām}\Afootnote{nṛṇā  N.}
} \\
\edtext{}{
  \Afootnote{doṣaṭo N H.}
}doṣato vābhighātād \edtext{vā\edlabel{SS.1.16.9--12}}{
  \Dfootnote{sandhānaṃ  A.}
} \edtext{sandhānān}{
  \linenum{|\xlineref{SS.1.16.9--12}}\lemma{vā 
  sandhānān}\Afootnote{sandhānāntasya N.}
} tasya me \edtext{śṛṇu}{
  \linenum{|\xlineref{SS.1.16.9--1}}\lemma{evaṃ\ldots śṛṇu}\Afootnote{\omit\ K.}
} ||8|| 
\end{verse}
\pend


\pstart
 tatra\edlabel{SS.1.16.10--1} samāsena \edtext{pañcadaśasandhānākṛtayo}{\lemma{°sandhānākṛtayo}
  \Afootnote{°sandhākṛtayo N.}\lemma{°daśasandhānākṛtayo}\Dfootnote{°daśakarṇabandhākṛtayaḥ A.}
} bhavanti\edlabel{SS.1.16.10--4} \edtext{|}{
  \linenum{|\xlineref{SS.1.16.10--4}}\lemma{bhavanti}\Dfootnote{\omit\  A.}
} tad yathā | nemīsandhānakaḥ\edlabel{SS.1.16.10--9} \edtext{|}{
  \linenum{|\xlineref{SS.1.16.10--9}}\lemma{nemī° 
  }\Dfootnote{nemi°  A.}
} utpalabhedyakaḥ | vallūrakaḥ | āsaṅgimaḥ | gaṇḍakarṇṇaḥ | āhāryaḥ | nirvvedhimaḥ | vyāyojimaḥ | kapāṭasandhikaḥ | 
arddhakapāṭasandhikaḥ \edtext{|}{\lemma{arddhakapāṭasandhikaḥ}\lemma{arddhakapāṭasandhikaḥ}\Afootnote{\omit\ N.}\lemma{°sandhikaḥ | arddha°}\Dfootnote{°sandhiko 'rddha°A}
} saṅkṣiptaḥ | hīnakarṇṇaḥ | vallīkarṇṇaḥ | yaṣṭīkarṇṇaḥ \edtext{|}{\lemma{yaṣṭī°}\Dfootnote{yaṣṭi°  A.}
} kākauṣṭhaḥ \edtext{|}{\lemma{kākauṣṭhaḥ}\Afootnote{kākauṣṭha\uwave{bhaḥ} H.}\Dfootnote{kākauṣṭhaka A.}
} iti \edtext{|}{\lemma{iti}\Afootnote{ti H.}
} teṣu \edtext{tatra}{
  \Dfootnote{\omit\  A.}
} \edtext{pṛthulāyatasamobhayapālir}{\lemma{pṛthulāyatasamo°} \Afootnote{pṛthulāyasamo° H; pṛthulātasamo N.}
} nemīsandhānakaḥ \edtext{|} {\lemma{nemī°}\Dfootnote{nemi°  A.}
} vṛttāyatasamobhayapālir utpalabhedyakaḥ\edtext{|}{\lemma{°bhedyakaḥ}\Afootnote{°bhedyaḥ N; °bhedakaḥ H.}
} hrasvavṛttasamobhayapālir vallūrakarṇṇakaḥ\edtext{|}{\lemma{vallūra°}\Afootnote{valūra° N.}\Dfootnote{vallūrakaḥ  A.}
} abhyantaradīrghaikapālir āsaṅgimaḥ | \edtext{bāhya}{\Afootnote{bāhyaika N H.}
}dīrghaikapālir ggaṇḍakarṇṇakaḥ | apālir \edtext{ubhayato'py}{\Afootnote{°to py N.}
} āhāryaḥ | \edtext{pīṭhopamapālir}{\Dfootnote{\add\ ubhayataḥ kṣīṇaputrikāśrito  A.}
} nirvvedhimaḥ | \edtext{aṇusthūlasamaviṣamapālir}{\lemma{aṇusthūla°}\Afootnote{aśusthūla° H}
  \Dfootnote{sthūlāṇu°  A.}
} vyāyojimaḥ | abhyantaradīrghaikapālir itarālpapāliḥ 
kapāṭasandhikaḥ \edtext{|}{\lemma{kapāṭa°}\Afootnote{kavāṭā° H.}
} bāhyadīrghaikapālir itarālpapāliś cārddhakapāṭasandhikaḥ \edtext{|}{
 \lemma{cārddhakapāṭa° \emph{em.}}\Afootnote{vārddhakavāṭa° H; cārddhakavāpa° N.}
\lemma{cārddha°} \Dfootnote{ardha° A.}
} \edtext{tatraite}{
  \Dfootnote{tatra  A.}
} \edtext{daśakarṇṇasandhivikalpā}{
  \Dfootnote{daśaite karṇabandhavikalpāḥ  A.}
} bandhyā bhavanti \edtext{|}{\lemma{bandhyā bhavanti}\Dfootnote{sādhyāḥ  A.}
} \edtext{teṣān}{\Dfootnote{teṣāṃ  A.}
} \edtext{nāmabhir}{
  \Dfootnote{svanā°  A.}
} evākṛtayaḥ prāyeṇa vyākhyātāḥ | saṃkṣiptādayaḥ pañcāsādhyāḥ | tatra \edtext{śuṣkaśaṣkulir}{
  \Dfootnote{\add\  utsannapālir  A.}
} itarālpapāliḥ saṃkṣiptaḥ | anadhiṣṭhānapāliḥ 
paryantayoś \edtext{ca}{
\lemma{paryantayoś ca}\Afootnote{\omit\ N.}
\lemma{ca}  \Dfootnote{\omit\  A.}
} kṣīṇamāṃso hīnakarṇṇaḥ | \edtext{tanuviṣamapālir}{
  \Dfootnote{°ṣamālpapālir  A.}
} vallīkarṇṇaḥ | \edtext{granthitamāṃsaḥ}{\Afootnote{granthitamānsaḥ N H.}\Dfootnote{granthitamāṃsa° A.}
} \edtext{stabdhasirātatasūkṣmapālir}{\lemma{stabdhasirātatasūkṣma°}\Dfootnote{stabdhasirāsaṃtatasūkṣma°  A.}
} yaṣṭīkarṇṇaḥ | \edtext{nirmāṃsasaṃkṣiptāgrālpaśoṇitapāliḥ}{\lemma{nirmāṃsa°}\Afootnote{nimāsa° N; nirmmānsa° H.}
} \edtext{kākauṣṭha}{\Dfootnote{kākauṣṭhaka  A.}
} iti | baddheṣv \edtext{api}{\Dfootnote{\add\  tu śopha  A.}
} \edtext{dāha}{\lemma{°dāha°}\Dfootnote{\add\  °rāga°  A.}
}\edtext{pāka}{\lemma{°pāka°}
  \Dfootnote{\add\  °piḍakā°  A.}
}\edtext{srāva}{\lemma{°srāva°}
  \Afootnote{°śrāva° H.}
}\edtext{śopha}{\lemma{°śopha°}
  \Afootnote{°sopha° N.}\Dfootnote{\omit\ A.} 
}yuktā na siddhim upayānti \edtext{||9||}{
  \linenum{|\xlineref{SS.1.16.10--1}}\lemma{tatra\ldots ||9||}\Afootnote{\omit\ K.}
} 
\footnoteD{Cakrapāṇi (1.16.9–13) and Ḍalhaṇa (1.16.10) point out that others read \emph{pañcadaśakarṇakṛtayaḥ} (instead of \emph{pañcadaśasandhānākṛtayaḥ}). Ḍalhaṇa (1.16.10) also mentions that some read \emph{samunnatasamobhayapāliḥ} (instead of \emph{vṛttāyatasamobhayapālir}) and others do not read \emph{saṃkṣiptādayaḥ pañcāsādhyāḥ}.}
 \pend


\pstart
\edtext{}{
  \lemma{\emph{inserted passage}}\Dfootnote{bhavanti cātra | yasya pālidvayam api karṇasya 
  na bhaved iha | karṇapīṭhaṃ same madhye 
  tasya viddhvā vivardhayet || bāhyāyām iha dīrghāyāṃ 
  sandhir ābhyantaro bhavet | ābhyantarāyāṃ dīrghāyāṃ 
  bāhyasandhir udāhṛtaḥ || ekaiva tu bhavet pāliḥ sthūlā 
  pṛthvī sthirā ca yā | tāṃ dvidhā pāṭayitvā tu chittvā 
  copari sandhayet || gaṇḍād utpāṭya māṃsena 
  sānubandhena jīvatā | karṇapālīm āpāles tu kuryān 
  nirlikhya śāstravit || A.}
}

\pend


\pstart
 \edtext{ato\edlabel{SS.1.16.15--1}}{
  \Afootnote{tato N.}
} \edtext{'nyatamasya}{
  \Dfootnote{'nyatamaṃ  A.}
} bandhañ cikīrṣuḥ \edtext{agropaharaṇīyoktopasambhṛtasambhāraḥ}{\lemma{agropaharaṇīyoktopa°}
  \Afootnote{agropasaṃharaṇīyoktopa° N.}\lemma{°sambhāraḥ}\Dfootnote{°sambhāraṃ  A.}
} viśeṣataś \edtext{cātropaharet}{
  \Afootnote{cāgropaharaṇīyāt \uline{N} \uline{H}.}
} \edtext{surāmaṇḍakṣīram}{
  \Dfootnote{surāmaṇḍaṃ kṣīram  A.}
} udakaṃ \edtext{dhānyāmlakapālacūrṇṇañ}{
  \Dfootnote{dhānyāmlaṃ ka°  A.}
} ceti | tato 'ṅganāṃ \edtext{puruṣam}{
  \Afootnote{puruṣañ N.}
} vā grathitakeśāntaṃ laghubhuktavantam āptaiḥ \edtext{suparigṛhītaṃ}{
  \Dfootnote{\add\  ca  A.}
} kṛtvā\edlabel{SS.1.16.15--21} \edtext{ca}{
  \linenum{|\xlineref{SS.1.16.15--21}}\lemma{kṛtvā ca}\Afootnote{\omit\ N H.}
\lemma{ca}  \Dfootnote{\omit\  A.}
} \edtext{bandhān}{
  \Dfootnote{bandham  A.}
} \edtext{upadhārya}{
  \Afootnote{upapādya H.}
} \edtext{chedyabhedyalekhyavyadhanair}{
  \Dfootnote{\add\  upapannair  A.}
} upapādya karṇṇaśoṇitam\edlabel{SS.1.16.15--27} \edtext{avekṣyaitad}{
  \linenum{|\xlineref{SS.1.16.15--27}}\lemma{karṇṇaśoṇitam 
  avekṣyaitad}\Afootnote{°ṇitata avekṣyetad N.}
\lemma{avekṣyaitad}  \Dfootnote{avekṣya  A.}
} duṣṭam \edtext{aduṣṭam}{
  \Afootnote{aduṣṭaś N.}
} \edtext{veti}{
  \Afootnote{ceti | N H.}
} \edtext{tato}{
  \Dfootnote{tatra  A.}
} vātaduṣṭe \edtext{dhānyāmlodakābhyāṃ}{
  \Dfootnote{dhānyāvloda° N; dhānyām loda° H; dhānyāmloṣṇoda°  A.}
} pittaduṣṭe \edtext{śītodakapayobhyāṃ}{
  \Afootnote{śītodakopa° N.}
} śleṣmaduṣṭe \edtext{surāmaṇḍodakābhyāṃ}{
  \Dfootnote{°ḍoṣṇodakābhyāṃ  A.}
} prakṣālya \edtext{karṇṇam}{
  \Dfootnote{karṇau  A.}
} punar avalikhet\edlabel{SS.1.16.15--42} | \edtext{anunnatam}{
  \linenum{|\xlineref{SS.1.16.15--42}}\lemma{avalikhet\ldots 
  anunnatam}\Afootnote{avalikhyānun° A; °kheta | anunnatam N.}
} ahīnam aviṣamañ ca \edtext{karṇṇasandhin}{
  \Afootnote{karṇasandhiṃ A \uline{N}.}
} \edtext{niveśya}{
  \Dfootnote{sanni°  A.}
} sthitaraktaṃ \edtext{sandarśya}{
  \Dfootnote{sandadhyāt | tato  A.}
} madhughṛtenābhyajya picuplotayor \edtext{anyatareṇāvaguṇṭhya}{
  \Afootnote{°gu\textsc{(l. 4)}ṇṭhyo H.}
} \edtext{nātigāḍhan}{
  \Afootnote{sūtreṇānavagāḍhaman A; nātigāḍhaṃ N.}
} \edtext{nātiśithilaṃ}{
  \Dfootnote{ati°  A.}
} \edtext{sūtreṇāvabadhya}{
  \Afootnote{ca baddhvā A; °baddha N.}
} kapālacūrṇṇenāvakīryācārikam upadiśet | 
dvivraṇīyoktena\edlabel{SS.1.16.15--61} 
cānnenopacaret\edlabel{SS.1.16.15--62} \edtext{||10||}{
  \linenum{|\xlineref{SS.1.16.15--1}}\lemma{ato\ldots ||10||}\Afootnote{\omit\ K.}
  \linenum{|\xlineref{SS.1.16.15--61}}\lemma{dvivraṇīyoktena\ldots 
  ||10||}\Afootnote{dvivaṇīyoktena \uwave{upapocaret} N.}
  \linenum{|\xlineref{SS.1.16.15--62}}\lemma{cānnenopacaret 
  ||10||}\Dfootnote{ca vidhāne°  A.}
}
\pend


\pstart
\begin{verse}
\edtext{}{
  \lemma{\emph{inserted passage}}\Dfootnote{bhavati cātra | A; \add\
  || bha || N.}
}%
%\pend
%
%
%\pstart
\edtext{vighaṭṭanan\edlabel{SS.1.16.16x-1}}{
  \Afootnote{vighaṭṭanaṃ N.}\Dfootnote{vighaṭṭanaṃ A.}
} divāsvapnaṃ vyāyāmam atibhojanam |\\  
\edtext{vyavāyam}{
  \Afootnote{āgnisantā\textsc{(f. 14v)}pa N.}
} agnisantāpam vākśramañ ca\edlabel{SS.1.16.16x-10} \edtext{vivarjjayet}{
  \linenum{|\xlineref{SS.1.16.16x-1}}\lemma{vighaṭṭanan\ldots 
  vivarjjayet}\Afootnote{\omit\ K.}
  \linenum{|\xlineref{SS.1.16.16x-10}}\lemma{ca vivarjjayet}\Afootnote{varjayet || N.}
} ||11|| 
\end{verse}
\pend


\pstart
 \edtext{nātiśuddharaktam\edlabel{SS.1.16.17--1}}{
  \Afootnote{\textsc{[sū.16.17]} na cāśu° A; nātisuddha° N.}
} \edtext{atipravṛttaraktaṃ}{
  \Afootnote{°vṛttaṃ raktaṃ N.}
} kṣīṇaraktaṃ vā sandadhyāt | sa hi vātaduṣṭe \edtext{raktabaddho'rūḍho}{
  \Afootnote{rakte rūḍho 'pi A; raktavaddho ruḍho N; raktabaddho rūḍho H.}
} \edtext{paripuṭanavān}{
  \Afootnote{°ṭavām N; °navā\textsc{(f. 33r)}m H.}
} bhavati\edlabel{SS.1.16.17--12} \edtext{|}{
  \linenum{|\xlineref{SS.1.16.17--12}}\lemma{bhavati |}\Dfootnote{\omit\  A.}
} \edtext{pittaduṣṭe}{
  \Afootnote{pittaduṣṭai N.}
} gāḍhapākarāgavān\edlabel{SS.1.16.17--15} \edtext{|}{
  \linenum{|\xlineref{SS.1.16.17--15}}\lemma{gāḍhapākarāgavān 
  |}\Dfootnote{dāhapākarāgavedanāvān  A.}
} \edtext{śleṣmaduṣṭe}{
  \Afootnote{śleṣaduṣṭe N.}
} \edtext{stabdhakarṇṇaḥ}{
  \Afootnote{stabdhaḥ A; stabdhavarṇṇaḥ N.}
} kaṇḍūmān \edtext{atipravṛttasrāvaḥ\edlabel{SS.1.16.17--20}}{
  \Afootnote{°ttaśrāvaḥ H.}
} \edtext{śophavān}{
  \linenum{|\xlineref{SS.1.16.17--20}}\lemma{atipravṛttasrāvaḥ 
  śophavān}\Dfootnote{°ttarakte śyāvaśophavān  A.}
} \edtext{kṣīṇālpamāṃso}{
  \Afootnote{kṣīṇo lpa° \uline{A} N.}
} na vṛddhim upaiti \edtext{||12||}{
  \linenum{|\xlineref{SS.1.16.17--1}}\lemma{nātiśuddharaktam\ldots 
  ||12||}\Afootnote{\omit\ K.}
}
\pend


\pstart
\edtext{}{
  \lemma{\emph{inserted passage}}\Dfootnote{āmatailena trirātraṃ 
  pariṣecayet trirātrāc ca picuṃ parivartayet | A.}
} sa\edlabel{SS.1.16.18--1} yadā \edtext{rūḍho}{
  \Afootnote{surūḍho A; ruḍho N.}
} nirupadravaḥ \edtext{karṇṇo}{
  \Dfootnote{savarṇo  A.}
} bhavati tadainaṃ śanaiḥ śanair abhivarddhayet | \edtext{anyathā}{
  \Dfootnote{ato 'nyathā  A.}
} \edtext{saṃrambhadāhapākavedanāvān}{
  \Afootnote{°karāgavedanāvān A N; °nāvām H.}
} bhavati\edlabel{SS.1.16.18--14} \edtext{|}{
  \linenum{|\xlineref{SS.1.16.18--14}}\lemma{bhavati |}\Dfootnote{\omit\  A.}
} punar \edtext{api}{
  \Dfootnote{\omit\  A.}
} chidyeta\edlabel{SS.1.16.18--18} \edtext{||13||}{
  \linenum{|\xlineref{SS.1.16.18--1}}\lemma{sa\ldots ||13||}\Afootnote{\omit\ K.}
  \linenum{|\xlineref{SS.1.16.18--18}}\lemma{chidyeta 
  ||13||}\Dfootnote{chidyate 
  vā ||  A.}
}
\pend


\pstart
 \edtext{athāpraduṣṭasyābhivarddhanārtham\edlabel{SS.1.16.19--1}}{
  \Afootnote{athāsyāḥ\textsc{(l. 3)} pra° H; \textsc{[sū.16.19]} athāsyāpra° A; °syāvivardhanārtham N.}
} abhyaṅgaḥ \edtext{|}{
  \Dfootnote{\add\  tad yathā  A.}
} \edtext{godhāpratudaviṣkirānūpaudakavasāmajjāpayastailaṃ}{
  \Dfootnote{°jjānau payaḥ sarpis tailaṃ  A.}
} \edtext{gaurasarṣapajañ}{
  \Afootnote{\omit\ gaura° N.}
} ca yathālābhaṃ \edtext{saṃbhṛtyārkālarkabalātibalānantāvidārīmadhukajalaśūkaprativāpan}{
  \Dfootnote{°lakavalātibalānantāvidārīmadhukajalaśūkaprativāpaṃ N; 
  °tāpāmārgāśvagandhāvidārigandhākṣīraśuklājalaśūkamadhuravargapayasyāprativāpaṃ
    A.}
} \edtext{tailam}{
  \Dfootnote{\add\  vā  A.}
} pācayitvā \edtext{svanuguptan}{
  \Afootnote{svanuguptaṃ A N.}
} nidadhyāt\edlabel{SS.1.16.19--12} \edtext{||14||}{
  \linenum{|\xlineref{SS.1.16.19--1}}\lemma{athāpraduṣṭasyābhivarddhanārtham\ldots
   ||14||}\Afootnote{\omit\ K.}
  \linenum{|\xlineref{SS.1.16.19--12}}\lemma{nidadhyāt 
  ||14||}\Afootnote{nidadyāt || N.}
} 
\footnoteD{Ḍalhaṇa (1.16.18) notes that some read \emph{rājasarṣapajaṃ} in the place of \emph{gaurasarṣapajaṃ}. This reading appears to have been accepted by Cakrapāṇi (1.16.18–20), who glosses \emph{rājasarṣapaja} as \emph{śvetasarṣapa}. Cakrapāṇi also says that some read sarpis in the place of \emph{payas}. In the compound beginning with \emph{arka}, Ḍalhaṇi notes that some read \emph{arkapuṣpī}.}
 \pend


\pstart
\begin{verse}
 \edtext{svedito\edlabel{SS.1.16.20--1}}{
  \Afootnote{svadito N.}
} \edtext{marditaṅ}{
  \linenum{|\xlineref{SS.1.16.20--1}}\lemma{svedito 
  marditaṅ}\Dfootnote{\textsc{[sū.16.20]}sveditonma°  A.}
} karṇṇam \edtext{anena\edlabel{SS.1.16.20--4}}{
  \Afootnote{ane \kakapada\ N.}
} \edtext{mrakṣayed}{
  \linenum{|\xlineref{SS.1.16.20--4}}\lemma{anena 
  mrakṣayed}\Dfootnote{snehenaitena yojayet |   A.}
} budhaḥ\edlabel{SS.1.16.20--6} | \edtext{}{
  \linenum{|\xlineref{SS.1.16.20--6}}\lemma{budhaḥ\ldots }
  \Dfootnote{ato 'nu°  
  A.}
\lemma{}  \Afootnote{tato nu° H; tato nupadravam N.}
}\\ 
tato'nupadravaḥ samyag balavāṃś ca \edtext{vivarddhate}{
  \linenum{|\xlineref{SS.1.16.20--1}}\lemma{svedito\ldots 
  vivarddhate}\Afootnote{\omit\ K.}
} ||15||
\footnoteD{N has a \emph{kākapāda} after \emph{ane}, but the missing letter (one would expect '\emph{na}') has not been supplied in a margin or elsewhere.}
 \end{verse}
\pend
%

\pstart
\begin{verse}
\edtext{}{
  \lemma{\emph{inserted passage}}\Dfootnote{yavāśvagandhāyaṣṭyāhvais tilaiś 
  codvartanaṃ hitam |  śatāvaryaśvagandhābhyāṃ payasyair aṇḍajīvanaiḥ || A.}
}%
%\pend
%
%
%\pstart
\edtext{}{
  \lemma{\emph{inserted passage}}\Dfootnote{tailaṃ vipakvaṃ sakṣīram 
  abhyaṅgāt pālivardhanam |  A.}
}ye\edlabel{SS.1.16.22--1} tu karṇṇā na varddhante 
snehasvedopapāditāḥ\edlabel{SS.1.16.22--6} \edtext{|}{
  \linenum{|\xlineref{SS.1.16.22--1}}\lemma{ye\ldots |}\Afootnote{\omit\ K.}
  \linenum{|\xlineref{SS.1.16.22--6}}\lemma{snehasvedopapāditāḥ 
  |}\Dfootnote{svedasnehopa°  A.}
}
\\
 teṣām\edlabel{SS.1.16.23--1} \edtext{apāṅge}{
  \Dfootnote{apāṅgadeśe  A.}
} tv\edlabel{SS.1.16.23--3} \edtext{abahiḥ}{
  \linenum{|\xlineref{SS.1.16.23--3}}\lemma{tv abahiḥ}\Dfootnote{tu  A.}
\lemma{abahiḥ}  \Afootnote{avarhi N.}
} \edtext{kuryāt}{
  \Afootnote{kuyāt N.}
} \edtext{prachānam}{
  \Afootnote{prachannam H.}
} eva ca\edlabel{SS.1.16.23--8} \edtext{||16||
\footnoteD{Ḍalhaṇa (1.16.23) notes that some read \emph{teṣām apāṅgacchedyaṃ hi kāryam ābhyantaraṃ bhavet}.}
}{
  \linenum{|\xlineref{SS.1.16.23--1}}\lemma{teṣām\ldots ||16||}\Afootnote{\omit\ 
  K.}
  \linenum{|\xlineref{SS.1.16.23--8}}\lemma{ca ||16||}\Dfootnote{tu |  A.}
}
\end{verse}
\pend


\pstart\begin{verse}
\edtext{}{
  \lemma{\emph{inserted passage}}\Dfootnote{bāhyacchedaṃ na kurvīta 
  vyāpadaḥ syus tato dhruvāḥ || A.}
}%
%\pend
%
%
%\pstart
\edtext{}{
  \lemma{\emph{inserted passage}}\Dfootnote{baddhamātraṃ tu yaḥ karṇaṃ 
  sahasaivābhivardhayet |  āmakośī samādhmātaḥ kṣipram eva vimucyate || A.}
}%
%\pend
%
%
%\pstart
amitāḥ\edlabel{SS.1.16.26.0--1} \edtext{karṇṇabandhās}{
  \Afootnote{karṇṇabandho H.}
} \edtext{tu}{
  \Afootnote{stu H.}
} vijñeyāḥ kuśalair iha |\\  
yo \edtext{yathā}{
  \Dfootnote{suviśiṣṭaḥ  A.}
} suniviṣṭaḥ \edtext{syāt}{
  \Dfootnote{taṃ  A.}
} tat \edtext{tathā\edlabel{SS.1.16.26.0--14}}{
  \Afootnote{yojaye N.}
} yojayed \edtext{bhiṣak}{
  \linenum{|\xlineref{SS.1.16.26.0--1}}\lemma{amitāḥ\ldots 
  bhiṣak}\Afootnote{\omit\ K.}
  \linenum{|\xlineref{SS.1.16.26.0--14}}\lemma{tathā\ldots 
  bhiṣak}\Dfootnote{viniyojayet ||  A.}
} ||17||
\footnoteD{Ḍalhaṇa (1.16.26) states that some read \emph{suniviṣṭaḥ} (the reading of the Nepalese version) instead of \emph{suviśiṣṭaḥ}.}
 \end{verse} 
\pend


\pstart
\edtext{}{
  \lemma{\emph{inserted passage}}\Dfootnote{(karṇapālyāmayānnṇnāṃ punar 
  vakṣyāmi suśruta |  karṇapalyāṃ prakupaitā vātapittakaphās trayaḥ || A.}
}
%\pend
%
%\pstart
\edtext{}{
  \lemma{\emph{inserted passage}}\Dfootnote{dvidhā vāpyatha saṃsṛṣṭāḥ 
  kurvanti vividhā rujaḥ |  visphoṭaḥ stabdhatā śophaḥ pālyāṃ doṣe tu vātike ||  
  dāhavisphiṭajananaṃ śophaḥ pākaś ca paittike |  kaṇḍūḥ saśvayathuḥ stambho 
  gurutvaṃ ca kaphātmake || A.}
}
%\pend
%
%\pstart
\edtext{}{
  \lemma{\emph{inserted passage}}\Dfootnote{yathādoṣaṃ ca saṃśodhya 
  kuryātteṣāṃ cikitsitam |  svedābhyaṅgaparīṣekaiḥ pralepāsṛgvimokṣaṇaiḥ || A.}
}
%\pend
%
%\pstart
\edtext{}{
  \lemma{\emph{inserted passage}}\Dfootnote{mṛdvīṃ kriyāṃ bṛṃhaṇīyair 
  yathāsvaṃ bhojanais tathā |  ya evaṃ vetti doṣāṇāṃ cikitsāṃ kartum arhati || 
  A.}
}
%\pend
%
%
%\pstart
\edtext{}{
  \lemma{\emph{inserted passage}}\Dfootnote{ata ūrdhvaṃ nāmaliṅgair vakṣye 
  pālyām upadravān |  atpāṭakaś cotpuṭakaḥ śyāvaḥ kaṇḍūyuto bhṛśam || A.}
}
%\pend
%
%
%\pstart
\edtext{}{
  \lemma{\emph{inserted passage}}\Dfootnote{avamanthaḥ sakaṇḍūko 
  granthiko jambulas tathā |  srāvī ca dāhavāṃś caiva śṛṇveṣāṃ kramaśaḥ kriyām 
  || A.}
}
%\pend
%
%
%\pstart
\edtext{}{
  \lemma{\emph{inserted passage}}\Dfootnote{apāmārgaḥ sarjarasaḥ 
  pāṭalālakucatvacau |  utpāṭake pralepaḥ syāttailamebhiś ca pācayet || A.}
}
%\pend
%
%
%\pstart
\edtext{}{
  \lemma{\emph{inserted passage}}\Dfootnote{śampākaśigrupūtīkān godāmedo 
  'tha tadvasām |  vārāhaṃ gavyamaiṇeyaṃ pittaṃ sarpiś ca saṃsṛjet || A.}
}
%\pend
%
%
%\pstart
\edtext{}{
  \lemma{\emph{inserted passage}}\Dfootnote{lepam utpuṭake 
  dadyāttailamebhiś ca sādhitam |  gaurīṃ sugandhāṃ saśyāmāmanantāṃ 
  taṇḍulīyakam | A.}
}
%\pend
%
%
%\pstart
\edtext{}{
  \lemma{\emph{inserted passage}}\Dfootnote{śyāve pralepanaṃ 
  dadyāttailamebhiś ca sādhitam |  pāṭhāṃ rasāñjanaṃ kṣaudraṃ tathā 
  syāduṣṇakāñjikam || A.}
}
%\pend
%
%
%\pstart
\edtext{}{
  \lemma{\emph{inserted passage}}\Dfootnote{dadyāl lepaṃ sakaṇḍūke 
  tailamebhiś ca sādhitam |  vraṇībhūtasya deyaṃ syādidaṃ tailaṃ vijānatā || A.}
}
%\pend
%
%
%\pstart
\edtext{}{
  \lemma{\emph{inserted passage}}\Dfootnote{madhukakṣīrakākolījīvakādyair 
  vipācitam |  godhāvarāhasarpāṇāṃ vasāḥ syuḥ kṛtabṛṃhaṇe || A.}
}
%\pend
%
%
%\pstart
\edtext{}{
  \lemma{\emph{inserted passage}}\Dfootnote{pralepanam idaṃ dadyād 
  avasicyāvamanthake |  prapauṇḍarīkaṃ madhukaṃ samaṅgāṃ dhavam eva ca 
  || A.}
}
%\pend
%
%
%\pstart
\edtext{}{
  \lemma{\emph{inserted passage}}\Dfootnote{tailam ebhiś ca saṃpakvaṃ 
  śṛṇu kaṇḍūmataḥ kriyām |  sahadevā viśvadevā ajākṣīraṃ sasaindhavam |  etair 
  ālepanaṃ dadyāt tailam ebhiś ca sādhitam || A.}
}
%\pend
%
%
%\pstart
\edtext{}{
  \lemma{\emph{inserted passage}}\Dfootnote{granthike guṭikāṃ pūrvaṃ 
  srāvayed avapāṭya tu |  tataḥ saindhavacūrṇaṃ tu ghṛṣṭvā lepaṃ pradāpayet || 
  A.}
}
%\pend
%
%
%\pstart
\edtext{}{
  \lemma{\emph{inserted passage}}\Dfootnote{likhitvā tatsrutaṃ ghṛṣṭvā 
  cūrṇair lodhrasya jambule |  kṣīreṇa pratisāryainaṃ śuddhaṃ saṃropayet tataḥ 
  || A.}
}
%\pend
%
%
%\pstart
\edtext{}{
  \lemma{\emph{inserted passage}}\Dfootnote{madhuparṇī madhūkaṃ ca ma 
  madhukaṃ madhunā saha |  lepaḥ srāviṇi dātavyas tailam ebhiś ca sādhitam || 
  A.}
}
%\pend
%
%
%\pstart
\edtext{}{
  \lemma{\emph{inserted passage}}\Dfootnote{pañcavalkaiḥ samadhukaiḥ 
  piṣṭais taiś ca ghṛtānvitaiḥ |  jīvakādyaiḥ sasarpiṣkair dahyamānaṃ pralepayet ||) 
  A.}
}
\pend


\pstart
\begin{verse}
jātaromā\emph{\edlabel{SS.1.16.25--0}} \edtext{suvartmā}{
  \Afootnote{suparmā N; suvarmmā H.}
} ca \edtext{śliṣṭasandhiḥ}{
  \Afootnote{śliṣṭasandhim N.}
} samaḥ sthiraḥ |\\
surūḍho 'vedano \edtext{yas}{
  \Dfootnote{ca  A.}
} \edtext{tu}{
  \Afootnote{tat N H.}
} taṃ karṇṇaṃ varddhayec \edtext{chanaiḥ}{
  \linenum{|\xlineref{SS.1.16.25--0}}\lemma{jātaromā\ldots 
  chanaiḥ}\Afootnote{\omit\ K.}
} ||18|| 
\end{verse}
\pend


\pstart
\begin{verse}
\edtext{viśleṣitāyām\emph{\edlabel{SS.1.16.27--0}}}{
  \Dfootnote{°tāyās tv  A.}
} atha \edtext{nāsikāyāṃ}{
  \Afootnote{nāsikāyā  A N.}
}  vakṣyāmi sandhānavidhiṃ yathāvat
\edtext{|}{
  \Afootnote{°māṇa N.}
} \edtext{}{
  \Afootnote{°hāṇam N.}
}\\ 
\edtext{nāsāpramāṇaṃ}{
  \Afootnote{patra N.}
} pṛthivīruhāṇāṃ  patraṃ gṛhītvā \edtext{tv}{
  \linenum{|\xlineref{SS.1.16.27--0}}\lemma{viśleṣitāyām\ldots 
  tv}\Afootnote{\omit\ K.}
} avalambi tasya ||19||
\footnoteD{Cakrapāṇidatta says that others read \emph{nāsāsandhānavidhim} here. Ḍalhaṇa (1.16.27–31) states that some read, \emph{chinnāṃ tu nāsikāṃ dṛṣṭvā vayaḥsthasya śarīriṇaḥ | nāsānurūpaṃ saṃcchidya patraṃ gaṇḍe niveśayet ||}}
 \end{verse}
\pend


\pstart
\begin{verse}
 tena\edlabel{SS.1.16.28--1} pramāṇena hi gaṇḍapārśvād  \edtext{utkṛtya}{
  \Afootnote{baddhaṃ A; vandhra H.}
} vadhraṃ tv atha nāsikāgram |\\  
vilikhya \edtext{cāśu}{
  \Dfootnote{tat  A.}
} \edtext{pratisandadhīta}{
  \Afootnote{sādhubandhair A; sādhuvaddha N.}
}  taṃ \edtext{sādhubaddham}{
  \linenum{|\xlineref{SS.1.16.28--1}}\lemma{tena\ldots 
  sādhubaddham}\Afootnote{\omit\ K.}
} bhiṣag apramattaḥ ||20|| 
\end{verse}
\pend


\pstart
\begin{verse}
\edtext{susīvitaṃ\emph{\edlabel{SS.1.16.29--0}}}{
  \Afootnote{\textsc{[sū.16.29]} susaṃhitaṃ A; susīvita N; suśīvitaṃ H.}
} samyag ato yathāvan  nāḍīdvayenābhisamīkṣya\edlabel{SS.1.16.29--5} 
\edtext{nahyet}{
  \linenum{|\xlineref{SS.1.16.29--5}}\lemma{nāḍīdvayenābhisamīkṣya 
  nahyet}\Dfootnote{baddhvā |   A.}
} |\edlabel{SS.1.16.29--7} \edtext{}{
  \linenum{|\xlineref{SS.1.16.29--7}}\lemma{| }\Dfootnote{prānnamya cainām  
  A.}
}\\ 
\edtext{unnāmayitvā}{
  \Dfootnote{°rṇayet tu   A.}
} \edtext{tv}{
  \Afootnote{pattrāṅga° H; pataṅga° A \uline{N}.}
} avacūrṇṇayīta \edtext{}{
  \linenum{|\xlineref{SS.1.16.29--0}}\lemma{susīvitaṃ\ldots }\Afootnote{\omit\ 
  K.}
} pattāṅgayaṣṭīmadhukāñjanaiś ca ||21|| 
\end{verse}
\pend


\pstart
\begin{verse}
saṃchādya\emph{\edlabel{SS.1.16.30--0}} samyak picunā 
\edtext{vraṇan\edlabel{SS.1.16.30--3}}{
  \Afootnote{vraṇa N.}
} \edtext{tu}{
  \linenum{|\xlineref{SS.1.16.30--3}}\lemma{vraṇan tu}\Dfootnote{sitena   A.}
\lemma{tu}  \Afootnote{tun  N.}
}  tailena siñced asakṛt tilānām |\\  
ghṛtañ ca pāyyaḥ sa naraḥ \edtext{sujīrṇṇe}{
  \Afootnote{virecya N H.}
} \edlabel{SS.1.16.30--18} \edtext{snigdho}{
  \linenum{|\xlineref{SS.1.16.30--0}}\lemma{saṃchādya\ldots 
  snigdho}\Afootnote{\omit\ K.}
  \linenum{|\xlineref{SS.1.16.30--18}}\lemma{ snigdho}\Afootnote{sa ya° A; 
  °deśaḥ || N.}
} virecyaḥ svayathopadeśam ||22|| 
\end{verse}
\pend


\pstart
\begin{verse}
rūḍhañ\emph{\edlabel{SS.1.16.31--0}} ca \edtext{sandhānam}{
  \Afootnote{sandhām N.}
} \edtext{upāgataṃ}{
  \Afootnote{upāgataś H.}
} \edtext{vai}{
  \Afootnote{syāt  A; cai  H.}
} \edtext{}{
  \Afootnote{tad ardhaśeṣaṃ A; tadvadhraseṣan N.}
} tadvadhraśeṣaṃ tu punar nikṛntet | 
\\
\edtext{hīnam}{
    \Dfootnote{hīnāṃ  A.}
}   \edtext{punar}{
  \Afootnote{yatetaḥ  N.}
} \edtext{varddhayituṃ}{
  \Dfootnote{samāṃ  A.}
} yateta  \edtext{samañ\edlabel{SS.1.16.31--17}}{
  \Afootnote{°ddhamānsam N.}
} ca kuryād\edlabel{SS.1.16.31--19} ativṛddhamāṃsam \edtext{iti}{
  \linenum{|\xlineref{SS.1.16.31--0}}\lemma{rūḍhañ\ldots iti}\Afootnote{\omit\ 
  K.}
  \linenum{|\xlineref{SS.1.16.31--17}}\lemma{samañ\ldots 
  iti}\Dfootnote{°māṃsām ||  A.}
  \linenum{|\xlineref{SS.1.16.31--19}}\lemma{kuryād\ldots iti}\Afootnote{\omit\ 
  N.}
} ||23|| om || 
\end{verse}
\pend


\pstart
\edtext{}{
  \lemma{\emph{inserted passage}}\Dfootnote{nāḍīyogaṃ 
  vinauṣṭhasya 
  nāsāsandhānavad vidhim |  ya evam eva jānīyāt sa rājñaḥ kartum arhati || A.}
}
\pend



\endnumbering
\endgroup
\end{document}