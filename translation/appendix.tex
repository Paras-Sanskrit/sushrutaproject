\section*{Appendix}
\subsection{On digital critical editions}

\begin{itemize}
    \item \fullcite{pric-2013}. \\ A survey of the field in 2013, with a focus on
    the presentation of electronic texts rather than on critical editing as such.
    
    \item \fullcite{mour-2015}. \\ Useful discussion about the \emph{apparatus criticus}
    in general, and an evaluation of the plus and minus points of positive and
    negative apparatuses. 
    
    \item \fullcite{burg-2016}. \\ Discussion of a software tool, including the
    handling of positive and negative apparatus.  Makes the assumption that online
    displays are notational variants only.
    
    \item \fullcite{burg-2017}.  \\ Discussion of how to express various kinds of apparatus in 
    TEI.
    
    \item \fullcite{baus-2015b}. \\ A huge book that disappointingly says nothing at all about 
    Sanskrit manuscripts.  Nevertheless there are many interesting case studies and remarks 
    applicable to the Indian manuscript tradition.
    
    \item \fullcite{roel-2020}. \\ A major collection of studies.  The materials on Sanskrit 
    manuscripts is unfortunately influenced by some inadequate recent studies on the 
    \emph{Mahābhārata}.  Nevertheless, the volume remains important for its many 
    studies of general method and theory.
    
\end{itemize}

