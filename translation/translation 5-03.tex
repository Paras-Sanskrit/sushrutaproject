% !TeX root = incremental_SS_Translation.tex
\section{Kalpasthāna, adhyāya 3}

\subsection{Introduction}

\subsection{Translation}

\begin{translation}
    \item[1] And now we shall explain the \se{kalpa}{rule} that is the required
knowledge about mobile poisons.\footnote{In contrast to stationary, plant
    poisons.  No reference is made to Dhanvantari
    \citep[see][]{birc-2021}.}\q{Come back to the issue of "kalpa".  Look up
        passages in the Kośa.}


\item[3] 

The full explanation about the sixteen \se{adhiṣṭhāna}{carriers} of the mobile
poisons, that have been mentioned by me in brief, will be
stated.\footnote{“Carrier” for \se{adhiṣṭhāna}{base, foundation} aims to capture
    the idea that the author will describe the creatures in which poisons inhere.}
 
 \item[4] In that context, they are:\footnote{The content of this section is
    presented as a table, for clarity for the contemporary reader and mindful of the
    theoretical issues surrounding notational variation, including the “symbolic
    rewriting” and the modification of “expressive capacities” discussed by
    \citet[321\,ff]{saru-2016}.  For further discussion, see
    \cite[81--83]{wuja-2021}.}
 \begin{multicols}{2}
 \begin{itemize}
     \item gaze and breath,
     \item teeth, nails, and bites
     \item  urine and faeces,
     \item \diff{menstrual blood},
     \item semen,
     \item \diff{tail}, % lāṅgūla
     \item \diff{contact with saliva}, % lālāsparśa
     \item \se{mukhasaṃdaṃśā}{nipping with the mouth},
     \item \se{avaśardhita}{fart},\footnote{This interpretation comes from
     Ḍalhaṇa on \Su{5.3.4}{567}, but he reads \dev{viśardhita}.}
     \item \diff{anus},\footnote{Ḍalhaṇa on \Su{5.3.4}{567} noted this reading.}
     \item bones,
     \item bile,
     \item \se{śūka}{bristles},
and 
     \item corpses.
     \end{itemize}
    \end{multicols}
\item[5] 
In that context,


{\centering
\begin{longtable}{
        >{\raggedright\arraybackslash}p{.3\textwidth}
        >{\raggedright\arraybackslash}p{.7\textwidth}}
\toprule
\emph{location of the poison} & \emph{creatures}\footnotemark\\
\midrule    
\endfirsthead

\toprule
\emph{location of the poison} & \emph{creatures}\\
    \midrule    
\endhead
\footnotetext{Many of these names are mere dubious placeholders.}%
in their breath and gaze    & divine snakes \\[2ex]
%
in their fangs    & the ones on earth\footnote{Ḍalhaṇa on \Su{5.3.5}{567} cited
    the otherwise unknown authority Sāvitra on the topic of poisonous snakes 
    \pvolcite{???}[???]{meul-hist}.} \\[2ex]
%
in their nails, mouths and fangs  & cats, dogs, monkeys,
\se{nara}{men},\footnote{Probably dittography from the previous word,
    \se{vānara}{monkey}. But it is supported in both Nepalese witnesses, so it
    must go back to an earlier exemplar.} crocodiles, frogs,
    \se{pākamatsya}{`cook-fish'},\footnote{MS KL 699 separates the words
        \dev{pāka} and \dev{matsya} with a daṇḍa, indicating that the scribe thought
        they were separate terms. Ḍalhaṇa thought this was a kind of fiery insect
        (\Su{5.3.5}{567}).} monitor lizards, \se{śambūka}{cone snails},
        \se{pracalāka}{`poisonous snakes'},\footnote{\emph{Arthaśāstra} 14.1.14, 23
            \citep[448]{oliv-2013}, where it might also be a chameleon, but the latter are
            not venomous.} \se{gṛhagoḍikā}{geckos},\footnote{The scribe of MS NAK 
            5-333
                noted in the margin that some of his sources read \dev{galagoḍikā}, which 
                is
                the name of a snake known also in the \CS\ and elsewhere in literature. 
                Hemacandra's \emph{Abhidhānacintāmaṇi} (4.364) mentions that 
                \dev{gṛhagodhikā}
                and \dev{gṛhagolikā} are synonyms \citep[691a, \emph{sub
                māṇikyā}]{radh-1876}.}
                                            four-footed insects and others \\[2ex] 

 in their urine and faeces       &     \se{kiṭipa}{lice},
                                                 \se{picciṭā}{`flat insects'},
                                                \se{kaṣāyavāsika}{`orange-dwellers'},
                                                \se{sarṣapaka}{`pepper snakes'},
                                                \se{toṭaka}{`angry beetles'},
                                                \se{varcaḥkīṭa}{dung beetles},
                                                 and     
                                                \se{kauṇḍinya}{`pot insects'}\\[2ex]
%
in their semen 						&  mice \\[2ex]
%
  in their \se{śūla}{stings}      &	scorpions, 
                                                    \se{viśvambhara}{`earth scorpions'}, 
                                                    \se{varaki}{wasps},\footnote{\dev{varaṭī} is a wasp; 
                                                    \dev{varaki} in the Nepalese MSS may be an alternant 
                                                    of this word. Ḍalhaṇa on \Su{5.3.5}{568} remarked 
                                                    that some interpreted \dev{varakimatsya} as two items,
                                                    “wasp and fish,” 
                                                    others as a single one, “wasp-fish.”} 
                                                    fish,
                                                    \se{ucciṭiṅga}{crabs}, and
                                                    \se{patravṛścika}{`wing-scorpions'}
                                                    \\[2ex]
%
   in their saliva, nails, urine, 
   feces, blood, semen 
   and fangs     							& spiders  \\[2ex]
%   
 in the bites of their mouths        &   flies, \se{kaṇabha}{wasps} and leeches \\[2ex]
%
in the bites of their mouths, 
in their fangs, faces,  \dag, 
farts, anuses and feces     &   \se{citraśīrṣa}{`speckle-heads'}, 
                                                \se{śārava}{`lids'},
                                                \se{kukṣita}{`bellied'},
                                                \se{dārukāri}{`wood-enemies'},
                                                \se{medaka}{`liquors'},
                                                and
                                                \se{śārikā}{`darts'}.
                                                \\[2ex]
%                                                
        &  \\[2ex]
%
        &  \\[2ex]
%
        &  \\[2ex]
%
        &  \\[2ex]
%
        &  \\
    \bottomrule
    \caption{Passage 5, expressed in tabular format.}
\end{longtable}
\par} % end centering

%\footnote{\cite{kaur-2018} is unhelpful, in spite of a section on the \SS\ (pp.\,61--63).}

% got to here - 2023-01 continue with table for #5
    
    \item [6]
    \diff{The enemies of the king pollute the 
    waters, roads and foodstuffs 
    in enemy territory. The experienced physician, 
    who has learned how to purify things, 
    should clean up those polluted things.} 
    
    \item[7]
    
    Polluted water is slimy and smells of tears.\footnote{\dev{asra} normally
means “tears,” but rarely means “blood.” }  It is covered with froth and
covered with streaks. The frogs and fish die, the birds are crazed and, along with
the wetland creatures, they wander about aimlessly.
    
     \item [8]
     
     Men, horses and elephants who swim in it experience vomiting, delusion,
fever, swelling and sharp pains.\footnote{On the polysemy of \se{nāga}{elephant\slash 
snake}, see \cite{seme-1979}.} 
    He should try to purify that polluted water, after curing their ailments.
     
\item [9]

And so, he should burn
    \gls{dhava}  and 
    \gls{aśvakarṇa}, % Dipterocarpa turbanitus Gaertn., GVDB 28
        as well as
    \gls{pāribhadra}, %Erythrina suberosa, GVDB 245
        with
    \gls{pāṭalā} % GVDB 242 Stereospermum suaveolens DC.
        and 
    \gls{sidhraka} % 
        and 
   \gls{muṣkaka}, % GVDB 312, Schrebera swietenioides Roxb and 
   %Elaeodendron glaucum Pers.
        and with
    \gls{rājadruma} % āragvadha, GVDB 37 Cassia fistula Linn
        and 
    \gls{somavalka}.
Then he should sprinkle that ash, cold, on the waters.

\item [10--11]

And in the same way, putting a handful of the ash in a pot, one may also purify water that 
one wants.

If any one of the limbs of cows, horses, elephants, men or women, touch a place on
the ground that enemies have spoiled with poison, or a ford or rock or a flat
surface, then it swells up and burns and its hair and nails fall out on that
place.\footnote{``Swells up" translates an unclear reading that was probably
    \dev{śūyati}, which may be an irregular form of $\surd$\dev{śū, śvā, śvi} 
    \citep[see][175--176]{whit-root}.}

\item [12]

In that situation, he should grind up \gls{anantā} together with all the aromatic
items, with alcoholic drinks.  And then  he should sprinkle the paths that need to
be used with waters mixed with mud.\footnote{Our “alcoholic drinks” translates
    \dev{surā}.  For a discussion of this term at our period see \cite[37--39
    \emph{et passim}]{mchu-2021a}.} \diff{And if there exists another path, he should 
    go by
    that.}\footnote{Ḍalhaṇa on \Su{5.3.12}{568} cited a similar reading for the fourth
        pāda, but with a negative particle,  “and if there is no other way, one should go
        by that.”}
    
    

\item [13]

When grasses and foods are polluted, people collapse, fall unconscious. And others
vomit. They get \se{viḍbheda}{loose stool} or they die.
%\footnote{In “they get
%loose stool,” the verb \dev{ārcchanti} ($\surd$\dev{ṛ}), transmitted in both
%Nepalese manuscripts, has an irregular initial strong vowel.  Alternatively, and perhaps
%more likely, it is a combination of \dev{ā+$\surd$ṛ}, 
%conjugated unusually as a class 6 verb, but with an appropriate sense of “to fall into 
%(misfortune).”} 
One should apply to them the therapy as described.

\item [14--15]

Alternatively, one should wipe various musical instruments with antidotes that
remove poison and then play them.   What is called the most excellent paste for a
musical instrument is \gls{tārāvitāra}\footnote{“Certain minerals” translates
    \dev{tārāvitāra}, the unanimous reading of the Nepalese witnesses.  But the
    meaning of this expression is not clear and may even refer to plants, like the
    other ingredients.  The vulgate reads \dev{tāraḥ sutāraḥ}, which is also not very
    clear.  However, Ḍalhaṇa on \Su{5.3.14}{568} identified these as “silver” and
    “mercury.” This is highly unlikely to be a correct understanding of the passage. 
    Historically, mercury is not naturally present in the South Asian peninsula
    \pvolcite{5}[233]{watt-1896} and the word \dev{pārada} that Ḍalhaṇa used is
    probably a loan-word from Persian \citep[sub \emph{paranda,
    parranda}][244b]{stei-pers}.  Mercurial compounds are not reliably attested in
    South Asia until two or three centuries after the composition of the \SS\ at the
    earliest.  The currently available “śāstric” recension of the \emph{Arthaśāstra}
    that is datable to 175--300 \CE\ \citep[29--31]{oliv-2013} does not mention
    mercury (\emph{ibid}, 534).  See further the study by \citet[17, \emph{et
    passim}]{wuja-2013b}.} together with \gls{surendragopa}, and a portion of of
    \gls{kuruvinda} equal to that, together with the bile called “brown
    cow”.\footnote{\dev{surendragopa} and \dev{kuruvinda} are both uncertain, see
        index. Ḍalhaṇa's opinion has been followed here, but it seems fair to say that all
        commentators were guessing.} By the sound of the musical instrument, even 
        terrible
        poisons that may be present at that place are destroyed.
        
\item[16]

If there is smoke or wind that is affected by poison then birds are dazed and fall
to the ground.  People get coughs, colds, and head illnesses, and acute eye
diseases.\footnote{The syntax of this verse is somewhat loose; the vulgate has
    regularized it, smoothing out the difficulties.}

\item[17]

The smoke and air can be purified by putting into the air: 
\gls{lākṣā},
\gls{haridrā},
\gls{ativiṣā},
and
\gls{abhayā},
with
\gls{vakra},
\gls{kuṣṭha},
\gls{elā},\footnote{}\q{write footnote:  don't repeat ativiṣā; vulgate similar to H.}
and
\gls{hareṇu},
and
\gls{priyaṅgu}.

\end{translation}

\subsubsection{The origin of poison}

\begin{translation}[resume]

\item [18]

As it is told, the arrogant demon called Kaiṭabha created an obstacle for
lotus-born Brahmā, at the very time that he was creating these creatures.\footnote{At 
this point, the text seems to make a new beginning to the topic of toxicology, as if 
starting a new chapter.  It is notable that no reference is made here to 
the famous origin story of poison in the churning of the primal milk ocean; for 
discussion of the sources of this account, see \cite{bede-1967}. For  
reflections on this passage, connecting it with Rudra and the 
\emph{Śatapathabrāhmaṇa}, see \cite{mana-2019}.}

\item[19]
Pitiless Fury took a body and burst out of the mouth of furious Brahmā's store of fiery 
energy.\footnote{“Fury” is here anthropomorphised.}

\item[20]

He burned that great, thundering, apocalyptic demon.   Then, after bringing about
the annihilation of that demon, his amazing fiery energy increased.

\item [21]

And so, there was a sinking down (\emph{viṣāda}) of the Daityas.   Observing that,
it was named “poison (\emph{viṣa})” because of it's ability to produce a “sinking
down.”

\item [22] After that, the Lord created beings and subsequently made that fury
enter into creatures still and moving.

\item [23--24] Water that falls from the sky to the earth has no obvious flavour.
The savour of the different places it lands on enters into it.  In the same way,
whatever substance a poison reaches, it establishes itself there and by its nature it 
takes on that substance's savour.\footnote{The scribal emendation in 
\MScite{Kathmandu NAK 
5-333} of \dev{niyacchati} to \dev{nigacchati} suggests that the scribe had more than 
one manuscript before him, one of them representing the reading of the vulgate 
recension.}

\item [25] Generally speaking, in a poison, all the qualities are really sharp. 
For this reason, every poison is known to irritate all of the humours.

\item[26]

Irritated and afflicted by the poison, they leave their natural functions.
Poison does not get digested, so it blocks the breaths.\footnote{Probably a
    reference to the five breaths.  Ḍalhaṇa referred to winds (\dev{vāta}), but this
    does not seem correct since it is a reference to humours rather than breaths.}

\item[27] 

Breathing is obstructed because its pathway is blocked by phlegm. Even
if life continues, a man remains without consciousness.

\item [28]

Similar to semen, the poison of all angry snakes pervades the whole body, and goes 
to the limbs like semen because of being stirred up. 

% Cakra on Ca.ci.2.4.49; in 46 śukra is all-pervading where there is saṃsparśana

\item [29]

The fang of snakes is like a hook.  When it gets there, it sticks inside them. That
is why the unagitated poison of a snake is not released.

\item [30] 

Sprinkling with very cold water is traditional for all cases of poisoning,
because poison is declared to be extremely hot and sharp.\footnote{The verb
    \dev{paṭh} “is declared, read aloud” here could possibly suggest that the
    author is working within a written, not oral, tradition.}

\item [31]

Poison in insects is slow and not very hot, having a lot of wind and phlegm. 
So in cases of insect poisoning, sweating is not forbidden.

\item [32cd]

In cases of a strike or a bite, the poison may, of its own accord, stay there. 

\item[33--35ab]

\dag Having come upon a body,\footnote{“Having come upon” translates
  \dev{prakhyāpya}, which is hard to interpret unless it is a rare form
  connected with the sense “to see.”} in the case of corpses that 
that have been pierced by a poisoned arrow and bitten by a snake,
%
someone who eats the poisoned flesh of a recent corpse out of carelessness 
will suffer with illness according to the poison, or even die. 
%
And therefore, the flesh of those should not be eaten when they have just died. 

It is admissable after three quarters of an hour, but without the poisoned arrow and 
the snakebite.

\item[35 margin] [At this point an Upajāti verse is added in the margin of K
but is not fully legible; the version of the text in H is also incomplete and not
fully comprehensible.] \footnote{\emph{Mādhavanidāna},  69.20--21
    \citep[480]{madhava1} has verses that are directly parallel to this section:\\
    \emph{darvīkarāṇāṃ viṣam āśughāti sarvāṇi coṣṇe dviguṇībhavanti
    ajīrṇapittātapapīḍiteṣu bāleṣu vṛddheṣu bubhukṣiteṣu 20\\ kṣīṇakṣate mohini
    kuṣṭhayukte rūkṣe ’bale garbhavatīṣu cāpi\\ śastrakṣate yasya na raktam eti
    rājyo latābhiś ca na saṃbhavanti 21 }}


\item[35.1]
\dag
When, in a wound, the poison  that is connected with these qualities 
runs, 
\ldots
Therefore, everything that is damaged by poison and eaten does not cause death.



\footnote{At this point, witness H inserts a marginal Indravajrā verse about
    diseases that afflict immoral women.}

\item[35.1]  [ślokas in the MSS that aren't in the vulgate.  The first line doesn't 
scan.  Witness K adds a part of the start of this in the
bottom margin.  This material is repeated at 3.39.2 in MS H. ]

\item [35cd \& 36cd]  

One designates a person who has diarrhoea of feces looking like
\se{gṛhadhūma}{soot} with wind,\footnote{\dev{gṛhadhūma} is not a plant in
    this context \emph{pace} \cite[362]{moni-sans}.} and who vomits foam, as
    “someone who has drunk poison.”

\item[37]

Therefore, fire burns a heart that is pervaded by poison. For, having pervaded
of its own accord  the location of consciousness, it abides.\footnote{Ḍalhaṇa
    said that someone who has died from drinking poison has a heart that cannot
    be burned because it is pervaded by poison (\Su{5.3.37}{570}).
    But the sense of the Nepalese MSS is the opposite.}

\subsubsection{Patients beyond help}

\item[38] Patients who should not be accepted include: those who have been
bitten under a \gls{aśvattha}, in a temple, in a cemetery, at an ant-hill, at
dawn or dusk, at a crossroads, in Yama's direction,\footnote{\dev{yāmye} means
    “southerly” but Ḍalhaṇa on \Su{5.3.38}{570} interpreted “\se{yāmya}{in Yama's
    direction}” as “under the seventh asterism.”} under the Great Bear and 
    those who have been bitten in the veins or lethal spots.

\item[39]

The poison of cobras kills rapidly.  During the summer, they all become twice as 
potent in those who have indigestion, those who are afflicted by bile or wind, old 
people, children and the hungry.

\item[39.1]

In those whose who are mad or intoxicated, or who suffer from anxiety, or who
are unable to tolerate its various strengths, it becomes sharp.
\dag \ldots

\item [3.40cd--3.41]

One should reject someone overcome by poison who does not bleed when cut with
a knife, where weals do not appear as a result of lashes,\footnote{Ḍalhaṇa, on
    \Su{5.3.40}{570}, glossed \dev{latābhis} “by means of whips,” as “when the
    body is struck by whips.”} %
    or where there is no horripilation because of cold water, \dag\ whose
    tongue is a mouth, whose hair is falling out, whose \dag nose\dag is
    exhausted and whose neck is broken.


\item[3.42]

\item[3.43ab]


\end{translation}