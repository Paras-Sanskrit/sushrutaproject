\RequirePackage{snapshot}
% !TeX encoding = UTF-8
% !TeX spellcheck = en_GB
\documentclass[12pt]{article}
\usepackage{xelatex-generic-biblatex-indexing,
    booktabs, % better-looking tables
    wasysym, %for the blob symbols in the table
}
\usepackage{hyperref}

% just print hyperlinked “URL” in the bibliography, not the long, ugly URL.
\DeclareFieldFormat{url}{%
    \ifhyperref
   % {\textsc{url: }\href{#1}{URL}}
    {\href{#1}{\textsc{url}}}
    {\textsc{url: }\nolinkurl{#1}}} %DW bug here
\DeclareFieldFormat{eprint:ark}{%
%    \textsc{ark:}\space
    \ifhyperref  
    {\href{https://n2t.net/#1}{\nolinkurl{#1}}}
    {\nolinkurl{#1}}}

\addbibresource {\jobname.bib}
\setlist[enumerate]{noitemsep} % enumitem global setting for \items
\setlist[itemize]{noitemsep} 

%% Make a new environment for the translation with tweaked spacing.
%% The old-fashioned LaTeX way:
%\newenvironment{translation}
%    {\begin{itemize}[noitemsep,
%            left=0pt .. \parindent]}
%    {\end{itemize}}

%% Make a new environment for the translation with tweaked spacing.
%% Do it the enumitem way:
\newlist{translation}{enumerate}{2}
\setlist[translation]{label=\textbullet,
   noitemsep,
    left=0pt .. \parindent}


\newcommand{\VN}[2]{%\cite[#1]{susr-trikamji2004}
SS\,#1}
% later, when we've got our own numbers for our edition, we'll add them 
% as the second argument.

%
%\newcommand{\item}[1]{$^{#1}$}

%\newenvironment{translation}{\list }{\endlist }

\renewcommand{\SS}{\emph{Suśrutasaṃhitā}}

\title{A Translation of the New Edition of the \SS}
\author{Jason Birch and Dominik Wujastyk}
\date{\texttt{Draft of \today }}

\begin{document}
       % Manual hyphenation points for Sanskrit words and compounds.
% By Dominik Wujastyk.
% Copyright Dominik Wujastyk 2021.
% Released under a BY-SA Creative Commons license 
% (Attribution-ShareAlike 4.0 International http://creativecommons.org/licenses/by-sa/4.0/).
% This file is still actively growing, slowly but steadily (March 2021) .
%
% These special hyphenations have to be loaded after
% \begin{document}. See
% http://www.tug.org/pipermail/xetex/2008-July/010362.html
% Or use 
% \AtBeginDocument{% Manual hyphenation points for Sanskrit words and compounds.
% By Dominik Wujastyk.
% Copyright Dominik Wujastyk 2021.
% Released under a BY-SA Creative Commons license 
% (Attribution-ShareAlike 4.0 International http://creativecommons.org/licenses/by-sa/4.0/).
% This file is still actively growing, slowly but steadily (March 2021) .
%
% These special hyphenations have to be loaded after
% \begin{document}. See
% http://www.tug.org/pipermail/xetex/2008-July/010362.html
% Or use 
% \AtBeginDocument{% Manual hyphenation points for Sanskrit words and compounds.
% By Dominik Wujastyk.
% Copyright Dominik Wujastyk 2021.
% Released under a BY-SA Creative Commons license 
% (Attribution-ShareAlike 4.0 International http://creativecommons.org/licenses/by-sa/4.0/).
% This file is still actively growing, slowly but steadily (March 2021) .
%
% These special hyphenations have to be loaded after
% \begin{document}. See
% http://www.tug.org/pipermail/xetex/2008-July/010362.html
% Or use 
% \AtBeginDocument{\input{sanskrit-hyphenations}}% should work, but doesn't
% special hyphenations for Sanskrit words tagged in
% Polyglossia.
% *English,\textenglish{},text,and
% *Sanskrit,\textsanskrit{},text.
%
% English (see below for \textsanskrit)
%
\hyphenation{%
    dhanva-ntariṇopa-diṣ-ṭaḥ
    suśruta-nāma-dheyena
    tac-chiṣyeṇa
    kāśyapa-saṃ-hitā
    cikitsā-sthāna
    su-śruta-san-dīpana-bhāṣya
    dṛṣṭi-maṇḍala
    uc-chiṅga-na
    sarva-siddhānta-tattva-cūḍā-maṇi
    tulya-sau-vīrāñja-na
    indra-gopa
    śrī-mad-abhi-nava-guptā-cārya-vi-ra-cita-vi-vṛti-same-tam
    viśva-nātha
śrī-mad-devī-bhāga-vata-mahā-purāṇa
    siddhā-n-ta-sun-dara
    brāhma-sphuṭa-siddh-ānta
    bhū-ta-saṅ-khyā
    bhū-ta-saṃ-khyā
    kathi-ta-pada
    devī-bhā-ga-vata-purāṇa
    devī-bhā-ga-vata-mahā-purāṇa
    Siddhānta-saṃ-hitā-sāra-sam-uc-caya
    sau-ra-pau-rāṇi-ka-mata-sam-artha-na
    Pṛthū-da-ka-svā-min
    Brah-ma-gupta
    Brāh-ma-sphu-ṭa-siddhānta
    siddhānta-sun-dara
    vāsa-nā-bhāṣya
    catur-veda
    bhū-maṇḍala
    jñāna-rāja
    graha-gaṇi-ta-cintā-maṇi
    Śiṣya-dhī-vṛd-dhi-da-tan-tra
    brah-māṇḍa-pu-rā-ṇa
    kūr-ma-pu-rā-ṇa
    jam-bū-dvī-pa
    bhā-ga-vata-pu-rā-ṇa
    kupya-ka
    nandi-suttam
    nandi-sutta
    su-bodhiā-bāī
    asaṅ-khyāta
    saṅ-khyāta
    saṅ-khyā-pra-māṇa
    saṃ-khā-pamāṇa
    nemi-chandra
    anu-yoga-dvāra
    tattvārtha-vārtika
    aka-laṅka
    tri-loka-sāra
    gaṇi-ma-pra-māṇa
    gaṇi-ma-ppa-māṇa
    eka-pra-bhṛti
gaṇaṇā-saṃ-khā
gaṇaṇā-saṅ-khyā
dvi-pra-bhṛti
duppa-bhi-ti-saṃ-khā
vedanābhi-ghāta
Viṣṇu-dharmottara-pu-rāṇa
abhaya-deva-sūri-vi-racita-vṛtti-vi-bhūṣi-tam
abhi-dhar-ma
abhi-dhar-ma-ko-śa
abhi-dhar-ma-ko-śa-bhā-ṣya
abhi-dharma-kośa-bhāṣya
abhi-dharma-kośa-bhāṣyam
abhi-nava
abhyaṃ-karopāhva-vāsu-deva-śāstri-vi-ra-ci-ta-yā
ācārya-śrī-jina-vijayālekhitāgra-vacanālaṃ-kṛtaś-ca
ācāry-opā-hvena
ādhāra
adhi-kāra
adhi-kāras
ādi-nātha
agni-besha
agni-veśa
ahir-budhnya
ahir-budhnya-saṃ-hitā
aita-reya-brāhma-ṇa
akusī-dasya
amara-bharati
Amar-augha-pra-bo-dha
amṛ-ta-siddhi
ānanda-kanda
ānan-da-rā-ya
ānand-āśra-ma-mudraṇā-la-ya
ānand-āśra-ma-saṃ-skṛta-granth-āva-liḥ
anna-pāna-mūlā
anu-ban-dhya-lakṣaṇa-sam-anv-itās
anu-bhav-ād
anu-bhū-ta-viṣayā-sam-pra-moṣa
anu-bhū-ta-viṣayā-sam-pra-moṣaḥ
aparo-kṣā-nu-bhū-ti
app-proxi-mate-ly
ardha-rātrika-karaṇa
ārdha-rātrika-karaṇa
ariya-pary-esana-sutta
arun-dhatī
ārya-bhaṭa
ārya-bhaṭā-cārya-vi-racitam
ārya-bhaṭīya
ārya-bhaṭīyaṃ
ārya-lalita-vistara-nāma-mahā-yāna-sūtra
ārya-mañju-śrī-mūla-kalpa
ārya-mañju-śrī-mūla-kalpaḥ
asaṃ-pra-moṣa
aṣṭāṅga-hṛdaya-saṃ-hitā
aṣṭāṅga-saṃ-graha
asura-bhavana
aśva-ghoṣa
ātaṅka-darpaṇa-vyā-khyā-yā
atha-vā
ava-sāda-na
āyār-aṅga-suttaṃ
ayur-ved
ayur-veda
āyur-veda
āyur-veda-dīpikā
āyur-veda-dīpikā-vyā-khyayā
āyur-ve-da-ra-sā-yana
āyur-veda-sū-tra
ayur-vedic
āyur-vedic
ayur-yog
bādhirya
bahir-deśa-ka
bala-bhadra
bala-kot
bala-krishnan
bāla-kṛṣṇa
bau-dhā-yana-dhar-ma-sūtra
bel-valkar
bhadra-kālī-man-tra-vi-dhi-pra-karaṇa
bhadrā-sana
bhadrā-sanam
bha-ga-vat-pāda
bhaiṣajya-ratnāvalī
bhan-d-ar-kar
bhartṛhari-viracitaḥ
bhaṭṭā-cārya
bhaṭṭot-pala-vi-vṛti-sahitā
Bhiṣag-varāḍha-malla-vi-racita-dīpikā-Kāśī-rāma-vaidya-vi-raci-ta-gūḍhā-rtha-dīpikā-bhyāṃ
bhiṣag-varāḍha-malla-vi-racita-dīpikā-Kāśī-rāma-vaidya-vi-racita-gūḍhārtha-dīpikā-bhyāṃ
bhoja-deva-vi-raci-ta-rāja-mārtaṇḍā-bhi-dha-vṛtti-sam-e-tāni
bhu--va-na-dī-pa-ka
bīja-pallava
bi-kaner
bodhi-sat-tva-bhūmi
brahma-gupta
brahmā-nanda
brahmāṇḍa-mahā-purā-ṇa
brahmāṇḍa-mahā-purā-ṇam
brahma-randhra
brahma-siddh-ānta
brāhma-sphuṭa-siddh-ānta
brāhma-sphu-ṭa-siddhānta
brahma-vi-hāra
brahma-vi-hāras
brahma-yā-mala-tan-tra
Bra-ja-bhāṣā
bṛhad-āraṇya-ka
bṛhad-yā-trā
bṛhad-yogi-yājña-valkya-smṛti
bṛhad-yogī-yājña-valkya-smṛti
bṛhaj-jāta-kam
bṛhat-khe-carī-pra-kāśa
buddhi-tattva-pra-karaṇa
cak-ra-dat-ta
cakra-datta
cakra-pāṇi-datta
cā-luk-ya
caraka-prati-saṃ-s-kṛta
caraka-prati-saṃ-s-kṛte
caraka-saṃ-hitā
casam-ul-lasi-tāmaharṣiṇāsu-śrutenavi-raci-tāsu-śruta-saṃ-hitā
cau-kham-ba
cau-luk-yas
chandi-garh
chara-ka
cha-rīre
chatt-opa-dh-ya-ya
chau-kham-bha
chi-ki-tsi-ta
cid-ghanā-nanda-nātha
ci-ka-ner
com-men-taries
com-men-tary
com-pre-hen-sive-ly
daiva-jñālaṃ-kṛti
daiva-jñālaṅ-kṛti
dāmo-dara-sūnu-Śārṅga-dharācārya-vi-racitā
Dāmodara-sūnu-Śārṅga-dharācārya-vi-racitā
darśanā-ṅkur-ābhi-dhayā
das-gupta
deha-madhya
deha-saṃ-bhava-hetavaḥ
deva-datta
deva-nagari
deva-nāgarī
devā-sura-siddha-gaṇaiḥ
dha-ra-ni-dhar
dharma-megha
dharma-meghaḥ
dhru-vam
dhru-va-sya
dhru-va-yonir
dhyā-na-grahopa-deśā-dhyā-yaś
dṛḍha-śūla-yukta-rakta
dvy-ulbaṇaikolba-ṇ-aiḥ
four-fold
gan-dh-ā-ra
gārgīya-jyoti-ṣa
gārgya-ke-rala-nīla-kaṇṭha-so-ma-sutva-vi-racita-bhāṣyo-pe-tam
garuḍa-mahā-purāṇa
gaurī-kāñcali-kā-tan-tra
gau-tama
gauta-mādi-tra-yo-da-śa-smṛty-ātma-kaḥ
gheraṇḍa-saṃ-hitā
gorakṣa-śata-ka
go-tama
granth-ā-laya
grantha-mālā
gran-tha-śreṇiḥ
grāsa-pramāṇa
guru-maṇḍala-grantha-mālā
gyatso
hari-śāstrī
haṭhābhyāsa-paddhati
haṭha-ratnā-valī
Haṭha-saṅ-keta-candri-kā
haṭha-tattva-kau-mudī
haṭha-yoga
hāyana-rat-na
haya-ta-gran-tha
hema-pra-bha-sūri
hetu-lakṣaṇa-saṃ-sargād
hīna-madhyādhi-kaiś
hindī-vyā-khyā-vi-marśope-taḥ
hoern-le
ijya-rkṣa
ikka-vālaga
indra-dhvaja
indrāṇī-kalpa
indria
Īśāna-śiva-guru-deva-pad-dhati
jābāla-darśanopa-ni-ṣad
jadav-ji
jagan-nā-tha
jala-basti
jal-pa-kal-pa-tāru
jam-bū-dvī-pa-pra-jña-pti
jam-bū-dvī-pa-pra-jña-pti-sūtra
jana-pad-a-sya
jāta-ka-kar-ma-pad-dhati
jaya-siṃha
jinā-agama-grantha-mālā
jin-en-dra-bud-dhi
jīvan-muk-ti-vi-veka
jñā-na-nir-mala
jñā-na-nir-malaṃ
joga-pra-dīpya-kā
jya-rkṣe
Jyo-tiḥ-śās-tra
jyo-ti-ṣa-rāya
jyoti-ṣa-rāya
jyotiṣa-siddhānta-saṃ-graha
jyotiṣa-siddhānta-saṅ-graha
kāka-caṇḍīśvara-kal-pa-tan-tra
kakṣa-puṭa
kali-kāla-sarva-jña
kali-kāla-sarva-jña-śrī-hema-candrācārya-vi-raci-ta
kali-kāla-sarva-jña-śrī-hema-candrācārya-vi-raci-taḥ
kali-yuga
kal-pa
kal-pa-sthāna
kalyāṇa-kāraka
Kāmeśva-ra-siṃha-dara-bhaṅgā-saṃ-skṛta-viśva-vidyā-layaḥ
kapāla-bhāti
karaṇa-tilaka
kar-ma
kar-man
kāṭhaka-saṃ-hitā
kavia-rasu
kavi-raj
keśa-va-śāstrī
ke-vala--rāma
keva-la-rāma
khaṇḍa-khādyaka-tappā
khe-carī-vidyā
knowl-edge
kol-ka-ta
kriyā-krama-karī
kṛṣṇa-pakṣa
kṛtti-kā
kṛtti-kās
kubji-kā-mata-tantra
kula-pañji-kā
kul-karni
ku-māra-saṃ-bhava
kuṭi-pra-veśa
kuṭi-pra-veśika
lakṣ-mī-veṅ-kaṭ-e-ś-va-ra
lit-era-ture
lit-era-tures
locana-roga
mādha-va
mādhava-kara
mādhava-ni-dāna
mādhava-ni-dā-nam
madh-ūni
madhya
mādhyan-dina
madhye
mahā-bhāra-ta
mahā-deva
mahā-kavi-bhartṛ-hari-praṇīta-tvena
maha-mahopa-dhyaya
mahā-maho-pā-dhyā-ya-śrī-vi-jñā-na-bhikṣu-vi-raci-taṃ
mahā-mati-śrī-mādhava-kara-pra-ṇī-taṃ
mahā-mudrā
mahā-muni-śrī-mad-vyāsa-pra-ṇī-ta
mahā-muni-śrī-mad-vyāsa-pra-ṇī-taṃ
maharṣiṇā
maha-rṣi-pra-ṇīta-dharma-śāstra-saṃ-grahaḥ
Maha-rṣi-varya-śrī-yogi-yā-jña-valkya-śiṣya-vi-racitā
mahā-sacca-ka-sutta
mahā-sati-paṭṭhā-na-sutta
mahā-vra-ta
mahā-yāna-sūtrālaṅ-kāra
maitrāya-ṇī-saṃ-hitā
maktab-khānas
māla-jit
māli-nī-vijayot-tara-tan-tra
manaḥ-sam-ā-dhi
mānasol-lāsa
mānava-dharma-śāstra
mandāgni-doṣa
mannar-guḍi
mano-har-lal
mano-ratha-nandin
man-u-script
man-u-scripts
mataṅga-pārame-śvara
mater-ials
matsya-purāṇam
medh-ā-ti-thi
medhā-tithi
mithilā-stha
mithilā-stham
mithilā-sthaṃ
mṛgendra-tantra-vṛtti
mud-rā-yantr-ā-laye
muktā-pīḍa
mūla-pāṭha
muṇḍī-kalpa
mun-sh-ram
Nāda-bindū-pa-ni-ṣat
nāga-bodhi
nāga-buddhi
nakṣa-tra
nara-siṃha
nārā-yaṇa-dāsa
nārā-yaṇa-dāsa
nārā-yaṇa-kaṇṭha
nārā-yaṇa-paṇḍi-ta-kṛtā
nar-ra-tive
nata-rajan
nava-pañca-mayor
nava-re
naya-na-sukho-pā--dhyāya
ni-ban-dha-saṃ-grahā-khya-vyākhya-yā
niban-dha-san-graha
ni-dā-na
nidā-na-sthā-na-sya
ni-dāna-sthānasyaśrī-gaya-dāsācārya-vi-racitayānyāya-candri-kā-khya-pañjikā-vyā-khyayā
nir-anta-ra-pa-da-vyā-khyā
nir-guṇḍī-kalpa
nir-ṇaya-sā-gara
Nir-ṇaya-sāgara
nir-ṇa-ya-sā-gara-mudrā-yantrā-laye
nir-ṇa-ya-sā-ga-ra-yantr-āla-ya
nir-ṇaya-sā-gara-yantr-ā-laye
niśvāsa-kārikā
nīti-śṛṅgāra-vai-rāgyādi-nāmnāsamākhyā-tānāṃ
nityā-nanda
nya-grodha
nya-grodho
nyā-ya-candri-kā-khya-pañji-kā-vyā-khya-yā
nyāya-śās-tra
okaḥ-sātmya
okaḥ-sātmyam
okaḥ-sātmyaṃ
oka-sātmya
oka-sātmyam
oka-sātmyaṃ
oris-sa
oṣṭha-saṃ-puṭa
ousha-da-sala
padma-pra-bha-sūri
Padma-prā-bhṛ-ta-ka
padma-sva-sti-kārdha-candrādike
paitā-maha-siddhā-nta
pañca-karma
pañca-karman
pāñca-rātrā-gama
pañca-siddh-āntikā
paṅkti-śūla
Paraśu-rāma
paraśu-rāma
pari-likh-ya
pāśu-pata-sū-tra-bhāṣya
pātañ-jala-yoga-śās-tra
pātañ-jala-yoga-śās-tra-vi-varaṇa
pat-añ-jali
pat-na
pāva-suya
phiraṅgi-can-dra-cchedyo-pa-yogi-ka
pim-pal-gaon
pipal-gaon
pitta-śleṣ-man
pit-ta-śleṣ-ma-śoṇi-ta
pitta-śoṇi-ta
prā-cīna-rasa-granthaḥ
prā-cya
prā-cya-hindu-gran-tha-śreṇiḥ
prācya-vidyā-saṃ-śodhana-mandira
pra-dhān-in
pra-ka-shan
pra-kaṭa-mūṣā
pra-kṛ-ti-bhū-tāḥ
pra-mā-ṇa-vārt-tika
pra-ṇītā
pra-saṅ-khyāne
pra-śas-ta-pāda-bhāṣya
pra-śna-pra-dīpa
pra-śnārṇa-va-plava
praśnārṇava-plava
pra-śna-vai-ṣṇava
pra-śna-vaiṣṇava
prati-padyate
pra-yatna-śaithilyānan-ta-sam-āpatti-bhyām
prei-sen-danz
punar-vashu
puṇya-pattana
pūrṇi-mā-nta
raghu-nātha
rāja-kīya
rāja-kīya-mudraṇa-yantrā-laya
rāja-śe-khara
rajjv-ābhyas-ya
raj-put
rāj-put
rakta-mokṣa-na
rāma-candra-śāstrī
rāma-kṛṣṇa
rāma-kṛṣṇa-śāstri-ṇā
rama-su-bra-manian
rāmā-yaṇa
rasa-ratnā-kara
rasa-ratnākarāntar-ga-taś
rasa-ratna-sam-uc-caya
rasa-ratna-sam-uc-ca-yaḥ
rasa-vīry-auṣa-dha-pra-bhāvena
rasā-yana
rasendra-maṅgala
rasendra-maṅgalam
rāṣṭra-kūṭa
rāṣṭra-kūṭas
sādhana
śākalya-saṃ-hitā
śāla-grāma-kṛta
śāla-grāma-kṛta
sāmañña-pha-la-sutta
sāmañña-phala-sutta
sama-ran-gana-su-tra-dhara
samā-raṅga-ṇa-sū-tra-dhāra
sama-ra-siṃ-ha
sama-ra-siṃ-haḥ
sāmba-śiva-śāstri
same-taḥ
saṃ-hitā
śāṃ-ka-ra-bhāṣ-ya-sam-etā
sam-rāṭ
saṃ-rāṭ
Sam-rāṭ-siddhānta
Sam-rāṭ-siddhānta-kau-stu-bha
sam-rāṭ-siddhānta-kau-stu-bha
saṃ-sargam
saṃ-sargaṃ
saṃ-s-kṛta
saṃ-s-kṛta-pārasī-ka-pra-da-pra-kāśa
saṃ-śo-dhana
saṃ-śodhitā
saṃ-sthāna
sam-ullasitā
sam-ul-lasi-tam
saṃ-valitā
saṃ-valitā
śāndilyopa-ni-ṣad
śaṅ-kara
śaṅ-kara-bha-ga-vat-pāda
śaṅ-karā-cārya
san-kara-charya
Śaṅ-kara-nārā-yaṇa
sāṅ-kṛt-yā-yana
san-s-krit
śāra-dā-tila-ka-tan-tra
śa-raṅ-ga-deva
śār-dūla-karṇā-va-dāna
śār-dūla-karṇā-va-dāna
śā-rī-ra-sthāna
śārṅga-dhara-saṃ-hitā
Śārṅga-dhara-saṃ-hitā
sar-va-dar-śana-saṅ-gra-ha
sarva-kapha-ja
sarv-arthāvi-veka-khyā-ter
sar-va-śa-rīra-carās
sarva-siddhānta-rāja
Sarva-siddhā-nt-rāja
sarva-vyā-dhi-viṣāpa-ha
sarva-yoga-sam-uc-caya
sar-va-yogeśvareśva-ram
śāstrā-rambha-sam-artha-na
śatakatrayādi-subhāṣitasaṃgrahaḥ
sati-paṭṭhā-na-sutta
ṣaṭ-karma
ṣaṭ-karman
sat-karma-saṅ-graha
sat-karma-saṅ-grahaḥ
ṣaṭ-pañcā-śi-kā
saun-da-ra-nanda
sa-v-āī
schef-tel-o-witz
scholars
sharī-ra
sheth
sid-dha-man-tra
siddha-nanda-na-miśra
siddha-nanda-na-miśraḥ
siddha-nitya-nātha-pra-ṇītaḥ
Siddhānta-saṃ-hitā-sāra-sam-uc-caya
Siddhā-nta-sār-va-bhauma
siddhānta-sindhu
siddhānta-śiro-maṇ
Siddhānta-śiro-maṇi
Siddhā-nta-tat-tva-vi-veka
sid-dha-yoga
siddha-yoga
sid-dhi
sid-dhi-sthā-na
sid-dhi-sthāna
śikhi-sthāna
śiraḥ-karṇā-kṣi-vedana
śiro-bhūṣaṇam
Śivā-nanda-saras-vatī
śiva-saṃ-hitā
śiva-yo-ga-dī-pi-kā
ska-nda-pu-rā-ṇa
śleṣ-man
śleṣ-ma-śoni-ta
sodā-haraṇa-saṃ-s-kṛta-vyā-khyayā
śodha-ka-pusta-kaa
śoṇi-ta
spaṣ-ṭa-krānty-ādhi-kāra
śrī-cakra-pāṇi-datta
śrī-cakra-pāṇi-datta-viracitayā
śrī-ḍalhaṇācārya-vi-raci-tayāni-bandha-saṃ-grahākhya-vyā-khyayā
śrī-dayā-nanda
śrī-hari-kṛṣṇa-ni-bandha-bhava-nam
śrī-hema-candrā-cārya-vi-raci-taḥ
śrī-kaṇtha-dattā-bhyāṃ
śrī-kṛṣṇa-dāsa
śrī-kṛṣṇa-dāsa-śreṣṭhinā
śrīmac-chaṅ-kara-bhaga-vat-pāda-vi-raci-tā
śrī-mad-amara-siṃha-vi-racitam
śrī-mad-bha-ga-vad-gī-tā
śrī-mad-bhaṭṭot-pala-kṛta-saṃ-s-kṛta-ṭīkā-sahitam
śrī-mad-dvai-pā-yana-muni-pra-ṇītaṃ
śrī-mad-vāg-bhaṭa-vi-raci-tam
śrī-maṃ-trī-vi-jaya-siṃha-suta-maṃ-trī-teja-siṃhena
śrī-mat-kalyāṇa-varma-vi-racitā
śrī-mat-sāyaṇa-mādhavācārya-pra-ṇītaḥsarva-darśana-saṃ-grahaḥ
śrī-nitya-nātha-siddha-vi-raci-taḥ
śrī-rāja-śe-khara
śrī-śaṃ-karā-cārya-vi-raci-tam
śrī-vā-cas-pati-vaidya-vi-racita-yā
śrī-vatsa
śrī-veda-vyāsa-pra-ṇīta-mahā-bhā-ratāntar-ga-tā
śrī-veṅkaṭeś-vara
śrī-vi-jaya-rakṣi-ta
sruta-rakta
sruta-raktasya
stambha-karam
sthānāṅga-sūtra
sthira-sukha
sthira-sukham
stra-sthā-na
subhāṣitānāṃ
su-brah-man-ya
su-bra-man-ya
śukla-pakṣa
śukrā-srava
suk-than-kar
su-pariṣkṛta-saṃgrahaḥ
sura-bhi-pra-kash-an
sūrya-dāsa
sūrya-siddhānta
su-shru-ta
su-śru-ta
su-shru-ta-saṃ-hitā
su-śru-ta-saṃ-hitā
su-śru-tena
sutra
sūtra
sūtra-neti
sūtra-ni-dāna-śā-rīra-ci-ki-tsā-kal-pa-sthānot-tara-tan-trātma-kaḥ
sūtra-sthāna
su-varṇa-pra-bhāsot-tama-sū-tra
Su-var-ṇa-pra-bhās-ot-tama-sū-tra
su-varṇa-pra-bhāsotta-ma-sūtra
su-vistṛta-pari-cayātmikyāṅla-prastāvanā-vividha-pāṭhān-tara-pari-śiṣṭādi-sam-anvitaḥ
sva-bhāva-vyādhi-ni-vāraṇa-vi-śiṣṭ-auṣa-dha-cintakās
svā-bhāvika
svā-bhāvikās
sva-cchanda-tantra
śvetāśva-taropa-ni-ṣad
taila-sarpir-ma-dhūni
tait-tirīya-brāhma-ṇa
tājaka-muktā-valeḥ
tājika-kau-stu-bha
tājika-nīla-kaṇṭhī
tājika-yoga-sudhā-ni-dhi
tapo-dhana
tapo-dhanā
tārā-bhakti-su-dhārṇava
tārtīya-yoga-su-sudhā-ni-dhi
tegi-ccha
te-jaḥ-siṃ-ha
ṭhāṇ-āṅga-sutta
ṭīkā-bhyāṃ
ṭīkā-bhyāṃ
tiru-mantiram
tiru-ttoṇṭar-purāṇam
tiru-va-nanta-puram
trai-lok-ya
trai-lokya-pra-kāśa
tri-bhāga
tri-kam-ji
tri-pita-ka
tri-piṭa-ka
tri-vik-ra-mātma-jena
ud-ā-haraṇa
un-mārga-gama-na
upa-ca-ya-bala-varṇa-pra-sādādī-ni
upa-laghana
upa-ni-ṣads
upa-patt-ti
ut-sneha-na
utta-rā-dhya-ya-na
utta-rā-dhya-ya-na-sūtra
uttara-khaṇḍa-khādyaka
uttara-sthāna
uttara-tantra
vācas-pati-miśra-vi-racita-ṭīkā-saṃ-valita
vācas-pati-miśra-vi-racita-ṭīkā-saṃ-valita-vyā-sa-bhā-ṣya-sam-e-tāni
vag-bhata-rasa-ratna-sam-uc-caya
vāg-bhaṭa-rasa-ratna-sam-uc-caya
vaidya-vara-śrī-ḍalhaṇā-cārya-vi-racitayā
vai-śā-kha
vai-śeṣ-ika-sūtra
vāja-sa-neyi-saṃ-hitā
vājī-kara-ṇam
vākya-śeṣa
vākya-śeṣaḥ
vaṅga-sena
vaṅga-sena-saṃ-hitā
varā-ha-mihi-ra
vārāhī-kalpa
vā-rāṇa-seya
va-ra-na-si
var-mam
var-man
var-ṇa-saṃ-khyā
var-ṇa-saṅ-khyā
vā-si-ṣṭha
vasiṣṭha-saṃ-hitā
vā-siṣṭha-saṃ-hitā
Va-sistha-Sam-hita-Yoga-Kanda-With-Comm-ent-ary-Kai-valya-Dham
vastra-dhauti
vasu-bandhu
vāta-pit-ta
vāta-pit-ta-kapha
vāta-pit-ta-kapha-śoṇi-ta
vāta-pitta-kapha-śoṇita-san-nipāta-vai-ṣamya-ni-mittāḥ
vāta-pit-ta-śoṇi-ta
vāta-śleṣ-man
vāta-śleṣ-ma-śoṇi-ta
vāta-śoṇi-ta
vātā-tapika
vātsyā-ya-na
vāya-vīya-saṃ-hitā
vedāṅga-rāya
veezhi-nathan
venkat-raman
vid-vad-vara-śrī-gaṇeśa-daiva-jña-vi-racita
vidya-bhu-sana
vi-jaya-siṃ-ha
vi-jñāna-bhikṣu
Vijñāneśvara-vi-racita-mitākṣarā-vyā-khyā-sam-alaṅ-kṛtā
vi-mā-na
vi-mā-na-sthāna
vimāna-sthā-na
vi-racitā
vi-racita-yāmadhu-kośākhya-vyā-khya-yā
vi-recana
vishveshvar-anand
vi-śiṣṭ-āṃśena
vi-suddhi-magga
vi-vi-dha-tṛṇa-kāṣṭha-pāṣāṇa-pāṃ-su-loha-loṣṭāsthi-bāla-nakha-pūyā-srāva-duṣṭa-vraṇāntar-garbha-śalyo-ddharaṇārthaṃ
vṛd-dha-vṛd-dha-tara-vṛd-dha-tamaiḥ
vṛddha-vṛddha-tara-vṛddha-tamaiḥ
vṛnda-mādhava
vyāḍī-ya-pa-ri-bhā-ṣā-vṛtti
vyā-khya-yā
vy-akta-liṅgādi-dharma-yuk-te
vyā-sa-bhā-ṣya-sam-e-tāni
vyati-krāmati
Xiuyao
yādava-bhaṭṭa
yāda-va-śarma-ṇā
yādava-sūri
yājña-valkya-smṛti
yājña-valkya-smṛtiḥ
yantrā-dhyāya
Yantra-rāja-vicāra-viṃśā-dhyāyī
yavanā-cā-rya
yoga-bhā-ṣya-vyā-khyā-rūpaṃ
yoga-cintā-maṇi
yoga-cintā-maṇiḥ
yoga-ratnā-kara
yoga-sāra-mañjarī
yoga-sāra-sam-uc-caya
yoga-sāra-saṅ-graha
yoga-śikh-opa-ni-ṣat
yoga-tārā-valī
yoga-yājña-val-kya
yoga-yājña-valkya-gītāsūpa-ni-ṣatsu
yogi-yājña-valkya-smṛti
yoshi-mizu
yukta-bhava-deva
}
%%%%%%%%%%%%%%%%%%%%
%Sanskrit:
%%%%%%%%%%%%%%%%%%%%
\textsanskrit{\hyphenation{%
    dhanva-ntariṇopa-diṣ-ṭaḥ
suśruta-nāma-dheyena
tac-chiṣyeṇa
    su-śruta-san-dīpana-bhāṣya
    cikitsā-sthāna
tulya-sau-vīrāñjana
indra-gopa
dṛṣṭi-maṇḍala
uc-chiṅga-na
vi-vi-dha-tṛṇa-kāṣṭha-pāṣāṇa-pāṃ-su-loha-loṣṭāsthi-bāla-nakha-pūyā-srāva-duṣṭa-vraṇāntar-garbha-śalyo-ddharaṇārthaṃ
śrī-ḍalhaṇācārya-vi-raci-tayāni-bandha-saṃ-grahākhya-vyā-khyayā
ni-dāna-sthānasyaśrī-gaya-dāsācārya-vi-racitayānyāya-candri-kā-khya-pañjikā-vyā-khyayā
casam-ul-lasi-tāmaharṣiṇāsu-śrutenavi-raci-tāsu-śruta-saṃ-hitā
bhartṛhari-viracitaḥ
śatakatrayādi-subhāṣitasaṃgrahaḥ
mahā-kavi-bhartṛ-hari-praṇīta-tvena
nīti-śṛṅgāra-vai-rāgyādi-nāmnāsamākhyā-tānāṃ
subhāṣitānāṃ
su-pariṣkṛta-saṃgrahaḥ
su-vistṛta-pari-cayātmikyāṅla-prastāvanā-vividha-pāṭhān-tara-pari-śiṣṭādi-sam-anvitaḥ
ācārya-śrī-jina-vijayālekhitāgra-vacanālaṃ-kṛtaś-ca
abhaya-deva-sūri-vi-racita-vṛtti-vi-bhūṣi-tam
abhi-dhar-ma
abhi-dhar-ma-ko-śa
abhi-dhar-ma-ko-śa-bhā-ṣya
abhi-dharma-kośa-bhāṣyam
abhyaṃ-karopāhva-vāsu-deva-śāstri-vi-racita-yā
agni-veśa
āhā-ra-vi-hā-ra-pra-kṛ-tiṃ
ahir-budhnya
ahir-budhnya-saṃ-hitā
akusī-dasya
alter-na-tively
amara-bharati
amara-bhāratī
āmla
amlīkā
ānan-da-rā-ya
anna-mardanādi-bhiś
anu-bhav-ād
anu-bhū-ta-viṣayā-sam-pra-moṣa
anu-bhū-ta-viṣayā-sam-pra-moṣaḥ
anu-māna
anu-miti-mānasa-vāda
ariya-pary-esana-sutta
ārogya-śālā-karaṇā-sam-arthas
ārogya-śālām
ārogyāyopa-kal-pya
arś-āṃ-si
ar-tha
ar-thaḥ
ārya-bhaṭa
ārya-lalita-vistara-nāma-mahā-yāna-sūtra
ārya-mañju-śrī-mūla-kalpa
ārya-mañju-śrī-mūla-kalpaḥ
asaṃ-pra-moṣa
āsana
āsanam
āsanaṃ
asid-dhe
aṣṭāṅga-hṛdaya
aṣṭāṅga-hṛdaya-saṃ-hitā
aṣṭ-āṅga-saṅ-graha
aṣṭ-āṅgā-yur-veda
aśva-gan-dha-kalpa
aśva-ghoṣa
ātaṅka-darpaṇa
ātaṅka-darpaṇa-vyā-khyā-yā
atha-vā
ātu-r-ā-hā-ra-vi-hā-ra-pra-kṛ-tiṃ
aty-al-pam
auṣa-dha-pāvanādi-śālāś
ava-sāda-na
avic-chin-na-sam-pra-dāya-tvād
āyur-veda
āyur-veda-sāra
āyur-vedod-dhāra-ka-vaid-ya-pañc-ānana-vaid-ya-rat-na-rāja-vaid-ya-paṇḍi-ta-rā-ma-pra-sāda-vaid-yo-pādhyā-ya-vi-ra-ci-tā
bahir-deśa-ka
bala-bhadra
bāla-kṛṣṇa
bau-dhā-yana-dhar-ma-sūtra
bhadrā-sana
bhadrā-sanam
bha-ga-vad-gī-tā
bha-ga-vat-pāda
bhaṭṭot-pala-vi-vṛti-sahitā
bhṛtyāva-satha-saṃ-yuktām
bhū-miṃ
bhu--va-na-dī-pa-ka
bīja-pallava
bodhi-sat-tva-bhūmi
brāhmaṇa-pra-mukha-nānā-sat-tva-vyā-dhi-śānty-ar-tham
brāhmaṇa-pra-mukha-nānā-sat-tve-bhyo
brahmāṇḍa-mahā-purā-ṇa
brahmāṇḍa-mahā-purā-ṇam
brāhma-sphu-ṭa-siddhānta
brahma-vi-hāra
brahma-vi-hāras
bṛhad-āraṇya-ka
bṛhad-yā-trā
bṛhad-yogi-yājña-valkya-smṛti
bṛhad-yogī-yājña-valkya-smṛti
bṛhaj-jāta-kam
cak-ra-dat-ta
cak-ra-pā-ṇi-datta
cā-luk-ya
caraka-prati-saṃ-s-kṛta
caraka-prati-saṃ-s-kṛte
cara-ka-saṃ-hitā
ca-tur-thī-vi-bhak-ti
cau-kham-ba
cau-luk-yas
chau-kham-bha
chun-nam
cikit-sā-saṅ-gra-ha
daiva-jñālaṃ-kṛti
daiva-jñālaṅ-kṛti
darśa-nāṅkur-ābhi-dhayāvyā-khya-yā
deva-nagari
deva-nāgarī
dhar-ma-megha
dhar-ma-meghaḥ
dhyā-na-grahopa-deśā-dhyā-yaś
dṛṣṭ-ān-ta
dṛṣṭ-ār-tha
dvāra-tvam
evaṃ-gṛ-hī-tam
evaṃ-vi-dh-a-sya
gala-gaṇḍa
gala-gaṇḍādi-kar-tṛ-tvaṃ
gan-dh-ā-ra
gar-bha-śa-rī-ram
gaurī-kāñcali-kā-tan-tra
gauta-mādi-tra-yo-da-śa-smṛty-ātma-kaḥ
gheraṇḍa-saṃ-hitā
gran-tha-śreṇi
gran-tha-śreṇiḥ
guru-maṇḍala-grantha-mālā
hari-śāstrī
hari-śās-trī
haṭha-yoga
hāyana-rat-na
hema-pra-bha-sūri
hetv-ābhā-sa
hīna-mithy-āti-yoga
hīna-mithy-āti-yogena
hindī-vyā-khyā-vi-marśope-taḥ
hoern-le
idam
ijya-rkṣe
ikka-vālaga
ity-arthaḥ
jābāla-darśanopa-ni-ṣad
jal-pa-kal-pa-tāru
jam-bū-dvī-pa
jam-bū-dvī-pa-pra-jña-pti
jam-bū-dvī-pa-pra-jña-pti-sūtra
jāta-ka-kar-ma-pad-dhati
jinā-agama-grantha-mālā
jī-vā-nan-da-nam
jñā-na-nir-mala
jñā-na-nir-malaṃ
jya-rkṣe
kāka-caṇḍīśvara-kal-pa-tan-tra
kā-la-gar-bhā-śa-ya-pra-kṛ-tim
kā-la-gar-bhā-śa-ya-pra-kṛ-tiṃ
kali-kāla-sarva-jña
kali-kāla-sarva-jña-śrī-hema-candrācārya-vi-raci-ta
kali-kāla-sarva-jña-śrī-hema-candrācārya-vi-raci-taḥ
kali-yuga
kal-pa-sthāna
kar-ma
kar-man
kārt-snyena
katham
kāvya-mālā
keśa-va-śāstrī
kol-ka-ta
kṛṣṇa-pakṣa
kṛtti-kā
kṛtti-kās
kula-pañji-kā
ku-māra-saṃ-bhava
lab-dhāni
mada-na-phalam
mādha-va
Mādhava-karaaita-reya-brāhma-ṇa
Mādhava-ni-dāna
mādhava-ni-dā-nam
madhu-kośa
madhu-kośākhya-vyā-khya-yā
madhya
madhye
ma-hā-bhū-ta-vi-kā-ra-pra-kṛ-tiṃ
mahā-deva
mahā-mati-śrī-mādhava-kara-pra-ṇī-taṃ
mahā-muni-śrī-mad-vyāsa-pra-ṇī-ta
mahā-muni-śrī-mad-vyāsa-pra-ṇī-taṃ
maha-rṣi-pra-ṇīta-dharma-śāstra-saṃ-grahaḥ
mahā-sacca-ka-sutta
mahau-ṣadhi-pari-cchadāṃ
mahā-vra-ta
mahā-yāna-sūtrālaṅ-kāra
mano-ratha-nandin
matsya-purāṇam
me-dhā-ti-thi
medhā-tithi
mithilā-stha
mithilā-stham
mithilā-sthaṃ
mud-rā-yantr-ā-laye
muktā-pīḍa
mūla-pāṭha
nakṣa-tra
nandi-purāṇoktārogya-śālā-dāna-phala-prāpti-kāmo
nara-siṃha
nara-siṃha-bhāṣya
nārā-ya-ṇa-dāsa
nārā-yaṇa-kaṇṭha
nārā-yaṇa-paṇḍi-ta-kṛtā
nava-pañca-mayor
nidā-na-sthā-na-sya
ni-ghaṇ-ṭu
nir-anta-ra-pa-da-vyā-khyā
nir-ṇaya-sā-gara
nir-ṇaya-sā-gara-yantr-ā-laye
nirūha-vasti
niś-cala-kara
ni-yukta-vaidyāṃ
nya-grodha
nya-grodho
nyāya-śās-tra
nyāya-sū-tra-śaṃ-kar
okaḥ-sātmya
okaḥ-sātmyam
okaḥ-sātmyaṃ
oka-sātmya
oka-sātmyam
oka-sātmyaṃ
oṣṭha-saṃ-puṭa
ousha-da-sala
padma-pra-bha-sūri
padma-sva-sti-kārdha-candrādike
paitā-maha-siddhā-nta
pañca-karma
pañca-karma-bhava-rogāḥ
pañca-karmādhi-kāra
pañca-karma-vi-cāra
pāñca-rātrā-gama
pañca-siddh-āntikā
pari-bhāṣā
pari-likh-ya
pātañ-jala-yoga-śās-tra
pātañ-jala-yoga-śās-tra-vi-varaṇa
pat-añ-jali
pāṭī-gaṇita
pāva-suya
pim-pal-gaon
pipal-gaon
pit-ta-kṛt
pit-ta-śleṣma-ghna
pit-ta-śleṣma-medo-meha-hik-kā-śvā-sa-kā-sāti-sā-ra-cchardi-tṛṣṇā-kṛmi-vi-ṣa-pra-śa-ma-naṃ
prā-cya
prā-cya-hindu-gran-tha-śreṇiḥ
prācya-vidyā-saṃ-śodhana-mandira
pra-dhān-āṅ-gaṃ
pra-dhān-in
pra-ka-shan
pra-kṛ-ti
pra-kṛ-tiṃ
pra-mā-ṇa-vārt-tika
pra-saṅ-khyāne
pra-śas-ta-pāda-bhāṣya
pra-śna-pra-dīpa
pra-śnārṇa-va-plava
praśnārṇava-plava
pra-śna-vaiṣṇava
pra-śna-vai-ṣṇava
prati-padyate
pra-ty-akṣa
pra-yat-na-śai-thilyā-nan-ta-sam-ā-pat-ti-bhyām
pra-yat-na-śai-thilyā-nān-tya-sam-ā-pat-ti-bhyāṃ
pra-yatna-śai-thilya-sya
puṇya-pattana
pūrṇi-mā-nta
rāja-kīya
rajjv-ābhyas-ya
rāma-kṛṣṇa
rasa-ratnā-kara
rasa-vai-śeṣika-sūtra
rogi-svasthī-karaṇānu-ṣṭhāna-mātraṃ
rūkṣa-vasti
sād-guṇya
śākalya-saṃ-hitā
sam-ā-mnāya
sāmañña-pha-la-sutta
sama-ran-gana-su-tra-dhara
samā-raṅga-ṇa-sū-tra-dhāra
sama-ra-siṃ-ha
sama-ra-siṃ-haḥ
saṃ-hitā
sāṃ-sid-dhi-ka
saṃ-śo-dhana
sam-ul-lasi-tam
śāndilyopa-ni-ṣad
śaṅ-kara
śaṅ-kara-bha-ga-vat-pāda
Śaṅ-kara-nārā-yaṇa
saṅ-khyā
sāṅ-kṛt-yā-yana
san-s-krit
sap-tame
śāra-dā-tila-ka-tan-tra
śa-raṅ-ga-deva
śār-dūla-karṇā-va-dāna
śā-rī-ra
śā-rī-ra-sthāna
śārṅga-dhara
śārṅga-dhara-saṃ-hitā
sar-va
sarva-darśana-saṃ-grahaḥ
sar-va-dar-śāna-saṅ-gra-ha
sar-va-dar-śāna-saṅ-gra-haḥ
sarv-arthāvi-veka-khyā-ter
sar-va-tan-tra-sid-dhān-ta
sar-va-tan-tra-sid-dhān-taḥ
sarva-yoga-sam-uc-caya
sar-va-yogeśvareśva-ram
śāstrā-rambha-sam-artha-na
śāstrāram-bha-sam-arthana
ṣaṭ-pañcā-śi-kā
sat-tva
saunda-ra-na-nda
sid-dha
sid-dha-man-tra
sid-dha-man-trā-hvayo
sid-dha-man-tra-pra-kāśa
sid-dha-man-tra-pra-kāśaḥ
sid-dha-man-tra-pra-kāśaś
sid-dh-ān-ta
siddhānta-śiro-maṇ
sid-dha-yoga
sid-dhi-sthāna
śi-va-śar-ma-ṇā
ska-nda-pu-rā-ṇa
sneha-basty-upa-deśāt
sodā-haraṇa-saṃ-s-kṛta-vyā-khyayā
śodha-ka-pusta-kaṃ
śo-dha-na-ci-kitsā
so-ma-val-ka
śrī-mad-devī-bhāga-vata-mahā-purāṇa
srag-dharā-tārā-sto-tra
śrī-hari-kṛṣṇa-ni-bandha-bhava-nam
śrī-hema-candrā-cārya-vi-raci-taḥ
śrī-kaṇtha-datta
śrī-kaṇtha-dattā-bhyāṃ
śrī-kṛṣṇa-dāsa
śrī-mad-amara-siṃha-vi-racitam
śrī-mad-aruṇa-dat-ta-vi-ra-ci-tayā
śrī-mad-bhaṭṭot-pala-kṛta-saṃ-s-kṛta-ṭīkā-sahitam
śrī-mad-dvai-pā-yana-muni-pra-ṇītaṃ
śrī-mad-vāg-bha-ṭa-vi-ra-ci-tam
śrī-maṃ-trī-vi-jaya-siṃha-suta-maṃ-trī-teja-siṃhena
śrī-mat-kalyāṇa-varma-vi-racitā
śrīmat-sāyaṇa-mādhavācārya-pra-ṇītaḥ
śrī-vā-cas-pati-vaidya-vi-racita-yā
śrī-vatsa
śrī-vi-jaya-rakṣi-ta
sthānāṅga-sūtra
sthira-sukha
sthira-sukham
strī-niṣevaṇa
śukla-pakṣa
su-śru-ta-saṃ-hitā
sū-tra
sūtrārthānān-upa-patti-sūca-nāt
sūtra-sthāna
su-varṇa-pra-bhāsot-tama-sū-tra
svalpauṣadha-dāna-mā-tram
śvetāśva-taropa-ni-ṣad
tad-upa-karaṇa-tāmra-kaṭāha-kalasādi-pātra-pari-cchada-nānā-vidha-vyādhi-śānty-ucitauṣadha-gaṇa-yathokta-lakṣaṇa-vaidya-nānā-vidha-pari-cāraka-yutāṃ
tājaka-muktā-valeḥ
tājika-kau-stu-bha
tājika-nīla-kaṇṭhī
tājika-yoga-sudhā-ni-dhi
tāmra-paṭṭādi-li-khi-tāṃ
tan-nir-vāhāya
tapo-dhana
tapo-dhanā
tārā-bhakti-su-dhārṇava
tārtīya-yoga-su-sudhā-ni-dhi
tegi-ccha
te-jaḥ-siṃ-ha
trai-lok-ya
trai-lokya-pra-kāśa
tri-piṭa-ka
tri-var-gaḥ
un-mār-ga-gama-na
upa-de-śa
upa-patt-ti
ut-sneha-na
utta-rā-dhyā-ya-na
uttara-sthāna
uttara-tantra
vāchas-pati
vād-ā-valī
vai-śā-kha
vai-ta-raṇa-vasti
vai-ta-raṇok-ta-guṇa-gaṇa-yu-k-taṃ
vājī-kara-ṇam
vāk-patis
vākya-śeṣa
vākya-śeṣaḥ
varā-ha-mihi-ra
va-ra-na-si
vā-rā-ṇa-sī
var-mam
var-man
varṇa-sam-ā-mnāya
va-siṣṭha-saṃ-hitā
vā-siṣṭha-saṃ-hitā
vasu-bandhu
vasu-bandhu
vāta-ghna-pit-talāl-pa-ka-pha
vātsyā-ya-na
vidya-bhu-sana
vidyā-bhū-ṣaṇa
vi-jaya-siṃ-ha
vi-jñāna-bhikṣu
vi-kal-pa
vi-kamp-i-tum
vi-mā-na-sthāna
vi-racita-yā
vishveshvar-anand
vi-śiṣṭ-āṃśena
viṣṇu-dharmot-tara-purāṇa
viśrāma-gṛha-sahitā
vi-suddhi-magga
vopa-de-vīya-sid-dha-man-tra-pra-kāśe
vyādhi-pratī-kārār-tham
vyāḍī-ya-pa-ri-bhā-ṣā-vṛtti
vyati-krāmati
vy-ava-haranti
yādava-bhaṭṭa
yāda-va-śarma-ṇā
yādava-sūri
yājña-valkya-smṛti
yavanā-cā-rya
yoga-ratnā-kara
yoga-sāra-sam-uc-caya
yoga-sāra-sam-uc-cayaḥ
yoga-sūtra-vi-vara-ṇa
yoga-yājña-valkya
yoga-yājña-valkya-gītāsūpa-ni-ṣatsu
yoga-yājña-valkyaḥ
yogi-yājña-valkya-smṛti
yuk-tiḥ
yuk-tis
}}
\normalfontlatin
\endinput
}% should work, but doesn't
% special hyphenations for Sanskrit words tagged in
% Polyglossia.
% *English,\textenglish{},text,and
% *Sanskrit,\textsanskrit{},text.
%
% English (see below for \textsanskrit)
%
\hyphenation{%
    dhanva-ntariṇopa-diṣ-ṭaḥ
    suśruta-nāma-dheyena
    tac-chiṣyeṇa
    kāśyapa-saṃ-hitā
    cikitsā-sthāna
    su-śruta-san-dīpana-bhāṣya
    dṛṣṭi-maṇḍala
    uc-chiṅga-na
    sarva-siddhānta-tattva-cūḍā-maṇi
    tulya-sau-vīrāñja-na
    indra-gopa
    śrī-mad-abhi-nava-guptā-cārya-vi-ra-cita-vi-vṛti-same-tam
    viśva-nātha
śrī-mad-devī-bhāga-vata-mahā-purāṇa
    siddhā-n-ta-sun-dara
    brāhma-sphuṭa-siddh-ānta
    bhū-ta-saṅ-khyā
    bhū-ta-saṃ-khyā
    kathi-ta-pada
    devī-bhā-ga-vata-purāṇa
    devī-bhā-ga-vata-mahā-purāṇa
    Siddhānta-saṃ-hitā-sāra-sam-uc-caya
    sau-ra-pau-rāṇi-ka-mata-sam-artha-na
    Pṛthū-da-ka-svā-min
    Brah-ma-gupta
    Brāh-ma-sphu-ṭa-siddhānta
    siddhānta-sun-dara
    vāsa-nā-bhāṣya
    catur-veda
    bhū-maṇḍala
    jñāna-rāja
    graha-gaṇi-ta-cintā-maṇi
    Śiṣya-dhī-vṛd-dhi-da-tan-tra
    brah-māṇḍa-pu-rā-ṇa
    kūr-ma-pu-rā-ṇa
    jam-bū-dvī-pa
    bhā-ga-vata-pu-rā-ṇa
    kupya-ka
    nandi-suttam
    nandi-sutta
    su-bodhiā-bāī
    asaṅ-khyāta
    saṅ-khyāta
    saṅ-khyā-pra-māṇa
    saṃ-khā-pamāṇa
    nemi-chandra
    anu-yoga-dvāra
    tattvārtha-vārtika
    aka-laṅka
    tri-loka-sāra
    gaṇi-ma-pra-māṇa
    gaṇi-ma-ppa-māṇa
    eka-pra-bhṛti
gaṇaṇā-saṃ-khā
gaṇaṇā-saṅ-khyā
dvi-pra-bhṛti
duppa-bhi-ti-saṃ-khā
vedanābhi-ghāta
Viṣṇu-dharmottara-pu-rāṇa
abhaya-deva-sūri-vi-racita-vṛtti-vi-bhūṣi-tam
abhi-dhar-ma
abhi-dhar-ma-ko-śa
abhi-dhar-ma-ko-śa-bhā-ṣya
abhi-dharma-kośa-bhāṣya
abhi-dharma-kośa-bhāṣyam
abhi-nava
abhyaṃ-karopāhva-vāsu-deva-śāstri-vi-ra-ci-ta-yā
ācārya-śrī-jina-vijayālekhitāgra-vacanālaṃ-kṛtaś-ca
ācāry-opā-hvena
ādhāra
adhi-kāra
adhi-kāras
ādi-nātha
agni-besha
agni-veśa
ahir-budhnya
ahir-budhnya-saṃ-hitā
aita-reya-brāhma-ṇa
akusī-dasya
amara-bharati
Amar-augha-pra-bo-dha
amṛ-ta-siddhi
ānanda-kanda
ānan-da-rā-ya
ānand-āśra-ma-mudraṇā-la-ya
ānand-āśra-ma-saṃ-skṛta-granth-āva-liḥ
anna-pāna-mūlā
anu-ban-dhya-lakṣaṇa-sam-anv-itās
anu-bhav-ād
anu-bhū-ta-viṣayā-sam-pra-moṣa
anu-bhū-ta-viṣayā-sam-pra-moṣaḥ
aparo-kṣā-nu-bhū-ti
app-proxi-mate-ly
ardha-rātrika-karaṇa
ārdha-rātrika-karaṇa
ariya-pary-esana-sutta
arun-dhatī
ārya-bhaṭa
ārya-bhaṭā-cārya-vi-racitam
ārya-bhaṭīya
ārya-bhaṭīyaṃ
ārya-lalita-vistara-nāma-mahā-yāna-sūtra
ārya-mañju-śrī-mūla-kalpa
ārya-mañju-śrī-mūla-kalpaḥ
asaṃ-pra-moṣa
aṣṭāṅga-hṛdaya-saṃ-hitā
aṣṭāṅga-saṃ-graha
asura-bhavana
aśva-ghoṣa
ātaṅka-darpaṇa-vyā-khyā-yā
atha-vā
ava-sāda-na
āyār-aṅga-suttaṃ
ayur-ved
ayur-veda
āyur-veda
āyur-veda-dīpikā
āyur-veda-dīpikā-vyā-khyayā
āyur-ve-da-ra-sā-yana
āyur-veda-sū-tra
ayur-vedic
āyur-vedic
ayur-yog
bādhirya
bahir-deśa-ka
bala-bhadra
bala-kot
bala-krishnan
bāla-kṛṣṇa
bau-dhā-yana-dhar-ma-sūtra
bel-valkar
bhadra-kālī-man-tra-vi-dhi-pra-karaṇa
bhadrā-sana
bhadrā-sanam
bha-ga-vat-pāda
bhaiṣajya-ratnāvalī
bhan-d-ar-kar
bhartṛhari-viracitaḥ
bhaṭṭā-cārya
bhaṭṭot-pala-vi-vṛti-sahitā
Bhiṣag-varāḍha-malla-vi-racita-dīpikā-Kāśī-rāma-vaidya-vi-raci-ta-gūḍhā-rtha-dīpikā-bhyāṃ
bhiṣag-varāḍha-malla-vi-racita-dīpikā-Kāśī-rāma-vaidya-vi-racita-gūḍhārtha-dīpikā-bhyāṃ
bhoja-deva-vi-raci-ta-rāja-mārtaṇḍā-bhi-dha-vṛtti-sam-e-tāni
bhu--va-na-dī-pa-ka
bīja-pallava
bi-kaner
bodhi-sat-tva-bhūmi
brahma-gupta
brahmā-nanda
brahmāṇḍa-mahā-purā-ṇa
brahmāṇḍa-mahā-purā-ṇam
brahma-randhra
brahma-siddh-ānta
brāhma-sphuṭa-siddh-ānta
brāhma-sphu-ṭa-siddhānta
brahma-vi-hāra
brahma-vi-hāras
brahma-yā-mala-tan-tra
Bra-ja-bhāṣā
bṛhad-āraṇya-ka
bṛhad-yā-trā
bṛhad-yogi-yājña-valkya-smṛti
bṛhad-yogī-yājña-valkya-smṛti
bṛhaj-jāta-kam
bṛhat-khe-carī-pra-kāśa
buddhi-tattva-pra-karaṇa
cak-ra-dat-ta
cakra-datta
cakra-pāṇi-datta
cā-luk-ya
caraka-prati-saṃ-s-kṛta
caraka-prati-saṃ-s-kṛte
caraka-saṃ-hitā
casam-ul-lasi-tāmaharṣiṇāsu-śrutenavi-raci-tāsu-śruta-saṃ-hitā
cau-kham-ba
cau-luk-yas
chandi-garh
chara-ka
cha-rīre
chatt-opa-dh-ya-ya
chau-kham-bha
chi-ki-tsi-ta
cid-ghanā-nanda-nātha
ci-ka-ner
com-men-taries
com-men-tary
com-pre-hen-sive-ly
daiva-jñālaṃ-kṛti
daiva-jñālaṅ-kṛti
dāmo-dara-sūnu-Śārṅga-dharācārya-vi-racitā
Dāmodara-sūnu-Śārṅga-dharācārya-vi-racitā
darśanā-ṅkur-ābhi-dhayā
das-gupta
deha-madhya
deha-saṃ-bhava-hetavaḥ
deva-datta
deva-nagari
deva-nāgarī
devā-sura-siddha-gaṇaiḥ
dha-ra-ni-dhar
dharma-megha
dharma-meghaḥ
dhru-vam
dhru-va-sya
dhru-va-yonir
dhyā-na-grahopa-deśā-dhyā-yaś
dṛḍha-śūla-yukta-rakta
dvy-ulbaṇaikolba-ṇ-aiḥ
four-fold
gan-dh-ā-ra
gārgīya-jyoti-ṣa
gārgya-ke-rala-nīla-kaṇṭha-so-ma-sutva-vi-racita-bhāṣyo-pe-tam
garuḍa-mahā-purāṇa
gaurī-kāñcali-kā-tan-tra
gau-tama
gauta-mādi-tra-yo-da-śa-smṛty-ātma-kaḥ
gheraṇḍa-saṃ-hitā
gorakṣa-śata-ka
go-tama
granth-ā-laya
grantha-mālā
gran-tha-śreṇiḥ
grāsa-pramāṇa
guru-maṇḍala-grantha-mālā
gyatso
hari-śāstrī
haṭhābhyāsa-paddhati
haṭha-ratnā-valī
Haṭha-saṅ-keta-candri-kā
haṭha-tattva-kau-mudī
haṭha-yoga
hāyana-rat-na
haya-ta-gran-tha
hema-pra-bha-sūri
hetu-lakṣaṇa-saṃ-sargād
hīna-madhyādhi-kaiś
hindī-vyā-khyā-vi-marśope-taḥ
hoern-le
ijya-rkṣa
ikka-vālaga
indra-dhvaja
indrāṇī-kalpa
indria
Īśāna-śiva-guru-deva-pad-dhati
jābāla-darśanopa-ni-ṣad
jadav-ji
jagan-nā-tha
jala-basti
jal-pa-kal-pa-tāru
jam-bū-dvī-pa-pra-jña-pti
jam-bū-dvī-pa-pra-jña-pti-sūtra
jana-pad-a-sya
jāta-ka-kar-ma-pad-dhati
jaya-siṃha
jinā-agama-grantha-mālā
jin-en-dra-bud-dhi
jīvan-muk-ti-vi-veka
jñā-na-nir-mala
jñā-na-nir-malaṃ
joga-pra-dīpya-kā
jya-rkṣe
Jyo-tiḥ-śās-tra
jyo-ti-ṣa-rāya
jyoti-ṣa-rāya
jyotiṣa-siddhānta-saṃ-graha
jyotiṣa-siddhānta-saṅ-graha
kāka-caṇḍīśvara-kal-pa-tan-tra
kakṣa-puṭa
kali-kāla-sarva-jña
kali-kāla-sarva-jña-śrī-hema-candrācārya-vi-raci-ta
kali-kāla-sarva-jña-śrī-hema-candrācārya-vi-raci-taḥ
kali-yuga
kal-pa
kal-pa-sthāna
kalyāṇa-kāraka
Kāmeśva-ra-siṃha-dara-bhaṅgā-saṃ-skṛta-viśva-vidyā-layaḥ
kapāla-bhāti
karaṇa-tilaka
kar-ma
kar-man
kāṭhaka-saṃ-hitā
kavia-rasu
kavi-raj
keśa-va-śāstrī
ke-vala--rāma
keva-la-rāma
khaṇḍa-khādyaka-tappā
khe-carī-vidyā
knowl-edge
kol-ka-ta
kriyā-krama-karī
kṛṣṇa-pakṣa
kṛtti-kā
kṛtti-kās
kubji-kā-mata-tantra
kula-pañji-kā
kul-karni
ku-māra-saṃ-bhava
kuṭi-pra-veśa
kuṭi-pra-veśika
lakṣ-mī-veṅ-kaṭ-e-ś-va-ra
lit-era-ture
lit-era-tures
locana-roga
mādha-va
mādhava-kara
mādhava-ni-dāna
mādhava-ni-dā-nam
madh-ūni
madhya
mādhyan-dina
madhye
mahā-bhāra-ta
mahā-deva
mahā-kavi-bhartṛ-hari-praṇīta-tvena
maha-mahopa-dhyaya
mahā-maho-pā-dhyā-ya-śrī-vi-jñā-na-bhikṣu-vi-raci-taṃ
mahā-mati-śrī-mādhava-kara-pra-ṇī-taṃ
mahā-mudrā
mahā-muni-śrī-mad-vyāsa-pra-ṇī-ta
mahā-muni-śrī-mad-vyāsa-pra-ṇī-taṃ
maharṣiṇā
maha-rṣi-pra-ṇīta-dharma-śāstra-saṃ-grahaḥ
Maha-rṣi-varya-śrī-yogi-yā-jña-valkya-śiṣya-vi-racitā
mahā-sacca-ka-sutta
mahā-sati-paṭṭhā-na-sutta
mahā-vra-ta
mahā-yāna-sūtrālaṅ-kāra
maitrāya-ṇī-saṃ-hitā
maktab-khānas
māla-jit
māli-nī-vijayot-tara-tan-tra
manaḥ-sam-ā-dhi
mānasol-lāsa
mānava-dharma-śāstra
mandāgni-doṣa
mannar-guḍi
mano-har-lal
mano-ratha-nandin
man-u-script
man-u-scripts
mataṅga-pārame-śvara
mater-ials
matsya-purāṇam
medh-ā-ti-thi
medhā-tithi
mithilā-stha
mithilā-stham
mithilā-sthaṃ
mṛgendra-tantra-vṛtti
mud-rā-yantr-ā-laye
muktā-pīḍa
mūla-pāṭha
muṇḍī-kalpa
mun-sh-ram
Nāda-bindū-pa-ni-ṣat
nāga-bodhi
nāga-buddhi
nakṣa-tra
nara-siṃha
nārā-yaṇa-dāsa
nārā-yaṇa-dāsa
nārā-yaṇa-kaṇṭha
nārā-yaṇa-paṇḍi-ta-kṛtā
nar-ra-tive
nata-rajan
nava-pañca-mayor
nava-re
naya-na-sukho-pā--dhyāya
ni-ban-dha-saṃ-grahā-khya-vyākhya-yā
niban-dha-san-graha
ni-dā-na
nidā-na-sthā-na-sya
ni-dāna-sthānasyaśrī-gaya-dāsācārya-vi-racitayānyāya-candri-kā-khya-pañjikā-vyā-khyayā
nir-anta-ra-pa-da-vyā-khyā
nir-guṇḍī-kalpa
nir-ṇaya-sā-gara
Nir-ṇaya-sāgara
nir-ṇa-ya-sā-gara-mudrā-yantrā-laye
nir-ṇa-ya-sā-ga-ra-yantr-āla-ya
nir-ṇaya-sā-gara-yantr-ā-laye
niśvāsa-kārikā
nīti-śṛṅgāra-vai-rāgyādi-nāmnāsamākhyā-tānāṃ
nityā-nanda
nya-grodha
nya-grodho
nyā-ya-candri-kā-khya-pañji-kā-vyā-khya-yā
nyāya-śās-tra
okaḥ-sātmya
okaḥ-sātmyam
okaḥ-sātmyaṃ
oka-sātmya
oka-sātmyam
oka-sātmyaṃ
oris-sa
oṣṭha-saṃ-puṭa
ousha-da-sala
padma-pra-bha-sūri
Padma-prā-bhṛ-ta-ka
padma-sva-sti-kārdha-candrādike
paitā-maha-siddhā-nta
pañca-karma
pañca-karman
pāñca-rātrā-gama
pañca-siddh-āntikā
paṅkti-śūla
Paraśu-rāma
paraśu-rāma
pari-likh-ya
pāśu-pata-sū-tra-bhāṣya
pātañ-jala-yoga-śās-tra
pātañ-jala-yoga-śās-tra-vi-varaṇa
pat-añ-jali
pat-na
pāva-suya
phiraṅgi-can-dra-cchedyo-pa-yogi-ka
pim-pal-gaon
pipal-gaon
pitta-śleṣ-man
pit-ta-śleṣ-ma-śoṇi-ta
pitta-śoṇi-ta
prā-cīna-rasa-granthaḥ
prā-cya
prā-cya-hindu-gran-tha-śreṇiḥ
prācya-vidyā-saṃ-śodhana-mandira
pra-dhān-in
pra-ka-shan
pra-kaṭa-mūṣā
pra-kṛ-ti-bhū-tāḥ
pra-mā-ṇa-vārt-tika
pra-ṇītā
pra-saṅ-khyāne
pra-śas-ta-pāda-bhāṣya
pra-śna-pra-dīpa
pra-śnārṇa-va-plava
praśnārṇava-plava
pra-śna-vai-ṣṇava
pra-śna-vaiṣṇava
prati-padyate
pra-yatna-śaithilyānan-ta-sam-āpatti-bhyām
prei-sen-danz
punar-vashu
puṇya-pattana
pūrṇi-mā-nta
raghu-nātha
rāja-kīya
rāja-kīya-mudraṇa-yantrā-laya
rāja-śe-khara
rajjv-ābhyas-ya
raj-put
rāj-put
rakta-mokṣa-na
rāma-candra-śāstrī
rāma-kṛṣṇa
rāma-kṛṣṇa-śāstri-ṇā
rama-su-bra-manian
rāmā-yaṇa
rasa-ratnā-kara
rasa-ratnākarāntar-ga-taś
rasa-ratna-sam-uc-caya
rasa-ratna-sam-uc-ca-yaḥ
rasa-vīry-auṣa-dha-pra-bhāvena
rasā-yana
rasendra-maṅgala
rasendra-maṅgalam
rāṣṭra-kūṭa
rāṣṭra-kūṭas
sādhana
śākalya-saṃ-hitā
śāla-grāma-kṛta
śāla-grāma-kṛta
sāmañña-pha-la-sutta
sāmañña-phala-sutta
sama-ran-gana-su-tra-dhara
samā-raṅga-ṇa-sū-tra-dhāra
sama-ra-siṃ-ha
sama-ra-siṃ-haḥ
sāmba-śiva-śāstri
same-taḥ
saṃ-hitā
śāṃ-ka-ra-bhāṣ-ya-sam-etā
sam-rāṭ
saṃ-rāṭ
Sam-rāṭ-siddhānta
Sam-rāṭ-siddhānta-kau-stu-bha
sam-rāṭ-siddhānta-kau-stu-bha
saṃ-sargam
saṃ-sargaṃ
saṃ-s-kṛta
saṃ-s-kṛta-pārasī-ka-pra-da-pra-kāśa
saṃ-śo-dhana
saṃ-śodhitā
saṃ-sthāna
sam-ullasitā
sam-ul-lasi-tam
saṃ-valitā
saṃ-valitā
śāndilyopa-ni-ṣad
śaṅ-kara
śaṅ-kara-bha-ga-vat-pāda
śaṅ-karā-cārya
san-kara-charya
Śaṅ-kara-nārā-yaṇa
sāṅ-kṛt-yā-yana
san-s-krit
śāra-dā-tila-ka-tan-tra
śa-raṅ-ga-deva
śār-dūla-karṇā-va-dāna
śār-dūla-karṇā-va-dāna
śā-rī-ra-sthāna
śārṅga-dhara-saṃ-hitā
Śārṅga-dhara-saṃ-hitā
sar-va-dar-śana-saṅ-gra-ha
sarva-kapha-ja
sarv-arthāvi-veka-khyā-ter
sar-va-śa-rīra-carās
sarva-siddhānta-rāja
Sarva-siddhā-nt-rāja
sarva-vyā-dhi-viṣāpa-ha
sarva-yoga-sam-uc-caya
sar-va-yogeśvareśva-ram
śāstrā-rambha-sam-artha-na
śatakatrayādi-subhāṣitasaṃgrahaḥ
sati-paṭṭhā-na-sutta
ṣaṭ-karma
ṣaṭ-karman
sat-karma-saṅ-graha
sat-karma-saṅ-grahaḥ
ṣaṭ-pañcā-śi-kā
saun-da-ra-nanda
sa-v-āī
schef-tel-o-witz
scholars
sharī-ra
sheth
sid-dha-man-tra
siddha-nanda-na-miśra
siddha-nanda-na-miśraḥ
siddha-nitya-nātha-pra-ṇītaḥ
Siddhānta-saṃ-hitā-sāra-sam-uc-caya
Siddhā-nta-sār-va-bhauma
siddhānta-sindhu
siddhānta-śiro-maṇ
Siddhānta-śiro-maṇi
Siddhā-nta-tat-tva-vi-veka
sid-dha-yoga
siddha-yoga
sid-dhi
sid-dhi-sthā-na
sid-dhi-sthāna
śikhi-sthāna
śiraḥ-karṇā-kṣi-vedana
śiro-bhūṣaṇam
Śivā-nanda-saras-vatī
śiva-saṃ-hitā
śiva-yo-ga-dī-pi-kā
ska-nda-pu-rā-ṇa
śleṣ-man
śleṣ-ma-śoni-ta
sodā-haraṇa-saṃ-s-kṛta-vyā-khyayā
śodha-ka-pusta-kaa
śoṇi-ta
spaṣ-ṭa-krānty-ādhi-kāra
śrī-cakra-pāṇi-datta
śrī-cakra-pāṇi-datta-viracitayā
śrī-ḍalhaṇācārya-vi-raci-tayāni-bandha-saṃ-grahākhya-vyā-khyayā
śrī-dayā-nanda
śrī-hari-kṛṣṇa-ni-bandha-bhava-nam
śrī-hema-candrā-cārya-vi-raci-taḥ
śrī-kaṇtha-dattā-bhyāṃ
śrī-kṛṣṇa-dāsa
śrī-kṛṣṇa-dāsa-śreṣṭhinā
śrīmac-chaṅ-kara-bhaga-vat-pāda-vi-raci-tā
śrī-mad-amara-siṃha-vi-racitam
śrī-mad-bha-ga-vad-gī-tā
śrī-mad-bhaṭṭot-pala-kṛta-saṃ-s-kṛta-ṭīkā-sahitam
śrī-mad-dvai-pā-yana-muni-pra-ṇītaṃ
śrī-mad-vāg-bhaṭa-vi-raci-tam
śrī-maṃ-trī-vi-jaya-siṃha-suta-maṃ-trī-teja-siṃhena
śrī-mat-kalyāṇa-varma-vi-racitā
śrī-mat-sāyaṇa-mādhavācārya-pra-ṇītaḥsarva-darśana-saṃ-grahaḥ
śrī-nitya-nātha-siddha-vi-raci-taḥ
śrī-rāja-śe-khara
śrī-śaṃ-karā-cārya-vi-raci-tam
śrī-vā-cas-pati-vaidya-vi-racita-yā
śrī-vatsa
śrī-veda-vyāsa-pra-ṇīta-mahā-bhā-ratāntar-ga-tā
śrī-veṅkaṭeś-vara
śrī-vi-jaya-rakṣi-ta
sruta-rakta
sruta-raktasya
stambha-karam
sthānāṅga-sūtra
sthira-sukha
sthira-sukham
stra-sthā-na
subhāṣitānāṃ
su-brah-man-ya
su-bra-man-ya
śukla-pakṣa
śukrā-srava
suk-than-kar
su-pariṣkṛta-saṃgrahaḥ
sura-bhi-pra-kash-an
sūrya-dāsa
sūrya-siddhānta
su-shru-ta
su-śru-ta
su-shru-ta-saṃ-hitā
su-śru-ta-saṃ-hitā
su-śru-tena
sutra
sūtra
sūtra-neti
sūtra-ni-dāna-śā-rīra-ci-ki-tsā-kal-pa-sthānot-tara-tan-trātma-kaḥ
sūtra-sthāna
su-varṇa-pra-bhāsot-tama-sū-tra
Su-var-ṇa-pra-bhās-ot-tama-sū-tra
su-varṇa-pra-bhāsotta-ma-sūtra
su-vistṛta-pari-cayātmikyāṅla-prastāvanā-vividha-pāṭhān-tara-pari-śiṣṭādi-sam-anvitaḥ
sva-bhāva-vyādhi-ni-vāraṇa-vi-śiṣṭ-auṣa-dha-cintakās
svā-bhāvika
svā-bhāvikās
sva-cchanda-tantra
śvetāśva-taropa-ni-ṣad
taila-sarpir-ma-dhūni
tait-tirīya-brāhma-ṇa
tājaka-muktā-valeḥ
tājika-kau-stu-bha
tājika-nīla-kaṇṭhī
tājika-yoga-sudhā-ni-dhi
tapo-dhana
tapo-dhanā
tārā-bhakti-su-dhārṇava
tārtīya-yoga-su-sudhā-ni-dhi
tegi-ccha
te-jaḥ-siṃ-ha
ṭhāṇ-āṅga-sutta
ṭīkā-bhyāṃ
ṭīkā-bhyāṃ
tiru-mantiram
tiru-ttoṇṭar-purāṇam
tiru-va-nanta-puram
trai-lok-ya
trai-lokya-pra-kāśa
tri-bhāga
tri-kam-ji
tri-pita-ka
tri-piṭa-ka
tri-vik-ra-mātma-jena
ud-ā-haraṇa
un-mārga-gama-na
upa-ca-ya-bala-varṇa-pra-sādādī-ni
upa-laghana
upa-ni-ṣads
upa-patt-ti
ut-sneha-na
utta-rā-dhya-ya-na
utta-rā-dhya-ya-na-sūtra
uttara-khaṇḍa-khādyaka
uttara-sthāna
uttara-tantra
vācas-pati-miśra-vi-racita-ṭīkā-saṃ-valita
vācas-pati-miśra-vi-racita-ṭīkā-saṃ-valita-vyā-sa-bhā-ṣya-sam-e-tāni
vag-bhata-rasa-ratna-sam-uc-caya
vāg-bhaṭa-rasa-ratna-sam-uc-caya
vaidya-vara-śrī-ḍalhaṇā-cārya-vi-racitayā
vai-śā-kha
vai-śeṣ-ika-sūtra
vāja-sa-neyi-saṃ-hitā
vājī-kara-ṇam
vākya-śeṣa
vākya-śeṣaḥ
vaṅga-sena
vaṅga-sena-saṃ-hitā
varā-ha-mihi-ra
vārāhī-kalpa
vā-rāṇa-seya
va-ra-na-si
var-mam
var-man
var-ṇa-saṃ-khyā
var-ṇa-saṅ-khyā
vā-si-ṣṭha
vasiṣṭha-saṃ-hitā
vā-siṣṭha-saṃ-hitā
Va-sistha-Sam-hita-Yoga-Kanda-With-Comm-ent-ary-Kai-valya-Dham
vastra-dhauti
vasu-bandhu
vāta-pit-ta
vāta-pit-ta-kapha
vāta-pit-ta-kapha-śoṇi-ta
vāta-pitta-kapha-śoṇita-san-nipāta-vai-ṣamya-ni-mittāḥ
vāta-pit-ta-śoṇi-ta
vāta-śleṣ-man
vāta-śleṣ-ma-śoṇi-ta
vāta-śoṇi-ta
vātā-tapika
vātsyā-ya-na
vāya-vīya-saṃ-hitā
vedāṅga-rāya
veezhi-nathan
venkat-raman
vid-vad-vara-śrī-gaṇeśa-daiva-jña-vi-racita
vidya-bhu-sana
vi-jaya-siṃ-ha
vi-jñāna-bhikṣu
Vijñāneśvara-vi-racita-mitākṣarā-vyā-khyā-sam-alaṅ-kṛtā
vi-mā-na
vi-mā-na-sthāna
vimāna-sthā-na
vi-racitā
vi-racita-yāmadhu-kośākhya-vyā-khya-yā
vi-recana
vishveshvar-anand
vi-śiṣṭ-āṃśena
vi-suddhi-magga
vi-vi-dha-tṛṇa-kāṣṭha-pāṣāṇa-pāṃ-su-loha-loṣṭāsthi-bāla-nakha-pūyā-srāva-duṣṭa-vraṇāntar-garbha-śalyo-ddharaṇārthaṃ
vṛd-dha-vṛd-dha-tara-vṛd-dha-tamaiḥ
vṛddha-vṛddha-tara-vṛddha-tamaiḥ
vṛnda-mādhava
vyāḍī-ya-pa-ri-bhā-ṣā-vṛtti
vyā-khya-yā
vy-akta-liṅgādi-dharma-yuk-te
vyā-sa-bhā-ṣya-sam-e-tāni
vyati-krāmati
Xiuyao
yādava-bhaṭṭa
yāda-va-śarma-ṇā
yādava-sūri
yājña-valkya-smṛti
yājña-valkya-smṛtiḥ
yantrā-dhyāya
Yantra-rāja-vicāra-viṃśā-dhyāyī
yavanā-cā-rya
yoga-bhā-ṣya-vyā-khyā-rūpaṃ
yoga-cintā-maṇi
yoga-cintā-maṇiḥ
yoga-ratnā-kara
yoga-sāra-mañjarī
yoga-sāra-sam-uc-caya
yoga-sāra-saṅ-graha
yoga-śikh-opa-ni-ṣat
yoga-tārā-valī
yoga-yājña-val-kya
yoga-yājña-valkya-gītāsūpa-ni-ṣatsu
yogi-yājña-valkya-smṛti
yoshi-mizu
yukta-bhava-deva
}
%%%%%%%%%%%%%%%%%%%%
%Sanskrit:
%%%%%%%%%%%%%%%%%%%%
\textsanskrit{\hyphenation{%
    dhanva-ntariṇopa-diṣ-ṭaḥ
suśruta-nāma-dheyena
tac-chiṣyeṇa
    su-śruta-san-dīpana-bhāṣya
    cikitsā-sthāna
tulya-sau-vīrāñjana
indra-gopa
dṛṣṭi-maṇḍala
uc-chiṅga-na
vi-vi-dha-tṛṇa-kāṣṭha-pāṣāṇa-pāṃ-su-loha-loṣṭāsthi-bāla-nakha-pūyā-srāva-duṣṭa-vraṇāntar-garbha-śalyo-ddharaṇārthaṃ
śrī-ḍalhaṇācārya-vi-raci-tayāni-bandha-saṃ-grahākhya-vyā-khyayā
ni-dāna-sthānasyaśrī-gaya-dāsācārya-vi-racitayānyāya-candri-kā-khya-pañjikā-vyā-khyayā
casam-ul-lasi-tāmaharṣiṇāsu-śrutenavi-raci-tāsu-śruta-saṃ-hitā
bhartṛhari-viracitaḥ
śatakatrayādi-subhāṣitasaṃgrahaḥ
mahā-kavi-bhartṛ-hari-praṇīta-tvena
nīti-śṛṅgāra-vai-rāgyādi-nāmnāsamākhyā-tānāṃ
subhāṣitānāṃ
su-pariṣkṛta-saṃgrahaḥ
su-vistṛta-pari-cayātmikyāṅla-prastāvanā-vividha-pāṭhān-tara-pari-śiṣṭādi-sam-anvitaḥ
ācārya-śrī-jina-vijayālekhitāgra-vacanālaṃ-kṛtaś-ca
abhaya-deva-sūri-vi-racita-vṛtti-vi-bhūṣi-tam
abhi-dhar-ma
abhi-dhar-ma-ko-śa
abhi-dhar-ma-ko-śa-bhā-ṣya
abhi-dharma-kośa-bhāṣyam
abhyaṃ-karopāhva-vāsu-deva-śāstri-vi-racita-yā
agni-veśa
āhā-ra-vi-hā-ra-pra-kṛ-tiṃ
ahir-budhnya
ahir-budhnya-saṃ-hitā
akusī-dasya
alter-na-tively
amara-bharati
amara-bhāratī
āmla
amlīkā
ānan-da-rā-ya
anna-mardanādi-bhiś
anu-bhav-ād
anu-bhū-ta-viṣayā-sam-pra-moṣa
anu-bhū-ta-viṣayā-sam-pra-moṣaḥ
anu-māna
anu-miti-mānasa-vāda
ariya-pary-esana-sutta
ārogya-śālā-karaṇā-sam-arthas
ārogya-śālām
ārogyāyopa-kal-pya
arś-āṃ-si
ar-tha
ar-thaḥ
ārya-bhaṭa
ārya-lalita-vistara-nāma-mahā-yāna-sūtra
ārya-mañju-śrī-mūla-kalpa
ārya-mañju-śrī-mūla-kalpaḥ
asaṃ-pra-moṣa
āsana
āsanam
āsanaṃ
asid-dhe
aṣṭāṅga-hṛdaya
aṣṭāṅga-hṛdaya-saṃ-hitā
aṣṭ-āṅga-saṅ-graha
aṣṭ-āṅgā-yur-veda
aśva-gan-dha-kalpa
aśva-ghoṣa
ātaṅka-darpaṇa
ātaṅka-darpaṇa-vyā-khyā-yā
atha-vā
ātu-r-ā-hā-ra-vi-hā-ra-pra-kṛ-tiṃ
aty-al-pam
auṣa-dha-pāvanādi-śālāś
ava-sāda-na
avic-chin-na-sam-pra-dāya-tvād
āyur-veda
āyur-veda-sāra
āyur-vedod-dhāra-ka-vaid-ya-pañc-ānana-vaid-ya-rat-na-rāja-vaid-ya-paṇḍi-ta-rā-ma-pra-sāda-vaid-yo-pādhyā-ya-vi-ra-ci-tā
bahir-deśa-ka
bala-bhadra
bāla-kṛṣṇa
bau-dhā-yana-dhar-ma-sūtra
bhadrā-sana
bhadrā-sanam
bha-ga-vad-gī-tā
bha-ga-vat-pāda
bhaṭṭot-pala-vi-vṛti-sahitā
bhṛtyāva-satha-saṃ-yuktām
bhū-miṃ
bhu--va-na-dī-pa-ka
bīja-pallava
bodhi-sat-tva-bhūmi
brāhmaṇa-pra-mukha-nānā-sat-tva-vyā-dhi-śānty-ar-tham
brāhmaṇa-pra-mukha-nānā-sat-tve-bhyo
brahmāṇḍa-mahā-purā-ṇa
brahmāṇḍa-mahā-purā-ṇam
brāhma-sphu-ṭa-siddhānta
brahma-vi-hāra
brahma-vi-hāras
bṛhad-āraṇya-ka
bṛhad-yā-trā
bṛhad-yogi-yājña-valkya-smṛti
bṛhad-yogī-yājña-valkya-smṛti
bṛhaj-jāta-kam
cak-ra-dat-ta
cak-ra-pā-ṇi-datta
cā-luk-ya
caraka-prati-saṃ-s-kṛta
caraka-prati-saṃ-s-kṛte
cara-ka-saṃ-hitā
ca-tur-thī-vi-bhak-ti
cau-kham-ba
cau-luk-yas
chau-kham-bha
chun-nam
cikit-sā-saṅ-gra-ha
daiva-jñālaṃ-kṛti
daiva-jñālaṅ-kṛti
darśa-nāṅkur-ābhi-dhayāvyā-khya-yā
deva-nagari
deva-nāgarī
dhar-ma-megha
dhar-ma-meghaḥ
dhyā-na-grahopa-deśā-dhyā-yaś
dṛṣṭ-ān-ta
dṛṣṭ-ār-tha
dvāra-tvam
evaṃ-gṛ-hī-tam
evaṃ-vi-dh-a-sya
gala-gaṇḍa
gala-gaṇḍādi-kar-tṛ-tvaṃ
gan-dh-ā-ra
gar-bha-śa-rī-ram
gaurī-kāñcali-kā-tan-tra
gauta-mādi-tra-yo-da-śa-smṛty-ātma-kaḥ
gheraṇḍa-saṃ-hitā
gran-tha-śreṇi
gran-tha-śreṇiḥ
guru-maṇḍala-grantha-mālā
hari-śāstrī
hari-śās-trī
haṭha-yoga
hāyana-rat-na
hema-pra-bha-sūri
hetv-ābhā-sa
hīna-mithy-āti-yoga
hīna-mithy-āti-yogena
hindī-vyā-khyā-vi-marśope-taḥ
hoern-le
idam
ijya-rkṣe
ikka-vālaga
ity-arthaḥ
jābāla-darśanopa-ni-ṣad
jal-pa-kal-pa-tāru
jam-bū-dvī-pa
jam-bū-dvī-pa-pra-jña-pti
jam-bū-dvī-pa-pra-jña-pti-sūtra
jāta-ka-kar-ma-pad-dhati
jinā-agama-grantha-mālā
jī-vā-nan-da-nam
jñā-na-nir-mala
jñā-na-nir-malaṃ
jya-rkṣe
kāka-caṇḍīśvara-kal-pa-tan-tra
kā-la-gar-bhā-śa-ya-pra-kṛ-tim
kā-la-gar-bhā-śa-ya-pra-kṛ-tiṃ
kali-kāla-sarva-jña
kali-kāla-sarva-jña-śrī-hema-candrācārya-vi-raci-ta
kali-kāla-sarva-jña-śrī-hema-candrācārya-vi-raci-taḥ
kali-yuga
kal-pa-sthāna
kar-ma
kar-man
kārt-snyena
katham
kāvya-mālā
keśa-va-śāstrī
kol-ka-ta
kṛṣṇa-pakṣa
kṛtti-kā
kṛtti-kās
kula-pañji-kā
ku-māra-saṃ-bhava
lab-dhāni
mada-na-phalam
mādha-va
Mādhava-karaaita-reya-brāhma-ṇa
Mādhava-ni-dāna
mādhava-ni-dā-nam
madhu-kośa
madhu-kośākhya-vyā-khya-yā
madhya
madhye
ma-hā-bhū-ta-vi-kā-ra-pra-kṛ-tiṃ
mahā-deva
mahā-mati-śrī-mādhava-kara-pra-ṇī-taṃ
mahā-muni-śrī-mad-vyāsa-pra-ṇī-ta
mahā-muni-śrī-mad-vyāsa-pra-ṇī-taṃ
maha-rṣi-pra-ṇīta-dharma-śāstra-saṃ-grahaḥ
mahā-sacca-ka-sutta
mahau-ṣadhi-pari-cchadāṃ
mahā-vra-ta
mahā-yāna-sūtrālaṅ-kāra
mano-ratha-nandin
matsya-purāṇam
me-dhā-ti-thi
medhā-tithi
mithilā-stha
mithilā-stham
mithilā-sthaṃ
mud-rā-yantr-ā-laye
muktā-pīḍa
mūla-pāṭha
nakṣa-tra
nandi-purāṇoktārogya-śālā-dāna-phala-prāpti-kāmo
nara-siṃha
nara-siṃha-bhāṣya
nārā-ya-ṇa-dāsa
nārā-yaṇa-kaṇṭha
nārā-yaṇa-paṇḍi-ta-kṛtā
nava-pañca-mayor
nidā-na-sthā-na-sya
ni-ghaṇ-ṭu
nir-anta-ra-pa-da-vyā-khyā
nir-ṇaya-sā-gara
nir-ṇaya-sā-gara-yantr-ā-laye
nirūha-vasti
niś-cala-kara
ni-yukta-vaidyāṃ
nya-grodha
nya-grodho
nyāya-śās-tra
nyāya-sū-tra-śaṃ-kar
okaḥ-sātmya
okaḥ-sātmyam
okaḥ-sātmyaṃ
oka-sātmya
oka-sātmyam
oka-sātmyaṃ
oṣṭha-saṃ-puṭa
ousha-da-sala
padma-pra-bha-sūri
padma-sva-sti-kārdha-candrādike
paitā-maha-siddhā-nta
pañca-karma
pañca-karma-bhava-rogāḥ
pañca-karmādhi-kāra
pañca-karma-vi-cāra
pāñca-rātrā-gama
pañca-siddh-āntikā
pari-bhāṣā
pari-likh-ya
pātañ-jala-yoga-śās-tra
pātañ-jala-yoga-śās-tra-vi-varaṇa
pat-añ-jali
pāṭī-gaṇita
pāva-suya
pim-pal-gaon
pipal-gaon
pit-ta-kṛt
pit-ta-śleṣma-ghna
pit-ta-śleṣma-medo-meha-hik-kā-śvā-sa-kā-sāti-sā-ra-cchardi-tṛṣṇā-kṛmi-vi-ṣa-pra-śa-ma-naṃ
prā-cya
prā-cya-hindu-gran-tha-śreṇiḥ
prācya-vidyā-saṃ-śodhana-mandira
pra-dhān-āṅ-gaṃ
pra-dhān-in
pra-ka-shan
pra-kṛ-ti
pra-kṛ-tiṃ
pra-mā-ṇa-vārt-tika
pra-saṅ-khyāne
pra-śas-ta-pāda-bhāṣya
pra-śna-pra-dīpa
pra-śnārṇa-va-plava
praśnārṇava-plava
pra-śna-vaiṣṇava
pra-śna-vai-ṣṇava
prati-padyate
pra-ty-akṣa
pra-yat-na-śai-thilyā-nan-ta-sam-ā-pat-ti-bhyām
pra-yat-na-śai-thilyā-nān-tya-sam-ā-pat-ti-bhyāṃ
pra-yatna-śai-thilya-sya
puṇya-pattana
pūrṇi-mā-nta
rāja-kīya
rajjv-ābhyas-ya
rāma-kṛṣṇa
rasa-ratnā-kara
rasa-vai-śeṣika-sūtra
rogi-svasthī-karaṇānu-ṣṭhāna-mātraṃ
rūkṣa-vasti
sād-guṇya
śākalya-saṃ-hitā
sam-ā-mnāya
sāmañña-pha-la-sutta
sama-ran-gana-su-tra-dhara
samā-raṅga-ṇa-sū-tra-dhāra
sama-ra-siṃ-ha
sama-ra-siṃ-haḥ
saṃ-hitā
sāṃ-sid-dhi-ka
saṃ-śo-dhana
sam-ul-lasi-tam
śāndilyopa-ni-ṣad
śaṅ-kara
śaṅ-kara-bha-ga-vat-pāda
Śaṅ-kara-nārā-yaṇa
saṅ-khyā
sāṅ-kṛt-yā-yana
san-s-krit
sap-tame
śāra-dā-tila-ka-tan-tra
śa-raṅ-ga-deva
śār-dūla-karṇā-va-dāna
śā-rī-ra
śā-rī-ra-sthāna
śārṅga-dhara
śārṅga-dhara-saṃ-hitā
sar-va
sarva-darśana-saṃ-grahaḥ
sar-va-dar-śāna-saṅ-gra-ha
sar-va-dar-śāna-saṅ-gra-haḥ
sarv-arthāvi-veka-khyā-ter
sar-va-tan-tra-sid-dhān-ta
sar-va-tan-tra-sid-dhān-taḥ
sarva-yoga-sam-uc-caya
sar-va-yogeśvareśva-ram
śāstrā-rambha-sam-artha-na
śāstrāram-bha-sam-arthana
ṣaṭ-pañcā-śi-kā
sat-tva
saunda-ra-na-nda
sid-dha
sid-dha-man-tra
sid-dha-man-trā-hvayo
sid-dha-man-tra-pra-kāśa
sid-dha-man-tra-pra-kāśaḥ
sid-dha-man-tra-pra-kāśaś
sid-dh-ān-ta
siddhānta-śiro-maṇ
sid-dha-yoga
sid-dhi-sthāna
śi-va-śar-ma-ṇā
ska-nda-pu-rā-ṇa
sneha-basty-upa-deśāt
sodā-haraṇa-saṃ-s-kṛta-vyā-khyayā
śodha-ka-pusta-kaṃ
śo-dha-na-ci-kitsā
so-ma-val-ka
śrī-mad-devī-bhāga-vata-mahā-purāṇa
srag-dharā-tārā-sto-tra
śrī-hari-kṛṣṇa-ni-bandha-bhava-nam
śrī-hema-candrā-cārya-vi-raci-taḥ
śrī-kaṇtha-datta
śrī-kaṇtha-dattā-bhyāṃ
śrī-kṛṣṇa-dāsa
śrī-mad-amara-siṃha-vi-racitam
śrī-mad-aruṇa-dat-ta-vi-ra-ci-tayā
śrī-mad-bhaṭṭot-pala-kṛta-saṃ-s-kṛta-ṭīkā-sahitam
śrī-mad-dvai-pā-yana-muni-pra-ṇītaṃ
śrī-mad-vāg-bha-ṭa-vi-ra-ci-tam
śrī-maṃ-trī-vi-jaya-siṃha-suta-maṃ-trī-teja-siṃhena
śrī-mat-kalyāṇa-varma-vi-racitā
śrīmat-sāyaṇa-mādhavācārya-pra-ṇītaḥ
śrī-vā-cas-pati-vaidya-vi-racita-yā
śrī-vatsa
śrī-vi-jaya-rakṣi-ta
sthānāṅga-sūtra
sthira-sukha
sthira-sukham
strī-niṣevaṇa
śukla-pakṣa
su-śru-ta-saṃ-hitā
sū-tra
sūtrārthānān-upa-patti-sūca-nāt
sūtra-sthāna
su-varṇa-pra-bhāsot-tama-sū-tra
svalpauṣadha-dāna-mā-tram
śvetāśva-taropa-ni-ṣad
tad-upa-karaṇa-tāmra-kaṭāha-kalasādi-pātra-pari-cchada-nānā-vidha-vyādhi-śānty-ucitauṣadha-gaṇa-yathokta-lakṣaṇa-vaidya-nānā-vidha-pari-cāraka-yutāṃ
tājaka-muktā-valeḥ
tājika-kau-stu-bha
tājika-nīla-kaṇṭhī
tājika-yoga-sudhā-ni-dhi
tāmra-paṭṭādi-li-khi-tāṃ
tan-nir-vāhāya
tapo-dhana
tapo-dhanā
tārā-bhakti-su-dhārṇava
tārtīya-yoga-su-sudhā-ni-dhi
tegi-ccha
te-jaḥ-siṃ-ha
trai-lok-ya
trai-lokya-pra-kāśa
tri-piṭa-ka
tri-var-gaḥ
un-mār-ga-gama-na
upa-de-śa
upa-patt-ti
ut-sneha-na
utta-rā-dhyā-ya-na
uttara-sthāna
uttara-tantra
vāchas-pati
vād-ā-valī
vai-śā-kha
vai-ta-raṇa-vasti
vai-ta-raṇok-ta-guṇa-gaṇa-yu-k-taṃ
vājī-kara-ṇam
vāk-patis
vākya-śeṣa
vākya-śeṣaḥ
varā-ha-mihi-ra
va-ra-na-si
vā-rā-ṇa-sī
var-mam
var-man
varṇa-sam-ā-mnāya
va-siṣṭha-saṃ-hitā
vā-siṣṭha-saṃ-hitā
vasu-bandhu
vasu-bandhu
vāta-ghna-pit-talāl-pa-ka-pha
vātsyā-ya-na
vidya-bhu-sana
vidyā-bhū-ṣaṇa
vi-jaya-siṃ-ha
vi-jñāna-bhikṣu
vi-kal-pa
vi-kamp-i-tum
vi-mā-na-sthāna
vi-racita-yā
vishveshvar-anand
vi-śiṣṭ-āṃśena
viṣṇu-dharmot-tara-purāṇa
viśrāma-gṛha-sahitā
vi-suddhi-magga
vopa-de-vīya-sid-dha-man-tra-pra-kāśe
vyādhi-pratī-kārār-tham
vyāḍī-ya-pa-ri-bhā-ṣā-vṛtti
vyati-krāmati
vy-ava-haranti
yādava-bhaṭṭa
yāda-va-śarma-ṇā
yādava-sūri
yājña-valkya-smṛti
yavanā-cā-rya
yoga-ratnā-kara
yoga-sāra-sam-uc-caya
yoga-sāra-sam-uc-cayaḥ
yoga-sūtra-vi-vara-ṇa
yoga-yājña-valkya
yoga-yājña-valkya-gītāsūpa-ni-ṣatsu
yoga-yājña-valkyaḥ
yogi-yājña-valkya-smṛti
yuk-tiḥ
yuk-tis
}}
\normalfontlatin
\endinput
}% should work, but doesn't
% special hyphenations for Sanskrit words tagged in
% Polyglossia.
% *English,\textenglish{},text,and
% *Sanskrit,\textsanskrit{},text.
%
% English (see below for \textsanskrit)
%
\hyphenation{%
    dhanva-ntariṇopa-diṣ-ṭaḥ
    suśruta-nāma-dheyena
    tac-chiṣyeṇa
    kāśyapa-saṃ-hitā
    cikitsā-sthāna
    su-śruta-san-dīpana-bhāṣya
    dṛṣṭi-maṇḍala
    uc-chiṅga-na
    sarva-siddhānta-tattva-cūḍā-maṇi
    tulya-sau-vīrāñja-na
    indra-gopa
    śrī-mad-abhi-nava-guptā-cārya-vi-ra-cita-vi-vṛti-same-tam
    viśva-nātha
śrī-mad-devī-bhāga-vata-mahā-purāṇa
    siddhā-n-ta-sun-dara
    brāhma-sphuṭa-siddh-ānta
    bhū-ta-saṅ-khyā
    bhū-ta-saṃ-khyā
    kathi-ta-pada
    devī-bhā-ga-vata-purāṇa
    devī-bhā-ga-vata-mahā-purāṇa
    Siddhānta-saṃ-hitā-sāra-sam-uc-caya
    sau-ra-pau-rāṇi-ka-mata-sam-artha-na
    Pṛthū-da-ka-svā-min
    Brah-ma-gupta
    Brāh-ma-sphu-ṭa-siddhānta
    siddhānta-sun-dara
    vāsa-nā-bhāṣya
    catur-veda
    bhū-maṇḍala
    jñāna-rāja
    graha-gaṇi-ta-cintā-maṇi
    Śiṣya-dhī-vṛd-dhi-da-tan-tra
    brah-māṇḍa-pu-rā-ṇa
    kūr-ma-pu-rā-ṇa
    jam-bū-dvī-pa
    bhā-ga-vata-pu-rā-ṇa
    kupya-ka
    nandi-suttam
    nandi-sutta
    su-bodhiā-bāī
    asaṅ-khyāta
    saṅ-khyāta
    saṅ-khyā-pra-māṇa
    saṃ-khā-pamāṇa
    nemi-chandra
    anu-yoga-dvāra
    tattvārtha-vārtika
    aka-laṅka
    tri-loka-sāra
    gaṇi-ma-pra-māṇa
    gaṇi-ma-ppa-māṇa
    eka-pra-bhṛti
gaṇaṇā-saṃ-khā
gaṇaṇā-saṅ-khyā
dvi-pra-bhṛti
duppa-bhi-ti-saṃ-khā
vedanābhi-ghāta
Viṣṇu-dharmottara-pu-rāṇa
abhaya-deva-sūri-vi-racita-vṛtti-vi-bhūṣi-tam
abhi-dhar-ma
abhi-dhar-ma-ko-śa
abhi-dhar-ma-ko-śa-bhā-ṣya
abhi-dharma-kośa-bhāṣya
abhi-dharma-kośa-bhāṣyam
abhi-nava
abhyaṃ-karopāhva-vāsu-deva-śāstri-vi-ra-ci-ta-yā
ācārya-śrī-jina-vijayālekhitāgra-vacanālaṃ-kṛtaś-ca
ācāry-opā-hvena
ādhāra
adhi-kāra
adhi-kāras
ādi-nātha
agni-besha
agni-veśa
ahir-budhnya
ahir-budhnya-saṃ-hitā
aita-reya-brāhma-ṇa
akusī-dasya
amara-bharati
Amar-augha-pra-bo-dha
amṛ-ta-siddhi
ānanda-kanda
ānan-da-rā-ya
ānand-āśra-ma-mudraṇā-la-ya
ānand-āśra-ma-saṃ-skṛta-granth-āva-liḥ
anna-pāna-mūlā
anu-ban-dhya-lakṣaṇa-sam-anv-itās
anu-bhav-ād
anu-bhū-ta-viṣayā-sam-pra-moṣa
anu-bhū-ta-viṣayā-sam-pra-moṣaḥ
aparo-kṣā-nu-bhū-ti
app-proxi-mate-ly
ardha-rātrika-karaṇa
ārdha-rātrika-karaṇa
ariya-pary-esana-sutta
arun-dhatī
ārya-bhaṭa
ārya-bhaṭā-cārya-vi-racitam
ārya-bhaṭīya
ārya-bhaṭīyaṃ
ārya-lalita-vistara-nāma-mahā-yāna-sūtra
ārya-mañju-śrī-mūla-kalpa
ārya-mañju-śrī-mūla-kalpaḥ
asaṃ-pra-moṣa
aṣṭāṅga-hṛdaya-saṃ-hitā
aṣṭāṅga-saṃ-graha
asura-bhavana
aśva-ghoṣa
ātaṅka-darpaṇa-vyā-khyā-yā
atha-vā
ava-sāda-na
āyār-aṅga-suttaṃ
ayur-ved
ayur-veda
āyur-veda
āyur-veda-dīpikā
āyur-veda-dīpikā-vyā-khyayā
āyur-ve-da-ra-sā-yana
āyur-veda-sū-tra
ayur-vedic
āyur-vedic
ayur-yog
bādhirya
bahir-deśa-ka
bala-bhadra
bala-kot
bala-krishnan
bāla-kṛṣṇa
bau-dhā-yana-dhar-ma-sūtra
bel-valkar
bhadra-kālī-man-tra-vi-dhi-pra-karaṇa
bhadrā-sana
bhadrā-sanam
bha-ga-vat-pāda
bhaiṣajya-ratnāvalī
bhan-d-ar-kar
bhartṛhari-viracitaḥ
bhaṭṭā-cārya
bhaṭṭot-pala-vi-vṛti-sahitā
Bhiṣag-varāḍha-malla-vi-racita-dīpikā-Kāśī-rāma-vaidya-vi-raci-ta-gūḍhā-rtha-dīpikā-bhyāṃ
bhiṣag-varāḍha-malla-vi-racita-dīpikā-Kāśī-rāma-vaidya-vi-racita-gūḍhārtha-dīpikā-bhyāṃ
bhoja-deva-vi-raci-ta-rāja-mārtaṇḍā-bhi-dha-vṛtti-sam-e-tāni
bhu--va-na-dī-pa-ka
bīja-pallava
bi-kaner
bodhi-sat-tva-bhūmi
brahma-gupta
brahmā-nanda
brahmāṇḍa-mahā-purā-ṇa
brahmāṇḍa-mahā-purā-ṇam
brahma-randhra
brahma-siddh-ānta
brāhma-sphuṭa-siddh-ānta
brāhma-sphu-ṭa-siddhānta
brahma-vi-hāra
brahma-vi-hāras
brahma-yā-mala-tan-tra
Bra-ja-bhāṣā
bṛhad-āraṇya-ka
bṛhad-yā-trā
bṛhad-yogi-yājña-valkya-smṛti
bṛhad-yogī-yājña-valkya-smṛti
bṛhaj-jāta-kam
bṛhat-khe-carī-pra-kāśa
buddhi-tattva-pra-karaṇa
cak-ra-dat-ta
cakra-datta
cakra-pāṇi-datta
cā-luk-ya
caraka-prati-saṃ-s-kṛta
caraka-prati-saṃ-s-kṛte
caraka-saṃ-hitā
casam-ul-lasi-tāmaharṣiṇāsu-śrutenavi-raci-tāsu-śruta-saṃ-hitā
cau-kham-ba
cau-luk-yas
chandi-garh
chara-ka
cha-rīre
chatt-opa-dh-ya-ya
chau-kham-bha
chi-ki-tsi-ta
cid-ghanā-nanda-nātha
ci-ka-ner
com-men-taries
com-men-tary
com-pre-hen-sive-ly
daiva-jñālaṃ-kṛti
daiva-jñālaṅ-kṛti
dāmo-dara-sūnu-Śārṅga-dharācārya-vi-racitā
Dāmodara-sūnu-Śārṅga-dharācārya-vi-racitā
darśanā-ṅkur-ābhi-dhayā
das-gupta
deha-madhya
deha-saṃ-bhava-hetavaḥ
deva-datta
deva-nagari
deva-nāgarī
devā-sura-siddha-gaṇaiḥ
dha-ra-ni-dhar
dharma-megha
dharma-meghaḥ
dhru-vam
dhru-va-sya
dhru-va-yonir
dhyā-na-grahopa-deśā-dhyā-yaś
dṛḍha-śūla-yukta-rakta
dvy-ulbaṇaikolba-ṇ-aiḥ
four-fold
gan-dh-ā-ra
gārgīya-jyoti-ṣa
gārgya-ke-rala-nīla-kaṇṭha-so-ma-sutva-vi-racita-bhāṣyo-pe-tam
garuḍa-mahā-purāṇa
gaurī-kāñcali-kā-tan-tra
gau-tama
gauta-mādi-tra-yo-da-śa-smṛty-ātma-kaḥ
gheraṇḍa-saṃ-hitā
gorakṣa-śata-ka
go-tama
granth-ā-laya
grantha-mālā
gran-tha-śreṇiḥ
grāsa-pramāṇa
guru-maṇḍala-grantha-mālā
gyatso
hari-śāstrī
haṭhābhyāsa-paddhati
haṭha-ratnā-valī
Haṭha-saṅ-keta-candri-kā
haṭha-tattva-kau-mudī
haṭha-yoga
hāyana-rat-na
haya-ta-gran-tha
hema-pra-bha-sūri
hetu-lakṣaṇa-saṃ-sargād
hīna-madhyādhi-kaiś
hindī-vyā-khyā-vi-marśope-taḥ
hoern-le
ijya-rkṣa
ikka-vālaga
indra-dhvaja
indrāṇī-kalpa
indria
Īśāna-śiva-guru-deva-pad-dhati
jābāla-darśanopa-ni-ṣad
jadav-ji
jagan-nā-tha
jala-basti
jal-pa-kal-pa-tāru
jam-bū-dvī-pa-pra-jña-pti
jam-bū-dvī-pa-pra-jña-pti-sūtra
jana-pad-a-sya
jāta-ka-kar-ma-pad-dhati
jaya-siṃha
jinā-agama-grantha-mālā
jin-en-dra-bud-dhi
jīvan-muk-ti-vi-veka
jñā-na-nir-mala
jñā-na-nir-malaṃ
joga-pra-dīpya-kā
jya-rkṣe
Jyo-tiḥ-śās-tra
jyo-ti-ṣa-rāya
jyoti-ṣa-rāya
jyotiṣa-siddhānta-saṃ-graha
jyotiṣa-siddhānta-saṅ-graha
kāka-caṇḍīśvara-kal-pa-tan-tra
kakṣa-puṭa
kali-kāla-sarva-jña
kali-kāla-sarva-jña-śrī-hema-candrācārya-vi-raci-ta
kali-kāla-sarva-jña-śrī-hema-candrācārya-vi-raci-taḥ
kali-yuga
kal-pa
kal-pa-sthāna
kalyāṇa-kāraka
Kāmeśva-ra-siṃha-dara-bhaṅgā-saṃ-skṛta-viśva-vidyā-layaḥ
kapāla-bhāti
karaṇa-tilaka
kar-ma
kar-man
kāṭhaka-saṃ-hitā
kavia-rasu
kavi-raj
keśa-va-śāstrī
ke-vala--rāma
keva-la-rāma
khaṇḍa-khādyaka-tappā
khe-carī-vidyā
knowl-edge
kol-ka-ta
kriyā-krama-karī
kṛṣṇa-pakṣa
kṛtti-kā
kṛtti-kās
kubji-kā-mata-tantra
kula-pañji-kā
kul-karni
ku-māra-saṃ-bhava
kuṭi-pra-veśa
kuṭi-pra-veśika
lakṣ-mī-veṅ-kaṭ-e-ś-va-ra
lit-era-ture
lit-era-tures
locana-roga
mādha-va
mādhava-kara
mādhava-ni-dāna
mādhava-ni-dā-nam
madh-ūni
madhya
mādhyan-dina
madhye
mahā-bhāra-ta
mahā-deva
mahā-kavi-bhartṛ-hari-praṇīta-tvena
maha-mahopa-dhyaya
mahā-maho-pā-dhyā-ya-śrī-vi-jñā-na-bhikṣu-vi-raci-taṃ
mahā-mati-śrī-mādhava-kara-pra-ṇī-taṃ
mahā-mudrā
mahā-muni-śrī-mad-vyāsa-pra-ṇī-ta
mahā-muni-śrī-mad-vyāsa-pra-ṇī-taṃ
maharṣiṇā
maha-rṣi-pra-ṇīta-dharma-śāstra-saṃ-grahaḥ
Maha-rṣi-varya-śrī-yogi-yā-jña-valkya-śiṣya-vi-racitā
mahā-sacca-ka-sutta
mahā-sati-paṭṭhā-na-sutta
mahā-vra-ta
mahā-yāna-sūtrālaṅ-kāra
maitrāya-ṇī-saṃ-hitā
maktab-khānas
māla-jit
māli-nī-vijayot-tara-tan-tra
manaḥ-sam-ā-dhi
mānasol-lāsa
mānava-dharma-śāstra
mandāgni-doṣa
mannar-guḍi
mano-har-lal
mano-ratha-nandin
man-u-script
man-u-scripts
mataṅga-pārame-śvara
mater-ials
matsya-purāṇam
medh-ā-ti-thi
medhā-tithi
mithilā-stha
mithilā-stham
mithilā-sthaṃ
mṛgendra-tantra-vṛtti
mud-rā-yantr-ā-laye
muktā-pīḍa
mūla-pāṭha
muṇḍī-kalpa
mun-sh-ram
Nāda-bindū-pa-ni-ṣat
nāga-bodhi
nāga-buddhi
nakṣa-tra
nara-siṃha
nārā-yaṇa-dāsa
nārā-yaṇa-dāsa
nārā-yaṇa-kaṇṭha
nārā-yaṇa-paṇḍi-ta-kṛtā
nar-ra-tive
nata-rajan
nava-pañca-mayor
nava-re
naya-na-sukho-pā--dhyāya
ni-ban-dha-saṃ-grahā-khya-vyākhya-yā
niban-dha-san-graha
ni-dā-na
nidā-na-sthā-na-sya
ni-dāna-sthānasyaśrī-gaya-dāsācārya-vi-racitayānyāya-candri-kā-khya-pañjikā-vyā-khyayā
nir-anta-ra-pa-da-vyā-khyā
nir-guṇḍī-kalpa
nir-ṇaya-sā-gara
Nir-ṇaya-sāgara
nir-ṇa-ya-sā-gara-mudrā-yantrā-laye
nir-ṇa-ya-sā-ga-ra-yantr-āla-ya
nir-ṇaya-sā-gara-yantr-ā-laye
niśvāsa-kārikā
nīti-śṛṅgāra-vai-rāgyādi-nāmnāsamākhyā-tānāṃ
nityā-nanda
nya-grodha
nya-grodho
nyā-ya-candri-kā-khya-pañji-kā-vyā-khya-yā
nyāya-śās-tra
okaḥ-sātmya
okaḥ-sātmyam
okaḥ-sātmyaṃ
oka-sātmya
oka-sātmyam
oka-sātmyaṃ
oris-sa
oṣṭha-saṃ-puṭa
ousha-da-sala
padma-pra-bha-sūri
Padma-prā-bhṛ-ta-ka
padma-sva-sti-kārdha-candrādike
paitā-maha-siddhā-nta
pañca-karma
pañca-karman
pāñca-rātrā-gama
pañca-siddh-āntikā
paṅkti-śūla
Paraśu-rāma
paraśu-rāma
pari-likh-ya
pāśu-pata-sū-tra-bhāṣya
pātañ-jala-yoga-śās-tra
pātañ-jala-yoga-śās-tra-vi-varaṇa
pat-añ-jali
pat-na
pāva-suya
phiraṅgi-can-dra-cchedyo-pa-yogi-ka
pim-pal-gaon
pipal-gaon
pitta-śleṣ-man
pit-ta-śleṣ-ma-śoṇi-ta
pitta-śoṇi-ta
prā-cīna-rasa-granthaḥ
prā-cya
prā-cya-hindu-gran-tha-śreṇiḥ
prācya-vidyā-saṃ-śodhana-mandira
pra-dhān-in
pra-ka-shan
pra-kaṭa-mūṣā
pra-kṛ-ti-bhū-tāḥ
pra-mā-ṇa-vārt-tika
pra-ṇītā
pra-saṅ-khyāne
pra-śas-ta-pāda-bhāṣya
pra-śna-pra-dīpa
pra-śnārṇa-va-plava
praśnārṇava-plava
pra-śna-vai-ṣṇava
pra-śna-vaiṣṇava
prati-padyate
pra-yatna-śaithilyānan-ta-sam-āpatti-bhyām
prei-sen-danz
punar-vashu
puṇya-pattana
pūrṇi-mā-nta
raghu-nātha
rāja-kīya
rāja-kīya-mudraṇa-yantrā-laya
rāja-śe-khara
rajjv-ābhyas-ya
raj-put
rāj-put
rakta-mokṣa-na
rāma-candra-śāstrī
rāma-kṛṣṇa
rāma-kṛṣṇa-śāstri-ṇā
rama-su-bra-manian
rāmā-yaṇa
rasa-ratnā-kara
rasa-ratnākarāntar-ga-taś
rasa-ratna-sam-uc-caya
rasa-ratna-sam-uc-ca-yaḥ
rasa-vīry-auṣa-dha-pra-bhāvena
rasā-yana
rasendra-maṅgala
rasendra-maṅgalam
rāṣṭra-kūṭa
rāṣṭra-kūṭas
sādhana
śākalya-saṃ-hitā
śāla-grāma-kṛta
śāla-grāma-kṛta
sāmañña-pha-la-sutta
sāmañña-phala-sutta
sama-ran-gana-su-tra-dhara
samā-raṅga-ṇa-sū-tra-dhāra
sama-ra-siṃ-ha
sama-ra-siṃ-haḥ
sāmba-śiva-śāstri
same-taḥ
saṃ-hitā
śāṃ-ka-ra-bhāṣ-ya-sam-etā
sam-rāṭ
saṃ-rāṭ
Sam-rāṭ-siddhānta
Sam-rāṭ-siddhānta-kau-stu-bha
sam-rāṭ-siddhānta-kau-stu-bha
saṃ-sargam
saṃ-sargaṃ
saṃ-s-kṛta
saṃ-s-kṛta-pārasī-ka-pra-da-pra-kāśa
saṃ-śo-dhana
saṃ-śodhitā
saṃ-sthāna
sam-ullasitā
sam-ul-lasi-tam
saṃ-valitā
saṃ-valitā
śāndilyopa-ni-ṣad
śaṅ-kara
śaṅ-kara-bha-ga-vat-pāda
śaṅ-karā-cārya
san-kara-charya
Śaṅ-kara-nārā-yaṇa
sāṅ-kṛt-yā-yana
san-s-krit
śāra-dā-tila-ka-tan-tra
śa-raṅ-ga-deva
śār-dūla-karṇā-va-dāna
śār-dūla-karṇā-va-dāna
śā-rī-ra-sthāna
śārṅga-dhara-saṃ-hitā
Śārṅga-dhara-saṃ-hitā
sar-va-dar-śana-saṅ-gra-ha
sarva-kapha-ja
sarv-arthāvi-veka-khyā-ter
sar-va-śa-rīra-carās
sarva-siddhānta-rāja
Sarva-siddhā-nt-rāja
sarva-vyā-dhi-viṣāpa-ha
sarva-yoga-sam-uc-caya
sar-va-yogeśvareśva-ram
śāstrā-rambha-sam-artha-na
śatakatrayādi-subhāṣitasaṃgrahaḥ
sati-paṭṭhā-na-sutta
ṣaṭ-karma
ṣaṭ-karman
sat-karma-saṅ-graha
sat-karma-saṅ-grahaḥ
ṣaṭ-pañcā-śi-kā
saun-da-ra-nanda
sa-v-āī
schef-tel-o-witz
scholars
sharī-ra
sheth
sid-dha-man-tra
siddha-nanda-na-miśra
siddha-nanda-na-miśraḥ
siddha-nitya-nātha-pra-ṇītaḥ
Siddhānta-saṃ-hitā-sāra-sam-uc-caya
Siddhā-nta-sār-va-bhauma
siddhānta-sindhu
siddhānta-śiro-maṇ
Siddhānta-śiro-maṇi
Siddhā-nta-tat-tva-vi-veka
sid-dha-yoga
siddha-yoga
sid-dhi
sid-dhi-sthā-na
sid-dhi-sthāna
śikhi-sthāna
śiraḥ-karṇā-kṣi-vedana
śiro-bhūṣaṇam
Śivā-nanda-saras-vatī
śiva-saṃ-hitā
śiva-yo-ga-dī-pi-kā
ska-nda-pu-rā-ṇa
śleṣ-man
śleṣ-ma-śoni-ta
sodā-haraṇa-saṃ-s-kṛta-vyā-khyayā
śodha-ka-pusta-kaa
śoṇi-ta
spaṣ-ṭa-krānty-ādhi-kāra
śrī-cakra-pāṇi-datta
śrī-cakra-pāṇi-datta-viracitayā
śrī-ḍalhaṇācārya-vi-raci-tayāni-bandha-saṃ-grahākhya-vyā-khyayā
śrī-dayā-nanda
śrī-hari-kṛṣṇa-ni-bandha-bhava-nam
śrī-hema-candrā-cārya-vi-raci-taḥ
śrī-kaṇtha-dattā-bhyāṃ
śrī-kṛṣṇa-dāsa
śrī-kṛṣṇa-dāsa-śreṣṭhinā
śrīmac-chaṅ-kara-bhaga-vat-pāda-vi-raci-tā
śrī-mad-amara-siṃha-vi-racitam
śrī-mad-bha-ga-vad-gī-tā
śrī-mad-bhaṭṭot-pala-kṛta-saṃ-s-kṛta-ṭīkā-sahitam
śrī-mad-dvai-pā-yana-muni-pra-ṇītaṃ
śrī-mad-vāg-bhaṭa-vi-raci-tam
śrī-maṃ-trī-vi-jaya-siṃha-suta-maṃ-trī-teja-siṃhena
śrī-mat-kalyāṇa-varma-vi-racitā
śrī-mat-sāyaṇa-mādhavācārya-pra-ṇītaḥsarva-darśana-saṃ-grahaḥ
śrī-nitya-nātha-siddha-vi-raci-taḥ
śrī-rāja-śe-khara
śrī-śaṃ-karā-cārya-vi-raci-tam
śrī-vā-cas-pati-vaidya-vi-racita-yā
śrī-vatsa
śrī-veda-vyāsa-pra-ṇīta-mahā-bhā-ratāntar-ga-tā
śrī-veṅkaṭeś-vara
śrī-vi-jaya-rakṣi-ta
sruta-rakta
sruta-raktasya
stambha-karam
sthānāṅga-sūtra
sthira-sukha
sthira-sukham
stra-sthā-na
subhāṣitānāṃ
su-brah-man-ya
su-bra-man-ya
śukla-pakṣa
śukrā-srava
suk-than-kar
su-pariṣkṛta-saṃgrahaḥ
sura-bhi-pra-kash-an
sūrya-dāsa
sūrya-siddhānta
su-shru-ta
su-śru-ta
su-shru-ta-saṃ-hitā
su-śru-ta-saṃ-hitā
su-śru-tena
sutra
sūtra
sūtra-neti
sūtra-ni-dāna-śā-rīra-ci-ki-tsā-kal-pa-sthānot-tara-tan-trātma-kaḥ
sūtra-sthāna
su-varṇa-pra-bhāsot-tama-sū-tra
Su-var-ṇa-pra-bhās-ot-tama-sū-tra
su-varṇa-pra-bhāsotta-ma-sūtra
su-vistṛta-pari-cayātmikyāṅla-prastāvanā-vividha-pāṭhān-tara-pari-śiṣṭādi-sam-anvitaḥ
sva-bhāva-vyādhi-ni-vāraṇa-vi-śiṣṭ-auṣa-dha-cintakās
svā-bhāvika
svā-bhāvikās
sva-cchanda-tantra
śvetāśva-taropa-ni-ṣad
taila-sarpir-ma-dhūni
tait-tirīya-brāhma-ṇa
tājaka-muktā-valeḥ
tājika-kau-stu-bha
tājika-nīla-kaṇṭhī
tājika-yoga-sudhā-ni-dhi
tapo-dhana
tapo-dhanā
tārā-bhakti-su-dhārṇava
tārtīya-yoga-su-sudhā-ni-dhi
tegi-ccha
te-jaḥ-siṃ-ha
ṭhāṇ-āṅga-sutta
ṭīkā-bhyāṃ
ṭīkā-bhyāṃ
tiru-mantiram
tiru-ttoṇṭar-purāṇam
tiru-va-nanta-puram
trai-lok-ya
trai-lokya-pra-kāśa
tri-bhāga
tri-kam-ji
tri-pita-ka
tri-piṭa-ka
tri-vik-ra-mātma-jena
ud-ā-haraṇa
un-mārga-gama-na
upa-ca-ya-bala-varṇa-pra-sādādī-ni
upa-laghana
upa-ni-ṣads
upa-patt-ti
ut-sneha-na
utta-rā-dhya-ya-na
utta-rā-dhya-ya-na-sūtra
uttara-khaṇḍa-khādyaka
uttara-sthāna
uttara-tantra
vācas-pati-miśra-vi-racita-ṭīkā-saṃ-valita
vācas-pati-miśra-vi-racita-ṭīkā-saṃ-valita-vyā-sa-bhā-ṣya-sam-e-tāni
vag-bhata-rasa-ratna-sam-uc-caya
vāg-bhaṭa-rasa-ratna-sam-uc-caya
vaidya-vara-śrī-ḍalhaṇā-cārya-vi-racitayā
vai-śā-kha
vai-śeṣ-ika-sūtra
vāja-sa-neyi-saṃ-hitā
vājī-kara-ṇam
vākya-śeṣa
vākya-śeṣaḥ
vaṅga-sena
vaṅga-sena-saṃ-hitā
varā-ha-mihi-ra
vārāhī-kalpa
vā-rāṇa-seya
va-ra-na-si
var-mam
var-man
var-ṇa-saṃ-khyā
var-ṇa-saṅ-khyā
vā-si-ṣṭha
vasiṣṭha-saṃ-hitā
vā-siṣṭha-saṃ-hitā
Va-sistha-Sam-hita-Yoga-Kanda-With-Comm-ent-ary-Kai-valya-Dham
vastra-dhauti
vasu-bandhu
vāta-pit-ta
vāta-pit-ta-kapha
vāta-pit-ta-kapha-śoṇi-ta
vāta-pitta-kapha-śoṇita-san-nipāta-vai-ṣamya-ni-mittāḥ
vāta-pit-ta-śoṇi-ta
vāta-śleṣ-man
vāta-śleṣ-ma-śoṇi-ta
vāta-śoṇi-ta
vātā-tapika
vātsyā-ya-na
vāya-vīya-saṃ-hitā
vedāṅga-rāya
veezhi-nathan
venkat-raman
vid-vad-vara-śrī-gaṇeśa-daiva-jña-vi-racita
vidya-bhu-sana
vi-jaya-siṃ-ha
vi-jñāna-bhikṣu
Vijñāneśvara-vi-racita-mitākṣarā-vyā-khyā-sam-alaṅ-kṛtā
vi-mā-na
vi-mā-na-sthāna
vimāna-sthā-na
vi-racitā
vi-racita-yāmadhu-kośākhya-vyā-khya-yā
vi-recana
vishveshvar-anand
vi-śiṣṭ-āṃśena
vi-suddhi-magga
vi-vi-dha-tṛṇa-kāṣṭha-pāṣāṇa-pāṃ-su-loha-loṣṭāsthi-bāla-nakha-pūyā-srāva-duṣṭa-vraṇāntar-garbha-śalyo-ddharaṇārthaṃ
vṛd-dha-vṛd-dha-tara-vṛd-dha-tamaiḥ
vṛddha-vṛddha-tara-vṛddha-tamaiḥ
vṛnda-mādhava
vyāḍī-ya-pa-ri-bhā-ṣā-vṛtti
vyā-khya-yā
vy-akta-liṅgādi-dharma-yuk-te
vyā-sa-bhā-ṣya-sam-e-tāni
vyati-krāmati
Xiuyao
yādava-bhaṭṭa
yāda-va-śarma-ṇā
yādava-sūri
yājña-valkya-smṛti
yājña-valkya-smṛtiḥ
yantrā-dhyāya
Yantra-rāja-vicāra-viṃśā-dhyāyī
yavanā-cā-rya
yoga-bhā-ṣya-vyā-khyā-rūpaṃ
yoga-cintā-maṇi
yoga-cintā-maṇiḥ
yoga-ratnā-kara
yoga-sāra-mañjarī
yoga-sāra-sam-uc-caya
yoga-sāra-saṅ-graha
yoga-śikh-opa-ni-ṣat
yoga-tārā-valī
yoga-yājña-val-kya
yoga-yājña-valkya-gītāsūpa-ni-ṣatsu
yogi-yājña-valkya-smṛti
yoshi-mizu
yukta-bhava-deva
}
%%%%%%%%%%%%%%%%%%%%
%Sanskrit:
%%%%%%%%%%%%%%%%%%%%
\textsanskrit{\hyphenation{%
    dhanva-ntariṇopa-diṣ-ṭaḥ
suśruta-nāma-dheyena
tac-chiṣyeṇa
    su-śruta-san-dīpana-bhāṣya
    cikitsā-sthāna
tulya-sau-vīrāñjana
indra-gopa
dṛṣṭi-maṇḍala
uc-chiṅga-na
vi-vi-dha-tṛṇa-kāṣṭha-pāṣāṇa-pāṃ-su-loha-loṣṭāsthi-bāla-nakha-pūyā-srāva-duṣṭa-vraṇāntar-garbha-śalyo-ddharaṇārthaṃ
śrī-ḍalhaṇācārya-vi-raci-tayāni-bandha-saṃ-grahākhya-vyā-khyayā
ni-dāna-sthānasyaśrī-gaya-dāsācārya-vi-racitayānyāya-candri-kā-khya-pañjikā-vyā-khyayā
casam-ul-lasi-tāmaharṣiṇāsu-śrutenavi-raci-tāsu-śruta-saṃ-hitā
bhartṛhari-viracitaḥ
śatakatrayādi-subhāṣitasaṃgrahaḥ
mahā-kavi-bhartṛ-hari-praṇīta-tvena
nīti-śṛṅgāra-vai-rāgyādi-nāmnāsamākhyā-tānāṃ
subhāṣitānāṃ
su-pariṣkṛta-saṃgrahaḥ
su-vistṛta-pari-cayātmikyāṅla-prastāvanā-vividha-pāṭhān-tara-pari-śiṣṭādi-sam-anvitaḥ
ācārya-śrī-jina-vijayālekhitāgra-vacanālaṃ-kṛtaś-ca
abhaya-deva-sūri-vi-racita-vṛtti-vi-bhūṣi-tam
abhi-dhar-ma
abhi-dhar-ma-ko-śa
abhi-dhar-ma-ko-śa-bhā-ṣya
abhi-dharma-kośa-bhāṣyam
abhyaṃ-karopāhva-vāsu-deva-śāstri-vi-racita-yā
agni-veśa
āhā-ra-vi-hā-ra-pra-kṛ-tiṃ
ahir-budhnya
ahir-budhnya-saṃ-hitā
akusī-dasya
alter-na-tively
amara-bharati
amara-bhāratī
āmla
amlīkā
ānan-da-rā-ya
anna-mardanādi-bhiś
anu-bhav-ād
anu-bhū-ta-viṣayā-sam-pra-moṣa
anu-bhū-ta-viṣayā-sam-pra-moṣaḥ
anu-māna
anu-miti-mānasa-vāda
ariya-pary-esana-sutta
ārogya-śālā-karaṇā-sam-arthas
ārogya-śālām
ārogyāyopa-kal-pya
arś-āṃ-si
ar-tha
ar-thaḥ
ārya-bhaṭa
ārya-lalita-vistara-nāma-mahā-yāna-sūtra
ārya-mañju-śrī-mūla-kalpa
ārya-mañju-śrī-mūla-kalpaḥ
asaṃ-pra-moṣa
āsana
āsanam
āsanaṃ
asid-dhe
aṣṭāṅga-hṛdaya
aṣṭāṅga-hṛdaya-saṃ-hitā
aṣṭ-āṅga-saṅ-graha
aṣṭ-āṅgā-yur-veda
aśva-gan-dha-kalpa
aśva-ghoṣa
ātaṅka-darpaṇa
ātaṅka-darpaṇa-vyā-khyā-yā
atha-vā
ātu-r-ā-hā-ra-vi-hā-ra-pra-kṛ-tiṃ
aty-al-pam
auṣa-dha-pāvanādi-śālāś
ava-sāda-na
avic-chin-na-sam-pra-dāya-tvād
āyur-veda
āyur-veda-sāra
āyur-vedod-dhāra-ka-vaid-ya-pañc-ānana-vaid-ya-rat-na-rāja-vaid-ya-paṇḍi-ta-rā-ma-pra-sāda-vaid-yo-pādhyā-ya-vi-ra-ci-tā
bahir-deśa-ka
bala-bhadra
bāla-kṛṣṇa
bau-dhā-yana-dhar-ma-sūtra
bhadrā-sana
bhadrā-sanam
bha-ga-vad-gī-tā
bha-ga-vat-pāda
bhaṭṭot-pala-vi-vṛti-sahitā
bhṛtyāva-satha-saṃ-yuktām
bhū-miṃ
bhu--va-na-dī-pa-ka
bīja-pallava
bodhi-sat-tva-bhūmi
brāhmaṇa-pra-mukha-nānā-sat-tva-vyā-dhi-śānty-ar-tham
brāhmaṇa-pra-mukha-nānā-sat-tve-bhyo
brahmāṇḍa-mahā-purā-ṇa
brahmāṇḍa-mahā-purā-ṇam
brāhma-sphu-ṭa-siddhānta
brahma-vi-hāra
brahma-vi-hāras
bṛhad-āraṇya-ka
bṛhad-yā-trā
bṛhad-yogi-yājña-valkya-smṛti
bṛhad-yogī-yājña-valkya-smṛti
bṛhaj-jāta-kam
cak-ra-dat-ta
cak-ra-pā-ṇi-datta
cā-luk-ya
caraka-prati-saṃ-s-kṛta
caraka-prati-saṃ-s-kṛte
cara-ka-saṃ-hitā
ca-tur-thī-vi-bhak-ti
cau-kham-ba
cau-luk-yas
chau-kham-bha
chun-nam
cikit-sā-saṅ-gra-ha
daiva-jñālaṃ-kṛti
daiva-jñālaṅ-kṛti
darśa-nāṅkur-ābhi-dhayāvyā-khya-yā
deva-nagari
deva-nāgarī
dhar-ma-megha
dhar-ma-meghaḥ
dhyā-na-grahopa-deśā-dhyā-yaś
dṛṣṭ-ān-ta
dṛṣṭ-ār-tha
dvāra-tvam
evaṃ-gṛ-hī-tam
evaṃ-vi-dh-a-sya
gala-gaṇḍa
gala-gaṇḍādi-kar-tṛ-tvaṃ
gan-dh-ā-ra
gar-bha-śa-rī-ram
gaurī-kāñcali-kā-tan-tra
gauta-mādi-tra-yo-da-śa-smṛty-ātma-kaḥ
gheraṇḍa-saṃ-hitā
gran-tha-śreṇi
gran-tha-śreṇiḥ
guru-maṇḍala-grantha-mālā
hari-śāstrī
hari-śās-trī
haṭha-yoga
hāyana-rat-na
hema-pra-bha-sūri
hetv-ābhā-sa
hīna-mithy-āti-yoga
hīna-mithy-āti-yogena
hindī-vyā-khyā-vi-marśope-taḥ
hoern-le
idam
ijya-rkṣe
ikka-vālaga
ity-arthaḥ
jābāla-darśanopa-ni-ṣad
jal-pa-kal-pa-tāru
jam-bū-dvī-pa
jam-bū-dvī-pa-pra-jña-pti
jam-bū-dvī-pa-pra-jña-pti-sūtra
jāta-ka-kar-ma-pad-dhati
jinā-agama-grantha-mālā
jī-vā-nan-da-nam
jñā-na-nir-mala
jñā-na-nir-malaṃ
jya-rkṣe
kāka-caṇḍīśvara-kal-pa-tan-tra
kā-la-gar-bhā-śa-ya-pra-kṛ-tim
kā-la-gar-bhā-śa-ya-pra-kṛ-tiṃ
kali-kāla-sarva-jña
kali-kāla-sarva-jña-śrī-hema-candrācārya-vi-raci-ta
kali-kāla-sarva-jña-śrī-hema-candrācārya-vi-raci-taḥ
kali-yuga
kal-pa-sthāna
kar-ma
kar-man
kārt-snyena
katham
kāvya-mālā
keśa-va-śāstrī
kol-ka-ta
kṛṣṇa-pakṣa
kṛtti-kā
kṛtti-kās
kula-pañji-kā
ku-māra-saṃ-bhava
lab-dhāni
mada-na-phalam
mādha-va
Mādhava-karaaita-reya-brāhma-ṇa
Mādhava-ni-dāna
mādhava-ni-dā-nam
madhu-kośa
madhu-kośākhya-vyā-khya-yā
madhya
madhye
ma-hā-bhū-ta-vi-kā-ra-pra-kṛ-tiṃ
mahā-deva
mahā-mati-śrī-mādhava-kara-pra-ṇī-taṃ
mahā-muni-śrī-mad-vyāsa-pra-ṇī-ta
mahā-muni-śrī-mad-vyāsa-pra-ṇī-taṃ
maha-rṣi-pra-ṇīta-dharma-śāstra-saṃ-grahaḥ
mahā-sacca-ka-sutta
mahau-ṣadhi-pari-cchadāṃ
mahā-vra-ta
mahā-yāna-sūtrālaṅ-kāra
mano-ratha-nandin
matsya-purāṇam
me-dhā-ti-thi
medhā-tithi
mithilā-stha
mithilā-stham
mithilā-sthaṃ
mud-rā-yantr-ā-laye
muktā-pīḍa
mūla-pāṭha
nakṣa-tra
nandi-purāṇoktārogya-śālā-dāna-phala-prāpti-kāmo
nara-siṃha
nara-siṃha-bhāṣya
nārā-ya-ṇa-dāsa
nārā-yaṇa-kaṇṭha
nārā-yaṇa-paṇḍi-ta-kṛtā
nava-pañca-mayor
nidā-na-sthā-na-sya
ni-ghaṇ-ṭu
nir-anta-ra-pa-da-vyā-khyā
nir-ṇaya-sā-gara
nir-ṇaya-sā-gara-yantr-ā-laye
nirūha-vasti
niś-cala-kara
ni-yukta-vaidyāṃ
nya-grodha
nya-grodho
nyāya-śās-tra
nyāya-sū-tra-śaṃ-kar
okaḥ-sātmya
okaḥ-sātmyam
okaḥ-sātmyaṃ
oka-sātmya
oka-sātmyam
oka-sātmyaṃ
oṣṭha-saṃ-puṭa
ousha-da-sala
padma-pra-bha-sūri
padma-sva-sti-kārdha-candrādike
paitā-maha-siddhā-nta
pañca-karma
pañca-karma-bhava-rogāḥ
pañca-karmādhi-kāra
pañca-karma-vi-cāra
pāñca-rātrā-gama
pañca-siddh-āntikā
pari-bhāṣā
pari-likh-ya
pātañ-jala-yoga-śās-tra
pātañ-jala-yoga-śās-tra-vi-varaṇa
pat-añ-jali
pāṭī-gaṇita
pāva-suya
pim-pal-gaon
pipal-gaon
pit-ta-kṛt
pit-ta-śleṣma-ghna
pit-ta-śleṣma-medo-meha-hik-kā-śvā-sa-kā-sāti-sā-ra-cchardi-tṛṣṇā-kṛmi-vi-ṣa-pra-śa-ma-naṃ
prā-cya
prā-cya-hindu-gran-tha-śreṇiḥ
prācya-vidyā-saṃ-śodhana-mandira
pra-dhān-āṅ-gaṃ
pra-dhān-in
pra-ka-shan
pra-kṛ-ti
pra-kṛ-tiṃ
pra-mā-ṇa-vārt-tika
pra-saṅ-khyāne
pra-śas-ta-pāda-bhāṣya
pra-śna-pra-dīpa
pra-śnārṇa-va-plava
praśnārṇava-plava
pra-śna-vaiṣṇava
pra-śna-vai-ṣṇava
prati-padyate
pra-ty-akṣa
pra-yat-na-śai-thilyā-nan-ta-sam-ā-pat-ti-bhyām
pra-yat-na-śai-thilyā-nān-tya-sam-ā-pat-ti-bhyāṃ
pra-yatna-śai-thilya-sya
puṇya-pattana
pūrṇi-mā-nta
rāja-kīya
rajjv-ābhyas-ya
rāma-kṛṣṇa
rasa-ratnā-kara
rasa-vai-śeṣika-sūtra
rogi-svasthī-karaṇānu-ṣṭhāna-mātraṃ
rūkṣa-vasti
sād-guṇya
śākalya-saṃ-hitā
sam-ā-mnāya
sāmañña-pha-la-sutta
sama-ran-gana-su-tra-dhara
samā-raṅga-ṇa-sū-tra-dhāra
sama-ra-siṃ-ha
sama-ra-siṃ-haḥ
saṃ-hitā
sāṃ-sid-dhi-ka
saṃ-śo-dhana
sam-ul-lasi-tam
śāndilyopa-ni-ṣad
śaṅ-kara
śaṅ-kara-bha-ga-vat-pāda
Śaṅ-kara-nārā-yaṇa
saṅ-khyā
sāṅ-kṛt-yā-yana
san-s-krit
sap-tame
śāra-dā-tila-ka-tan-tra
śa-raṅ-ga-deva
śār-dūla-karṇā-va-dāna
śā-rī-ra
śā-rī-ra-sthāna
śārṅga-dhara
śārṅga-dhara-saṃ-hitā
sar-va
sarva-darśana-saṃ-grahaḥ
sar-va-dar-śāna-saṅ-gra-ha
sar-va-dar-śāna-saṅ-gra-haḥ
sarv-arthāvi-veka-khyā-ter
sar-va-tan-tra-sid-dhān-ta
sar-va-tan-tra-sid-dhān-taḥ
sarva-yoga-sam-uc-caya
sar-va-yogeśvareśva-ram
śāstrā-rambha-sam-artha-na
śāstrāram-bha-sam-arthana
ṣaṭ-pañcā-śi-kā
sat-tva
saunda-ra-na-nda
sid-dha
sid-dha-man-tra
sid-dha-man-trā-hvayo
sid-dha-man-tra-pra-kāśa
sid-dha-man-tra-pra-kāśaḥ
sid-dha-man-tra-pra-kāśaś
sid-dh-ān-ta
siddhānta-śiro-maṇ
sid-dha-yoga
sid-dhi-sthāna
śi-va-śar-ma-ṇā
ska-nda-pu-rā-ṇa
sneha-basty-upa-deśāt
sodā-haraṇa-saṃ-s-kṛta-vyā-khyayā
śodha-ka-pusta-kaṃ
śo-dha-na-ci-kitsā
so-ma-val-ka
śrī-mad-devī-bhāga-vata-mahā-purāṇa
srag-dharā-tārā-sto-tra
śrī-hari-kṛṣṇa-ni-bandha-bhava-nam
śrī-hema-candrā-cārya-vi-raci-taḥ
śrī-kaṇtha-datta
śrī-kaṇtha-dattā-bhyāṃ
śrī-kṛṣṇa-dāsa
śrī-mad-amara-siṃha-vi-racitam
śrī-mad-aruṇa-dat-ta-vi-ra-ci-tayā
śrī-mad-bhaṭṭot-pala-kṛta-saṃ-s-kṛta-ṭīkā-sahitam
śrī-mad-dvai-pā-yana-muni-pra-ṇītaṃ
śrī-mad-vāg-bha-ṭa-vi-ra-ci-tam
śrī-maṃ-trī-vi-jaya-siṃha-suta-maṃ-trī-teja-siṃhena
śrī-mat-kalyāṇa-varma-vi-racitā
śrīmat-sāyaṇa-mādhavācārya-pra-ṇītaḥ
śrī-vā-cas-pati-vaidya-vi-racita-yā
śrī-vatsa
śrī-vi-jaya-rakṣi-ta
sthānāṅga-sūtra
sthira-sukha
sthira-sukham
strī-niṣevaṇa
śukla-pakṣa
su-śru-ta-saṃ-hitā
sū-tra
sūtrārthānān-upa-patti-sūca-nāt
sūtra-sthāna
su-varṇa-pra-bhāsot-tama-sū-tra
svalpauṣadha-dāna-mā-tram
śvetāśva-taropa-ni-ṣad
tad-upa-karaṇa-tāmra-kaṭāha-kalasādi-pātra-pari-cchada-nānā-vidha-vyādhi-śānty-ucitauṣadha-gaṇa-yathokta-lakṣaṇa-vaidya-nānā-vidha-pari-cāraka-yutāṃ
tājaka-muktā-valeḥ
tājika-kau-stu-bha
tājika-nīla-kaṇṭhī
tājika-yoga-sudhā-ni-dhi
tāmra-paṭṭādi-li-khi-tāṃ
tan-nir-vāhāya
tapo-dhana
tapo-dhanā
tārā-bhakti-su-dhārṇava
tārtīya-yoga-su-sudhā-ni-dhi
tegi-ccha
te-jaḥ-siṃ-ha
trai-lok-ya
trai-lokya-pra-kāśa
tri-piṭa-ka
tri-var-gaḥ
un-mār-ga-gama-na
upa-de-śa
upa-patt-ti
ut-sneha-na
utta-rā-dhyā-ya-na
uttara-sthāna
uttara-tantra
vāchas-pati
vād-ā-valī
vai-śā-kha
vai-ta-raṇa-vasti
vai-ta-raṇok-ta-guṇa-gaṇa-yu-k-taṃ
vājī-kara-ṇam
vāk-patis
vākya-śeṣa
vākya-śeṣaḥ
varā-ha-mihi-ra
va-ra-na-si
vā-rā-ṇa-sī
var-mam
var-man
varṇa-sam-ā-mnāya
va-siṣṭha-saṃ-hitā
vā-siṣṭha-saṃ-hitā
vasu-bandhu
vasu-bandhu
vāta-ghna-pit-talāl-pa-ka-pha
vātsyā-ya-na
vidya-bhu-sana
vidyā-bhū-ṣaṇa
vi-jaya-siṃ-ha
vi-jñāna-bhikṣu
vi-kal-pa
vi-kamp-i-tum
vi-mā-na-sthāna
vi-racita-yā
vishveshvar-anand
vi-śiṣṭ-āṃśena
viṣṇu-dharmot-tara-purāṇa
viśrāma-gṛha-sahitā
vi-suddhi-magga
vopa-de-vīya-sid-dha-man-tra-pra-kāśe
vyādhi-pratī-kārār-tham
vyāḍī-ya-pa-ri-bhā-ṣā-vṛtti
vyati-krāmati
vy-ava-haranti
yādava-bhaṭṭa
yāda-va-śarma-ṇā
yādava-sūri
yājña-valkya-smṛti
yavanā-cā-rya
yoga-ratnā-kara
yoga-sāra-sam-uc-caya
yoga-sāra-sam-uc-cayaḥ
yoga-sūtra-vi-vara-ṇa
yoga-yājña-valkya
yoga-yājña-valkya-gītāsūpa-ni-ṣatsu
yoga-yājña-valkyaḥ
yogi-yājña-valkya-smṛti
yuk-tiḥ
yuk-tis
}}
\normalfontlatin
\endinput

    
    %\nocite{forb-1856}
    \maketitle
    
    
        \noindent The \href{http://sushrutaproject.org}{Suśruta Project} is
producing
\href{https://saktumiva.org/wiki/wujastyk/susrutasamhita/01-su.su/provisional-edition_sutrasthana}{a
 new Sanskrit text edition} of the \emph{Suśrutasaṃhitā} based on the early 
Nepalese manuscripts.  As we gradually transcribe and edit the manuscripts, we are 
producing this new translation of the classic work.
\MSnocite{Kathmandu, KL 699}
\MSnocite{Kathmandu, NAK 1-1079}
\MSnocite{Kathmandu, NAK 5-333}
        
        \tableofcontents
        
        \newpage
        \section{The Manuscripts used in the Vulgate editions by 
        Yādavaśarma Trivikrama Ācārya}
    
        Yādavaśarma Trivikrama Ācārya produced three successive editions of the \SS\ 
        with the commentary of Ḍalhaṇa, in 1915, 1931 and 1938.  These editions, 
        especially the last, are considered the most scholarly and reliable editions of the 
        work, and have been constantly reprinted up to the present day.
        
        The 1915 edition was based on three manuscripts.  The 1931 edition used 
        another nine.  For his final 1938 edition, Ācārya used a further 
        three.\footnote{\cite[22]{susr-trikamji3}.}
        
        \subsection{The manuscripts of the 1915 edition}
        
        \begin{enumerate}
            \item[1] Calcutta, Royal Asiatic Society.  Covers the sūtra, nidāna, śārīra and 
            kalpa sthānas.  
            
            \item [2] Jaipur, Pandit Gaṅgādharabhaṭṭaśarman, lecturer at the Royal 
            Sanskrit University.  Covers the cikitsāsthānna and the uttaratantra.
            
            \item [3]  Bundi, my great friend the royal physician Paṃ.\ Śrīprasādaśarman  
            Covers the uttaratantra.
        \end{enumerate}
        
        \subsection{The manuscripts of the 1931 edition}

\begin{enumerate}

    \item[1] Vārāṇasī, professor of literature, the great Gaurīnāthapāṭhaka.  With the 
    \emph{Nibandhasaṅgraha}. Covers the nidānasthāna and uttaratantra.
    
    \item [2]  Ahmedabad.  My friend Sva.\ Vā.\ Vaidya Raṇachoḍalāla Motīlālaśarman.  
    With the \emph{Nibandhasaṅgraha}.  Covers the śārīrasthāna.
    
    \item [3] From the library of my great friend Sva.\ Vā.\ Vaidya Murārajīśarman.  
    Extremely old. No commentary.  Covers the śārīrasthāna. 
    
    \item [4]  Puṇe, BORI library.  With the \emph{Nibandhasaṅgraha}. Covers the
śārīrasthāna.\footnote{Not one of the three MSS of the
\emph{śārīrasthāna}described in \cite{shar-vaid}.}
    
    \item [5]  Puṇe, BORI library.  With the \emph{Nibandhasaṅgraha}. Complete.  
    With some damaged folia.
    
    \item [6]  Bombay, Asiatic Society.  Incomplete.\footnote{Possibly \MScite{Mumbai, 
    AS B.I.3} or \MScite{Mumbai, AS B.D.109} \citep[v.\,1, \# 212 and 
    213]{vela-1930}.  But both these have the \emph{Nibandhasaṅgraha}.  The first 
    covers only the śārīrasthāna; the second may be complete, but Velankar calls it 
    only “disorderly.”}
    
    \item [7] 
    
    \item [8]
    
    \item [9]

\end{enumerate}
    
        \subsection{The manuscripts of the 1938 edition}
        
        \begin{enumerate}
    \item [1]
            \item[2]
            
            \item [3]
        \end{enumerate}
    
        \begin{tabular}{c|ccc|ccccccccc|ccc}
            \toprule
              \multicolumn{16}{c}{\emph{Manuscripts}} \\
 \emph{edition}            &\multicolumn{3}{c}{1915}
&                \multicolumn{9}{c}{1935} 
  &              \multicolumn{3}{c}{1938} \\
                 \emph{sthāna}  & 1 & 2 & 3 & 1 &2  &3  &4  &5  &6  &7  &8  &9  &1  
                 &2 &3 \\
            \midrule

             \emph{sū}. &  \newmoon&  &  &
               &  &  &  & \newmoon & ? &  & \newmoon & \newmoon 1--43 &  
             \newmoon & &\newmoon \\
            
             \emph{ni}. &\newmoon  &  &  &
              \newmoon &  &  &  &  \newmoon&  ?&  & \newmoon &  &  
              \newmoon&\newmoon & \newmoon\\
            
             \emph{śā}. &  \newmoon&  &  &
              & \newmoon & \newmoon & \newmoon & \newmoon &  ? &  &  
              \newmoon&  &  
              \newmoon& &\newmoon \\
            
             \emph{ci}. &  & \newmoon &  &
               &  &  &  &\newmoon & ? &  \newmoon&\newmoon  &  &
              \newmoon & &\newmoon 1--9 \\
            
             \emph{ka}.  &\newmoon  &  &  &
               &  &  &  &\newmoon  &  ?&  & \newmoon &  &  
             \newmoon  & & \\

            \emph{utt}.  &  & \newmoon &\newmoon  &
            \newmoon  &  &  &  & \newmoon & ? &  & \newmoon &  &  
            & & \\
            \bottomrule
        \end{tabular}
        
        
        \newpage
    
    \section{Sūtrasthāna, adhyāya 1}
    
\begin{translation}

    \item[1] Now I shall narrate the chapter on the origin of this
knowledge.\footnote{Ḍalhaṇa understood the word "\saneng{veda}{knowledge}" as
specifically "medical knowledge." He said that the word "longevity"
(\emph{āyur}) \ssaneng{āyur}{life, longevity} had been elided.
%    
%    Notes Dec 8:
%    Dominik: N's ādhyāyaṃ corruption of H's nāmādhyāyaṃ: possible evidence that N was created 
%after H
%    Check: Ācārya 1931 footnote on vedotpattim
%    Commentary Ḍalhaṇa (c. 1200 CE) notes āyur dropped from veda in vedotpattim
%    
%    
After this opening statement, later manuscripts and commentaries include
the attribution, "as the venerable Dhanvantari stated."  The absence of this
statement in the early Nepalese MSS is highly noteworthy because it removes
the outer narrative frame of the \SS\
\parencites[148]{wuja-2013}[\S\,3.1.2]{kleb-2021b}.  On the figure of Dhanvatari in 
medical literature, see \cite[IA 358--361]{meul-hist}.} %     <!-- Notes Dec 8:
%    Dom: note the omission of Dhanvantari, which is in the edition.
%    On Dhanvantari, see Meulenbeld HIM, authorities associated with Suśruta. He's
% an authority on
%surgery or toxicology-->
    
    \item[2] Now, as is well-known, Aupadhenava, Vaitaraṇa, Aurabhra, Puṣkalāvata,
Karavīra, Gopurarakṣita, Bhoja, Suśruta and others addressed Lord Divodāsa,
king of Kāśi, the best of the immortals, who was in his ashram surrounded by
an entourage of sages.\footnote{On these persons, see \cite[IA
361--363, 369\,ff.]{meul-hist}. The authority Bhoja does not appear in the list as
published in the vulgate edition \citep[1]{susr-trikamji2}, and was not
included in \cite{meul-hist} amongst “authorities mentioned in the \SS.” 
\citeauthor{meul-hist} gathered textual evidence about Bhoja at \cite[IA
690--691]{meul-hist}. \citet{kleb-2021a} has discussed these authors in the
context of an anonymous commentary on the \SS\ that cites them.}

\nocite{emen-1969}
    
%    Notes Dec 8:
%    Dom: Check these names in Meulenbeld 
%    Bhoja is an early lost authority on medicine. Not the same person as King Bhoja, commentator 
%on the Yogasūtras.
%    Ḍalhaṇa's comm. mentions Bhoja as also included in prabhṛtayaḥ: so the version of the text he 
%was using did not mention Bhoja, but he was aware of him: His provenance makes it possible that 
%he knew the Nepalese version of the SŚ
    
    \item[3]
%O Lord, after seeing people who are assailed by the impingements of various pains 
%caused by 
%physical, mental and accidental diseases, who have the support of friends [but] 
%feeling as if they 
%were alone, and acting frantically, shouting out, we have been distressed. 

“O Lord, distress arose in our minds after witnessing people thrashing about with
cries, assailed by different kinds of \saneng{vedanābhighāta}{pain and injury}, 
feeling helpless in spite 
of having friends, because of diseases arising from the body, the mind and
external sources.


    
%    Notes Dec 8:
%    āgantu - caused by something from outside the body
%    abhighāta - threats, impingements 
%    vedanābhighāta - tatpuruṣa 
%    anātha - among a list of people who shouldn't be treated.
%    Ḍalhaṇa- sanātha: samitra someone with a friend -->

    \let\uncertain\texttt
    
\item[4]    
“To quell the illnesses of those who seek happiness and for our own purpose of
prolonging life, we desire \saneng{āyurveda}{the science of life} that is being
taught.  Welfare\ssaneng{śreyas}{welfare}, both in this world and in
    the next, depends upon it. Therefore, we have come to the Lord in pupillage." %
% reading bhagavan (voc.) and tam (it, ayurveda, masc. acc.)

% we think upasannāḥ smaḥ is probably wrong, but we can't see how to improve it.
% upapannā sma ?

\item[5] The Lord said to them:

“Welcome to you!  My children, all of you are beyond reproach and worthy 
to be taught.
   
    
    \item[6] 
%    "As is well-known in this world, before creating people, Brahmā composed 
%    what is called Āyurveda.\footnote{The relative pronoun \emph{yad}, that has 
%no  
%    correlative \citep[\P 461]{spei-1886}, is omitted.} 
%    It is taught as part of the \emph{Atharvaveda}, in hundreds of 
%    thousands of verses and a thousand chapters and, after observing the short 
%    lifespan and low intelligence of people, made it again in eight parts. 
%    %infer tat as the object of kṛtavān
%    
    
    “As is well known, Ayurveda is the name of what is said to be the subsidiary
part of the Atharvaveda.   Before creating people, Svayambhū composed it in
hundreds of thousands of verses and a thousand chapters and, after observing the
short lifespan and low intelligence of people, he presented it again in eight
parts.\footnote{Svayambhū is another name for Brahmā, the creator.}
    
    \item[7] “Surgery, treatment of body parts above the clavicle, general medicine, 
    knowledge of spirits, care of children, and the disciplines of antidotes, rejuvenation 
    and aphrodisiacs.
    % why do some of the auxiliaries end in tantra? Dom: Some were disciplines that had a separate life outside āyurveda. The others were more particular to vaidyas.
  
        \item[8] “Now,  a collection of the characteristics of each component of 
        Āyurveda.
    
        \item[9] “Among them, [the component] called surgery has the goal of 
        extracting 
        various grasses, wood, stone, dust, iron (?), soil, bone, hair, nails, discharge of 
        pus, malignant wounds and foreign bodies inside the womb, and of determining 
        the application of surgical instruments, knives, caustics and fire by means of 
        sixty definitions.
        %Ḍalhaṇa seems to read duṣṭavraṇāntar, and glosses antar as madhyāt. He then reads garbhaśalya (HIM - foetuses stuck in the womb). Ḍalhaṇa is aware of the reading ṣaṣṭyā vidhānaiḥ (following uddhraṇārtha), and says some explain it as apatarpaṇādyai rakṣāvidhānāntair dvivraṇīyoktair ity arthaḥ. However, mss. 699 and 533 read abhi° not vi°
        
        \item[10] “[The component] named the doctrine of treating body parts above 
        the clavicles has the aim of curing diseases situated above clavicles that is,  
        diseases located in ears, eyes, mouth, nose and so on.
        
        \item[11] “[The component] called general medicine has the goal of curing 
        illnesses established in the whole body and [diseases] such as fever, tumour, 
        swelling, hemorrhagic disorders, insanity, epilepsy, urinary diseases, diarrhoea 
        and the like.
        
        \item[12] “[The component] called knowledge of spirits is for appeasing
demons by pacification rites and making food offerings for those whose
minds have been possessed by gods, their enemies,\footnote{Dānavas.  The
insertion marks (\emph{kākapada}s) below the text at this point appears to
be by the original scribe.} Gandharvas, Yakṣas, demons, deceased
ancestors, Piśācas, Vināyakas, \footnote{The vulgate doesn't have
\emph{vināyaka}s but does add \emph{asura}s, probably under the influence
of Ḍalhaṇa.  Cite Paul Courtright, Ganesha book.} Nāgas and evil spirits
that possess children. % Notes: vināyaka is omitted from the vulgate. In
% Mahābhārata, etc. It refers to
%a class of demons.-->
          
        
        \item[13] “[The component] called care of children is for bearing children and 
        purifying defects in a wet-nurse's milk, and curing diseases that have arisen 
        from bad breast milk and demons.
        
        \item[14] “[The component] called the discipline of toxicology is for
[knowing] the signs of poison from snake and insect bites and for
neutralising various combinations of poisons.\footnote{The scribal
insertion marks (crosses) above the line at this point in MS K appear to
be in a later hand and their referent is lost in the damaged part of the
folio.  Although MSS N and H include \saneng{lūtā}{spiders} and
\saneng{sarīsṛpa}{creepy-crawlies} in the list, it does seem that MS K had
a shorter list, and the vulgate edition adds \saneng{mūṣika}{rodents}.}
        
        \item[15] “[The component] called the discipline of rejuvenation is 
        maintaining 
        youth, bringing about a long life and mental vigour and for curing diseases.
        
        % Got to here 2021-01-13
        
        \item[16] “[The component] called the discipline of aphrodisiacs brings about 
        the 
        increase, purity, accumulation and  production of semen for those whose semen 
        is minimal, bad, depleted, and dry [respectively] and for inducing an erection.
        
        \item[17] “Thus, this Āyurveda is taught with eight components."
        
        "Among these [components], tell us which is for whom."
        
        \item[18] They said, "After you have conveyed the knowledge of surgery, 
        teach 
        us everything."
        
        \item[19] He said, "so be it."
        
        \item[20] They then said, "Having considered the view of all of us, when we 
        are 
        unanimous, Suśruta will question you. We too will learn what is being taught to 
        him."
        
        \item[21] He said, "so be it.
        
        \item[22] “Now, as is well-known, the aim of Āyurveda is eliminating the 
        disease of one who have been assailed by disease and protecting the healthy;  
        āyurveda is [that knowledge] in which they find a long life, or that by which 
        long life is known. Learn its best component (i.e., surgery), which is being 
        taught in accordance with tradition, perception, inference and analogy.
        
        \item[23] "For this component is first, the most important, because it is 
        referred to first; it cures wounds and joins together the most important thing, 
        Yajña's head. For, just as it has been said of old, 'the head that had been cut off 
        by Rudra was joined again by the two Aśvins.'
        
        \item[24] "And also, of the eight disciplines of Āyurveda, [surgery] alone is 
        the best because of the quick action of its \saneng{kriyā}{procedures}, its 
        application of blunt 
        instruments, knives, caustics and fire, and it is common to all disciplines.
        
        \item[25] "Therefore, [surgery] is eternal, meritorious, leads to heaven, 
        brings renown, bestows a long life, and affords a livelihood.
        
        \item[26] "Brahmā said this, 'Prajāpati learned it. From him, the Aśvins. From 
        the Aśvins, 
        Indra. From Indra, I. In this world, I will transmit to those who desire it for the benefit of 
        people.' [And in this regard, there is this verse].\footnote{This is an expansion 
        of the scribe's abbreviation \emph{bha} for \emph{bhavanti cātra ślokāḥ} 
        “There are some verses about this.”\label{bha}}
        
        \item[27]           \begin{quote}
            For, I (i.e., Brahmā) am Dhanvantari, the first god, the remover of old age, 
            pain and 
            death of mortals. Having understood surgery, the best of the great 
            knowledge 
            systems, I arrived on earth again to teach it here.
        \end{quote}    
        
        % draft tr.
        
        \item[28] In this context, as far as this discipline is concerned, a 
        \saneng{puruṣa}{human being} is called an amalgam of the five elements 
        and the embodied soul.  This is where \saneng{kriyā}{procedures} apply. This 
        is the locus. 
        
        Why?
        
        Because of the duality of the world, the world is twofold: the stationary
and the moving. Its \saneng{ātmaka}{nature} is twofold, depending on the
preponderance of Agni and Soma.\footcite[See][]{wuja-2004}  Alternatively,
it can be considered as being fivefold.  The multitude of beings in it are
fourfold: they are termed “sweat-born, stone-born, caul-born and
egg-born”.\footnote{This fourfold classification of beings is paralleled
with closely-related vocabulary in  \emph{Bhelasaṃhitā} 4.4.4	
\parencites[206]{kris-2000}[81]{mook-1921}.}  Where they are concerned, the
human being is the main thing; others are his support.  Therefore, the
\saneng{puruṣa}{human being} is the locus.

\item[29]  Diseases are said to be the conjunction of the person and 
\saneng{duḥkha}{suffering}.
 There are four of them: invasive, bodily, mental and inherent.  The invasive ones 
 are caused by an injury.  The bodily ones are based on food, caused by 
 \saneng{vaiṣamya}{irregularities} in wind, bile, phlegm and blood.\footnote{Note 
 that four humoral substances are assumed here.} 
 
The \saneng{mānasa}{mental} ones, caused by 
\saneng{icchā}{desire} and 
\saneng{dveṣa}{hatred}, 
include:  
\saneng{krodha}{anger}, 
\saneng{āśoka}{grief}, 
\saneng{dainya}{misery}, 
\saneng{harṣa}{overexcitement}, 
 \saneng{kāma}{lust}, 
 \saneng{viṣāda}{depression},
 \saneng{īrṣyā}{envy},
 \saneng{asūyā}{jealousy},
 \saneng{mātsarya}{malice}, 
 and
 \saneng{lobha}{greed}.
 
 The \saneng{svābhāvika}{inherent} ones are hunger, thirst, old age, death, 
 sleep and  those of the \saneng{prakṛti}{temperament}.
 
 These too are \saneng{adhiṣṭhāna}{located} in the mind and body.
 
\saneng{lekhana}{Scarification},
\saneng{bṛṃhaṇa}{nourishment},
\saneng{saṃśodhana}{purification},
 \saneng{saṃśamana}{pacification},
 \saneng{āhāra}{diet} and
 \saneng{ācāra}{regimen}, 
 properly employed, bring about their cure.
    
 
 \item [30] Furthermore, food is the  \saneng{mūla}{root} of living beings as well
as of \saneng{bala}{strength}, \saneng{varṇa}{complexion} and 
\saneng{ojas}{vital
    energy}. It \saneng{āyatta}{depends on} the six \saneng{rasa}{flavours}.
Flavours, furthermore, have substances as their \saneng{āśrayin}{substrate}.  And
substances are \saneng{oṣadhī-}{remedies}.\footnote{Pāṇini 6.3.132 provides that
the final vowel of the noun \emph{oṣadhi} may be lengthened
(\emph{$\rightarrow$oṣadhī}) under certain conditions.  These conditions require
that the word be used in a Vedic mantra and not in the nominative.  Neither
condition is met in this passage, yet the author uses the form \emph{oṣadhī}. 
This form is in fact not uncommon in medical literature as well as in epics,
purāṇas, smṛtis, and other parts of Sanskrit literature.} There are 
two types:
\saneng{sthāvara}{stationary} and \saneng{jaṅgama}{moving}.



\item [31]  Of these, there are four types of stationary ones:
\saneng{vanaspati}{fruit trees}, \saneng{vṛkṣa}{flowering trees},
\saneng{oṣadhi}{herbs} and \saneng{vīrudh}{shrubs}.\footnote{Ca.sū.1.71--72 
also
describes these four types of medicinal plant in similar terms but with slightly
differing names: \emph{oṣadhi} is a plant that ends after fruiting, \emph{vīrudh}
is a plant that branches out, \emph{vanaspati} is a tree with fruit, and
\emph{vānaspatya} is a tree with fruit and flowers.}
Amongst these, the “fruit trees” have fruit but no flowers.\footnote{The MSS agree 
in reading \emph{phalavantyaḥ} “having flowers” which is grammatically 
non-standard. This form is also found in the  \emph{Viṣṇudharmottarapurāṇa} 
(1.92.27, \cite[1.92.27][56r]{sarm-1912}).}  The “flowering trees” 
have flowers and fruit.  The “herbs” die when the fruit is ripe. “Shrubs” put out 
shoots.

% \citep{ober-2003} didn't have anything on phalavantyo that I could find quickly.


   \item[32]  As is well known, moving remedies are also of four types: those
\saneng{jarāyuja}{born in in a caul}, those \saneng{aṇḍaja}{born from eggs},
those \saneng{svedaja}{born of sweat}, and \saneng{udbhid}{shoots}. Amongst
these, those born in a caul include \saneng{paśu}{animals}, humans, and
\saneng{vyāla}{wild animals}.  Birds, \saneng{sarīsṛpa}{creepy-crawlies} and
snakes are “born of eggs.” \saneng{kṛmi}{Worms}, \saneng{kunta}{small insects}
and \saneng{pipīlika}{ants} and others are born of sweat.\footnote{The word
\emph{kunta}, though marked as “lexical” in most dictionaries, is in fact found
in literature, commonly as a compound with \emph{pipīlika}; the compound
sometimes seems to be understood a type of ant (\emph{tatpuruṣa} compound)
rather than as a pair of insects (\emph{dvandva} compound).}  Shoots include
\saneng{indragopa}{red velvet mites} and \saneng{maṇḍūka}{frogs}.\footnote{On
\emph{indragopa}, see \cite{lien-1978}.}|

\item[33] In this context, among the stationary remedies, 
\saneng{tvak}{skin}, 
\saneng{patra}{leaves}, 
\saneng{puṣpa}{flowers}, 
\saneng{phala}{fruits},
\saneng{mūla}{roots},
\saneng{kanda}{bulbs},
\saneng{kṣīra}{sap},
\saneng{niryāsa}{resin},
\saneng{sāra}{essence},
\saneng{sneha}{oil}, 
and
\saneng{svarasa}{juice extract}\footnote{On \saneng{svarasa}{juice extract} see
CS 1.1.73, 1.4.7; \VN{4.10.12}{}.} 
are useful; among the moving remedies 
\saneng{carman}{pelt}, hair, nails, and 
\saneng{rudhira}{blood} and so forth. 

 \item[34] And \saneng{pārthiva}{earthen products} include gold and 
 silver.\footnote{The flow of concepts in the treatise seems to be interrupted here.}
 
\item[35] The \saneng{kālakṛta}{items created by time} are \saneng{samplava}{clusters} 
as far as wind and \saneng{nivāta}{no wind}, heat and shade, darkness and light
and the cold, hot and \saneng{varṣā}{rainy seasons} are concerned. 
The divisions of time are the
\saneng{nimeṣa}{blink of the eye}, a
\saneng{kāṣṭhā}{trice}, 
\saneng{kalā}{minutes}, 
\saneng{muhūrta}{three-quarters of an hour}, a
\saneng{ahorātra}{day and night}, a
\saneng{pakṣa}{fortnight}, a
\saneng{māsa}{month}, a
\saneng{ṛtu}{season}, a
\saneng{ayana}{half-year}, a
\saneng{saṃvatsara}{year},
and
\saneng{yuga}{yuga}.\footnote{These units are presented at 
\VN{1.6.5}{} and discussed by \citet[\S\,59]{haya-2017}.}


\item[36]  These naturally cause 
\saneng{sañcaya}{accumulation}, 
\saneng{prakopa}{irritation}, 
\saneng{upaśama}{pacification} 
and 
\saneng{pratīkāra}{alleviation} of the \saneng{doṣa}{humours}. And they have
    \saneng{prayojanavat}{practical purposes}.
    
[There are verses on this:]\footnote{See footnote \ref{bha}.}

\item[38] 
This fourfold category is taught by physicians as a cause for the agitation and 
quelling of bodily diseases.%
%
\footnote{On the topic of the “group of four,” the commentator Ḍalhaṇa
considers them to be “food, behaviour, earthen products and items created by
time.”  He refers to the author of the lost commentary entitled \emph{Pañjikā},
and to Jejjaṭa \citep[IA, 372--3, 192]{meul-hist}.  In his view, these early
commentators  do not agree that the \saneng{caturvarga}{fourfold grouping} refers
to the quartet of \saneng{sthāvara}{stationary}, \saneng{jaṅgama}{moving},
\saneng{pārthiva}{earthen products} and \saneng{kālakṛta}{items created by 
time} \citep[9a]{susr-trikamji2004}.}

\item[39] There are two kinds of invasive diseases. Some certainly 
\saneng{nipat-}{affect} the mind, 
others the body.\footnote{The text uses an archaic expletive here, \emph{ha}.} 
Their \saneng{kriyā}{treatment} is of two kinds too. 

\item[40] For those that affect the body there is \saneng{śārīravad}{physical} 
therapy, whereas for those that affect the mind there is the 
\saneng{varga}{collection} of desirable sensory experiences like sound that 
bring \saneng{sukha }{comfort}.


Those that affect the body have therapy that is  
\saneng{śārīravat}{physical}, whereas for those of the mind it is 
 

    
\end{translation}    
    
    \newpage
% % % % % % % % % % % % % % % % % % % % % % % SS 1.28
\section{Sūtrasthāna, adhyāya 28}
% Adhyāya 28  
    
\begin{translation}    
    \item [1] Thus, living creatures and their strength,
\saneng{varṇa}{complexion} and \saneng{ojas}{energy} are rooted in food.  That
(food) depends on the six \saneng{rasa}{flavours}. Thus, the flavours depend
on \saneng{dravya}{substance}, and substances depend on medicinal herbs. 
There are two kinds of them (herbs):  stationary and mobile.\footnote{\VN{1.1.28}{} \cite[I, 
21]{shar-susr}, \cite[7]{susr-trikamji2004}.}

\end{translation}

%\section{Nidānasthāna}
%Cf.\ \cite{adri-engl}.
\nocite{*} % include everything from the bib file in the bibliography

\newpage % now the end matter
    \printshorthands
    \printbibliography[notkeyword=edition,
        notkeyword=shorthand]
    \printindex[lexical]
    \indexprologue{\emph{\footnotesize The numbers after the colon refer to pages
        in this document.}} 
    \printindex[manuscripts]
\end{document}
