% !TeX root = incremental_SS_Translation.tex

\section{Uttaratantra, adhyāya 38}

\subsection*{Introductory remarks}

\paragraph*{Summary of the Content}
The chapter talks about various diseases of the female reproductive system and, in doing so, combines both aspects that go into a representation of diseases in āyurvedic literature: signs, symptoms and pathogenesis (\textit{nidāna}), on the one hand, and medical treatment (\textit{cikitsā}), on the other. In chapters of the \textit{Uttaratantra}, these two aspects are sometime dealt with in two different chapters X-\textit{vijñānīya} and X-\textit{pratiṣedha}. There are, however, many examples where this distinction is not made. 


\paragraph*{Placement of the Chapter}
While in \cite{vulgate} the current chapter is found at the end of the section on paediatrics (\textit{Kumāratantra}, or \textit{Kumārabhṛtya} as this section is styled in K), in the Nepalese version, this is chapter 6.58, and it is chapter 23 of an entirely different section, namely, the \textit{Kāyācikitsā}.

Several things are noteworthy in this regard:
\begin{itemize}
	\item In the placement of the vulgate, this chapter follows upon 6.37 \textit{Grahotpatti} (6.35 in the Nepalese version), a chapter that talks about the origination of nine \se{graha}{planetary deities?} that are responsible for all children's diseases described in previous chapters of the \textit{Kumāratantra}. In this way, the current chapter retains the general focus on the \se{kaumārabhṛtya}{child bearing}, but, at the same time, marks a change to a distinct, less mystical approach to the topic at hand (that could originate in a cultural milieu different from that of the preceding 11 chapters). Ḍalhaṇa  \parencite[668b]{vulgate} explains how the chapter fits its context in the following way: 
		\begin{quote}
		It is appropriate that for the sake of treating the \se{yonivyāpat}{disorders of the female reproductive system}, the chapter called \se{yonivyāpatpratiṣedha}{Countermeasures Against Disorders of the Female Reproductive System} (SS.6.38) is taught immediately after the chapter called \se{grahotpatti}{Origination of Planetary Deities} (SS.6.37). It is because (1) there is an explicit mention of the word “\textit{yoni}” in the statement  “born in the \se{yoni}{womb} of animal and human” (in SS.6.37.13bc) and because (2) the \se{yonivyāpat}{disorders of the female reproductive system} are the causes for the inborn disorders of children.%
			\footnote{%
			Ḍalhaṇa on SS.6.38.1: \textit{grahotpattyadhyāyanantaraṃ ‘tityagyoniṃ mānuṣaṃ ca’ iti vacanena yoner nāmasaṃkīrtanāt kuṃārajanmavikārakāraṇatvāc ca, yonivyāpaccikitsitārthaṃ yonivyāpatpratiṣedhādhyāyārambho yujyate [\ldots]/}%
			}
		\end{quote}

	\item In the placement of the Nepalese version, \textit{Yonivyāpatpratiṣedha} is preceded by 6.56 \textit{Mūtrāghātapratiṣedha} (6.58 in \cite{vulgate}) and 6.57 \textit{Mūtrakṛcchrapratiṣedha} (6.59 in \cite{vulgate}), two chapters dealing with the diseases of the urinary tract. The current chapter carries on with the topic of diseases that affect genitalia. In its Nepalese version, the chapter opens with two verses that explain the reasons for treating the particular set of diseases. These lack any reference to the \se{kumārajanmavikāra}{inborn disorders of children} mentioned by Ḍalhaṇa, and instead highlights the importance of curing female diseases for the satisfaction of male partner. 
	\item SS.1.3 in both \cite{vulgate} and the Nepalese version lists the chapter at the place, where it is found in the vulgate (Cf.\ S.1.3.37ab: \textit{naigameṣacikitsā ca grahotpattiḥ sayonijāḥ}).
	\item Parallel chapters in the \textit{Aṣṭāṅgasaṃgraha} and the \textit{Aṣṭāṅgahṛdayasaṃhitā} form a part of the \textit{Śalyatantra} (not ) section of each text.
\end{itemize} 
%In the Nepalese version, this is chapter 6.58 (\textit{Kāyācikitsā} 23) that follows upon 6.56 \textit{Mūtrāghātapratiṣedha} (6.58 in the vulgate) and 6.57 \textit{Mūtrakṛcchrapratiṣedha} (6.59 in the vulgate). In the vulgate, on the other hand, this chapter concludes another section, the \textit{Kumāratantra} (\textit{Kumārabhṛtya} in K), and follows upon 6.36 \textit{Naigameṣapratiṣedha} (6.34 in the Nepalese version) and 6.37 \textit{Grahotpatti} (6.35 in the Nepalese version).

\paragraph*{Parallels}
The current chapter is parallel in its content to \textit{Aṣṭāṅgasaṃgraha} 6.38 and 6.39 as well as \textit{Aṣṭāṅgahṛdayasaṃhitā} 6.33 and 6.34 (\textit{Guhyarogavijñāna} and \textit{Guhyarogapratiṣedha} respectively).%, which form a part of the \textit{Śalyatantra} section of each text.

A close literary parallel to the first part of the chapter is found in \textit{Mādhavanidāna} (MN) 62, or at least its version printed in \citet{}. The readings of the MN as it stands now usually side with the vulgate version rather than with the Nepalese. In addition to the basic text, there are several valuable pointers made in the \textit{Madhukośa}, an early commentary on the MN. This part of the text is authored by Śrīkaṇṭhadatta, who was most like a direct student of Vijarakṣita. The latter wrote the first part of the \textit{Madhukośa}, up to chapter 32, and, what is more, can be dated to the second half of the 11th -- first half of the 12th centuries \citep[22--26]{meul-1974}.

Another most interesting parallel is found in Carakasaṃhitā 6(Ci).30.

\subsection*{Translation}
\begin{translation}

\item [1] And now I shall explain the countermeasures against \se{yonivyāpat}{disorders of the female reproductive system}.%
	\footnote{%
	On this broad understanding of the term \emph{yoni}, see \cite[pp.\ 
	572--5]{das-orig}}


\item [2] Since for good men, a woman is the most pleasurable thing, therefore a physician should diligently attend to the diseases located in the \se{yoni}{female reproductive system}, because he is entirely devoted to it (that is, to curing these diseases) for the sake of (people's) happiness.%
	% DW: because he really is dependent ... for the sake of... 
	% yasmāt pramāda .... , ataḥ (= tasmāt) vaidyaḥ... samuprakrameta, yasmāt ... tadadhīna... 
	% Jason suggests: to parse ``yasmāt sukhārtham [asti], [tasmāt] tadadhīnaḥ''
	\footnote{%
	As our translation indicates, the sentence construction does not allow an 
	unambiguous identification of who or what is the referent of the pronoun 
	\textit{tad} in the compound form \emph{tadadhīna} ‘devoted to it.’ Our 
	current understanding is that \emph{tad} refers to the ‘most pleasurable thing’ 
	mentioned in pāda a. It could, however, also refer to ‘them,’ that is, the ‘good 
	men.’%
	}

\item [3] A corrupted \se{yoni}{female reproductive system} cannot consume \se{bīja}{semen}, and therefore, the woman cannot take a fetus (that is, become pregnant). She gets severe 
\se{arśas}{prolapses}, 
\se{gulma}{abdominal lump} and similarly many other 
\se{roga}{diseases}.
	% think about ``praduṣṭa-'' (!!!):
	%% Martha suggested ``ruined'' 
	%%% COMM (ak): it fits really well here, but what to do about praduṣṭa- doṣas?
	%% spoiled, corrupted, “vitiated”?
\end{translation}

\paragraph*{Philological Notes}
The first two verses (2 and 3) in the Nepalese version are written in a classical variety of the \emph{upajāti} metre: 
	\begin{center}
	 \underline{$\cup$} $\_ \cup\ \_\ \_\ \cup\ \cup\ \_\ \cup\ \_\ \_$.
	\end{center} 
In content, they are only approximately parallel to three hemistichs in \textit{anuṣṭubh} metre found in \cite{vulgate}. The latter verses lack the apologetic explanation concerning the reasons for this chapter being taught.

\begin{translation}

\item [4] \se{doṣa}{Humours}, \se{vāta}{wind}, etc., corrupted due to 
\se{mithyopacāra}{faulty medical treatment},%
	\footnote{%
	In our translation of the compound \textit{mithyopacāra}, we decided for the technical meaning of the term \textit{upacāra}, that is, `medical application' or `treatment'. The combination \textit{mithyā + upa-$\sqrt{car}$} is attested several times in medical literature. At least once, at CS Vi.3.38, it is given an explicit commentarial gloss (by Cakrapāṇidatta): ``\textit{mithopacaritān iti asamyak cikitsitān}''. In the SS \parencite{vulgate}, it is used once in Ut.18.30, where it refers specifically to the wrong application of \se{tarpaṇa}{?} and \se{putapāka}{?}, both of which are mentioned in the previous verse. Another use of the compound in a seemingly conforming meaning is found in a citation from Bhoja's text quoted by Gayadāsa at SS Ni.5.17: ``\textit{śvitraṃ tu dvividhaṃ proktaṃ doṣajaṃ vraṇajaṃ tathā/ tatra mithyopacārād dhi vraṇasya vraṇajaṃ smṛtam // \ldots}''. In contrast to this, the parallel verse in \cite{vulgate} = CS Ci.30.7 = MN 62.1 reads \textit{mithyācāra} ‘wrong conduct’. All commentators (Cakrapāṇidatta on the CS, Śrīkaṇṭhadatta on the MN, and Ḍalhaṇa on the SS) explain that the wrong conduct stands here specifically for unwholesome diet. The parallel in AH Ut.33.28 = AS Ut.38.34 plainly reads \textit{duṣtabhojana} `corrupted food' instead.  
	}
sexual activity, fate, and also \se{doṣa}{defects} of \se{ārtava}{menstrual blood} and \se{bīja}{semen}, produce various diseases in the \se{yoni}{female reproductive organ}. 
These 20 diseases are taught here distinctly and one by one along with their \se{bheṣaja}{treatment}, \se{hetu}{causes} and \se{cihna}{signs}. 
	%% mithyopacāra-/ mithyācāra -->technically, this usually refers to “wrong treatment”/ “faulty 
	%% mithyācāra- is glossed in Caraka + Mādhavanidāna --- asamyagāhārācara
	%% Aṣṭāṅga... use ``duṣṭabhojana-''
	%%% Mithyopacāra in Caraka Vi.3.38 is glossed with “asamyak cikitsita-” 
	% Cf.:
	%% \emph{Mādhavanidāna} 62.1-2ab: \textit{viṃśatir vyāpado yonau nirdiṣṭā rogasaṃgrahe/ mithyācāreṇa tāḥ strīṇāṃ praduṣṭenārtavena ca// jāyante bījadoṣāc ca daivāc ca śṛṇu tāḥ pṛthak}
	% Cf.:
	%% \textit{Aṣṭāṅgahṛdaya} 6.33.28ab-29ab = \textit{Aṣṭāṅgasaṃgraha} 6.38.34:
	%% \textit{{viṃśatir vyāpado yoner jāyante duṣṭabhojanāt/ viṣamasthāṅgaśayanabhṛśamaithunasevanaiḥ/ duṣṭārtavād apadravyair bījadoṣeṇa daivataḥ //}
	% NOTE: Several other interpretations are possible: 
	%% - suratakriyāyāḥ can be Genitive
	%% - mithyopacāra- can wrong treatment specifically
	%% - chose ``amorous'' rather than ``sexual'' to capture the feeling of \emph{surata} better.
	%% the intro verses are identical with Carakasaṃhitā 6.30.7--8. There, in Cakra comments on ``rogasaṃgraha'' saying that it refers to 1.19.3 !!!
\end{translation}

\paragraph*{Philological Notes} 
The Nepalese version of the SS continues here with 3 hemistichs in classical \textit{upajāti} metre (see the syllabic pattern above). On the other hand, \cite{vulgate} contains two complete verses (4 hemistichs) in the \textit{anuṣṭubh}. Three final hemistichs are found verbatim in CS Ci.30.7cd--8. It is very likely that the these verses were borrowed from the CS into SS (and not the other way around), because CS Ci.30.7cd = SS Ut.38.5ab says that the 20 kinds of diseases were already taught in the \se{rogasaṃgraha}{Collection of Diseases}. In the context of the SS, this reference does not make any sense and is left uncommented by Ḍalhaṇa. In case of the CS, however, Cakrapāṇidatta explains that this reference points back to CS Sū.19, a chapter that does, in fact, lists all the diseases dealt with in later sections of the text. 20 diseases of \se{yoni}{female reproductive system} as mentioned in Sū.19.3.

The above three hemistichs in \textit{anuṣṭubh} are also repeated in MN 62.1--2ab. Given that all following verses stem from the SS, it is likely that MN 62.1--2ab too was incorporated into the text from the SS (and not its original location in the CS).   

\begin{translation}
\item [5.1] Because of \se{vāta}{wind}, \se{yoni}{female reproductive organ} becomes:
	%All feminines refer to {yoni}, and hence are various kinds of adjectival forms (ktānta-s, bahuvrīhi-s, or other kinds of adjectives). Perhaps, instead of “occur” one could say smth like “\se{yoni}{} becomes...”, There is actually “bhavet” in 6.

	\begin{enumerate}
		\item \se{udāvartā}{?},
		\item called \se{vandhyā}{Infertile}, and
		\item \se{plutā}{Sprung},
		\item \se{pariplutā}{Flooded}, and
		\item \se{vātalā}{Windy}.
	\end{enumerate}

\item [5.2] And because of \se{pitta}{choler}, occur:
	\begin{enumerate}
		\item \se{raktakṣayā}{With bloodloss},
			% This is, according to the analysis given in a later verse, a rare case of a ``genetive bahuvrīhi'' (A reads it as a grammatically more correct Locative bahuvrīhi!).
		\item \se{vāminī}{Vomiting}, and
		\item \se{sraṃsanī}{Causing a Fall},
			% this one is most likely formed with some kind od LyuṬ (or smth like that) + ṄīP. See similar formations like jananī, hananī, śāmanī, śamanī
			% the Aborting One (?)
		\item \se{putraghnī}{Child-murderess}, and also
		\item \se{pittalā}{Bilious / Choleric}.
	\end{enumerate}


\item [6.1] And because of \se{kapha}{phlegm} occur:
	\begin{enumerate}
		\item \se{atyānandā}{Extremely Excited},
		\item \se{karṇinī}{Protuberant}, and
		% @@NOTE a conjunctive error (errores coniunctivi)
		\item[3.\ \& 4.] two \se{caraṇī}{}, and
		\item[5.] other \se{śleṣmalā}{Phlegmatic}.
	\end{enumerate}

\item [6.2] And similarly there are other (kinds of morbid female reprodctive system) involving all \emph{doṣa}s:
	\begin{enumerate}
		\item \se{śaṇḍī}{Impotent},
		\item \se{aṇḍīnī}{With testicles},
		\item two \se{mahatī}{Huge},
		\item \se{sūcīvaktrā}{With a needle-like opening},
		\item \se{sarvātmikā}{}.
	\end{enumerate}
\end{translation}

\paragraph*{Philological Notes}
Verses 5 and 6 consist of four hemistichs written in a kind of \textit{triṣṭubh} metre --- that is, of eight unequal \textit{pāda}s containing 11 syllables each --- and correspond to six hemistichs in \textit{anuṣṭubh} in \cite{vulgate} (Ut.38.6cd--9cd).
By the standards of classical Sanskrit prosody, the metre in all four hemistichs is irregular. However, considering the wide range of metrical variations of the \textit{triṣṭubh} permissible in Epic Sanskrit, the concerned verses can be considered to fall well within metrical norm. Based on the metrical analysis of a large sample of \textit{triṣṭubh} passages in the \textit{Mahābhārata}, \textcite[108]{fitz-2009} postulated the following general metrical structure: 

\begin{table}[h!]
\centering
\caption{\small Summary of table 3 in \cite{fitz-2009}.}
\resizebox{\linewidth}{!}{%
\begin{tabular}{l|c|c|c|c|c}
syllable nr. & 1 & 2,3,4    & 5,6,7       & 8,9,10 & 11  \\ 
\cline{2-6}
            & x & ra (~$\_\ \cup\ \_$~), ma (~$\_\ \_\ \_$~) & bha (~$\_\ \cup\ \cup$~), ra (~$\_\ \cup\ \_$~), sa (~$\cup\ \cup\ \_$~) & ra (~$\_\ \cup\ \_$~)  & x  
\end{tabular}
}
\end{table}

Our verses scan:

\begin{table}[h!]
\centering
\caption{\small Metrical structure of vss.\ Ut.38.5--6 in the Nepalese version of the \SS.}
\begin{tabular}{ c || c | c }
 5 & $\cup\ \_\ \_\ \_\ , \_\ \cup\ \_\ \_\ \cup\ \_\ \cup$ & $\cup\ \_\ \cup\ \_\ , \_\ \cup\  \_\ \_\ \cup\ \_\ \_$ \\ 
 6 & $\_\ \_\ \cup\ \_\ , \_\ \cup\ \_\ \_\ \cup\ \_\ \cup$ & $\_\ \_\ \cup\ \_\ , \_\ \cup\ \_\ \_\ \cup\ \_\ \_$     
\end{tabular}
\end{table}

Following Fitzgerald's hypothesis (F-2009 p 99) formulated explicitly with regard to the \textit{Mahābhārata} that ``the more variable a \textit{triṣṭubh} passage of the Mbh is, the older it is likely to be'', one may speculate that the current passage in the Nepalese version may go back to an ancient textual layer that, at the time when the hyparchetype of the Nepalese version was produced, was not yet fully ``Sanskritized'' and harmonized with the surrounding passages. 
Alternatively and, perhaps, less likely, vss.\ 5--6 of the Nepalese version could have been composed as an attempt to harmonize the text of the SS --- that is, to recast the list of diseases originally written in \textit{anuṣṭubh} into \textit{triṣṭubh}. %428813269the author of which was familiar with the Epic forms.

Note that so far we have not come across any other examples of non-Classical metres used either in \cite{vulgate} or in the Nepalese version. 

\begin{translation}
\item [7] The \se{udāvartā}{Retaining} releases foamy \se{rajas}{menstrual blood} with pain. 
	% Cf.:
	%% AS.Utt.38.39	vegodāvartanād yoniṃ prapīḍayati mārutaḥ | 
	%% sā phenilaṃ rajaḥ kṛcchrād udāvṛttaṃ vimuñcati ||
	%% AS.Utt.38.40	iyaṃ vyāpad udāvṛttā
One should diagnose the \se{vandhyā}{Infertile} by the absence of \se{ārtava}{menstrual blood}, and the \se{utplutā}{?} by chronic pain.
	% DW: The \se{vandhyā}{} is symptomized by the absence of \se{ārtava}{menstrual blood}, and the \se{utplutā} by chornic pain.
	% in the above list we had `plutā' instead. 
	% A has `viplutā'
	% CS 1.19.4.(9)| has pariplutā + upaplutā
In the case of \se{pariplutā}{Flooded}, there is an extreme appetite for sex.
	% Cf.:
	%% \emph{Madhukośa} ad MaNi 62.3: 
	%%% \emph{‘grāmyadharmeṇa rug bhṛśam’ ity atra ‘grāmyadharme rucir bhṛśam’ iti pāṭgāntaram, tatra rucir abhilāṣaḥ, grāmyadharme maithune/}
	%% From VP: vyavāye maithune amaraḥ.

\item [8] The \se{vātalā}{Windy} is hard, stiff, afflicted by stabbing and pricking pain.
And in four former types too, there are \se{vedanā}{painful sensations} associated with the \se{anila}{wind}.

\item [9] The \se{lohitakṣayā}{Bloodloss} is the one that has blood that diminishes with a burning sensation.
	\q{NOTE say smth. about yasyāḥ}
	\q{add a FOOTNOTE explaining that Skt gives a kind of grammatical analysis}
And the \se{vāminī}{Vomiting}, flooded with \se{rajas}{menstrual blood}, ejects the \se{bīja}{semen} in the flow. 
	\q{ADD smth about \textit{srutau}}

\item [10] The \se{prasraṃsanī}{Falling} protrudes, there are contractions, and delivery is hard.
	% conjunctive error
	%%@@ NOTE! This is a very interesting case! The anusvāra after prasraṃsanī_ṃ_ in both K and H does not look like an anusvāra. In fact, in K, there is clearly a syllable missing after °sanī [saṃ]. It seems possible that the mark above the line is an INSERTION SIGN in the template. 
	%% Both K and H have it, which suggest that they share a common template !!! in which the original INSERTIONI MARK was already MISINTERPRETED !!!!
	%% H is again either improvising, or using another MS. !
	%%@@ the anusvāra after duḥprajāyinīṃ looks original though. So, perhaps, one should try making sense of this reading.
The \se{putraghnī}{Child-Murdress} destroys an embryo that repeatedly lodges in the womb with flows of \se{rakta}{blood}.
	% WHO: repeated abortions

\item [11] The \se{pittalā}{Choleric} is intensely afflicted by \se{dāha}{burning sensation} and \se{pāka}{inflammation}.
And one should add signs of \se{pitta}{choler} to four former kinds too. 

\item [12] % At the moment I kept “atyānandād” and supplied “atyānandā”. It is however equally possible to emend to “atyānandā” SEE ALSO 13
Because of strong excitement and inability to obtain satisfaction, (a woman whose yoni is \se{atyānandā}{Extremely Excited}) engages in \se{grāmyadharma}{sex}. 
Now, in the case of \se{karṇinī}{?}, from \se{śleṣman}{phlegm} and \se{āsṛk}{menstrual blood} a \se{karṇikā}{lump} develops in the \se{yoni}{?}. 

\item[13]
During \se{maithuna}{sexual intercourse}, the \se{caraṇī}{?} is first, she dominates the man.
% caraṇī pūrvā -- first kind of caraṇī?
% caraṇī or acaraṇī ?!?
% perhaps, “reaches climax” instead of “is”. 
%% DISCUSS: what to do with the reading of the VULGATE ?! (caraṇī is ACTUALLY Ḍalhaṇa's reading)
Because of frequent excessive intercourse, the \se{bīja}{semen} does not lodge afterwards.
% perhaps, smth like Aticaraṇī is called aticaraṇāt, and because of that 

\item [14]
\se{śleṣmalā}{Phlegmatic} \se{yoni}{female reproductive organ} is slimy, it is itchy and very cold. And in all four former types too, one should add signs of \se{kapha}{phlegm}.

\item [15]
The breasts of \se{ṣaṇḍī}{?} lack \se{ārtava}{female reproductive fluid}, and during sex, it is rough to the touch.
	% More likely: ṣaṇḍī lacks breasts and ārtava.
And the \se{yoni}{?} of a young woman, taken by a man with a large body (that is, penis), may become \se{aṇḍānī}{?}

\item [16]
\se{mahāyoniḥ}{} is expanded and \se{sūcīvaktrā}{} is extremely closed.
	% MAYBE: rolled out and rolled in/ up ?
The \se{sarvadoṣasamanvitā}{Connected to all humours} is diagnosed in women/ yonis in which signs of all \se{doṣa}{humours} occur.

\item [17]
And in four former types too, one observes signs of all humours.
	% syntax is unlcear
These five (yonis/ vyāpats) are incurable. Diseases born from all humours.
	% what's going on with imāḥ <> vyādhayaḥ ?!?

\item [18]
But in case of curable types, medical protocol of oleation etc.\ in accordance with affected \se{doṣa}{humour} is recommended.
And one should especially administer \se{uttarabasti}{vaginal douching} according to instructions.

\item [19]
One should treat a yoni that is rough, cold, stiff and also insensible with \se{kumbhīsveda}{} filled with ānūpa-animals-meat and water.
	% Not sure about alpasparśa
	% see Ḍalhaṇa on kumbhīsveda
		% WHO says: sudation with buried pot filled with decoctions; making the person lay down or sit over a bed placed above a buried pot filled with hot decoction or liquid
		% CA: C.Su.14/39-40

\item [20]
A physician should place excellent clothes (?!?) along with sweet medicinal substances into the yonis. And they should apply \se{balātaila}{} sufficiently and gently.


\end{translation}
