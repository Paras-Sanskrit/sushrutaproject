% !TeX root = incremental_SS_Translation.tex

\section[Uttaratantra, adhyāya 38]{Uttaratantra, adhyāya 38%
	\footnote{%
	In the Nepalese version, this is chapter 6.58 (\textit{Kāyācikitsā} 23) that follows upon 6.56 \textit{Mūtrāghātapratiṣedha} (6.58 in the vulgate) and 6.57 \textit{Mūtrakṛcchrapratiṣedha} (6.59 in the vulgate). In the vulgate, on the other hand, this chapter concludes another section, the \textit{Kumāratantra} (\textit{Kumārabhṛtya} in K), and follows upon 6.36 \textit{Naigameṣapratiṣedha} (6.34 in the Nepalese version) and 6.37 \textit{Grahotpatti} (6.35 in the Nepalese version).

	Parallel passages are found in \textit{Aṣṭāṅgasaṃgraha} 6.38 and 6.39 as well as \textit{Aṣṭāṅgahṛdayasaṃhitā} 6.33 and 6.34 (\textit{Guhyarogavijñāna} and \textit{Guhyarogapratiṣedha} respectively), which form a part of the \textit{Śalyatantra} section (not the \textit{Kaumāratantra}) of each text.

	Further close parallel to the first part of the chapter is found in \textit{Mādhavanidāna} 62.%
	}%
}

\begin{translation}

\item [1] And now I shall explain the countermeasures against \se{yonivyāpat}{disorders of the female reproductive system}.%
	\footnote{%
	On this broad understanding of the term \emph{yoni}, see \cite[pp.\ 
	572--5]{das-orig}}

\item [2] Since for good men, a woman is the most pleasurable thing, therefore a physician should diligently attend to the diseases located in the \se{yoni}{female reproductive system}, because he is entirely engaged in (i.e., curing these diseases) for the sake of (people's?) happiness.%
	% DW: because he really is dependent ... for the sake of... 
	% yasmāt pramāda .... , ataḥ (= tasmāt) vaidyaḥ... samuprakrameta, yasmāt ... tadadhīna... 
	% Jason suggests: to parse ``yasmāt sukhārtham [asti], [tasmāt] tadadhīnaḥ''
	\footnote{%
	As our translation indicates, the sentence construction does not allow an 
	unambiguous identification of who or what is the referent of the pronoun 
	\textit{tad} in the compound form \emph{tadadhīna} ‘devoted to it.’ Our 
	current understanding is that \emph{tad} refers to the ‘most pleasurable thing’ 
	mentioned in pāda a. It could, however, also refer to ‘them,’ that is, the ‘good 
	men.’%
	}

\item [3] A corrupted female reproductive system cannot consume semen, and 
therefore, the woman cannot hold the fetus. She gets severe 
\se{arśas}{prolapses}, 
\se{gulma}{abdominal lump} and similarly many other 
\se{roga}{diseases}.
	% think about ``praduṣṭa-'' (!!!):
	%% Martha suggested ``ruined'' 
	%%% COMM (ak): it fits really well here, but what to do about praduṣṭa- doṣas?
	%% spoiled, corrupted, “vitiated”?

\item [4] \emph{Doṣa}s \si{vāta}{wind}, etc., corrupted due to misconduct, amorous activity, fate, and also corruption of \se{ārtava}{?} and \se{bīja}{semen}, produce various diseases in the female reproductive system. These 20 diseases are taught here spearately and one by one along with their treatment, causes and signs.
	% Cf.:
	%% \emph{Mādhavanidāna} 62.1-2ab: \textit{viṃśatir vyāpado yonau nirdiṣṭā rogasaṃgrahe/ mithyācāreṇa tāḥ strīṇāṃ praduṣṭenārtavena ca// jāyante bījadoṣāc ca daivāc ca śṛṇu tāḥ pṛthak}
	% Cf.:
	%% \textit{Aṣṭāṅgahṛdaya} 6.33.28ab-29ab = \textit{Aṣṭāṅgasaṃgraha} 6.38.34:
	%% \textit{{viṃśatir vyāpado yoner jāyante duṣṭabhojanāt/ viṣamasthāṅgaśayanabhṛśamaithunasevanaiḥ/ duṣṭārtavād apadravyair bījadoṣeṇa daivataḥ //}
	% NOTE: Several other interpretations are possible: 
	%% - suratakriyāyāḥ can be Genitive
	%% - mithyopacāra- can wrong treatment specifically
	%% - chose ``amorous'' rather than ``sexual'' to capture the feeling of \emph{surata} better.

\item [5.1] Because of \emph{vāta}, occur:
<<<<<<< HEAD
	\begin{itemize}
		\item \se{Udāvartā}{Retaining(?)},
		\item the one called \se{Vandhyā}{Infertile}, and
		\item \se{Plutā}{Soaked},
		\item \se{Pariplutā}{Flooded}, and
		\item \se{Vātalā}{Flatulent/ Producing \emph{vāta}}.
	\end{itemize}

\item [5.2] And because of \emph{pitta}, occur:
	\begin{itemize}
		\item \se{Raktakṣayā}{Bloodloss},
		\item \se{Vāminī}{Vomiting}, and
		\item \se{Sraṃsanī}{Miscarrying},
		\item \se{Putraghnī}{Child-murderess}, and also
		\item \se{Pittalā}{Bilious/ Producing \emph{pitta}}.
	\end{itemize}


\item [6.1] And because of \emph{kapha} occur:
	\begin{itemize}
		\item \se{atyānanāa}{?},
		\item \se{karṇinī}{}, and
		\item two \se{caraṇī}{}, and
		\item another \se{śleṣmalā}{producing \emph{kapha}}.
	\end{itemize}

\item [6.2] And similarly there are other diseases involving all \emph{doṣa}s:
	\begin{itemize}
		\item \se{śaṇḍī}{},
		\item \se{aṇḍinī}{},
		\item \se{mahatī}{},
		\item \se{sūcīvaktrā}{},
		\item \e{sarvātmikā}{}.
	\end{itemize}
=======
	\begin{enumerate}
		\item \se{udāvartā}{Retaining},
		\item the one called \se{vandhyā}{Infertile}, and
		\item \se{plutā}{Soaked},
		\item \se{pariplutā}{Flooded}, and
		\item \se{vātalā}{Flatulent}.
	\end{enumerate}
\item [5.2] And because of \emph{pitta}, occur:
	\begin{enumerate}
		\item \se{Raktakṣayā}{Bloodloss},
		\item \se{vāminī}{Vomiting one}, and
		\item \se{Sraṃsanī}{Miscarrying},
		\item \se{putraghnī}{Child-murderess}, and also
		\item \se{pittalā}{Bilious}.
	\end{enumerate}


\item [6.1] And because of \emph{kapha} occur:
	\begin{enumerate}
		\item \se{atyānanda}{},
		\item \se{karṇinī}{}, and
		\item two \se{caraṇī}{},
		\item \se{putraghnī}{}, and also
		\item \se{pittalā}{bilious}.
	\end{enumerate}

\item [6.2] And similarly there are other diseases involving all \emph{doṣa}s:
	\begin{enumerate}
		\item \se{śaṇḍī}{},
		\item \se{aṇḍīnī}{},
		\item two \se{mahatī}{},
		\item \se{sūcīvaktrā}{},
		\item \se{sarvātmikā}{}.
	\end{enumerate}
>>>>>>> ed0ad3124576219dd582071ecd56976a2b6fee16

\item [7] The \se{Udāvartā}{} releases foamy \se{ārtava}{} with difficulty. 
	% Cf.:
	%% AS.Utt.38.39	vegodāvartanād yoniṃ prapīḍayati mārutaḥ | 
	%% sā phenilaṃ rajaḥ kṛcchrād udāvṛttaṃ vimuñcati ||
	%% AS.Utt.38.40	iyaṃ vyāpad udāvṛttā
One should know that the \se{Vandhyā}{} does not produce \se{ārtava}{}, and the \se{Utplutā}{} is constantly painful. 

\item [8] In case of the \se{Pariplutā}{}, there is extreme lust for the villagers' duty (sexual intercourse).
	% Cf.:
	%% \emph{Madhukośa} ad MaNi 62.3: 
	%%% \emph{‘grāmyadharmeṇa rug bhṛśam’ ity atra ‘grāmyadharme rucir bhṛśam’ iti pāṭgāntaram, tatra rucir abhilāṣaḥ, grāmyadharme maithune/}
	%% From VP: vyavāye maithune amaraḥ.
The \se{Vātalā}{} is hard, stiff, afflicted by stabbing and pricking pain.
In all four former types, there are painful sensations associated with the wind.
\end{translation}
