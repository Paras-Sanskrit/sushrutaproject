% !TeX root = incremental_SS_Translation.tex

\section{Uttaratantra, adhyāya 38}

\subsection*{Introductory remarks}

\paragraph*{Summary of the Content}
The chapter talks about various diseases of the female reproductive system and, 
in doing so, combines both aspects that go into a representation of diseases in 
āyurvedic literature: signs, symptoms and pathogenesis (\textit{nidāna}), on the 
one hand, and medical treatment (\textit{cikitsā}), on the other. In chapters of the 
\textit{Uttaratantra}, these two aspects are sometime dealt with in two different 
chapters X-\textit{vijñānīya} and X-\textit{pratiṣedha}. There are, however, many 
examples where this distinction is not made. 


\paragraph*{Placement of the Chapter}
While in \cite{vulgate} the current chapter is found at the end of the section on 
paediatrics (\textit{Kumāratantra}, or \textit{Kumārabhṛtya} as this section is 
styled in \MScite{Kathmandu KL 699}), in the Nepalese version, this is chapter 
6.58, and it is chapter 23 of an entirely different section, namely, the 
\textit{Kāyācikitsā}.

Several things are noteworthy in this regard:
\begin{itemize}
    \item In the placement of the vulgate, this chapter follows upon 6.37 
    \textit{Grahotpatti} (6.35 in the Nepalese version), a chapter that talks about 
    the origination of nine \se{graha}{planetary deities?} that are responsible for 
    all children's diseases described in previous chapters of the 
    \textit{Kumāratantra}. In this way, the current chapter retains the general 
    focus on the \se{kaumārabhṛtya}{child bearing}, but, at the same time, marks 
    a change to a distinct, less mystical approach to the topic at hand (that could 
    originate in a cultural milieu different from that of the preceding 11 chapters). 
    Ḍalhaṇa  \parencite[668b]{vulgate} explains how the chapter fits its context in 
    the following way: 
    \begin{quote}
        It is appropriate that for the sake of treating the \se{yonivyāpat}{disorders 
        of the female reproductive system}, the chapter called 
        \se{yonivyāpatpratiṣedha}{Countermeasures Against Disorders of the 
        Female Reproductive System} (SS.6.38) is taught immediately after the 
        chapter called \se{grahotpatti}{Origination of Planetary Deities} (SS.6.37). It 
        is because (1) there is an explicit mention of the word “\textit{yoni}” in the 
        statement  “born in the \se{yoni}{womb} of animal and human” (in 
        SS.6.37.13bc) and because (2) the \se{yonivyāpat}{disorders of the female 
        reproductive system} are the causes for the inborn disorders of children.%
        \footnote{%
        Ḍalhaṇa on SS.6.38.1: \textit{grahotpattyadhyāyanantaraṃ ‘tityagyoniṃ 
        mānuṣaṃ ca’ iti vacanena yoner nāmasaṃkīrtanāt 
        kumārajanmavikārakāraṇatvāc ca, yonivyāpaccikitsitārthaṃ 
        yonivyāpatpratiṣedhādhyāyārambho yujyate [\ldots]/}%
        }
    \end{quote}
    
    \item In the placement of the Nepalese version, \textit{Yonivyāpatpratiṣedha} 
    is preceded by 6.56 \textit{Mūtrāghātapratiṣedha} (6.58 in \cite{vulgate}) and 
    6.57 \textit{Mūtrakṛcchrapratiṣedha} (6.59 in \cite{vulgate}), two chapters 
    dealing with the diseases of the urinary tract. The current chapter carries on 
    with the topic of diseases that affect genitalia. In its Nepalese version, the 
    chapter opens with two verses that explain the reasons for treating the 
    particular set of diseases. These lack any reference to the 
    \se{kumārajanmavikāra}{inborn disorders of children} mentioned by Ḍalhaṇa, 
    and instead highlight the importance of curing female diseases for the 
    satisfaction of male partner. 
    \item SS.1.3 in both \cite{vulgate} and the Nepalese version lists the chapter at 
    the place, where it is found in the vulgate (Cf.\ Sū.3.37ab: 
    \textit{naigameṣacikitsā ca grahotpattiḥ sayonijāḥ}).
    \item Parallel chapters in the \textit{Aṣṭāṅgasaṃgraha} and the 
    \textit{Aṣṭāṅgahṛdayasaṃhitā} form a part of the \textit{Śalyatantra} section 
    of each text.
\end{itemize} 
%In the Nepalese version, this is chapter 6.58 (\textit{Kāyācikitsā} 23) that follows 
%upon 6.56 \textit{Mūtrāghātapratiṣedha} (6.58 in the vulgate) and 6.57 
%\textit{Mūtrakṛcchrapratiṣedha} (6.59 in the vulgate). In the vulgate, on the 
%other 
%hand, this chapter concludes another section, the \textit{Kumāratantra} 
%(\textit{Kumārabhṛtya} in \MScite{Kathmandu KL 699}), and follows upon 6.36 
%\textit{Naigameṣapratiṣedha} (6.34 in the Nepalese version) and 6.37 
%\textit{Grahotpatti} (6.35 in the Nepalese version).

\paragraph*{Parallels}
The current chapter is parallel in its content to \textit{Aṣṭāṅgasaṃgraha} 6.38 
and 6.39 as well as \textit{Aṣṭāṅgahṛdayasaṃhitā} 6.33 and 6.34 
(\textit{Guhyarogavijñāna} and \textit{Guhyarogapratiṣedha} respectively).%, 
%which form a part of the \textit{Śalyatantra} section of each text.

A close literary parallel to the first part of the chapter is found in 
\textit{Mādhavanidāna} (MN) 62, or at least its version printed in @@\citet{}. The 
readings of the MN as it stands now usually side with the vulgate version rather 
than with the Nepalese. In addition to the basic text, there are several valuable 
pointers made in the \textit{Madhukośa}, an early commentary on the MN. This 
part of the text is authored by Śrīkaṇṭhadatta, who was most like a direct student 
of Vijarakṣita. The latter wrote the first part of the \textit{Madhukośa}, up to 
chapter 32, and, what is more, can be dated to the second half of the 11th -- first 
half of the 12th centuries \citep[22--26]{meul-1974}.

Another most interesting parallel is found in Carakasaṃhitā 6(Ci).30.

\subsection*{Translation}
\begin{translation}
    
    \item [1] And now I shall explain the countermeasures against 
    \se{yonivyāpat}{disorders of the female reproductive system}.%
    \footnote{%
    On this broad understanding of the term \emph{yoni}, see \cite[pp.\ 
    572--5]{das-orig}}
    
    
    \item [*3] Since for good men, a woman is the most pleasurable thing, 
    therefore a physician should diligently attend to the diseases located in the 
    \se{yoni}{female reproductive system}, because he is entirely devoted to it 
    (that is, to curing these diseases) for the sake of (people's) happiness.%
    % DW: because he really is dependent ... for the sake of... 
    % yasmāt pramāda .... , ataḥ (= tasmāt) vaidyaḥ... samuprakrameta, yasmāt ... 
    %tadadhīna... 
    % Jason suggests: to parse ``yasmāt sukhārtham [asti], [tasmāt] tadadhīnaḥ''
    \footnote{%
    As our translation indicates, the sentence construction does not allow an 
    unambiguous identification of who or what is the referent of the pronoun 
    \textit{tad} in the compound form \emph{tadadhīna} ‘devoted to it.’ Our 
    current understanding is that \emph{tad} refers to the ‘most pleasurable thing’ 
    mentioned in pāda a. It could, however, also refer to ‘them,’ that is, the ‘good 
    men.’%
    }
    
    \item [*4] A corrupted \se{yoni}{female reproductive system} cannot consume 
    \se{bīja}{semen}, and therefore, the woman cannot take a fetus (that is, 
    become pregnant). She gets severe 
    \se{arśas}{prolapses}, 
    \se{gulma}{abdominal lump} and similarly many other 
    \se{roga}{diseases}.
    % think about ``praduṣṭa-'' (!!!):
    %% Martha suggested ``ruined'' 
    %%% COMM (ak): it fits really well here, but what to do about praduṣṭa- doṣas?
    %% spoiled, corrupted, “vitiated”?
\end{translation}

\paragraph*{Philological Notes}
The first two verses (2 and 3) in the Nepalese version are written in a classical 
variety of the \emph{upajāti} metre: 
\begin{center}
    \underline{$\cup$} $\_ \cup\ \_\ \_\ \cup\ \cup\ \_\ \cup\ \_\ \_$.
\end{center} 
In content, they are only approximately parallel to three hemistichs in 
\textit{anuṣṭubh} metre found in \cite{vulgate}. The latter verses lack the 
apologetic explanation concerning the reasons for this chapter being taught.

\begin{translation}
    
    \item [*5] \se{doṣa}{Humours}, \se{vāta}{wind}, etc., corrupted due to 
    \se{mithyopacāra}{faulty medical treatment},%
    \footnote{%
    In our translation of the compound \textit{mithyopacāra}, we decided for the 
    technical meaning of the term \textit{upacāra}, that is, `medical application' or 
    `treatment'. The combination \textit{mithyā + upa-$\sqrt{car}$} is attested 
    several times in medical literature. At least once, at CS Vi.3.38, it is given an 
    explicit commentarial gloss (by Cakrapāṇidatta): ``\textit{mithopacaritān iti 
    asamyak cikitsitān}''. In the SS \parencite{vulgate}, it is used once in Ut.18.30, 
    where it refers specifically to the wrong application of \se{tarpaṇa}{?} and 
    \se{putapāka}{?}, both of which are mentioned in the previous verse. Another 
    use of the compound in a seemingly conforming meaning is found in a citation 
    from Bhoja's text quoted by Gayadāsa at SS Ni.5.17: ``\textit{śvitraṃ tu 
    dvividhaṃ proktaṃ doṣajaṃ vraṇajaṃ tathā/ tatra mithyopacārād dhi vraṇasya 
    vraṇajaṃ smṛtam // \ldots}''. In contrast to this, the parallel verse in 
    \cite{vulgate} = CS Ci.30.7 = MN 62.1 reads \textit{mithyācāra} ‘wrong 
    conduct’. All commentators (Cakrapāṇidatta on the CS, Śrīkaṇṭhadatta on the 
    MN, and Ḍalhaṇa on the SS) explain that the wrong conduct stands here 
    specifically for unwholesome diet. The parallel in AH Ut.33.28 = AS Ut.38.34 
    plainly reads \textit{duṣtabhojana} `corrupted food' instead.  
}
sexual activity, fate, and also \se{doṣa}{defects} of \se{ārtava}{menstrual 
blood} and \se{bīja}{semen}, produce various diseases in the \se{yoni}{female 
reproductive organ}. 
These 20 diseases are taught here distinctly and one by one along with their 
\se{bheṣaja}{treatment}, \se{hetu}{causes} and \se{cihna}{signs}. 
%% mithyopacāra-/ mithyācāra -->technically, this usually refers to “wrong 
%%treatment”/ “faulty 
%% mithyācāra- is glossed in Caraka + Mādhavanidāna --- asamyagāhārācara
%% Aṣṭāṅga... use ``duṣṭabhojana-''
%%% Mithyopacāra in Caraka Vi.3.38 is glossed with “asamyak cikitsita-” 
% Cf.:
%% \emph{Mādhavanidāna} 62.1-2ab: \textit{viṃśatir vyāpado yonau nirdiṣṭā 
%%rogasaṃgrahe/ mithyācāreṇa tāḥ strīṇāṃ praduṣṭenārtavena ca// jāyante 
%%bījadoṣāc ca daivāc ca śṛṇu tāḥ pṛthak}
% Cf.:
%% \textit{Aṣṭāṅgahṛdaya} 6.33.28ab-29ab = \textit{Aṣṭāṅgasaṃgraha} 6.38.34:
%% \textit{{viṃśatir vyāpado yoner jāyante duṣṭabhojanāt/ 
%%viṣamasthāṅgaśayanabhṛśamaithunasevanaiḥ/ duṣṭārtavād apadravyair 
%%bījadoṣeṇa daivataḥ //}
% NOTE: Several other interpretations are possible: 
%% - suratakriyāyāḥ can be Genitive
%% - mithyopacāra- can wrong treatment specifically
%% - chose ``amorous'' rather than ``sexual'' to capture the feeling of 
%%\emph{surata} better.
%% the intro verses are identical with Carakasaṃhitā 6.30.7--8. There, in Cakra 
%%comments on ``rogasaṃgraha'' saying that it refers to 1.19.3 !!!
\end{translation}

\paragraph*{Philological Notes} 
The Nepalese version of the SS continues here with 3 hemistichs in classical 
\textit{upajāti} metre (see the syllabic pattern above). On the other hand, 
\cite{vulgate} contains two complete verses (4 hemistichs) in the 
\textit{anuṣṭubh}. Three final hemistichs are found verbatim in CS Ci.30.7cd--8. It 
is very likely that the these verses were borrowed from the CS into SS (and not 
the other way around), because CS Ci.30.7cd = SS Ut.38.5ab says that the 20 
kinds of diseases were already taught in the \se{rogasaṃgraha}{Collection of 
Diseases}. In the context of the SS, this reference does not make any sense and 
is left uncommented by Ḍalhaṇa. In case of the CS, however, Cakrapāṇidatta 
explains that this reference points back to CS Sū.19, a chapter that does, in fact, 
lists all the diseases dealt with in later sections of the text. 20 diseases of 
\se{yoni}{female reproductive system} as mentioned in Sū.19.3.

The above three hemistichs in \textit{anuṣṭubh} are also repeated in MN 
62.1--2ab. Given that all following verses stem from the SS, it is likely that MN 
62.1--2ab too was incorporated into the text from the SS (and not its original 
location in the CS).   

\begin{translation}
\item [*6.1] Because of \se{vāta}{wind}, \se{yoni}{female reproductive organ} 
becomes:
%All feminines refer to {yoni}, and hence are various kinds of adjectival forms 
%(ktānta-s, bahuvrīhi-s, or other kinds of adjectives). Perhaps, instead of “occur” 
%one could say smth like “\se{yoni}{} becomes...”, There is actually “bhavet” in 
%6.

\begin{enumerate}
    \item \se{udāvartā}{?},
    \item called \se{vandhyā}{Infertile}, and
    \item \se{plutā}{Sprung},
    \item \se{pariplutā}{Flooded}, and
    \item \se{vātalā}{Windy}.
\end{enumerate}

\item [*6.2] And because of \se{pitta}{choler}, occur:
\begin{enumerate}
    \item \se{raktakṣayā}{With bloodloss},
    % This is, according to the analysis given in a later verse, a rare case of a 
    %``genetive bahuvrīhi'' (A reads it as a grammatically more correct Locative 
    %bahuvrīhi!).
    \item \se{vāminī}{Vomiting}, and
    \item \se{sraṃsanī}{Causing a Fall},
    % this one is most likely formed with some kind od LyuṬ (or smth like that) + 
    %ṄīP. See similar formations like jananī, hananī, śāmanī, śamanī
    % the Aborting One (?)
    \item \se{putraghnī}{Child-murderess}, and also
    \item \se{pittalā}{Bilious / Choleric}.
\end{enumerate}


\item [*7.1] And because of \se{kapha}{phlegm} occur:
\begin{enumerate}
    \item \se{atyānandā}{Extremely Excited},
    \item \se{karṇinī}{Protuberant}, and
    % @@NOTE a conjunctive error (errores coniunctivi)
    \item[3.\ \& 4.] two \se{caraṇī}{}, and
    \item[5.] other \se{śleṣmalā}{Phlegmatic}.
\end{enumerate}

\item [*7.2] And similarly there are other (kinds of morbid female reproductive 
system) involving all \emph{doṣa}s:
\begin{enumerate}
    \item \se{śaṇḍhī}{Impotent},
    \item \se{aṇḍīnī}{With testicles},
    \item two \se{mahatī}{Huge},
    \item \se{sūcīvaktrā}{With a needle-like opening},
    \item \se{sarvātmikā}{}.
\end{enumerate}
\end{translation}

\paragraph*{Philological Notes}
Verses 5 and 6 consist of four hemistichs written in a kind of \textit{triṣṭubh} 
metre --- that is, of eight unequal \textit{pāda}s containing 11 syllables each --- 
and correspond to six hemistichs in \textit{anuṣṭubh} in \cite{vulgate} 
(Ut.38.6cd--9cd).
By the standards of classical Sanskrit prosody, the metre in all four hemistichs is 
irregular. However, considering the wide range of metrical variations of the 
\textit{triṣṭubh} permissible in Epic Sanskrit, the concerned verses can be 
considered to fall well within metrical norm. Based on the metrical analysis of a 
large sample of \textit{triṣṭubh} passages in the \textit{Mahābhārata}, 
\textcite[108]{fitz-2009} postulated the following general metrical structure: 

\begin{table}[h!]
\centering
\caption{\small Summary of table 3 in \cite{fitz-2009}.}
\resizebox{\linewidth}{!}{%
    \begin{tabular}{l|c|c|c|c|c}
        syllable nr. & 1 & 2,3,4    & 5,6,7       & 8,9,10 & 11  \\ 
        \cline{2-6}
        & x & ra (~$\_\ \cup\ \_$~), ma (~$\_\ \_\ \_$~) & bha (~$\_\ \cup\ \cup$~), ra 
        (~$\_\ \cup\ \_$~), sa (~$\cup\ \cup\ \_$~) & ra (~$\_\ \cup\ \_$~)  & x  
    \end{tabular}
}
\end{table}

Our verses scan:

\begin{table}[h!]
\centering
\caption{\small Metrical structure of vss.\ Ut.38.5--6 in the Nepalese version of the 
\SS.}
\begin{tabular}{ c || c | c }
    5 & $\cup\ \_\ \_\ \_\ , \_\ \cup\ \_\ \_\ \cup\ \_\ \cup$ & $\cup\ \_\ \cup\ \_\ , \_\ \cup\  
    \_\ \_\ \cup\ \_\ \_$ \\ 
    6 & $\_\ \_\ \cup\ \_\ , \_\ \cup\ \_\ \_\ \cup\ \_\ \cup$ & $\_\ \_\ \cup\ \_\ , \_\ \cup\ \_\ 
    \_\ \cup\ \_\ \_$     
\end{tabular}
\end{table}

Following Fitzgerald's hypothesis \parencite[99]{fitz-2009} formulated explicitly 
with regard to the \textit{Mahābhārata} that ``the more variable a 
\textit{triṣṭubh} passage of the Mbh is, the older it is likely to be'', one may 
speculate that the current passage in the Nepalese version may go back to an 
ancient textual layer that, at the time when the hyparchetype of the Nepalese 
version was produced, was not yet fully ``Sanskritized'' and harmonized with the 
surrounding passages. 
Alternatively and, perhaps, less likely, vss.\ 5--6 of the Nepalese version could 
have been composed as an attempt to harmonize the text of the SS --- that is, to 
recast the list of diseases originally written in \textit{anuṣṭubh} into 
\textit{triṣṭubh}. %428813269the author of which was familiar with the Epic forms.

Note that so far we have not come across any other examples of non-Classical 
metres used either in \cite{vulgate} or in the Nepalese version. 

\begin{translation}
\item [9] The \se{udāvartā}{Retaining} releases foamy \se{rajas}{menstrual 
blood} with pain. 
% Cf.:
%% AS.Utt.38.39	vegodāvartanād yoniṃ prapīḍayati mārutaḥ | 
%% sā phenilaṃ rajaḥ kṛcchrād udāvṛttaṃ vimuñcati ||
%% AS.Utt.38.40	iyaṃ vyāpad udāvṛttā
One should diagnose the \se{vandhyā}{Infertile} by the absence of 
\se{ārtava}{menstrual blood}, and the \se{utplutā}{?} by chronic pain.
% DW: The \se{vandhyā}{} is symptomized by the absence of 
%\se{ārtava}{menstrual blood}, and the \se{utplutā} by chornic pain.
% in the above list we had `plutā' instead. 
% A has `viplutā'
% CS 1.19.4.(9)| has pariplutā + upaplutā
In the case of \se{pariplutā}{Flooded}, there is an extreme appetite for sex.
% Cf.:
%% \emph{Madhukośa} ad MaNi 62.3: 
%%% \emph{‘grāmyadharmeṇa rug bhṛśam’ ity atra ‘grāmyadharme rucir 
%%%bhṛśam’ iti pāṭgāntaram, tatra rucir abhilāṣaḥ, grāmyadharme maithune/}
%% From VP: vyavāye maithune amaraḥ.

\item [11] The \se{vātalā}{Windy} is hard, stiff, afflicted by stabbing and pricking 
pain.
And in four former types too, there are \se{vedanā}{painful sensations} 
associated with the \se{anila}{wind}.

\item [12] The \se{lohitakṣayā}{Bloodloss} is the one that has blood that 
diminishes with a burning sensation.
%%todo Is the prakriyā with “yasyām” more correct? What is the reason behind 
%%this difference? 
And the \se{vāminī}{Vomiting}, flooded with \se{rajas}{menstrual blood}, ejects 
the \se{bīja}{semen} in the flow. 
\footnote{%
The exact force of \emph{srutau} ‘in the flow’ remains unclear.%
}

\item [13] The \se{prasraṃsanī}{Falling} protrudes, it is agitated, and delivery is 
hard.
% conjunctive error
% kṣobhitā ? --- agitated
%%@@ NOTE! This is a very interesting case! The anusvāra after 
%%prasraṃsanī_ṃ_ in both \MScite{Kathmandu KL 699} and 
%%\MScite{Kathmandu 
%%NAK 5-333} does not look like an anusvāra. In fact, in \MScite{Kathmandu 
%%KL 
%%699}, there is clearly a syllable missing after °sanī [saṃ]. It seems possible 
%%that 
%%the mark above the line is an INSERTION SIGN in the template. 
%% Both \MScite{Kathmandu KL 699} and \MScite{Kathmandu NAK 5-333} 
%%have it, which suggest that they share a common template !!! in which the 
%%original INSERTIONI MARK was already MISINTERPRETED !!!!
%% \MScite{Kathmandu NAK 5-333} is again either improvising, or using 
%%another MS. !
%%@@ the anusvāra after duḥprajāyinīṃ looks original though. So, perhaps, one 
%%should try making sense of this reading.
The \se{putraghnī}{Child-Murdress} kills a well-established fetus because of 
flows of \se{rakta}{blood}.%
\footnote{%
Note that our interpretation of the semantic value of the reduplication 
\emph{sthitaṃ sthitam} follows Ḍalhaṇa's comment: \emph{sthitaṃ sthitaṃ 
grabhaṃ hanti, notpannamātram}, ‘She kills a “\emph{sthitaṃ sthitam}” fetus, 
not the one that has just arisen.’ Note, however, that from a strict Pāṇinian point 
of view, this reduplication can be used to indicate either a permanent or a 
repeated character of an action or propererty (Cf.\ A 8.1.4: 
\emph{nityavīpsayoḥ}), thus ‘always established’ and ‘repeatedly established’ 
respectively. The second option seems contextually fitting as well and would point 
towards repeated miscarriage.%
}
% this follows Ḍalhaṇa's interpretation of “sthitaṃ sthitam [...] notpannamātram”.
% WHO: repeated abortions
% 
\end{translation}

\paragraph*{Philological Notes}
In 10ab, we introduced two minor corrections and deleted the final 
\emph{anusvāra}s in \emph{prasraṃsanīṃ} and \emph{duḥprajāyanīṃ} found 
in both MSS. In doing so, we effectively changed the Accusative ending to the 
Nominative ones. Apart from mere grammatical, that is, syntactic, reasons, we 
believe that it is possible to explain how this mistake could occur. Based on 
irregular forms of both \emph{anusvāra} signs (that is, in \MScite{Kathmandu KL 
699} and \MScite{Kathmandu NAK 5-333}) at the end of \emph{prasraṃsanīṃ}, 
and considering the fact that \MScite{Kathmandu KL 699} is missing one syllable, 
we believe that both MSS could have faithfully copied what initially was an 
insertion mark of their common ancestor. The addition of an \emph{anusvāra} 
after \emph{duḥprajāyanī}, on the other hand, is most likely deliberate and 
occured after the initial confusion between an insertion mark and 
\emph{anusvāra} in order to smooth out the syntax.  
% Once the confusion between an insertion mark and an \emph{anusvāra} has 
%happened, the change to \emph{duḥprajāyanī\textbf{ṃ}} was purposefully 
%introduced to smooth out the syntax of the individual hemistich. In this scenario, 
%it 
%seems the most plausible to postulate that the common template of 
%\MScite{Kathmandu KL 699} and \MScite{Kathmandu NAK 5-333} already 
%contained the initial confusion between the two signs. 

If we are correct in thinking that the omission of one syllable in 10a was already 
present in the common ancestor of \MScite{Kathmandu KL 699} and 
\MScite{Kathmandu NAK 5-333}, the question about the source of 
\MScite{Kathmandu NAK 5-333}'s reading \emph{\underline{saṃ}sraṃsate} 
arises. At the moment, it remains unclear to us whether the scribe of 
\MScite{Kathmandu NAK 5-333} had access to further textual sources or whether 
he conjectured the text on his own. Note also that this hemistich is written in an 
uncommon type of \emph{anuṣṭubh}, namely, a \emph{ta-vipulā}. Note, 
furthermore, that a reding parallel to the Nepalese edition is found, for example, 
in Mādhavanidāna 64.6ab. Here, however, the text readds \emph{sraṃsate 
\underline{ca}}, which bring the metre back to a regular \emph{anuṣtubh}.  

\begin{translation}
\item [14] The \se{pittalā}{Choleric} has intense \se{dāha}{burning sensation} 
and \se{pāka}{inflammation}.
%%todo check “atyartha” in \MScite{Kathmandu KL 699}, if it has ttha? ththa?
And in the case of the first four kinds as well,%
\footnote{%
The first four kinds are described in the preceding verses. They are 
\se{lohitakṣayā}{}, \se{vāminī}{}, \se{prasraṃsanī}{} and \se{putraghnī}{}.
} 
one should include the symptoms of \se{pitta}{choler}.
%% yuj could also mean “treat”

\item [15]
She overindulges in \se{grāmyadharma}{sex} because of 
\se{atyānanda}{excessive enjoyment} and dissatisfaction.%
\footnote{%
The syntax of 12ab differs from its parallel formulations beginning with 8ab. The 
most notable irregularity is that the concerned hemistich lacks the name of the 
described condition and, consequently, the Nominative subject of the short 
sentence. It seems likely, therefore, that the Ablative \textit{atyānadāt} ‘because 
of excessive enjoyment’ is meant additionally to explain the reasons behind the 
specific name of the disease, that is \se{atyānandā}{Excessive Enjoyment}.
}
And in the case of \se{karṇinī}{?}, from \se{śleṣman}{phlegm} and 
\se{āsṛk}{menstrual blood} a \se{karṇikā}{protuberance} develops in the 
\se{yoni}{?}. 

\item[16]
During \se{maithuna}{sexual intercourse}, the first \se{caraṇī}{?} is the one 
that surpasses the man.%
\footnote{%
Ḍalhaṇa's reports two readings of the hemistich (see the Philological Notes) and, 
accordingly, proposes two slightly different (though equally puzzling) explanations 
of the clause `to surpass the man during sexual intercourse'. In the first variant 
(identical with the Nepalese version), he explains that during the intercourse, the 
\se{yoni}{vulva?} afflicted by the condition becomes bigger, i.e., swells: 
\textit{pūrvā caraṇī atiricyate maithunācaraṇe' dhikā bhavati}. However, it 
remains unclear what syntactic role is ascribed to the Ablative of the word `man' 
(\textit{puruṣāt}).
The second explanation is similarly unclear: \textit{puruṣāt pūrvam atiricyate, 
atyarthaṃ kaṇḍūyata ity arthaḥ}. Taken literally, it says that a woman afflicted by 
the particular condition is scratched excessively (or, perhaps, feels excessive 
itchiness). \textit{Madhukośa} accepts Ḍalhaṇa's alternative reading as the main 
text of \textit{Mādhavanidāna} 62.9ab. Accordingly, it assumes that the condition 
is called \textit{a-caraṇā} and that it makes a woman in-capable of enjoying 
lovemaking so that she withdraws from it before the man (\textit{acaraṇā' 
samyaṅmaithunācaraṇāt pūrvaṃ prathamaṃ puruṣād atiricyate viramati}).%
}
Because of frequent excessive intercourse, the \se{bīja}{semen} then does not 
stay in place.%
\footnote{%
The syntactic structure of 13cd corresponds to that of 12ab, and, by the same 
token, it seems likely that the Ablative ‘because of frequent excessive intercourse’ 
(\textit{aticaraṇāt}) is meant to explain the name of the condition, namely, 
\se{aticaraṇā}{Excessive Intercourse}.
}
\end{translation}

\paragraph*{Philological Notes}
Note here that the reading of Ut.38.16ab printed in the \cite{vulgate} is the one 
given by Ḍalhaṇa as an alternative. The reading that he accepted in his main text 
(inferable from the text his commentary) must have been identical with 
Ut.38.13ab of the Nepalese version: \textit{maitunetyādi/ pūrvā caraṇī atiricyate 
maithunācaraṇe `dhikā bhavati} 

\begin{translation}
\item [17]
\se{śleṣmalā}{Phlegmatic} \se{yoni}{female genitals} are slimy, tormented by 
itchiness and very cold. 
And in the first four types too, one should include symptoms of 
\se{kapha}{phlegm}.

\item [18]
In the case of \se{ṣaṇḍhī}{?},%
\footnote{%
It is noteworthy that both MSS equally unambiguous in transmitting 
\textit{ṣaṇḍhī} as the name of the condition here, and calling it \textit{śaṇḍhī} in 
the list above (see verse 8). At the moment we preserve this orthographic 
variation in our provisional edition and in the translation. Note, furthermore, that 
in our printed sourses of the verse (\cite{vulgate} and 
@@\textit{Mādhavanidāna}@@) that condition is caleld \textit{ṣaṇḍī}.%
} 
the \se{ārtava}{mentrual blood} and breasts are missing,%
\footnote{%
From a strict grammatical point of view, the compound \textit{naṣṭārtvastanaḥ} 
(as well as the uncompounded reading of H, \textit{naṣṭārtavaḥ stanaḥ}) should 
mean ‘breasts that lack menstrual blood’. At the moment, we cannot make any 
sense of this translation and follow the \cite{vulgate} (\textit{anartvastanā 
ṣaṇḍī}) and the \textit{Mādhavanidāna} (MN 62.11: \textit{anartavā 'stanī ṣaṇḍī}) 
in thinking that the intention of the author was to express that both the breasts 
and the mentrual blood of a woman afflicted with the particular condition are 
reduced. From a grammatical point of view, however, the \textit{karmadhāraya} 
compound \textit{ārtavastana} should be either singular neuter or dual 
masculine.%
} 
and during sex, it is rough to the touch.

And the \se{yoni}{?} of a juvenile woman, taken by a copulent man,%
\footnote{%
Both medieval commentators, Ḍalhaṇa and Śrīkaṇṭhadatta, explain that a 
‘copulent’ is aneiphimism for a large penis. Cf.\ Ḍalhaṇa on SS.Ut.38.18: 
\textit{atikāyo bṛhatsādhano naraḥ} and Śrīkaṇṭhadatta on MN 62.11: 
\textit{atikāyagṛhītāyā mahāmehanena gṛhītāyaḥ}.%
%TODO ALSO MW and Āyurvedīya Śabdakośa report ṣaṇḍhī is an alternative 
%spelling
} 
may become \se{aṇḍanī}{?}.%
\footnote{%
Note that in the version of \cite{vulgate}, the condition \textit{aṇḍānī} is called 
\textit{phalinī}, or, according to a variant reading and its explanation offered by 
Ḍalhaṇa, \textit{aphalinī}.
@@MN@@ reads \textit{aṇḍalī} and reports \textit{aṇḍiṇī} as the reading of the 
MS `\textit{ka}'. Since the term presupposed by Śrīkaṇṭhadatta was based on the 
primary nominal stem \textit{aṇḍa} ‘egg’ (Cf.: \textit{aṇḍa\underline{l}ī aṇḍavan 
niḥsṛtā yoniḥ}), it is more likely that he read either \textit{aṇḍanī} or 
\textit{aṇḍinī}. Note, furthermore, that in Maithilī as well as in the modern Bengali 
script, letters \textit{n} and \textit{l} can be easily confused.
}
%TODO add MN to bibliography
%@@check@@ Āyurvedīya mahākoṣa (supplementary volume) ->> aṇḍinī <<- 
%quoting THIS verse, so he must have presupposed the reading of the \Nep

\item [19]
\se{mahāyoniḥ}{} is wide open and \se{sūcīvaktrā}{} is very closed.
For those women who have all the symptoms,%
their \se{yoni}{genitals} have all the humours.

\item [20]
And also in first four types, one observes the symptoms of all humours.%
%TODO @@add quote from the Bhāvaprakāśa in the provisional edition !!!! 
% @@TILL HERE@@ 
These five types of incurable \se{yonivyāpat}{diseases of female genitalia} are 
diseases that arise from all the humours.
% asādhya<>imā or vyadhayaḥ
%TODO add a footnote about "catasṛṣu"
%TODO change to the numbering as in the Edition
\end{translation}

\paragraph*{Philological Notes}
In this verse, we introduced a conjectural emendation and adopted the reading 
\textit{sarvaliṅganidarśanam} instead of the variant 
\textit{sarvaliṅgānidarśanam} that is supported by both MSS. We have two 
reasons for doing so. 

On the one hand, we think that the reading of the MSS is faulty. If parsed as 
\textit{sarvaliṅgāni darśanam}, we arrive at a faulty syntactic construction that 
we think is unlikely to occur in our text. The parsing 
\textit{sarvaliṅga-anidarśanam} ‘one does not observe the symptoms of all 
humours’, on its turn, is possible to sustain from the point of syntax. However, it 
violates the repetitive structure of the section, which consists of (1) a set of verses 
describing specific symptoms of the four types of \se{yonivyāpat}{} associated 
with a particular humour, (2) a description of a general type of 
\textit{yonivyāpat}{} caused by the particular humour, and (3) a statement that 
the general symptoms of the particular humour are observed in the case of the 
four specific types as well. This being the case, we expect that the current verse 
does exactly this.

On the other hand, our conjecture is supported by an external evidence of 
\textit{Bhāvaprakāśa} Ma Ci 70.16 that reads \textit{sarvaliṅgasamutthānā 
sarvadoṣaprakopajā | catasṛṣv api cādyāsu sarvaliṅganidarśanam ||} 


\begin{translation}
\item [21]
But in the case of the curable ones, the sequence beginning with oleation%
\footnote{%
Commenting on this, Ḍalhaṇa says: “In this way one understands the following 
meaning: in the case of curable diseases of female genitals, one should first use 
the type of oleation that counters the particular humour and then apply 
therapeutic emesis etc.”
(\textit{etena yasya doṣasya yaḥ pratyanīkaḥ snehas tena snehena saṃsnehya 
tato vamanādīn sādhyāsv avacārayed ity arthaḥ})
} 
for each \se{doṣa}{humour} is recommended.
And one should especially administer an \se{uttarabasti}{vaginal douche} 
according to the instructions.%
\footnote{%
SS Ci 37.100ff.\ give a detailed account of the therapeutic procedure called 
\textit{uttarabasti}. These verses also describe specific instruments and 
application methods that vary depending on the gender and age of the patient. 
Therefore, our translation ‘vaginal douche’ is called by the context. In other 
contexts, the same term could refer instead to smth.\ like ‘urethral douche’.
}

\item [22]
One should treat a \se{yoni}{female genital}  that is rough, cold, rigid and 
\se{alpasparśa}{lacking in sensation} with \se{kumbhīsveda}{pot-sweat}%
\footnote{%
The term \textit{kumbhīsveda} occurs several times in other āyurvedic works. 
Commenting on this verse, Ḍalhaṇa gives a detailed account of the procedure: 
``One should treat with \textit{kumbhīsveda}, that is to say, one should 
prepare a pot filled with decoction made from wind-reducing substances 
such as meat of aquatic animals and those living in marshes, bury it in earth, 
prepare a bed above it, add to the decoction globules of 
\se{lauhapāṣāṇa}{iron stones?} melted in the fire, and treat the woman 
with the \se{bāṣpasveda}{vapour} that arises from that pot and is directed 
only to the region of \se{yoni}{female genitalia}. However, others explain 
that one should take the heat that comes about when one adds water into 
the pot filled with meat of aquatic and marshy animals as well as substances 
reducing wind.''
(\textit{kumbhīsvedaiḥ, ānūpaudakamāṃsavātaghnadravyakvāthapūrṇāṃ 
kumbhīṃ kṛtvā bhūmau nikhanya tadupari śayyāṃ 
saṃsthāpyāgnisantaptalauhapāṣāṇaguḍakān kvāthe nikṣipya tadutthitair 
bāṣpasvedair yonipradeśamātragāmibhir upacaret; anye tu kumbhīṃ 
vātaharadravyānūpaudakamāṃsapūrṇāṃ sajalāṃ kṛtvā pravṛttoṣmāṇaṃ 
gṛhṇīyād iti vyākhyānayanti/} -- \Su{6.38.24ab}{669--670}.)
} 
filled with marsh water. 
%meat of animals living in the marshes as well as water.
%TODO add a footnote about audaka/ udaka issue (also anūpa/ ānūpa)
%TODO DISCUSS IS IT REALLY MARSH WATER?!?!

\item [23]
One should insert \se{veśavāra}{}%
\footnote{%
As Ḍalhaṇa points out here, \textit{veśavāra} is defined in SS.1.46.365--6 as 
follows: “Meat, boneless and steamed, is again pounded on stony slab and cooked 
after mixing pippalī, śuṇṭhī, marica, jaggery and ghee. This is known as vesavāra 
(curry). Vesavāra is heavy, unctuous, promotes strength and allays disorders of 
vāta.” (trl.\ PV Sharma, vol.\ 1, p.\ 530). 
\textit{māṃsaṃ nirasthi susvinnaṃ punardṛṣadi peṣitam | 
pippalīśuṇṭhimaricaguḍasarpiḥsamanvitam || aikadhyaṃ pācayet samyag 
vesavāra iti smṛtaḥ | vesavāro guruḥ snigdho balyo vātarujāpahaḥ ||}
} 
mixed with sweet drugs in the \se{yoni}{vaginas}. And they also should very 
gently keep \se{balā}{} oil.

\item [24]
One should also apply wholesome cleansings as well as emeses. 
In conditions associated with \se{uṣā}{burning sensations} and \se{coṣa}{heat}, 
a physician should apply cold treatments that were told.
% note uṣā in H/ ūṣā in \MScite{Kathmandu KL 699}/ oṣa in Ed

\item [25]
A physcian should fill \se{yoni}{vagina} that has a bad smell, or also the one that 
is slimy, with powders prepared from \se{pañcakaṣāya}{five astringent 
substances}.%
\footnote{%
According to Ḍalhaṇa, the \se{pañcakaṣāya}{five astringent substances} are 
\se{nyagrodha}{}, \se{udumbara}{}, \se{plakṣa}{}, \se{aśvattha}{} and 
\se{gardabhāṇḍa}.
} 
And in these conditions, one should employ the decoction made from ingredients 
such as the \se{rājavṛkṣa}{royal tree} as the cleansing agent.%
\footnote{%
Note that in translating this verse, we put a comma after \textit{pūrayet}.%
}

\item [26]
As for a vagina connected with \se{yonikrimi}{vaginal worms},%
\footnote{%
The condition \textit{yonikrimi} seems to be virtually unknown in currently 
preserved  āyurvedic literature. The term is neither recorded in 
\textcite{josi-maha} or found in any of the available electronic texts. As a matter 
of fact, the only occurrence of the term in an āyurvedic work we are aware of, is a 
footnote (!) that reports an alternative reading of Cakrapāṇidatta's commentary 
on CS Ci.30.18 (see \textcite[635b]{cara-trikamji3}).
Outside of medical literature, the \textit{Śabdakalpadruma} records a quote from 
the \textit{Brahmavaivarttapurāṇa} (\textit{śrīkṛṣṇajanmakhaṇḍe 83 adhyāyaḥ}). 
Among other things, it describes the hardships that await a \textit{śūdra}, who 
transgresses his \textit{dharma} and has sex with a Brahmin woman or with his 
mother. After suffering for a hundred Brahma-years in the hell, he is damned for 
all times to reincarnate on earth among presumably some of the lowest and most 
wretched beings. So, for example, he will be reborn for seven times as a 
\se{yonikrimi}{vaginal worm} of prostitutes (\textit{yaḥ śūdro brāhmaṇīgāmī 
mātṛgāmī sa pātakī | \ldots\ yonikrimiḥ puṃścalīnāṃ sa bhavet saptajanmasu 
||})     
} 
one should slowly fill it, along with cow urine and sea salt, with pastes composed 
of cleansing substances.

\item [27]
And he should wash the vagina that is itchy and \se{niḥsparśa}{lacking in 
sensation} with the water of \se{bṛhatī}{Indian barberries} and fill it with pastes 
made of them. He should also fumigate it.
% @@note my emendations
%% @@NOTE@@ if we don't change -pariśodhitā, it is the woman herself (rather 
%%than a physician), who should do all those things. 
%%% Given social stigmas etc., it is conceivable that male doctors didn't want to/ 
%%%could not perform these operation on female private parts.
%%%% SAME thing about {prasraṃsanī} and the change svinnā<>svinnāṃ
% should dhūpayet also go with kalkaiḥ ?!

\item [28]
Suppositories with cleansing substances should be administered into the 
\se{karṇinī}{Protuberant}.

One should anoint the \se{prasraṃsanī}{} with ghee, sweat it and enter milk 
into it. 
%@@NOTE@@ the change between svinnā>svinnām noted above
%% @@ At any rate, my choice of assigning the pradhāna- and the gauṇa- karma 
%%(i.e., prayojitakartṛ) is more or less random. 

\item [29] 
And then, a physician should cover it with \se{veśāvara}{} and make a bandage. 
And for each \se{doṣa}{humour}, he should place/ prescribe 
SURĀRIṢṬASAMĀM/N.

\item [30cd]
He should also prescribe food reach in milk and meat broth.
% blood ?

\item [31]
The \se{doṣa}{diseases} of \se{śukra}{semen}, \se{ārtava}{menstrual blood}, 
\se{stanya}{} as well as the diseases of \se{rasa}{} were told, so also the 
causes for impotence and the ways of an aborted fetus. 

\item [32] 
And also the treatment given to a pregnant woman in case of various diseases 
was also told. And a physician should also treat the subsequent diseases that 
arise after these.  

\end{translation}