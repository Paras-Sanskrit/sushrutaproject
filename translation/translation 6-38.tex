% !TeX root = incremental_SS_Translation.tex

\section{Uttaratantra, adhyāya 38}

\begin{translation}

\item [1] And now I shall explain the countermeasures against \se{yonivyāpat}{disorders of the female reproductive system}.%
	\footnote{%
	On this broad understanding of the term \textit{yoni}, see \cite[pp.\ 572--5]{das-orig}}

\item [2] Since for good men, a woman is the most pleasurable thing, therefore a physician should diligently attend to the diseases located in the \se{yoni}{female reproductive system}, because he is devoted to it for the sake of (people's?) happiness.%
	\footnote{%
	As our translation indicates, the sentence construction does not allow an inambigues identification of who or what is the referent of the prounoun \textit{tad} in the compound form \textit{tadhīna} ‘devoted to it’. Our current understanding is that \textit{tad} refers to the ‘most pleasurable thing’ mentioned in pāda a. It could, however, also refer to ‘them’, that is, the ‘good men’.%
	}

\item [3] Female reproductive system, when morbid, cannot consume the semen, and therefore, the woman does not get pregnant. She gets severe \se{arśas}{prolapses}, \se{gulma}{abdominal lump} and similarly many other \se{roga}{diseases}.


\end{translation}
