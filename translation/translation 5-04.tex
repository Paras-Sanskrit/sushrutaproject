% !TeX root = incremental_SS_Translation.tex
\section{Kalpasthāna, adhyāya 4}

\subsection{Introduction}
The fourth chapter of the Kalpasthāna of the \emph{Suśrutasaṃhitā} addresses 
the topic of snake bites and snake venom.  Unusually for the Nepalese version of 
the \SS, the discussion is framed as a question from Suśruta to the wise 
Dhanvantari.  Suśruta's questions are about the number of snakes, how they 
are classified, the symptoms of their bites and the pulses or stages of poisoning 
experienced by a victim of snakebite and related topics.  The taxonomy of snakes 
is presented in a presentational variant form in Figure~\ref{snakes}.


    
\subsection{Literature}
A brief survey of this chapter's contents and a detailed assessment of the
existing research on it to 2002 was provided by Meulenbeld.\footcite[IA,
292--294]{meul-hist}.\footnote{See also, \cites{doni-2015,ewar-1878}.}
%Translations of this chapter since 2000 have appeared by 
%\textcites[131--139]{wuja-2003}[3, 1--15]{shar-1999}{srik-2002}.\footnote{For a 
%    bibliography of translations to 2002, including Latin (1847), English (1877), 
%Gujarati (1963) 
%    and Japanese (1971), see \cite[IB, 314--315]{meul-hist}.}
    
A discussion of this chapter specifically in the light of the Nepalese
manuscripts was published by Harimoto.\footcite[101--104]{hari-2011} After a
close comparative reading of lists of poisonous snakes, Harimoto concluded
that, “the Nepalese version is internally consistent while the [vulgate]
editions are not.”  Harimoto showed how the vulgate editions had been
adjusted textually to smooth over inconsistencies, and gave insights into
these editorial processes.\footnote{The two editions that Harimoto noted,
    \cite{vulgate} and \cite{bhat-1889}, present identical texts.}


\subsection{Translation}

\begin{translation}
    \item[1] Now we shall explain the procedure\q{kalpa?} relating to the
knowledge concerning the venom in those who have been bitten by 
snakes.\footnote{\emph{Sarvāṅgasundarī} on 1.16.17 kalpa = prayoga.}
    
    \item[3] Suśruta, grasping his feet, questions the wise Dhanvantari, the 
    expert in all the sciences.
    
    \item[4]
    
    “My Lord, please speak about the number of snakes, and their divisions,
the symptoms of someone who has been bitten, and the knowledge
concerning the \se{vega}{waves}\q{?} of poisoning”.\footnote{The word
    “wave” translates \dev{vega}, which is other contexts may mean
    “(natural) urge.”  Here, it is rather the discrete stages or phases of
    physiological reaction to envenomation.}
        
    \item[5]
    
    On hearing his query, that distinguished physician spoke.
    
    “The venerable snakes such as Vāsukī and Takṣaka are uncountable. 
    
\item[6--9ab]

“They are snake-lords who support the earth, as bright as the ritual fire,
ceaselessly roaring, raining and scorching. They hold up the earth, with its
oceans, mountains and continents. If they are angered, they can destroy the
whole world with a breath and a look.  Honour to them. They have no role
here in medicine.

“The ones that I shall enumerate in due order are those mundane
ones with poison in their fangs who bite humans.\footnote{The next few
    verses are discussed in detail by \citet[101--104]{hari-2011}, who shows
    that in the taxonomy of snakes, the Nepalese version of the \SS\ has greater
    internal coherence than the vulgate recension.}

\item[9cd--10]    

“There are eighty kinds of snakes and they are divided in five ways:
Darvīkaras, Maṇḍalins, Rājimants, and Nirviṣās.  And Vaikarañjas that are
traditionally of three kinds.\footnote{\citet{hari-2011} translates these
    names as “hooded,” “spotted,” “striped,” “harmless,” and “hybrid.”}
\end{translation}
    %\newcomaand{\qlabelhook}{\framebox}
    \begin{figure}[t]
        \centering\small
        \Tree [.Snakes{ (80)}  
        [.Darvīkara {26 kinds} ]
        [.Maṇḍalin  {22 kinds} ]  
        [.Rājimant  {10 kinds} ]   
        [.Nirviṣa     {12 kinds} ]  
    [.Vaikarañja [.{3 kinds} {7 kinds} ] ]  ]
        
       \bigskip
          
            \Tree [.Snakes{ (80)}  
            [.Darvīkara {26 kinds} ]
            [.Maṇḍalin  {26 kinds} ]  
            [.Rājimant  {13 kinds} ]   
            [.Nirviṣa     {12 kinds} ]  
            [.Vaikarañja [.{3 kinds} ] ]  ]
        \caption{Top: the taxonomy of snakes in \Su{5.4.9--13ab}{571}. \\ Bottom: 
        the 
        taxonomy of snakes in the Nepalese version.}
        \label{snakes}
        \end{figure}
    
    \begin{translation}
    \item [11]
    “Twenty hooded snakes are known, six Maṇḍalins, and the same 
    number of Rājīmants. Niriviṣas and Karañjas are 
    
\end{translation}