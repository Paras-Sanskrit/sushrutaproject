% !TeX root = incremental_SS_Translation.tex
\section{Kalpasthāna, adhyāya 1}

\subsection{Literature}

A brief survey of this chapter's contents and a detailed assessment of the
existing research on it to 2002 was provided by Meulenbeld.\footcite[IA,
289--290]{meul-hist} Translations of this chapter since 2000 have appeared by 
\textcites[131--139]{wuja-2003}[3, 1--15]{shar-susr}{srik-2002}.\footnote{For a 
bibliography of translations to 2002, including Latin (1847), English (1877), Gujarati (1963) 
and Japanese (1971), see \cite[IB, 314--315]{meul-hist}.}

More recently, a discussion of the fourth chapter of this section in the light of
the Nepalese manuscripts was published by Harimoto.\footcite[101--104]{hari-2011}
After a close comparative reading of lists of poisonous snakes, Harimoto concluded
that, “the Nepalese version is internally consistent while the [vulgate] editions
are not.”  Harimoto showed how the vulgate editions, had been adjusted textually to smooth 
over inconsistencies, and gave
insights into these editorial processes.\footnote{The two editions
 that Harimoto noted, \cite{susr-trikamji3} and \cite{bhat-1889}, present identical
texts.}



\subsection{Manuscript notes}

\begin{itemize}
    \item \MScite{Kathmandu NAK 5-333} has foliation letter numerals, for example
on f.\,323a, that are similar to \MScite{Cambridge Add.\ 1693},\footnote{Scan
at 
\href{https://cudl.lib.cam.ac.uk/view/MS-ADD-01693/1}{cudl.lib.cam.ac.uk/view/MS-ADD-01693/1}.}
 dated to 1165\,\CE\, noted in Bendall's chart of Nepalese letter-numerals \cite[Lithograph V, 
after p.\,225]{bend-budd}
\end{itemize}

\newpage

\subsection{Translation}

\begin{translation}
 \item[1--2]  And now I shall explain the procedures for safeguarding food and
drink, as were declared by the Venerable Dhanvantari.\footnote{MS H adds in the
margin \dev{atha khalu vatsa suśrutaḥ} “Now begins Vatsa Suśruta.”  This phrase
has been copied here by the scribe from the beginning of the \SS\ chapter in the
\emph{sūtrasthāna} on the rules about food and drink (\Su{1.46.3}{214}).  The
scribe presumably felt, not unreasonably, that this section had common subject
matter with the present chapter.  Further, SS 1.46.3 is the only place in the Nepalese 
transmission of the \SS\ that names Dhanvantari and integrates him into the narrative of the 
\SS\ as the teacher of Suśruta. 
  
 The mention of Dhanvantari here is the only other time in the Nepalese
transmission that this authority is cited as the source of Ayurvedic teaching, and the unique 
occurrence of this actual phrase, “as was declared by the Venerable Dhanvantari.”
See the discussion by \citet[28--32]{kleb-2021b}, who concludes that the earliest
recoverable recension of the \SS\ may have had the phrase only at this point and
not elsewhere in the work.}
 
 \item[3] 

 Divodāsa, the king of the earth, was the foremost supporter of religious
discipline and virtue. With unblemished instruction he taught his students, of
whom Suśruta was the leader.\footnote{This is a quite different statement from
the vulgate \citep[559]{susr-trikamji3} that has Dhanvantari as the teacher, and
calls him the \se{kāśipati}{Lord of Kāśī}.  Ḍalhaṇa followed the vulgate but
explicitly noted the reading before us with small differences: \dev{divodāsaḥ
kṣitipatistapodharmaśrutākaraḥ} “Divodāsa, the king of the earth, was a mine of
traditions about discipline and virtue.”}

\subsection{[Threats to the king]}

\item[4--5]  

Evil-hearted enemies who have plucked up their courage, may seek to harm the king,
who knows nothing of it.  He may be assailed with poisons by or by his own people
who have been subverted, wishing to pour the poison of their anger into any
vulnerability they can find.\footnote{Verses about the use of Venemous Virgins as a weapon
do not appear in the Nepalese manuscripts. Cf.\ \cite[81\,f., 132]{wuja-2003}.  This material 
is present in the commentary of Gayadāsa.} 

\item[6] Therefore, a king should always be protected from poison by a physician.

%A king may be cunningly assailed with poisons by evil-hearted enemies who
%have plucked up their courage, or even by his own people turned traitor,
%wishing to pour the poison of their anger into any chink they can find. Or
%sometimes by women using various concoctions, hoping to make him love
%them.\footnote{On how women of ill-character mix their nail-clippings or
%menstrual blood, etc.\ with the king's food, see
%p.\,\pageref{dusyodara}.} Or again, if a Venomous Virgin is used, a man can
%lose his life instantly.\label{visakanya}
%% \footnote{\label{visakanya}On the `Venomous
%% Virgin', see p.\,\pageref{intro:visakanya}.}

\item [7] 

The racehorse-like fickleness of men's minds is well known. And for this reason, a
king should never trust anyone.\footnote{The verb \root śvas is conjugated as a
first class root in the Nepalese manuscripts.}

\item [8--11]

He should employ a doctor in his \se{mahānasa}{kitchen} who is respected by experts, who 
belongs to a good family, is orthodox, sympathetic, not emaciated, and always busy.

\item [12--13]

The kitchen should be constructed at a recommended location and orientation.  It should
have a lot of light,\footnote{We read \dev{mahacchuciḥ} with the Nepalese manuscripts and 
against the vulgate's \dev{mahacchuci}.  We understand \dev{śucis} as a neuter noun 
meaning “light” following \citet[1050a]{apte-prac}.} have clean utensils and be staffed by 
men 
and
women who have been vetted.\footnote{Verses detailing the ideal staff are omitted in the 
Nepalese manuscripts. 
Cf.\ \cites[560]{susr-trikamji3}[132]{wuja-2003}.}


\item[17--18ab]

The chefs, \se{voḍhāra}{bearers}, and makers of boiled rice soups and cakes and whoever
else might be there, must all be under the strict control of the
doctor.\footnote{The word \dev{saupodanaikapūpika} “chefs for the boiled rice soups
and cakes” is grammatically interesting.  The term \dev{sūpodana} (as opposed to
sūpaudana) is attested in the \emph{Bodhāyanīya\-gṛhyasūtra} 2.10.54 
\citep[68]{shas-1920}.  More pertinently, perhaps, \dev{sūpodana} is attested in
the Bower Manuscript, part II, leaf 11r, line 3 \citep[vol.\,1,
p.\,43]{hoer-bowe}.} 
% 2.11.54 supodana in the Bodh. (from Einoo's cards)
% sūpodana kṣīrodana
% Bower MS 328
% Kāty  otoṣthayoḥ samāse vā.

\item[18cd--19ab]

An expert  knows people's \se{iṅgita}{body language} 
through abnormalities
in voice, movement and facial expression. He should be able to identify 
a poisoner by the following signs.\q{Cf.\ Arthaśāstra 1.21.8.}


\item[19cd--23]

Wanting to speak, he gets confused, when asked a question, he never arrives at an
answer, and he talks a lot of confused nonsense, like a fool.  He laughs for no
reason, cracks his knuckles and scratches at the ground. He gets the shakes and
glances nervously from one person to another. His face is drained of colour, he is
\se{dhyāma}{grimy} and he cuts at things with his nails.\footnote{The word
\dev{dhyāma} is glossed by Ḍalhaṇa (in a variant reading) as someone who is the
colour of dirty clothes \Su{5.1}{560}.}  A poisoner goes the wrong way and is
absent-minded.

\item[25--27]

I shall explain the signs to look for in toothbrush twigs, in food and drink as
well as in \se{abhyaṅga}{massage oil} and \se{avalekhana}{combs}; in
\se{utsādana}{dry rubs} and showers, in \se{kaṣāya}{decoctions} and 
\se{anulepana}{massage ointment};
in \se{sraj}{garlands}, clothes, beds, armour and ornaments; in slippers and footstools, and
on the backs of elephants and horses; in \se{snuff}{nasya}, \se{dhūma}{inhaled
    smoke}, \se{añjana}{eye make-up}, etc., and any other things which are commonly 
    poisoned. Then, I shall also explain the remedy.

\item[28]

% My old Susruta.tex translation has \bird and \animal commands for making 
% indexes.  Convert them to the \se{}{} command that we're using in the 
% present document.
\newcommand\animal[4]{\se{#2}{#1}} 
\let\bird=\animal 

%28
Flies or crows or other creatures that eat 
a poisonous \se{bali}{morsel} served 
from the king's portion, die on the spot. 

\item [29] 

Such food makes a fire crackle violently, and gives it an overpowering colour like
a peacock's throat.

\item[30--33]

%Its flames sputter, it has acrid smoke, and before long it goes out. 

After a chukar partridge %\animal{chukar partridge}{cakora}{Alectoris
% chukar}{Collins 45}
looks at food which has poison mingled with it, its eyes are promptly drained of
colour; a peacock pheasant %\animal{peacock pheasant}{jīvajīvaka}{Polyplectron
% bicalcaratum}{Dave BSL 270, 273,
%274, 281}
drops dead.  A koel %\animal{koel}{kokila}{Eudynamys scolopacea}{Collins 66}
changes its song and the common crane %\animal{common crane}{kroñca}{Grus
% grus}{Collins 47}
rises up excitedly.\footnote{The verb \dev{arcchati} “rises up” is a rare form
best known from epic Sanskrit \citep[see][212, \S 7.6.1]{ober-2003}.   The
transmitted form \dev{kroñca} is obviously a colloquial version of Sanskrit
\dev{krauñca}.  Commenting on \Su{1.7.10}{31}, Ḍalhaṇa interestingly gives the
colloquial versions of several Sanskrit bird names, even singling out
pronunciation in the specific location of Kānyakubja.  For \dev{krauñca} he says
that people pronounce it \dev{kurañja} and \dev{koṃci}.  The form \dev{koñca}
is found in Pāli (see \cite[731]{cone-dict}, who notes that Ardhamāgadhī has the
same form). Elsewhere, Ḍalhaṇa calls the bird \dev{krauñcira},  \dev{krauñci}, and 
\dev{kaicara}
(\Su{1.46.105}{223}, \Su{6.31.154}{684} and
(\Su{6.58.44}{790} respectively).}  It will excite a peacock 
%\bird{peacock}{mayūra}, %{Pavo cristatus}{Collins 39}
and the terrified parakeet %\saneng{parakeet}{śuka}%{Psittacula krameri\slash
% eupatria\slash
%cyanocephala}{Collins 64}
and the hill myna %\ssaneng{hill myna}{sārikā}%{Acridotheres tristis tristis, L.,
% etc.}{Ali \#1006,
%\citet[28\,ff.]{Dave}, \citet[119]{Collins}}
screech. The swan %\animal{swan}{haṃsa}{?}{?}
trembles very much, and the racket-tailed drongo %\animal{racket-tailed
% drongo}{bhṛṅgarāja}{Dicrurus paradiseus}{Collins 123}
churrs.\footnote{Ḍalhaṇa seemed confused about the \sed{bhṛṅgarāja}{racket-tailed
drongo}.  He called it a generic \sed{bhramaraka}{drongo}, a word that can also mean 
“bee,” \citep[62]{dave}, and then said that it is like the
\sed{dhūmyāṭa}{black drongo} \citep[for a nice explanation of this name,
see][62--63]{dave} and that people call it “the king of birds.”} The chital deer
%\saneng{pṛṣaṭa}{chital}  
sheds tears and the
monkey releases excrement.\footnote{\MScite{Kathmandu KL 699} reads 
\sed{vṛṣabha}{bull} for
\sed{pṛṣata}{Chital deer}.  The latter may perhaps be mistaken for the former in
the Newa script, although the reading of \MScite{Kathmandu KL 699} is hard to 
read at this point.}

\item[34]

Vapour rising from tainted food gives rise to a pain in the heart,
it makes the eyes roll, and it gives one a headache.\footnote{ “Tainted” translates
\dev{upakṣipta}.  The word's semantic field includes “to hurl, throw against,” and
especially “to insult verbally, insinuate, accuse.”  The commentator Ḍalhaṇa
glossed the term as, “spoiled food given to be eaten” (\dev{vidūṣitasyānnasya
bhoktuṃ dattasya}), but he noted that some people read “\dev{ukhākṣipta}” or
“thrown into a pan.”  Other translators have commonly translated it as “served,” perhaps
influenced by Ḍalhaṇa's “\sed{datta}{given}.”}


\item[35, 36cd] 

In such a case, an errhine and a collyrium that are costus, \se{lāmajja}{lāmajja
    grass}, \se{nalada}{spikenard} and \se{madhus}{honey};\footnote{The vulgate 
    supplies another phrase and verb at this point that is not present in the Nepalese 
    transmission, but that makes the text flow more easily.}  a paste of sandalwood
on the heart may also provide relief.\footnote{\citet[350]{sing-1972} discussed
the difficulties in identifying \dev{lāmajja}, a plant cited more often in the
\SS\ than in the \CS; Ḍalhaṇa adopted the common view that it is a type of \emph{uśīra} or 
vetiver grass.  The grammatical neuter form 
\dev{madhus}  “sweetness” of the Nepalese
manuscripts is less common than neuter \dev{madhu} “honey, sweetness,
liquorice.”}

\item[37]

Held in the hand, it makes the hand burn, and the nails fall out. In such a case,
the \se{pralepa}{ointment} is \se{śyāmā}{beautyberry}, %{Callicarpa macrophylla,
% Vahl.}{AVS 1.334,    NK \#420},
\se{indragopa}{velvet-mite}, %{Kerria lacca
% (Kerr.)}{http://www.icar.org.in/ilri/de fault.htm},
soma and \se{utpala}{water-lily}.%
%{Nymphaea stellata, Willd.}{GJM 528, IGP 790; Dutt 110, NK \#1726}
\footnote{“Beautyberry” (\emph{Callicarpa macrophylla} Vahl.) is one
identification of \dev{śyāmā}, but vaidyas and commentators have different ideas
about the plant's identity (see \cites[410]{sing-1972}[1:
334]{avs}[\#420]{nadk-1954}).  On translating \dev{indragopa} as “velvet-mite,”
see \cite{lien-1978}. Ḍalhaṇa's remarks show that he had a reading
\dev{indrāgopā} before him, and he tries to explain \dev{indrā} and \dev{gopā} as
separate plants.  But he also says that some people read \dev{indragopa}.  Ḍalhaṇa
curiously parses the name \dev{somā} (f.) out of the compound; this feminine noun
is almost unknown to Ayurvedic literature.  Some dictionaries and commentators
consider it a synonym for \dev{guḍūcī}, others for \dev{brāhmī} or
\dev{candrataru}.  Ḍalhaṇa also mentions that some people think the word refers to
the \sed{somalatā}{soma creeper}, which might explain his choice to take the word as
feminine.  But the compounded word is far more likely to be \dev{soma} (m.), the
well-known mystery plant \citep[see][76--78, 125]{wuja-2003}.  If this can be
taken as rue (\emph{Ruta graveolens}, L.), as some assert, one can point to a
pleasing passage in Dioscorides where rue plays an antitoxic role: “\ldots it is a
counterpoison of serpents, the stinging of Scorpions, Bees, Hornets and Wasps; and
it is reported that if a man be anointed with the juice of the Rue, these will not
hurt him; and that the serpent is driven away at the smell thereof when it is
burned; insomuch that when the weasel is to fight with the serpent she armeth
herself by eating Rue, against the might of the serpent.” \parencites[cited 
from][262]{wren-1956}[not found in][]{osba-dios}.}
     
     \item [38--39] If he eats that food, through inattention or by mistake, then
his tongue will feel like a \se{aṣṭhīlā}{pebble} and it will lose its sense
of taste. It stings and %\sskt{stings}{tudyate},
burns, and his \se{śleṣman}{saliva}\label{saliva} dribbles out.\footnote{The word
\dev{aṣṭhīlā} is normally feminine.   The Nepalese manuscripts read it with a
short \dev{a-} ending.  Gayadāsa noticed that some manuscripts read 
\dev{aṣṭhīla}
with a short \dev{-a} ending (\MScite{Bikaner RORI 5157}, f.\,5v:7--8) and
Ḍalhaṇa reproduced his observation.  The vulgate reading “\sed{cāsyāt}{from his
mouth}” is more obvious (\emph{lectio facilior}), but is not attested in the
Nepalese manuscripts.} In such a case, he should apply the treatment prescribed
above for vapour, and what will be stated below under “toothbrush
twigs”.\footnote{Poisoned toothbrushes are discussed in verses 48\,ff.\ below.}
     
     \item[40]
     
     On reaching his stomach, it causes \se{mūrcchā}{stupor}, vomiting, the hair
stands on end, there is distension, a burning feeling and an impairment of
the senses.\footnote{I translate \dev{mūrcchā} in the light of the metaphors
discussed by \citet{meul-2011}, that include thickening and losing
consciousness.}

     \item[41] 
     
In this case, vomiting must quickly be induced using the fruits of
\se{madana}{emetic nut}, %{Randia dumetorum, Lamk.}{NK \#2091},
\se{alābu}{bitter gourd}, %{Lagenaria vulgaris, Seringe.}{NK \#1419},
\se{bimbī}{red gourd}, %{Coccinia indica, W. \& A.}{PVS 1994.4.715; NK 534}
and \se{koṣītakī}{luffa}, %{Luffa cylindrica, (L.) M. J. Roem.
% \textnormal{or}
% L. acutangula, (L.) Roxb.}{ADPS 252, NK \#1514 etc.}
taken with milk and \se{udaśvit}{watered buttermilk}, or alternatively with
rice-water.
     
     \item[42]
     
    
 Reaching the \se{pakvāśaya}{intestines}, it causes a burning feeling, stupor,
diarrhoea, thirst, impairment of the senses, \se{āṭopa}{flatulence} and it makes
him pallid and thin.
    
    % % % % % % % % % % % % % % % % % % % % %
    
      \item [43]
In such a case, purgation with the fruit of \se{nīlī}{indigo}, 
       %{Indigofera tinctoria, L.}{NK \#1309},
together with ghee, is best.  And  `\se{dūṣīviṣāri}{slow-acting poison antidote}'
should be drunk with honey and \se{dadhi}{curds}.\footnote{The `slow-acting
poison' is discussed at \Su{5.2.25\,ff.}{565}.}
     
     \item[44]
     
     When poison is in any liquid substances such as milk, wine or water, there are
     various streaks, and foam and bubbles form.  

     \item[45]
     
     Also, no reflections are visible or, however, if they can be seen, they are
distorted, fractured, or tenuous and not distorted.\footnote{Both Nepalese
witnesses read \se{vikṛta}{distorted} twice, which seems tautologous.  We
have read the sandhi as \se{vāvikṛtā}{or not distorted}. The scribe of
\MScite{Kathmandu NAK 5-333}, apparently the original hand, has added the
alternate reading in the margin “\se{yamalā}{double},” as in the vulgate.
Perhaps the scribe too was troubled by the possible tautology.  It is also
evidence that he may have had a witness with variant readings in it
similar to the vulgate. We retain the \emph{lectio difficilior}.} 
\q{Mention
    this in the introduction as an example of the scribe knowing the vulgate.}
     
\item[46]

Vegetables, soups, food and meat are soggy and tasteless.  They seem to go stale
suddenly, and they have no aroma.  
a
\item[47] 

All edibles lack aroma, colour or taste.  Ripe fruits
rapidly go bad, and unripe ones ripen.

\item[48]

When a toothbrush twig has poison on it, the bristles are damaged and the
flesh of the tongue, teeth and lips swells up.

\item[49]

 Then, once the swelling is 
 lanced, a \se{pratisāraṇa}{dressing} made with
 \se{dhātakī}{fire-flame bush}
 %{Woodfordia fruticosa (L.) Kurz}{AVS 5.412, NK \#2626}
 flowers, 
 %{Terminalia chebula Retz.}{NK \#2451}, 
 \se{jambū}{jambul},
 %{Syzygium cumini, (L.) Skeels}{ADPS 188, NK \#967, Potter 168}
 \se{āmra}{mango} stones and
 \se{harītakī}{chebulic myrobalan}
 fruit mixed with honey should be 
 applied.\footnote{This recipe is different from the vulgate.}
 
 \item[50] 

     
 
    \end{translation}

   
   

