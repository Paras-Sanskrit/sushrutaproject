\section{Sūtrasthāna, adhyāya 17}

\begin{translation}    
    
    \item [1] Now we shall expound the method for piercing the ear.\footnote{The topic of  \se{kaṛnavyadha}{piercing the ear} is not discussed in the \emph{Carakasaṃhitā} (\cite[IB, 326, n.\,175]{meul-hist}), but it is mentioned in some texts that followed the \emph{Suśrutasaṃhitā}, such as the \emph{Kaśāpyasaṃhitā} (\cite[IIA, 30]{meul-hist}). The instrument for piercing the ear is described in the \emph{Aṣṭāṅgahṛdayasūtra} (1.26.26). Both Ḍalhaṇa and Cakrapāṇi's versions (SS.1.16.1) describe the topic of this chapter as \se{karṇavyadhabandhavidhi}{the method of piercing and joining the ear}, instead of simply 'the method for piercing the ear.' Although it is omitted in the Nepalese's version's opening statement to this chapter, \se{karṇabandha}{joining the ear} is mentioned in passages 17–20. Also, the Nepalese version omits the opening remark on Dhanvantari recorded by both Cakrapāṇi and Ḍalhaṇa (1.16.2). For a discussion of the frame story in the Nepalese version, see \href{sushrutaproject.org/2021/07/11/dhanvantari2/}{SushrutaProject}, accessed July 26, 2021. Both commentators state that only the ears of healthy people, whose bodies are free from disease, should be pierced, and he quotes Bhoja to affirm this: 'When piercing the ears of children who are free of disease at these times, their ear flaps and apertures, as well as limbs, increase.'  (\emph{kāleṣv eteṣv adoṣāṇāṃ bālānāṃ karṇayor vyadhe | saha gātrair vivardhante karṇapālyaś ca khāni ca}~|| \emph{eteṣv adoṣāṇāṃ} ] \cite[16.2]{acar-1938} : \emph{eteṣu doṣāṇāṃ} \cite[16.2]{acar-1939}. \emph{vyadhe} ] \cite[16.2]{acar-1938} : \emph{vyadhaḥ} \cite[16.2]{acar-1939}).}
    \item [2] One may pierce a child's ears for the purpose of preserving and 
    decorating. On renowned days, half days, hours and constellations during the first half of the sixth or seventh lunar month, the boy who has \se{kṛtamaṅgala}{received a benediction}, -- \se{svastivācana}{blessings
        pronounced}\footnote{The
    syntax here is unclear. The expression \emph{svastivācana} may have been
    a gloss inserted into the text at an earlier period to clarify
    \emph{maṅgala}.  But as it stands, it is not syntactically connected to the rest of the sentence.  Both Cakrapāṇi and Ḍalhaṇa (1.16.3) record a reading in which the words are united in a compound that reads more naturally.} -- should be placed on the lap of a wet-nurse.\footnote{Cakrapāṇi and Ḍalhaṇa's versions (1.16.3) includes the option of placing the child in \se{kumāradharāṅka}{the lap of a man},  the gender of whom is made clear by  Ḍalhaṇa's gloss \se{bālagrāhipuruṣa}{'a man who holds the child'}. Both versions also add that the child should be enticed with \se{krīḍanaka}{toys}, which according to Ḍalhaṇa include \se{kṛtrimahastyaśvabalīvardaśukādi}{replica elephants, horses, bulls and parrots}. Ḍalhaṇa mentions that others read \se{bhakṣyaviśeṣair vā}{'or by special treats'} before this.} Then, while pacifying him and having pulled his ear with the left hand, the physician should use his right hand to pierce the ear straight through at a naturally occurring cleft.\footnote{Cakrapāṇi and Ḍalhaṇa's versions (1.16.3) add that this cleft is \se{ādityakarāvabhāsita}{illuminated by sunshine}.} For a boy do
    the right ear first; for a girl do the left one. Use a needle on a
    thin ear; an \se{ārā}{awl} on a thick one.\footnote{Ḍalhaṇa (1.16.3) clarifies that the awl is a shoe-maker's knife for \se{carma\-bhedana}{piercing leather}.}
    
    \item [3]  If there is excess blood or pain one should know that it was pierced
    in the wrong place. The absence of side-effects is a sign that it has been pierced 
    in the right place.\footnote{At this point, manuscript KL-699 is missing a folio, so the rest of this chapter
    is constructed on the basis of witnesses N and H.}
    
    \item [4] In this context, if an ignorant person accidentally pierces a 
    \se{sirā}{duct} there will 
    be fever, burning, \se{śvayathu}{swelling}, pain, \se{granthi}{lumps}, 
    \se{manyāstambhā}{paralysis 
        of the nape of the neck}, 
    \se{apatānaka}{convulsions}, headache or sharp pain in the ear.\footnote{This passage is significantly augmented in both Cakrapāṇi (1.16.4) and Ḍalhaṇa's (1.16.5) versions to outline the specific problems caused by piercing three ducts called \emph{kālikā}, \emph{marmikā} and \emph{lohitikā}. In fact, the order of the problems mentioned in the Nepalese version has been retained in the other versions and divided between each duct. Cakrapāṇi's commentary cites several verses attributed to Bhoja on the problems caused by piercing these three ducts in the ear flap: '\emph{Lohitikā}, \emph{marmikā} and the black ones are the ducts situated in the earflaps.  Listen in due order to the problems that arise when they are pierced. Paralysis of the nape of the neck and convulsions, or sharp pain arise from piercing \emph{lohitikā}. Pain and lumps are thought to arise from piercing \emph{marmikā}. Piercing \emph{kālikā} gives rise to swelling, fever and burning' (\emph{lohitā marmarī kṛṣṇāḥ karṇapāliśritāḥ sirāḥ} | \emph{tāsāṃ tu vyadhane doṣān anupūrveṇa me śrṇu} || \emph{manyāstambho 'patānaś ca śūlo vā lohitāvyadhāt} | \emph{vedanā granthayaś caiva marmarīvyadhanāt smṛtāḥ} || \emph{kālikāvyadhanāc chotho jvaro dāhaś ca jāyate}).}
    
    \item[5]     Having removed the \se{varti}{wick} in the hole because of the aggravation of humours or a culpable piercing,\footnote{In addition to these reasons, Cakrapāṇi (1.16.5) and Ḍalhaṇa's versions (1.16.6) add \se{kliṣṭajihmāpraśastasūcīvyadha}{'piercing with a painful, crooked and unrecommended needle'} and \se{gāḍhataravartitva}{'a wick that is too thick'}. Ḍalhaṇa was aware of the reading in the Nepalese version because he notes that some read \se{doṣasamudāyāt}{'because of the aggravation of humours'} rather than \emph{kliṣṭajihmāpraśastasūcīvyadhād gāḍhatara\-vartitvād}.} one should smear it with
    a paste of the roots of 
    barley, 
    liquorice, 
    \se{mañjiṣṭhā}{Indian madder}, and the
    \se{gandharvahasta}{castor oil tree},
    thickened with honey and ghee. When it has healed well, one should pierce it again.
    
    \item[6] One should treat the properly-pierced ear by sprinkling it with raw sesame
    oil.   After every three days one should apply a thicker \se{varti}{wick} and
    sprinkle oil right on it.\footnote{The manuscripts support the reading
    \emph{sthūlatarīṃ} that is either a non-standard form or a scribal error.}
    
    \item[7]
    Once the ear is free from humours or side-effects, one should 
    loosen it with a light \se{pravardhanaka}{dilator} in order to enlarge it.\footnote{Cakrapāṇi (1.16.6) and Ḍalhaṇa (1.16.8) point out that the dilator can be made of wood, such as that of the \se{apāmarga}{Prickly Chaff Flower}, the \se{nimba}{Neem tree} and the \se{kārpāsa}{Cotton Plant}. Ḍalhaṇa adds that it can also be made of \se{sīsaka}{lead} and should have the shape of the \se{dhattūrapuṣpa}{datura flower}.}
    
    \item[8]
    
    \begin{sloka}

A person's ear enlarged in this way can split in two, either as a result of the humours\footnote{Ḍalhaṇa (1.16.9) notes that the word \emph{doṣa} here can refer to either a humour, such as \se{vāta}{wind}, as we have understood it, or a disease generated from a humour.} or a blow. Listen to me about the \se{sandhāna}{joins} it can have.
        
    \end{sloka}
    
        \item[9]
    
Here, there are, in brief, fifteen ways of mending the ear flap.\footnote{The Nepalese version uses the word \emph{sandhāna} to refer to joining a split in an ear flap, which is consistent with the terminology in the verse cited above (8). However, the vulgate uses the term \emph{bandha} here and at the very beginning of the chapter (1.16.1) to introduce the topic of repairing the ear.}  They are as follows:
    \se{nemīsandhānakaḥ}{Rim-join}, \se{utpalabhedyaka}{Lotus-splittable}, \se{vallūraka}{Dried Flesh}, \se{āsaṅgima}{Fastening}, \se{gaṇḍakarṇa}{Cheek-ear}, \se{āhārya}{Take away}, \se{nirvedhima}{Ready-Split}, \se{vyāyojima}{Multi-joins}, \se{kapāṭasandhika}{Door-hinge}, \se{ardhakapāṭasandhika}{Half door-hinge}, 
    \se{saṃkṣipta}{Compressed}, \se{hīnakarṇa}{Reduced-ear},
    \se{vallīkarṇa}{Creeper-ear}, \se{yaṣṭīkarṇa}{Stick-ear}, and \se{kākauṣṭha}{Crow's lip}.\footnote{For an artist's impression of these different kinds of joins in the ear flap, see Figure 3.2 in \cites[136]{wuja-1998}.}
    
    In this context, among these, 
    \begin{description}
        
        \item[\mdseries``Rim-join'' (\emph{nemīsandhānaka}):]
        both flaps are wide, long, and equal.
        
        \item[\mdseries``Lotus-splittable'' (\emph{utpalabhedyaka}):]
        both flaps are round, long, and equal.
        
        \item[\mdseries``Dried flesh'' (\emph{vallūraka}):]
        both flaps are short, round, and equal.
        
        \item[\mdseries``Fastening'' (\emph{āsaṅgima}):]
        one flap is longer on the inside.
        
        \item[\mdseries``Cheek-ear'' (\emph{gaṇḍakarṇa}):]
        one flap is longer on the outside.\footnote{For an artist's impression of this join, see Figure 3.3 in \cites[137]{wuja-1998}.}
        
        \item[\mdseries``Take-away'' (\emph{āhārya}):]
        the flaps are missing, in fact, on both sides.
        
        \item[\mdseries``Ready-split'' (\emph{nirvedhima}):]
        the flaps are like a \se{pīṭha}{dais}.
        
        \item[\mdseries``Multi-joins'' (\emph{vyāyojima}):]
        one flap is small, the other thick, one flap is equal, the other unequal.
        
        \item[\mdseries``Door-hinge'' (\emph{kapāṭasandhika}):]
        the flap on the inside is long, the other is small.
        
        \item[\mdseries``Half door-hinge'' (\emph{ardhakapāṭasandhika}):]
        the flap on the outside is long, the other is small.
    \end{description}
    
    `These ten \se{vikalpa}{options} for \se{sandhi}{joins} of the ear should be
    bound.  They can mostly be explained as resembling their names.\footnote{Cakrapāṇi (1.16.1–13) and Ḍalhaṇa (1.16.10) provide examples of how the names of these joins describe their shapes. For example, the \se{nemīsandhānaka}{rim-join} is similar to the join of the \se{cakradhārā}{rim of a wheel}.}  The five from \se{saṃkṣipta}{compressed} on are incurable.\footnote{Ḍalhaṇa mentions that some do not read the statement that only five are incurable, and they understand the causes of unsuccessful joins given below (i.e., heat, inflammation, suppuration and swelling) as also pertaining to the first ten when they do heal.}  Among these, “compressed” has a dry ear canal and the other flap is small.   “Reduced ear” has 
    flaps that have no base and have wasted flesh on their edges. “Creeper-ear” has 
    flaps that are thin and uneven. “Stick-ear” has \se{granthita}{lumpy} flesh and the 
    flaps are stretched thin and have \se{stabdha}{stiff} \se{sirā}{ducts}.  “Crow-lip” 
    has a flap 
    without flesh with \se{saṃkṣipta}{compressed} tips and little blood. Even when 
    they are bound up, they do not heal because they are hot, inflamed, 
    \se{srāva}{suppurating}, or swollen.\footnote{The vulgate (SS.1.16.11–12) has four verses (\emph{śloka}) at this point that are not in the Nepalese manuscripts. The additional verses iterate the types of joins required for ear flaps that are missing, elongated, thick, wide, etc. All four verses were probably absent in the version of the \emph{Suśrutasaṃhitā} known to Cakrapāṇi. He cites the verses separately in his commentary, the \emph{Bhānumatī}, introducing each one as 'some people read' (\emph{ke cit paṭhanti}). The root text in Ācārya's edition of the \emph{Bhānumatī} (\cite[1.16]{acar-1939}) appears to be identical to the one commented on by Ḍalhaṇa (\cite[1.16]{acar-1938}).}
    
    \item[10]  
    
    % 15
    A person wishing to perform any of these joins should therefore gather together the
    supplies prepared according to the recommendations of the `Preparatory
    Supplies' chapter.\footnote{SS.1.5.}  And in particular, he should gather
    \se{surāmaṇḍa}{decanted liquor}, milk, water,
    \se{dhānyāmla}{fermented rice-water}, and \se{kapālacūrṇa}{powdered 
        earthenware crockery}.  
    
    Next, he should prepare the woman or man, who have had the ends of their hair tied 
    up, have eaten lightly, and are firmly supported by qualified 
    attendants.
    
    Then, he should ready the \se{bandha}{bindings} and carry out the procedure with
    \se{chedya}{cutting}, \se{bhedya}{splitting}, \se{lekhya}{scarification}, or
    \se{vyadhana}{piercing}. Then, he should examine the blood of the ear to know whether it is 
    \se{duṣṭa}{tainted} or not. If it is tainted by wind, the ear should be
    bathed with \se{dhānyāmla}{fermented rice-water} and water; if tainted by choler, 
    then cold water
    and milk should be used; if tainted by phlegm, then \se{surāmaṇḍa}{decanted 
        liquor} and water
    should be used, and then he should scarify it again.
    
    
    Then, arranging the join in the ear so that it is neither proud, depressed, nor
    uneven, one should make the join. Having seen that the bloood has stopped, one should anoint it with honey and ghee,
    bandage each ear with \se{picu}{cotton} and \se{prota}{gauze}, and
    bind it up with a thread, neither too tightly nor too loosely.  Then, the earthenware
    powder should be sprinkled on, and \se{ācārika}{medical advice} given.
    And he should supplement with food as taught in  the `Two Wound'
    chapter.\footnote{SS.4.1.}
    
    \item[11]
    \begin{sloka}
        One should avoid rubbing, sleeping during the day, exercise, overeating,
        sex, getting hot by a fire, or the effort of speaking.
    \end{sloka}
    
    \item[12]
    
    % 17
    One should not make a join when the blood is too pure, too copious, or too
    thin.\footnote{The vulgate reads “impure” for the Nepalese “too pure,” which would
    appear to make better medical sense.  Emending the text to \emph{nāśuddha-} for
    \emph{nātiśuddha-} in the Nepalese recension would yield the same meaning as the
    vulgate.} For when the ear is tainted by wind, then it is
    \se{raktabaddha}{obstructed by blood}, unhealed and will peel. When tainted with
    choler, is becomes \se{gāḍha}{pinched}, \se{pāka}{septic} and red.  When tainted
    by phlegm, it will be \se{stabdha}{stiff} and itchy.  It has excessively copious
    \se{srāva}{suppuration} and is \se{puffed up}{śopha}.  It has it has a small
    amount of \se{kṣīṇa}{wasted} flesh and it will not grow.
    
    \item[13] When the ear is properly healed and there are no complications,  one may
    very gradually start to expand it.  Otherwise, it may be \se{saṃrambha}{angry},
    burning, septic or painful.  It may even be split open again.
    
    
    \item [14]
    
    
    Now, massage for the healthy ear, in order to enlarge it. 
    
    \newcommand{\animal}[4]{#1 (\emph{#2}\footnote{#3 (#4)})}
    \newcommand{\plant}[4]{#1 (\emph{#2}\footnote{#3 (#4)})}
    
    One should gather as much as one can of the following: a
    \animal{monitor lizard}{godhā}{Varanus bengalensis, Schneider}{Daniel
        1983:58},
    \se{pratuda}{scavenging} and \se{viṣkira}{seed-eating} birds, and
    creatures that live in marshes or water,\footnote{For such classifications,
    see \citet{zimm-jung} and \citet{smit-clas}.} fat, marrow, milk, and sesame oil, and
    white mustard oil.  
    % note: think more about the compound structure here.
    Then cook the oil with 
    \newcommand\skt[2]{#1 (#2)}
    an
    \skt{admixture}{prativāpa} of the following:
    \plant{purple calotropis}{arka}{Calotropis gigantea, (L.) R. Br.}{ADPS 52, AVS
        1.341, NK \#427, Potter 57, ID 306},
    \plant{white calotropis}{alarka}{Calotropis procera, (Ait.) R. Br.}{NK
        \#428, GIMP 46b, ID 306},
    \plant{country mallow}{balā}{Sida cordifolia, L.}{ADPS 71, NK \#2297},
    \plant{`strong Indian mallow'}{atibalā}{Abutilon indicum, (L.) Sweet; Sida
        rhombifolia, L.?}{NK \#11, IGP ,4 1080; NK \#2300},
    %\plant{Indian sarsaparilla}{anantā}{Hemidesmus indicus, (L.) R.
    %  Br. \textnormal{and} Cryptolepis buchanani, Roemer \&
    %  Schultes}{ADPS 434, AVS 3.141, NK \#1210},
    %the \skt{Indian sarsaparillas}{sārive}
    \plant{country sarsaparilla}{anantā}{Hemidesmus indicus, (L.) R. Br.}{ADPS 434,
        AVS 3.141--5, NK \#1210}
    %and
    %\plant{black creeper}{pālindī}{Ichnocarpus frutescens, (L.)
    %    R.Br. \textnormal{or} Cryptolepis buchanani, Roemer \&
    %    Schultes}{AVS 3.141, 3.145, 3.203, NK \#1283, \#1210, ADPS
    %    434}),
    %
    %%\plant{prickly chaff-flower}{apāmārga}{Achyranthes aspera,
    %%    L.}{GJM 524f., IMP 1.39, ADPS 44f., IMP 3.2066f., Dymock 3.135},
    %\plant{Withania}{aśvagandhā}{Withania somnifera (L.) Dunal}{IMP
    %    5.409f., Dymock 2.566f., Chevallier 150.},
    \plant{beggarweed}{vidāri}{Desmodium
        gangeticum (L.) DC}{Dymock 1.428, GJM 602, cf.\ NK
        \#1192; ADPS 382, 414 and IMP 2.319, 4.366 are confusing},
    %\plant{giant potato}{kṣīraśukla  $\rightarrow$ kṣīravidārī}{Ipmoea mauritiana,
    %    Jacq.}{ADPS 510, AVS 3.222, IMP 3.1717ff.},
    liquorice (\emph{madhuka}),
    \plant{hornwort}{jalaśūka $\rightarrow$ jalanīlikā}{Ceratophyllum
        demersum, L.}{IMP 2371, AVS 2.56, IGP 232},\footnote{This name is not
    certain: in fact, the commentator Ḍalhaṇa notes that some people interpret
    it as a poisonous, hairy, air-breathing, underwater creature.}
    \skt{items having the `sweet' savour}{madhuravarga},\footnote{The
    items which exemplify the `sweet' savour \label{kakolyadi}
    (\emph{madhuravarga}) are enumerated at SS.1.42.11.} and
    \plant{`milk flower'}{payasyā  $\rightarrow$ vidārī}{Pueraria tuberosa (Willd.) 
        DC.}{ADPS
        510, IMP 1.792f., AVS 4.391; not Dymock 1.424f. See GJM supplement 444,
        451, IMP 1.187, but IMP 3.1719 = Ipmoea mauritiana, Jacq.}.
    %
    This should then be deposited in a well-protected spot.
    
    \item[15]% 20
    \begin{sloka}
        
        The wise man who been sweated should rub the \se{mardita}{massaged} ear with 
        it. 
        Then it will be free of complications, and will enlarge properly and be strong.
    \end{sloka}
    
    \item[16]
    % 22cd-23
    Ears which do not enlarge even when sweated and oiled, 
    should be scarified
    at the \se{apāṅga}{edge of the hole}, but not outside it.  
    
    
    \item[17]
    In this tradition, experts know countless repairs to ears.  So a 
    physician who is \se{suniviṣṭa}{very intent} on working in this way 
    \se{yojayed}{may repair} them.
    
    \item[18]
    % 25
    If an ear has grown hair, has a nice hole, a firm join, and is strong and
    even, well-healed, and free from pain, then one can enlarge it slowly.
    
    
    \item[19]
    
    %28, 29.
    Now I shall describe the proper method of repairing a severed nose.
    First, take from the trees a leaf the same size as the man's nose and hang it
    on him. 
    
    \item[20] Next, having cut a \se{vadhra}{slice of flesh} with the same
    measurements off the cheek, the end of the nose is then scarified.\footnote{The
    vulgate reads \se{baddham}{bound, connected} for \se{vadhra}{slice of flesh}.
    This is a critical variant from the surgical point of view.  If the slice remains
    connected, it will have a continuing blood supply.  This is one of the effective 
    techniques that so astonished surgeons witnessing a similar operation in Pune in
    the eighteenth century \citep[see][67--70]{wuja-roots3}.}
    
    Then the \se{apramatta}{diligent} physician, 
    should quickly \se{pratisandhā-}{put it back together} so that it is 
    \se{sādhubaddha}{well joined}.
    
    
    \footnote{Or
    `\ldots\ off the cheek, it is fixed to the end of the nose, which has been
    scarified'. The Sanskrit text is unfortunately not unambiguous on the
    important point of whether or not the flap of grafted skin remains connected
    to its original site on the cheek.} 
    
    % 30.
    Having carefully observed that it has been well sown up,
    two tubes should be fixed in place.  Then, having lifted them up,\footnote{The 
    Sanskrit here, \emph{unnāmayitvā} is  non-Pāṇinian.}
    the powder of
    \plant{sappanwood}{pattāṅga}{Caesalpinia sappan, L.}{AVS 1.323, IMP 
        2.847f.},\footnote{For {pattāṅga} there are manuscript variants 
    \emph{pattrāṅga} (MS H) and \emph{pattaṅga}    (N).  We read with H and K 
    (f.\,14r:1) on \citet[1.14.36]{vulgate}. The vulgate reads \emph{pataṅga} 
    and this reading is propagated in modern dictionaries.}
    \plant{liquorice}{yaṣṭīmadhuka}{Glycyrrhiza glabra, L.}{AVS 3.84, NK \#1136},
    and
    \plant{Indian barberry}{añjana}{Berberis aristata, DC.}{Dymock 1.65, NK
        \#685, GJM 562, IGP 141}\q{añjana}
    should be applied to it.
    
    \item[22] 
    The wound should be covered properly with \skt{cotton}{picu} and should be
    moistened repeatedly with sesame oil.  Ghee should be given to the man to
    drink.  His digestion being complete, he should be oiled and purged in
    accordance with the instructions specific to him.\footnote{The expression 
    \emph{svayathopadeśa} is ungrammatical but supported in all available 
    witnesses.}   
    
    \item[23] %32.
    And once healed and really come together, what is left of its \se{vadhra}{flesh}
    should then be trimmed. If it is  \se{hīna}{reduced}, however, one should make an
    effort to stretch it , and one should make its overgrown flesh smooth.
    
    
\end{translation}    
