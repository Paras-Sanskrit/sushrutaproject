% !TeX root = incremental_SS_Translation.tex

\section{Sūtrasthāna, adhyāya 13:  On Leeches}


\subsection{Literature} 

\subsubsection{Previous scholarship}

Meulenbeld offered an annotated
overview of this chapter and a bibliography of studies on Indian leeches and their 
application.\footcite[IA, 209; IB, 324, n.\,131]{meul-hist}

A Persian version of this chapter of the \SS\ was included in \emph{Sikandar
    Shāh's Mine of Medicine} (\emph{Ma`din al-\underbar{sh}ifā' i
    Sikandar-\underbar{Sh}āhī}) composed in 1512 by Miyān Bhūwah b.
\underline{Kh}awāṣṣ \underline{Kh}ān.\footcites[96--109]{sidd-1959}
{azee-1971} [231--232]{stor-1971} [IB, 324,
n.\,128]{meul-hist}[8--9]{spez-2019}

More recently, Brooks has explored the sense of touch in relation to
leeching and patient-physician
interactions.\footcite{%
    broo-2020,
    broo-2020b,
    %broo-2018,
    broo-2020c}

\subsection{Translation}

\begin{translation}    
\item [1] 
    And now we shall explain \diff{the chapter} about leeches.
    
\item [3] The leech is for the benefit of kings, rich people, delicate people,
children, the elderly, fearful people and women.  It is said to be the most
gentle means for letting blood.

\item [4]

In that context, one should let blood that is corrupted by wind, bile or
phlegm with a horn, a leech, or a \gls{alābu}, respectively.   Or, each kind
can be be made to flow by any of them in their particular way.\footnote{This
    sentence is hard to construe grammatically, although its meaning seems
    clear. In place of \dev{viśeṣastu}, Cakrapāṇidatta and Ḍalhaṇa both read
    \dev{viśeṣatas}, which helps interpretation (\cite[95]{acar-1939},
    \cite[55]{vulgate}). It is notworthy that the critical syllable \dev{stu} is
    smudged or corrected in both \MScite{Kathmandu NAK 1-1079} and in 1-1146, a
    much later Devanāgarī manuscript.\MSsilent{Kathmandu NAK 1-1146}
      
There is an insertion in the text, printed in parentheses in the
vulgate at \Su{1.13.4}{55} as  \dev{viśeṣatastu visrāvyaṃ
śṛṅgajalaukālābubhirgṛhṇīyāt}.  This insertion is not included in the
earlier edition of the vulgate, but is replaced by
\dev{snigdhaśītarūkṣatvāt} \citep[54]{susr-trikamji2}. Ḍalhaṇa noted that,
“this reading is discussed to some extent by some compilers
(\dev{nibandhakāra}), but it is definitely rejected by most of them,
including Jejjhaṭa.” }

\item[1.13.5x]  And there are the following about this:

\item [1.13.5]

The horn of cows is praised for being unctuous, \diff{smooth}, and very
sweet.  Therefore, when wind is troubled, that is good for
bloodletting.\footnote{The vulgate replaced “smooth” with “hot.”}

\item [1.13.5a]

Having a length of seven fingers and a large body the shape of a half moon, should 
first be placed into a cut.  A strong person should suck with the mouth. 

\footnote{This passage is not found in the vulgate, but it is similar to the passage 
cited by Ḍalhaṇa at \Su{1.13.8}{56} and attributed to Bhāluki.  Ācārya was a ware of 
this reading in the Nepalese manuscripts; see his note 4 on \Su{1.13.5}{55, note 4}.}

\item[1.13.6]

A leech lives in the cold, is sweet and is born in the water. So when
someone is afflicted by bile, they are suitable for
bloodletting.\footnote{Note that the particular qualities (\emph{guṇa}s) of
    the leech in this and the following verses counteract the quality of the
    affliction.  See \cite[113, table 1]{broo-2018}.}

\item[1.13.7]

A \gls{alābu} is well known for being pungent, dry and sharp.  So
when someone is afficted by phlegm it is suitable for bloodletting.

\item[1.13.8]

In that context, at the scarified location one should let blood using a
horn wrapped in a covering of a thin bladder.  Or with a \gls{alābu} with a
flame inside it because of the suction.\footnote{There are questions about
    the covering of the horn.  Other versions of the text, and the commentator,
    propose that there may be two coverings, or that cloth may be a constituent.
    Comparison with contemporary horn-bloodletting practice by traditional
    Sudanese healers suggests that a covering over the top hole in the horn is
    desirable when sucking, to prevent the patient's blood entering the mouth
    \citep{pbs-2020}.  Our understanding of this verse is that the bladder
    material is used to cover the mouthpiece and to block it to preserve suction in
    the horn for a few minutes while the blood is let. }

\item[1.13.9]

Leeches are called “\emph{jala-ayuka}” because \se{jala}{water} is their 
\se{āyur}{life}.





    
\end{translation}



% % % % % % % % % % % % % % % % % % % % % % % SS 1.28
