% !TeX root = incremental_SS_Translation.tex

\section{Sūtrasthāna, adhyāya 13:  On Leeches}


\subsection{Literature} 

\citet[IA, 209]{meul-hist} offered an annotated
overview of this chapter.  See also 
\cites{broo-2020,broo-2020b,broo-2018,broo-2020c}

\subsection{Translation}

\begin{translation}    
\item [1] 
    And now we shall explain \diff{the chapter} about leeches.
    
\item [3] The leech is for the benefit of kings, rich people, delicate people,
children, the elderly, fearful people and women.  It is said to be the most
gentle means for letting blood.

\item [4]

In that context, one should let blood that is corrupted by wind, bile or
phlegm with a horn, a leech, or a \gls{alābu}, respectively.   Or, each kind
can be be made to flow by any of them in their particular way.\footnote{This
    sentence is hard to construe grammatically, although its meaning seems
    clear. In place of \dev{viśeṣastu}, Cakrapāṇidatta and Ḍalhaṇa both read
    \dev{viśeṣatas}, which helps interpretation (\cite[95]{acar-1939},
    \cite[55]{vulgate}). It is notworthy that the critical syllable \dev{stu} is
    smudged or corrected in both \MScite{Kathmandu NAK 1-1079} and in 1-1146, a
    much later Devanāgarī manuscript.\MSsilent{Kathmandu NAK 1-1146}
      
There is an insertion in the text, printed in parentheses in the
vulgate at \Su{1.13.4}{55} as  \dev{viśeṣatastu visrāvyaṃ
śṛṅgajalaukālābubhirgṛhṇīyāt}.  This insertion is not included in the
earlier edition of the vulgate, but is replaced by
\dev{snigdhaśītarūkṣatvāt} \citep[54]{susr-trikamji2}. Ḍalhaṇa noted that,
“this reading is discussed to some extent by some compilers
(\dev{nibandhakāra}), but it is definitely rejected by most of them,
including Jejjhaṭa.” }

\item[1.13.5x]  And there are the following about this:

\item [1.13.5]

The horn of cows is praised for being unctuous, \diff{smooth}, and very
sweet.  Therefore, when wind is troubled, that is good for
bloodletting.\footnote{The vulgate replaced “smooth” with “hot.”}

\item [1.13.5ef]




    
\end{translation}



% % % % % % % % % % % % % % % % % % % % % % % SS 1.28
