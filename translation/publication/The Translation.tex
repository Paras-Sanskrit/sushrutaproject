% !TeX root = surgery.tex

% turn off the footnotes, for the conference handout
\renewcommand\footnote[1]{\relax }



%INTRODUCTION
%A. Preliminaries
%1. Aim of the Article
%2. Importance of 1.16 in the History of Āyurveda 
%B. Text
%1. The Nepalese Version
%2. Vulgate
%3. Differences between the Two (as exemplified by 1.16)
%C. Edition 
%1. Manuscripts
%2. Editorial Principles
%EDITION OF 1.16 

  \newcommand{\animal}[4]{#1 (\emph{#2}%\footnote{#3 (#4)})
}
\newcommand{\plant}[4]{#1 (\emph{#2}%\footnote{#3 (#4)})
}
\newcommand\skt[2]{#1 (#2)}


 \section{Translation of Sūtrasthāna 16}
%\subsection{Sūtrasthāna, adhyāya 16}

\begin{translation}    
    
    
    \item [1] Now we shall expound the method for piercing the ear.\footnote{The
    topic of  \se{kaṛnavyadha}{piercing the ear} is not discussed in the
    \emph{Carakasaṃhitā} (\cite[IB, 326, n.\,175]{meul-hist}), but it is mentioned
    in some texts that followed the \emph{Suśrutasaṃhitā}, such as the
    \emph{Kaśāpyasaṃhitā} \citep[IIA, 30]{meul-hist}. Also, the instrument for
    piercing the ear is described in the \emph{Aṣṭāṅgahṛdayasaṃhitā}
    \Ah{1.26.26}{321}. In the versions of the text known to Ḍalhaṇa
    \citep[76]{vulgate} and Cakrapāṇidatta \citep[125]{acar-1939}, the heading of
    this chapter is “the method of piercing and joining the ear”
    (\emph{karṇavyadhabandhavidhi}), instead of the Nepalese version's “the method
    of piercing the ear” (\emph{karṇavyadhavidhi}). The topic of joining the ear
    (\emph{karṇabandha}) is discussed in passages 17--20 of the Nepalese version.
    However, it appears that only subsequent redactors reflected its importance by
    including it in chapter headings.

 The Nepalese version also omits the opening remark on Dhanvantari that appears in
subsequent versions of the text. For a discussion of the frame story in the
Nepalese version, see \cite{birc-2021}. Ḍalhaṇa \citep[76]{vulgate} and
Cakrapāṇidatta \citep[125]{acar-1939} state that only the ears of healthy people
should be pierced, and they quote the lost authority Bhoja to affirm this: “When
piercing the ears of children who are free of disease at these times, their ear
flaps and apertures, as well as limbs, increase” (for the Sanskrit, see
\cite[76]{vulgate}).}

\item [2] One may pierce a child's ears for the purpose of preserving and
decorating. On renowned days, half days, hours and constellations during the first
half of the sixth or seventh lunar month, the physician sits the boy, who has
received a benediction and the recitation of a blessing,\footnote{The causative
    form \emph{vy\u adhayet} is known in Classical Sanskrit \citep[166]{whit-root}.

The compound \emph{kṛtamaṅgalasvastivācanaṃ} “who has received a benediction and
the recitation of a blessing” is an emendation based on the similar text at
\Su{3.2.25}{346}.  Cf.\ also \Su{3.10.8, 24}{388, 390} that have slightly
different formulations.} --, on the lap of a wet-nurse and pacifies
him.\footnote{The versions of 1.16.3 known to Cakrapāṇidatta
    \citep[126]{acar-1939} and Ḍalhaṇa \citep[76]{vulgate} have the additional
    compound \emph{kumāradharāṅke} (“on the lap of one who holds the child”) after
    \emph{dhātryaṅke}. The gender of \emph{kumāradhara} is made clear by  Ḍalhaṇa's
    gloss “a man who holds the child.” Also, both versions add \emph{bālakrīḍanakaiḥ
    pralobhya} (“having enticed with children's toys”) to indicate that the child
    should be tempted with toys to stay on the assistant's lap. According to Ḍalhaṇa
    on \Su{1.16.3}{76}, the toys include replica elephants, horses, bulls and parrots.
    Ḍalhaṇa further mentions that others read \emph{bhakṣyaviśeṣair vā} (“or by
    special treats”) before \emph{bālakrīḍanakaiḥ}, but we see no trace of these small
    kindnesses in our witnesses.} Then, he should pierce the ears, pulling the ear
    with his left hand  and piercing straighth through with his right hand at a
    naturally-occurring cleft.\footnote{The versions of 1.16.3 of Cakrapāṇidatta
        \citep[126]{acar-1939} and Ḍalhaṇa \citep[76]{vulgate} add that this naturally
        occurring cleft is illuminated by sunshine  (\emph{ādityakarāvabhāsite}).

The syntax of this slightly long sentence is unusual in beginning with the dual
object \emph{tau} “the two (ears)” at the start of the sentence, which is remote
from the main verb.  The other singular accusatives referring to the ear being
pierced are governed by absolutives.} For a boy, do the right ear first; for a
girl, do the left one. Use a needle on a thin ear; an \se{ārā}{awl} on a thick
one.\footnote{Ḍalhaṇa on 1.16.3 \citep[76]{vulgate} clarifies that the awl is a
    shoe-maker's knife for piercing leather.    He also cites the authority of “the
    notes of Lakṣmaṇa” (\emph{Lakṣmaṇaṭippaṇaka}) on the issue of the thickness of the
    needle. \textit{The Notes of Lakṣmaṇa} is not known from any earlier or
    contemporary sources and was presumably a collection of glosses on the \SS\ that
    was available to Ḍalhaṇa in twelfth-century Bengal. See \citet[IA,
    386]{meul-hist}.}
    
    \item [3]  One may know that it was pierced in the wrong place if there is
excess blood or pain. The absence of side-effects is a sign that it has been
pierced in the right place.\footnote{At this point, \MScite{Kathmandu KL 699}
    is missing a folio, so the rest of this chapter is constructed on the basis of
    witnesses \MScite{Kathmandu NAK 5-333} and \MScite{Kathmandu NAK 1-1079}.}
    
    \item [4] In this context, if an ignorant person accidentally pierces a
\se{sirā}{duct} there will be fever, burning, \se{śvayathu}{swelling}, pain,
\se{granthi}{lumps}, \se{manyāstambhā}{paralysis of the nape of the neck},
\se{apatānaka}{convulsions}, headache or sharp pain in the ear.\footnote{This
    passage is significantly augmented in Cakrapāṇidatta's and Ḍalhaṇa's versions,
    to outline the specific problems caused by piercing three ducts called
    \emph{kālikā}, \emph{marmikā} and \emph{lohitikā} (1.16.4
    \citep[126]{acar-1939} and 1.16.5 \citep[77]{vulgate} respectively). In fact,
    the order of the problems mentioned in the Nepalese version has been retained
    in the other versions and divided between each duct. Cakrapāṇidatta's
    commentary on 1.16.4 \citep[126]{acar-1939} cites several verses attributed to
    Bhoja on the problems caused by piercing these three ducts in the ear flap:
    '\emph{Lohitikā}, \emph{marmikā} and the black ones are the ducts situated in
    the earflaps.  Listen in due order to the problems that arise when they are
    pierced. Paralysis of the nape of the neck and convulsions, or sharp pain
    arise from piercing \emph{lohitikā}. Pain and lumps are thought to arise from
    piercing \emph{marmikā}. Piercing \emph{kālikā} gives rise to swelling, fever
    and burning.'}
    
    \item[5]     Having removed the \se{vartti}{wick} in the hole because of the
accumulation of humours or an unsatisfactory piercing,\footnote{In addition to
these reasons, 1.16.6 of Ḍalhaṇa at \Su{1.16.6}{77}, added “because of
piercing with a painful, crooked and unsatisfactory needle”
(\emph{kliṣṭajihmāpraśastasūcīvyadhāt}) and  “because of a wick that is too
thick” (\emph{gāḍhataravartitvāt}). Ḍalhaṇa was aware of the reading in the
Nepalese version because in his commentary on \Su{1.16.6}{77} he noted that
some read “because of the accummulation of humours” rather than “because of
piercing with a painful, crooked and unsatisfactory needle or because of a
wick that is too thick.” On the concept of humoral
\se{samudāya}{accumulation}, see the important analysis by \citet{meul-1992}.}
one should smear it with a paste of the roots of barley, liquorice,
\se{mañjiṣṭhā}{Indian madder}, and the \se{gandharvahasta}{castor oil tree},
thickened with honey and ghee. When it has healed well, one should pierce it
again.
    
    \item[6] One should treat the properly-pierced ear by sprinkling it with
unrefined sesame oil.   After every three days one should apply a thicker
\se{varti}{wick} and sprinkle oil right on it.\footnote{The manuscripts
    support the reading \emph{sthūlatarīṃ} that is either a non-standard form or a
    scribal error.}
    
    \item[7]
    Once the ear is free from humours or side-effects, one should 
    put in a light \se{pravardhanakā}{dilator} in order to enlarge 
    it.\footnote{Cakrapāṇidatta on 1.16.6 \citep[127]{acar-1939} and Ḍalhaṇa on 1.16.8 
    \citep[77]{vulgate} pointed out that the dilator can be made of wood, such as that of the 
    \se{apāmarga}{prickly chaff flower}, the \se{nimba}{neem tree} and the 
    \se{kārpāsa}{cotton plant}. Ḍalhaṇa added that it can also be made of \se{sīsaka}{lead} 
    and should have the shape of the \se{dhattūrapuṣpa}{datura flower}.}
    
    \item[8]
    
    \begin{quote}
        \itshape
A person's ear enlarged in this way can split in two, either as a result of the 
humours\footnote{Ḍalhaṇa on 1.16.9  \citep[77]{vulgate} notes that the word \emph{doṣa} 
here can refer to either a humour, such as \se{vāta}{wind}, as we have understood it, or a 
disease generated from a humour.} or a blow.\\ Listen to me about the 
\se{sandhāna}{joins} 
it can have. 
    \end{quote}
    
        \item[9]
    
Here, there are, in brief, fifteen ways of mending the ear flap.\footnote{The Nepalese version uses the word \emph{sandhāna} to refer to joining a split in an ear flap, which is consistent with the terminology in the verse cited above (8). However, 1.16.10 of Ḍalhaṇa's version \citep[77]{vulgate} uses the term \emph{bandha} here and at the very beginning of the chapter (i.e., 1.16.1) to introduce the topic of repairing the ear.}  They are as follows:
    \se{nemīsandhānakaḥ}{Rim-join}, \se{utpalabhedyaka}{Lotus-splittable}, \se{vallūraka}{Dried Flesh}, \se{āsaṅgima}{Fastening}, \se{gaṇḍakarṇa}{Cheek-ear}, \se{āhārya}{Take away}, \se{nirvedhima}{Ready-Split}, \se{vyāyojima}{Multi-joins}, \se{kapāṭasandhika}{Door-hinge}, \se{ardhakapāṭasandhika}{Half door-hinge}, 
    \se{saṃkṣipta}{Compressed}, \se{hīnakarṇa}{Reduced-ear},
    \se{vallīkarṇa}{Creeper-ear}, \se{yaṣṭīkarṇa}{Stick-ear}, and \se{kākauṣṭha}{Crow's lip}.\footnote{For an artist's impression of these different kinds of joins in the ear flap, see \cite[290]{majn-1975} (reproduced as Figure 3.2 in \cite[154]{wuja-2003}).}
    
    In this context, among these, 
    \begin{description}
        
        \item[\mdseries``Rim-join'' (\emph{nemīsandhānaka}):]
        both flaps are wide, long, and equal.
        
        \item[\mdseries``Lotus-splittable'' (\emph{utpalabhedyaka}):]
        both flaps are round, long, and equal.
        
        \item[\mdseries``Dried flesh'' (\emph{vallūraka}):]
        both flaps are short, round, and equal.
        
        \item[\mdseries``Fastening'' (\emph{āsaṅgima}):]
        one flap is longer on the inside.
        
        \item[\mdseries``Cheek-ear'' (\emph{gaṇḍakarṇa}):]
        one flap is longer on the outside.\footnote{For an artist's impression of this join, see \cite[291]{majn-1975} (reproduced as Figure 3.3 in \cites[155]{wuja-2003}).}
        
        \item[\mdseries``Take-away'' (\emph{āhārya}):]
        the flaps are missing, in fact, on both sides.
        
        \item[\mdseries``Ready-split'' (\emph{nirvedhima}):]
        the flaps are like a \se{pīṭha}{dais}.
        
        \item[\mdseries``Multi-joins'' (\emph{vyāyojima}):]
        one flap is small, the other thick, one flap is equal, the other unequal.
        
        \item[\mdseries``Door-hinge'' (\emph{kapāṭasandhika}):]
        the flap on the inside is long, the other is small.
        
        \item[\mdseries``Half door-hinge'' (\emph{ardhakapāṭasandhika}):]
        the flap on the outside is long, the other is small.
    \end{description}

    `These ten \se{vikalpa}{options} for \se{sandhi}{joins} of the ear should be
    bound.  They can mostly be explained as resembling their names.\footnote{Cakrapāṇidatta on 1.16.9–13 \citep[128–129]{acar-1939} and Ḍalhaṇa on \Su{1.16.10}{77–78} provide examples of how the names of these joins describe their shapes. For example, the \se{nemīsandhānaka}{rim-join} is similar to the join of the \se{cakradhārā}{rim of a wheel}.}  The five from \se{saṃkṣipta}{compressed} on are incurable.\footnote{Ḍalhaṇa on \Su{1.16.10}{77–78} mentions that some do not read the statement that only five are incurable, and they understand the causes of unsuccessful joins given below (i.e., heat, inflammation, suppuration and swelling) as also pertaining to the first ten when they do heal.}  Among these, “compressed” has a dry ear canal and the other flap is small.   “Reduced ear” has 
    flaps that have no base and have wasted flesh on their edges. “Creeper-ear” has 
    flaps that are thin and uneven. “Stick-ear” has \se{granthita}{lumpy} flesh and the 
    flaps are stretched thin and have \se{stabdha}{stiff} \se{sirā}{ducts}.  “Crow-lip” 
    has a flap 
    without flesh with \se{saṃkṣipta}{compressed} tips and little blood. Even when 
    they are bound up, they do not heal because they are hot, inflamed, 
    \se{srāva}{suppurating}, or swollen.\footnote{The version of 1.16.11–13 known to 
    Ḍalhaṇa \citep[78]{vulgate} has four verses (\emph{śloka}) at this point that are not in 
    the Nepalese manuscripts. The additional verses iterate the types of joins required for ear 
    flaps that are missing, elongated, thick, wide, etc. All four verses were probably absent in 
    the version of the \emph{Suśrutasaṃhitā} known to Cakrapāṇidatta. He cites the verses 
    separately in his commentary, the \emph{Bhānumatī} \citep[128–129]{acar-1939}, 
    introducing each one as 'some people read' (\emph{ke cit paṭhanti}). However,  in 
    Trikamajī Ācārya's edition of the \emph{Sūtrasthāna} of the \emph{Bhānumatī}, the root 
    text is largely identical to the one commented on by Ḍalhaṇa (\cite{vulgate}), even in 
    instances like this where Cakrapāṇidatta's commentary indicates that he was reading a 
    different version of the \emph{Suśrutasaṃhitā}.}\q{The vulgate verses missing in the 
    Nepalese version here are worth noting because they are explicit about a skin-flap graft 
    remaining connected to the site of removal.}
    
    \item[10]  
    
    % 15
    A person wishing to perform a join of any of these should therefore 
    have supplies prepared 
    according to the recommendations of the “Preparatory
    Supplies” chapter.\footnote{\emph{Suśrutasaṃhitā} \Su{1.5}{18–23}, probably verse 6 
    especially that lists the equipment 
    and medications that a surgeon should have ready.}  And in this regard, he 
    should particularly gather\footnote{The reading in the Nepalese manuscripts of 
    \emph{viśeṣataś cāgropaharaṇīyāt} has been emended to \emph{viśeṣataś 
    cātropaharet} to make sense of the list of ingredients, which is in the 
    accusative case. Also, the repetition of \emph{agropaharaṇīyāt} in the 
    Nepalese version suggests that its second occurrence, which does not make 
    good sense here, is a dittographic error.} \se{surāmaṇḍa}{decanted liquor}, 
    milk, water,
    \se{dhānyāmla}{fermented rice-water}, and \se{kapālacūrṇa}{powdered 
        earthenware crockery}.\footnote{The term \emph{kapālacūrṇa} is unusual. Ḍalhaṇa \citep[79]{vulgate} defines it as the powder of fragments of fresh earthen pots and Cakrapāṇidatta \citep[129]{acar-1939} as the powder of earthenware vessels.}  
    
    Next, having made the woman or man tie up the ends of their hair, eat lightly
and be firmly held by qualified attendants, the physician considers the
\se{bandha}{joins} and then applies them by means of \se{chedya}{cutting},
\se{bhedya}{splitting}, \se{lekhya}{scarification}, or
\se{vyadhana}{piercing}.\footnote{There are syntactic difficulties in this
    sentence.  We have %    It appears that a verb has
    %dropped out of this sentence in the Nepalese version because each word of the
    %sentence is in the accusative case with no apparent verb. Therefore, we have
    adopted the reading in Ḍalhaṇa's version \citep[78]{vulgate}, which has \emph{ca
    kṛtvā} following \emph{suparigṛhītaṃ}. It is likely that a verb, such as
    \emph{kṛtvā}, dropped out of the Nepalese transmission.}  Next, he should examine
    the blood of the ear to know whether it is \se{duṣṭa}{tainted} or not. If it is
    tainted by wind, the ear should be bathed with \se{dhānyāmla}{fermented
        rice-water} and water; if tainted by choler, then cold water and milk should be
    used; if tainted by phlegm, then \se{surāmaṇḍa}{decanted liquor} and water should
    be used, and then he should scarify it again.
    one
    
After arranging the join in the ear so that it is neither proud, depressed, nor
uneven, and observing tonehat the blood has stopped, one should anoint it with
honey and ghee, bandage each ear with \se{picu}{cotton} and \se{plota}{gauze}, and
bind it up with a thread, neither too tightly nor too loosely.  Then, the
physician should sprinkle earthenware powder on it and  provide
\se{ācārika}{medical advice}. And he should supplement with food as taught in  the
“Two Wound” chapter.\footnote{\emph{Suśrutasaṃhitā} 4.1 \citep[396–408]{vulgate}.}
    
    \item[11]
    \begin{verse}
        One should avoid rubbing, sleeping during the day, exercise, overeating,
        sex, getting hot by a fire, or the effort of speaking.
    \end{verse}
    
    \item[12]
    
    % 17
    One should not make a join when the blood is too pure, too copious, or too
    thin.\footnote{1.16.17 of Ḍalhaṇa's version \citep[79]{vulgate} reads “impure” for the Nepalese “too pure,” which would
    appear to make better medical sense.  Emending the text to \emph{nāśuddha-} for
    \emph{nātiśuddha-} in the Nepalese recension would yield the same meaning as the
    Ḍalhaṇa's version.} For when the ear is tainted by wind, then it is
    \se{raktabaddha}{obstructed by blood}, unhealed and will peel. When tainted with
    choler, is becomes \se{gāḍha}{pinched}, \se{pāka}{septic} and red.  When tainted
    by phlegm, it will be \se{stabdha}{stiff} and itchy.  It has excessively copious
    \se{srāva}{suppuration} and is \se{puffed up}{śopha}.  It has it has a small
    amount of \se{kṣīṇa}{wasted} flesh and it will not grow.\footnote{In his edition of \emph{Suśrutasaṃhitā}, Ācārya \citep[79 n. 1]{vulgate} includes in parentheses the following treatment for these conditions, which according to a footnote is not found in the palm-leaf manuscript he used: 'One should sprinkle it with raw sesame oil for three days and one should renew the cotton bandage after three days' (\emph{āmatailena trirātraṃ pariṣecayet trirātrāc ca picuṃ parivartayet}).}
    
    \item[13] When the ear is properly healed and there are no complications,  one may
    very gradually start to expand it.  Otherwise, it may be \se{saṃrambha}{inflamed},
    burning, septic or painful.  It may even split open again.
    
    
    \item [14]
    
    
    Now, massage for the healthy ear, in order to enlarge it. 
    
  
    
    One should gather as much as one can the following: a
    \se{godhā}{monitor lizard}, %{godhā}{Varanus bengalensis, Schneider}{Daniel 1983:58},
    \se{pratuda}{scavenging} and \se{viṣkira}{seed-eating} birds, and
    creatures that live in marshes or water,\footnote{For such classifications,
    see \citet{zimm-1999} and \citet{smit-1994}.} fat, marrow, milk, and sesame oil, and
    white mustard oil.\footnote{Ḍalhaṇa's version of \Su{1.16.19} includes 
    \se{sarpis}{ghee}. However, Ḍalhaṇa's remarks on this passage and Cakrapāṇidatta's on 
    1.16.18 \citep[130]{acar-1939} indicate that they knew a version of this recipe, perhaps 
    similar to the Nepalese one, that did not include ghee (). Ḍalhaṇa also noted that others 
    simply read four oils, beginning with fat and without milk, whereas Cakrapāṇidatta said 
    that some say it is made with four oils and milk.}\q{think more about the compound 
    structure here?}
Then cook the oil with an \se{prativāpa}{admixture} of the following:
    \se{arka}{purple calotropis}, %{Calotropis gigantea, (L.) R. Br.}{ADPS 52, AVS 1.341, NK \#427, Potter 57, ID 306},
    \se{alarka}{white calotropis}, %{Calotropis procera, (Ait.) R. Br.}{NK \#428, GIMP 46b, ID 306},
    \se{balā}{country mallow}, %{Sida cordifolia, L.}{ADPS 71, NK \#2297},
    \se{atibalā}{`strong Indian mallow'}, %{Abutilon indicum, (L.) Sweet; Sida rhombifolia, L.?}{NK \#11, IGP ,4 1080; NK \#2300},
    %\plant{Indian sarsaparilla}{anantā}{Hemidesmus indicus, (L.) R.
    %  Br. \textnormal{and} Cryptolepis buchanani, Roemer \&
    %  Schultes}{ADPS 434, AVS 3.141, NK \#1210},
    %the \skt{Indian sarsaparillas}{sārive}
    \se{anantā}{country sarsaparilla}, %{Hemidesmus indicus, (L.) R. Br.}{ADPS 434,AVS 3.141--5, NK \#1210}
    %and
    %\plant{black creeper}{pālindī}{Ichnocarpus frutescens, (L.)
    %    R.Br. \textnormal{or} Cryptolepis buchanani, Roemer \&
    %    Schultes}{AVS 3.141, 3.145, 3.203, NK \#1283, \#1210, ADPS
    %    434}),
    %
    %%\plant{prickly chaff-flower}{apāmārga}{Achyranthes aspera,
    %%    L.}{GJM 524f., IMP 1.39, ADPS 44f., IMP 3.2066f., Dymock 3.135},
    %\plant{Withania}{aśvagandhā}{Withania somnifera (L.) Dunal}{IMP
    %    5.409f., Dymock 2.566f., Chevallier 150.},
    \se{vidāri}{beggarweed}, %{Desmodium gangeticum (L.) DC}{Dymock 1.428, GJM 602, cf.\ NK \#1192; ADPS 382, 414 and IMP 2.319, 4.366 are confusing},
    %\plant{giant potato}{kṣīraśukla  $\rightarrow$ kṣīravidārī}{Ipmoea mauritiana,
    %    Jacq.}{ADPS 510, AVS 3.222, IMP 3.1717ff.},
    \se{madhuka}{liquorice} and
    hornwort (\emph{jalaśūka} $\rightarrow$ \emph{jalanīlikā}\footnote{Ceratophyllum
        demersum, L. 
        %(IMP 2371, AVS 2.56, IGP 232). 
        This name is not
    certain. In fact, Ḍalhaṇa on 1.16.19 \citep[79]{vulgate} notes that some people interpret it as a poisonous, hairy, air-breathing, underwater creature.}).%
%items having the `sweet' savour (\emph{madhuravarga}\footnote{The
%items which exemplify the `sweet' savour \label{kakolyadi}
% (\emph{madhuravarga}) are enumerated at SS.1.42.11.}) 
    %and `milk 
    %flower'(\emph{payasyā}  $\rightarrow$ \emph{vidārī}\footnote{Pueraria 
    %tuberosa (Willd.) DC. 
    %(ADPS
    %    510, IMP 1.792f., AVS 4.391; not Dymock 1.424f. See GJM supplement 444,
    %    451, IMP 1.187, but IMP 3.1719 = Ipmoea mauritiana, Jacq.). 
    \footnote{The version of 1.16.19 known to Ḍalhaṇa \citep[79]{vulgate} adds several 
    ingredients to this admixture, including \emph{apāmārga, aśvagandhā, 
    kṣīraśuklā, madhuravarga} and \emph{payasyā}. Also, it has 
    \emph{vidārigandhā} instead of \emph{vidāri}. When commenting on 1.16.19, 
    Ḍalhaṇa \citep[79]{vulgate} notes that some do not read 
    \emph{madhuravarga} and \emph{payasyā}. Therefore, there were probably 
    other versions of this recipe with fewer ingredients, as seen in the Nepalese 
    version.}
    %
    This should then be deposited in a well-protected spot.
    
    \item[15]% 20
    \begin{verse}
        
        The wise man who has been sweated should rub the \se{mardita}{massaged} ear with 
        it. 
        Then it will be free of complications, and will enlarge properly and be strong.\footnote{For these aims (i.e., healing and enlarging the ear), the text known to Ḍalhaṇa \citep[79]{vulgate} has an additional verse and a half describing an \se{udvartana}{ointment for rubbing the ear} and \se{taila}{sesame oil} cooked with various medicines for massage. Cakrapāṇidatta \citep[131]{acar-1939} does not comment on these verses, nor verse 15 of the Nepalese version, and so the version of the \emph{Suśrutasaṃhitā} known to him may not have included them.}
    \end{verse}
    
    \item[16]
    % 22cd-23
        \begin{verse}
    Ears which do not enlarge even when sweated and oiled, 
    should be scarified  at the \se{apāṅga}{edge of the hole}, but not outside 
    it.\footnote{Dalhaṇa's version of 
    \Su{1.16.23}{79--80} adds another hemistich that states more explicitly that the 
    scarification should not be done on the outside of hole as it will cause derangement.}  
        \end{verse}
    
    \item[17]
          \begin{verse}
    In this tradition, experts know countless repairs to ears.  So a 
    physician who is \se{suniviṣṭa}{very intent} on working in this way 
    \se{yojayed}{may repair} them.\footnote{After verse 17, the 1938 edition of Ācārya \citep[80]{vulgate} has in parentheses nineteen verses on diseases of the ear lobes, treatments and complications. It is possible that these verses were in some of the witnesses used by Ācārya to construct the text as they occur in other manuscripts, such as  MS Hyderabad Osmania 137-3 (b). However, Cakrapāṇidatta \citep[132]{acar-1939} and Ḍalhaṇa \citep[80]{vulgate} state that some read about the diseases of the ear lobes in this chapter whereas others read about them in the chapter on \se{miśrakacikitsa}{various treatments} (SS 5.25), which does indeed begin with a discussion of the disease \emph{paripoṭa}.  Ḍalhaṇa goes on to say that some believe that these verses were not composed by sages and, therefore, do not read them.}
        \end{verse}
    
    \item[18]
    % 25
           \begin{verse}
    If an ear has grown hair, has a nice hole, a firm join, and is strong and
    even, well-healed, and free from pain, then one can enlarge it slowly.\footnote{The order of verses 17 and 18 are reversed in Ḍalhaṇa's version \citep[80]{vulgate}.}
        \end{verse}
    
    \item[19]
    
    %28, 29.
    Now I shall describe the proper method of making a repair when a nose is severed.
    First, take from the trees a leaf the same size as the man's nose and hang it
    on him. 
    
    \item[20] Next, having cut a \se{vadhra}{slice of flesh}\footnote{The
    version of 1.16.28b known to Ḍalhaṇa \citep[81]{vulgate} reads “\se{baddham}{bound, 
    connected}” instead of “\se{vadhra}{slice of flesh}.”
    This is a critical variant from the surgical point of view.  If the slice remains
    connected, it will have a continuing blood supply.  This is one of the effective 
    techniques that so astonished surgeons witnessing a similar operation in Pune in
    the eighteenth century \citep[see][67--70]{wuja-2003}.} with the same
    measurements off the cheek, the end of the nose is then scarified.\footnote{Or 1.16.20 could be mean, 
    `\ldots\ off the cheek, it is fixed to the end of the nose, which has been
    scarified.' Unfortunately, the Sanskrit of the Nepalese version is not unambiguous on the
    important point of whether or not the flap of grafted skin remains connected
    to its original site on the cheek. However, Ḍalhaṇa \citep[81]{vulgate} clarifies the meaning of the vulgate here by stating that one should supply the word 'flesh' when reading 'connected,' thus indicating that he understood the flesh to be connected to the face.} % 
%
Then the \se{apramatta}{undistracted} physician, 
    should quickly put it back together so that it is 
    \se{sādhubaddha}{well joined}.
    \label{well-joined}
    
    \item[21] 
    % 30.
    Having carefully observed that it has been sewn up properly,
    he should then fasten it along with two tubes.\footnote{Ḍalhaṇa noted that the two tubes 
    should be made of \se{nala}{reed} or the \se{eraṇḍapatranāla}{stalk of the 
    leaf of castor oil plant} (on \Su{1.16.21}{81}). They should not be made of 
    lead or betel nut because the weight will cause them to slip down.}  
    Having caused it to be raised,\footnote{The 
    Sanskrit term \emph{unnāmayitvā} in 1.16.21 is non-Pāṇinian.}
    the powder of \se{pattāṅga}{sappanwood},\footnote{Caesalpinia sappan, L. 
    %(AVS 1.323, IMP 2.847f.). 
    For \emph{pattāṅga} there are manuscript variants 
    \emph{pattrāṅga} (\MScite{Kathmandu NAK 5-333}) and \emph{pattaṅga} 
    (\MScite{Kathmandu NAK 1-1079}).  Also, \MScite{Kathmandu KL 699} 
    (f.\,14r:1) has \emph{pattrāṅga} in a verse in 1.14 (cf.\ \Su{1.14.36}{66}). The 
    text known to Ḍalhaṇa  has \emph{pataṅga} (\Su{1.16.29}{81}) and this term 
    is propagated in modern dictionaries.}
    %\se{yaṣṭīmadhuka}
    {liquorice}
    %{Glycyrrhiza glabra, L.}{AVS 3.84, NK \#1136},
    and
    Indian barberry.\footnote{Berberis aristata, DC.
    % (Dymock 1.65, NK  \#685, GJM 562, IGP 141). 
Ḍalhaṇa understands it as \se{rasāñjana}{elixir 
salve} (\cite[81]{vulgate}).}\q{añjana}
    should be sprinkled on it.
    
    \item[22] 
    The wound should be covered properly with \se{picu}{cotton} and should be
    moistened repeatedly with sesame oil.  Ghee should be given to the man to
    drink.  His digestion being complete, he should be oiled and purged in
    accordance with the instructions specific to him.\footnote{The expression 
    \emph{svayathopadeśa} is ungrammatical but supported in all available 
    witnesses.}   
    
    \item[23] %32.
And once healed and really come together, what is left of that \se{vadhra}{slice
    of flesh} should then be trimmed.\footnote{The vulgate transmission has lost the
    word \emph{vadhra} and replaced it with \emph{ardha} “half,” which makes little
    sense in this surgical  context.} If it is \se{hīna}{reduced}, however, one should
    make an effort to stretch it, and one should make its overgrown flesh
    smooth.\footnote{Ḍalhaṇa  accepted a verse following this, \Su{1.16.32}{81}, which
        points out that the procedure for joining the nose is similar to that of joining
        the lips without fusing the ducts. He noted that earlier teachers did not think
        this statement on the nose and lips was made by sages, but he included it because
        it was accepted by Jejjaṭa, Gayadāsa and others, although they did not comment on
        it because it was easy to understand. Cakrapāṇidatta also did not comment on this
        additional verse \citep[133]{acar-1939}.}
    
    
\end{translation}    
