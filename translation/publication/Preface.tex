% !TeX root = surgery.tex
\chapter*{Preface}
\addcontentsline{toc}{chapter}{Preface}

The \emph{Compendium of Suśruta} (Sanskrit: \SS) is a world classic of ancient
medicine, comparable in age and importance to the Hippocratic Corpus and the
Huangdi Neijing. It is a substantial treatise, written in the Sanskrit
language, that presents a systematic, scholarly form of medicine for
diagnosing and treating the many ailments that patients presented in South
Asia two thousand years ago. \emph{The Compendium} is a text written by
physicians for physicans and is one of the founding treatises of the
indigenous medical system of India, Ayurveda.  As such, it still informs
indigenous medical practice in India and Ayurvedic complementary and
alternative medicine internationally.

Amongst historians of medicine, \textit{The Compendium} is perhaps most famous
for its passages describing remarkable forms of surgery. These techniques were
used in South Asia and beyond: \emph{The Compendium}'s method of couching for
cataract circulated in China in the seventh century and a form of facial
plastic surgery described in \emph{The Compendium} was witnessed by British
surgeons in India in the eighteenth century and subsequently formed the basis
of certain types of facial reconstruction as practised even today.

In 2007, a previously unknown manuscript of \emph{The Compendium} from the
uncatalogued collections of the Kaiser Library in Kathmandu was announced in a
scholarly publication.  This manuscript, \MScite{Kathmandu KL 699}, is datable
to 878 \CE, almost a thousand years before any other known manuscript of the
work. Furthermore, it became clear that two other manuscripts in Kathmandu
were close copies of KL 699, albeit from later dates.

These exciting discoveries provide a time machine by which we can directly
examine the medical thought of Nepalese physicians of the ninth century.  We
can also start to see the changes and additions to the text that have happened
in the last millennium.  Almost all of these later changes to the text have
tended to obscure its clarity and directness, banalizing the language and
inflating the recipes.  Through the critical study of  \MScite{Kathmandu KL
    699} and its companions we are beginning to recover an older, more authentic
and more meaningful version of this medical classic.

This book presents a single chapter from \emph{The Compendium} that is of
unique interest to medical history.  The book also lays out the basic
parameters and methods of the project that will also apply to future
publications. The members of the Suśruta Project have already read, edited and
translated several other chapters.  These materials are already available in
digital format through the project website.\footcite{wuja-2021b} It is our
intention to continue to publish both digitally and in print.  We are
currently preparing a critical edition and annotated translation of the
\emph{Kalpasthāna}, \emph{The Compendium}'s book on plant and animal poisons
and their remedies.

Beyond the present funded project, the important work of editing and
translating the other five books of the treatise remains for the future.
Therefore the present book aims to establish an academically sound model for
disseminating this older version of \emph{The Compendium} to an international
audience that will pave the way for future editions of the rest of the work.

When one sees an authorial collaboration such as the present book, it is natural to be 
curious about who did what.  The bulk of the words of this book were co-written by 
myself and Jason Birch.  Andrey Klebanov contributed an important section describing 
the manuscripts; Madhu K. Paramesvaran wrote about the evolution of recipes 
through time. The text of the translation was a fully collaborative effort, arising out of 
weekly seminar meetings that we all attended and to which we all contributed. 
Usually, Jason or I would present a first draft, and that was discussed word by word, 
together with re-visiting difficult manuscript readings and refining the critical edition. 
Additionally, Vandana Lele, Harshal Bhatt, Madhusudan Rimal and Deepro 
Chakraborty and Paras Mehta all spent many hours carefully transcribing the old 
Nepalese manuscripts so that the text could be edited using the methods of Digital 
Humanities.  The book could not have been written without the participation of the 
whole project team.


I would like to thank other project participants who have also contributed to
the project at different times, including Jane Allred, Devayani Shenoy and
Gauri Vyaghrambhare.  

On a personal note, I would like to thank my former colleagues and friends
from the University of Vienna, especially Karin Preisendanz, Philip Maas and
Alessandro Graheli, from whom I learned and continue to learn so much about
Indology and textual criticism, and my wife Dagmar Wujastyk, discussions with
whom inform and improve all my scholarly efforts.

The project owes its existence to Canada's Social Science and Humanities Research 
Council who have generously funded the work through an Insight Grant. I am grateful 
to my academic home, the University of Alberta, for hosting and supporting the 
project. 

{\flushright --- Dominik Wujastyk\\ Edmonton 2023\par}

%Since the nineteenth century, historians of medicine and practioners alike have 
%relied 
%on printed editions of the work that were hastily produced on the basis of a few 
%local 
%manuscripts.  One particular editor, Yādavaśarman Trivikrama Ācārya (1881--1956), 
%produced respected printed editions of \emph{The Compendium} in the years 
%between 1915 and 1939, and Ācārya's editions have become the basis for all 
%subsequent readings of the work, whether for historical or clinical purposes.

%
% I suppose the Preface is a good opportunity to summarise the Project’s other work 
% (i.e., other than 1.16) and mention future outputs, such as the critical edition and 
% annotated translation of the kalpasthāna. And perhaps you might even flag the 
%need 
% for 
% future funding to edit and translate the other sthānas, so that when some wealthy 
% business person picks up the book, they immediately know what they can do to 
% immortalise themselves in the future history of āyurvedic literature 

% 
%https://greenleafbookgroup.com/learning-center/book-creation/distinguishing-between-a-foreword-a-preface-and-an-introduction
%
%The preface gives you, as the author, the opportunity to introduce yourself to your 
%readers and explain to them why they should hear what you have to say. This is 
%where you build credibility, so you should give some insight into how you got to be 
%an expert on your subject. You can toot your own horn a bit here.
%
%You should use a preface to spark curiosity about your content and draw readers in. 
%Here, you speak directly about the purpose, creation, or importance of your book. 
%How did your book’s genesis come about? What was the pain point you were 
%seeking 
%to address when you decided you wanted to write a book, and why is that important 
%to your readers? The preface is where you explain the who, when, and where of it 
%all.