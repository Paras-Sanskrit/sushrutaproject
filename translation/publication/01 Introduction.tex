% !TeX root = surgery.tex
%\subsection{Preliminaries}
\chapter{Introduction}

%\section{The Aim of the Article}

The \emph{Compendium of Suśruta} (\emph{Suśrutasaṃhitā}) is amongst the most
important treatises on medicine to survive from the ancient world. It has been
studied seriously by historians since it first became available in print in the
mid-nineteenth century.\footnote{The edition princeps was that of
    \citet{gupt-1835}.  A selection of prominent studies includes:
    \cites{hoer-1897,hoer-1906a,hoer-1906b,hoer-1907a,hoer-1907,stra-1934,
    sing-1972a,shar-1975, ray-1980, adri-1984,
    yano-1986,meul-hist,shar-1999,vali-2007}[and Meulenbeld's magnum 
    opus][]{meul-hist}.} %Sharma 1975; ?
    Meulenbeld listed forty-four editions of the work since the first edition of 1835
    by \citeauthor{gupt-1835} in Calcutta, and eight translations, starting from the
    Latin translation of 1844 by \citeauthor{hess-1855}.\footcite[IB,
    311\,ff.]{meul-hist}   Many more translations have appeared in recent decades and
    reprints of the early twentieth-century editions continue to be reprinted frequently.

The study of this work has yielded rich historical discoveries about 
the earliest history of surgery, ancient pharmacology, toxicology and many other 
social and medical topics. Yet there remain fundamental unanswered questions 
about the history of the text itself and about related issues in the history of 
medicine in Asia.\footcite[IA, 203–389]{meul-hist}

In January 2007, a manuscript of the \SS\ in the Kaiser Library, Kathmandu,
previously unknown to contemporary scholarship, was brought to international
attention by \citeauthor{dimi-kais}.\footcite{dimi-kais} \MScite{Kathmandu KL 699} is a 
Nepalese palm-leaf
manuscript covering about two thirds of the Sanskrit text. It is dated to 878 \CE,
making it one of the earliest dated manuscripts known from South
Asia.\footcite[87–88]{hari-2011} The manuscript has been declared by UNESCO to be
part of the Memory of the World.\footcite{unes-2013}

The newly-discovered manuscript in Nepal is related to two other early
palm-leaf manuscripts in the National Archives in Kathmandu, \MScite{Kathmandu
    NAK 5-333} and \MScite{Kathmandu NAK 1-1079}. Klebanov has assembled
compelling evidence for believing that these Nepalese manuscripts present a
version of the text that was in wider circulation in northern India,
especially Bengal, in the period up to about 1200
\CE.\footcite{kleb-2010,kleb-2021b} Generally speaking, the Nepalese version
of the \SS\ is shorter and sometimes clearer than the versions commented on by
Cakrapāṇidatta (\emph{fl.}\ eleventh century) and Ḍalhaṇa (\emph{fl.}\ twelfth
century).  The version of the \SS\ commented on by Ḍalhaṇa has formed the
basis of modern printed editions and translations, such as those of
Yadavaśarman Trivikramātmaja Ācārya and
others.\footnote{\cite{susr-trikamji1,vulgate,shar-susr}. Note that Ācārya
    himself referred to this text as “Ḍalhaṇa's version” (see footnote
    \ref{dalhanaversion} below).} Some of the changes in the text between the
    Nepalese version and what we might call the Ḍalhaṇa version, or the vulgate
    version,\footnote{For discussion of “the vulgate,” represented by
        \cite{vulgate}, see p.\,\pageref{vulgate} below.} consist of the addition and
        loss of numerous verses, changes to medical recipes, and reordering of
        chapters, especially in the \emph{Uttaratantra} or last part of the work.
        \citeauthor{lari-2003} hypothesized long ago, in a different context, that
        Sanskrit texts tended to continue to expand through the addition of new
        materials,
\begin{quote}
The process of addition to these compilations must have gone on for centuries. The
hearers or readers of of these compilations must have known other verses \ldots\
and it would be natural for them to include these verses in the compilation. This
type of  addition may have continued until a commentary on the collection was
composed.   A commentary would have served to fix the text. and the expansion of
the text would have been more difficult after
that.\footnote{\cite[xii]{lari-2003}, cited with agreement by
    \citet[51]{oliv-manu} in the context of legal literature and by
    \citet[62--63]{bron-how} in the context of epic literature.  See the latter
    citation for further discussion of Sanskrit text formation between the empires. 
%    The idea 
%    was developed by \citet{bakk-2019}, who coined the descriptive expression 
%    ``composition-in-transmission.''
    }
\end{quote}
In the case of the \emph{Suśrutasaṃhitā}, the Nepalese manuscripts appear to present us 
with the last recoverable snapshot of this stage of the work when it was still open to 
absorbing new materials, most notably the \emph{Uttaratantra}, and before the text was 
fixed as a result of the authority of the major commentators, Cakrapāṇidatta and 
Ḍalhaṇa.\footnote{The roles of earlier commentators including Jejjaṭa, Gayadāsa and 
Candraṭa in closing the text and influencing Cakrapāṇidatta and Ḍalhaṇa remains an open 
research problem.}  It is in this sense that we use the expression, ``Ḍalhaṇa's version," when 
referring to the vulgate text of the \SS.

The present study offers a critical edition and annotated translation of the
sixteenth chapter of the \emph{Ślokasthāna}, the first book of the Nepalese
version of the \SS.\footnote{This book is called the \emph{Sūtrasthāna} in later
    versions of the \SS.  Note that the \SS\ itself used the name \emph{Ślokasthāna}
    at several places, e.g., \Su{6.42.61}{721}, \Su{6.65.30 and 31}{818}, usually
    referring to identifiable passages in that part of the work. The name is also used
    in the \emph{Ślokasthāna} itself, at 1.1.40 of the Nepalese version.} 
%  
This chapter is important in the history of Indian medicine because of its
discussion of surgical methods for repairing torn ears and severed noses. In
addition to discussing the manuscripts and published editions used in this new
edition, the introduction of this study addresses some of the challenges of
editing the Nepalese manuscripts and the salient differences between the Nepalese
version of the \SS\ 1.16 and the text as known to Cakrapāṇidatta and Ḍalhaṇa. The
notes to the edition incorporate alternative readings mentioned by the
commentators.  The annotations to the translation discuss the following topics:
instances where the text is uncertain; non-standard spellings and syntax; the
meaning of technical and obscure terms; relevant remarks by the commentators;
ambiguities in the identification of medical ingredients, in particular, plant
names; and the additional compounds, verses and passages in Ḍalhaṇa's version of
the text. In short, this is a pilot study for undertaking a complete edition and
translation of the Nepalese version of the \SS.


\section{Importance of SS.1.16 in the History of Medicine}


Simple forms of surgery have a long history in South Asia. In works datable to at
least 1200 \BC\ we learn how a reed was used as a catheter to cure urine
retention.\footcite[70--71]{zysk-1985} Cauterization too was described in the same
ancient sources, to prevent wounds from bleeding. The \emph{Atharvaveda}, in the
early first millennium \BC, described the bones of the human body, showing early
anatomical awareness in a religious context.\footnote{Translation by \citet[\S43,
    \S100]{hoer-1907}. Further bibliography: \cite[IIB, 819]{meul-hist}.} The Brāhmaṇa
    literature of the only slightly later period contained more detailed descriptions
    of animal butchery in the context of religious sacrifice that involved the
    enumeration of internal organs and bones.\footcite{mala-1996,saha-2015}   This
    exemplifies an early Sanskrit vocabulary for internal parts of bodies.  However,
    this is not the same as anatomical dissection, whose methods and intentions are
    quite different. As \citeauthor{keit-1908} pointed out long ago, the enumeration
    of the bones in the \emph{Brāhmaṇas} was derived from correspondences with the
    numbering of various verse forms, not from anatomical
    observation.\footcite{keit-1908}  With the \SS, we find ourselves in the presence
    of something quite different and far more sophisticated from the medical point of
    view, where the body was studied specifically for medical and surgical
    purposes.\footnote{\cite{zysk-1986}. The \CS\ too has brief descriptions of
        surgical techniques, but the \SS\ goes into greater detail.} The text gives us a
        historical window onto a school of professionalised medicine, including surgical
        practice, that existed almost two millennia ago and which in its day was perhaps
        the most advanced school of surgery in the world.

The author of the \SS\ described how a surgeon should be trained and how various
operations should be done.  There are descriptions of ophthalmic couching (the
dislodging of the lens of the eye), perineal lithotomy (cutting for stone in the
bladder), the removal of arrows and splinters, suturing, the examination of dead
human bodies for the study of anatomy, and other
procedures.\footnote{\cites{mukh-1913,
    desh-2000,nara-2011,wuja-2003,wils-1823,vali-2007} and many other
    studies.}\MSnocite{London Wellcome 3007} The author of the \SS\ claimed that
    surgery was the most ancient and most efficacious of the eight branches of medical
    knowledge.\footnote{\SS\ \Su{1.1.15--19}{4}.} Anecdotal discussion with
        contemporary surgeons suggests that many details in the descriptions could only
        have been written by a practising surgeon: it is beyond reasonable doubt that
        elaborate surgical techniques were a reality amongst those whose practices were
        recorded in the \SS.\footnote{\citet{leff-2020} provide a detailed discussion of
            the \SS's surgical technique in the case of ophthalmic cataract, with references
            for further reading. The manuscript account by \citet{jack-1884} provides a
            fascinating quantitative comparison of traditional couching operations with his
            contemporary nineteenth-century 
            methods (\MScite{London Wellcome 3007}).}

\vspace*{\fill}  %DW ugh

 %\q{see my comments in
% the input file}
%
%[Jason: Dominik, I think we need to discuss the importance of 1.16 specifically at 
%this point. The next five paragraphs are more suited to a general discussion on 
%surgery. Seeing that this is simply a journal article and not a book, we might 
%finish 
%this section with the paragraphs on 'Torn ear lobes' and 'Rhinplasty' and 
%perhaps 
%add a paragraph addressing the medical implications of the unique features of 
%the 
%chapter in the Nepalese version.]

%As argued elsewhere, in spite of Suśruta's elaborate descriptions there
%is little historical evidence to show that these practices persisted beyond the
%time of the composition of \SS.\footcite{wuja-indi} A few
%references to surgery found in Sanskrit literature between the fourth to the
%eighth centuries \AD\ were collected by
%\citeauthor{shar-1972}.\footcite[74--8]{shar-1972}  But the stereotypical nature of
%most of these references and the paucity of real detail, suggest that the practice
%of surgery was rare in this period.
%
%There is some evidence, however, that although surgery ceased to be part of
%the professional practice of traditional physicians of the \emph{vaidya}
%castes, it migrated to practitioners of the `barber-surgeon' type.  As such,
%it was no longer supported by the underpinning of Sanskrit literary
%tradition, and so it is harder to find historical data about the
%practice. \citet{sirc-raks} discussed some epigraphical evidence for the
%heritage and migrations of the `Ambaṣṭha' caste, who appear to have
%functioned as barber-surgeons in South India and later migrated to Bengal.
%There is also evidence from the eighteenth century of the practice of
%smallpox inoculation by traditional 
%`\emph{ṭīkādar}s.'\footcite{holw-acco,coul-acco} And some other surgical 
%techniques which sound
%similar to those described in \emph{The Compendium}, for example for removing 
%ulcers, were
%observed in the same period.\footcite[79]{babe-scie}

% and the letter of
% 1731 cited by \citet[141\,f., 276]{dhar-indi}.}

%While the theoretical aspects of surgery continued to appear in those
%medical textbooks that tried to be comprehensive, in practice those who
%applied the surgical techniques seem to have been increasingly isolated from
%mainstream of āyurvedic practice. It may be that as the caste system grew in
%rigidity through the first millennium \AD, taboos concerning physical
%contact became almost insurmountable and \emph{vaidya}s seeking to enhance
%their status may have resisted therapies that involved intimate physical
%contact with the patient, or cutting into the body.  On the other hand,
%against this hypothesis it may be argued that the examination of the pulse
%and urine gained in popularity, as did massage therapies.
%
%An example of this process may be the famous ophthalmic operation of couching 
%for
%cataract, which is first described in \emph{The Compendium}.\footcite[Well
%outlined by][378--9]{majn-1975} A description of this operation survives in the
%ninth-century \emph{Kalyāṇa\-kāraka} composed in eastern India by the Jaina 
%author
%Ugrāditya.\footcites[67, n.\,76]{meul-1984}[IIA, 151\,ff]{meul-hist}[366--368,
%375--378]{shas-kaly}. This procedure, or one very similar to it, also reached
%China, through transmission on the routes taken by traders and Buddhist
%pilgrims.\footcites[132--48]{unsc-medi}{desh-1999,desh-2000} But by the 
%beginning
%of the twentieth century it was described by \citet{elli-indi} as long having been
%carried out by traditional practitioners of the barber-surgeon type rather than by
%physicians trained in the Sanskrit texts.

%By the seventeenth century, foreign visitors to India began to remark on how
%surgery was virtually non-existent in India. The French traveller
%Tavernier\label{tavernier}, for example, reported in 1684 that once when the
%King of Golconda had a headache and his physicians prescribed that
%blood should be let in four places under his tongue, nobody could be found
%to do it, `for the Natives of the Country understand nothing of
%Chirurgery'.\footnote{\cite[1.2.103]{tave-trav}; see also
%\cite[1.130]{slee-ramb}.}


\subsection{Torn ear lobes}

The \SS's description of the repair of torn ear lobes is  unique for its
time.\footnote{The comprehensive study of ears in the history of Indian culture by
    \citet{boll-2010} oddly omits reference to \emph{Suśrutasaṃhitā}'s surgery,
    although it mentions the text's description of ear diseases.}
    \citeauthor{majn-1975}, a practising surgeon, noted that, “through the habit of
    stretching their earlobes, the Indians became masters in a branch of surgery that
    Europe ignored for another two thousand years”.\footcite[291]{majn-1975} The
    different types of mutilated ear lobe that the \SS\ described are not always easy
    to understand from the Sanskrit: the illustrations supplied in Majno's text
    greatly help with the visualization of the most likely
    scenarios.\footcites[290--291]{majn-1975}[reproduced with permission
    in][92--93]{wuja-2003}

%One of the subjects unfortunately not covered in the present book is Suśruta's 
%use
%of ants for suturing.\footnote{\Su{4.2.56}{412}.}  The technique, which is also
%described in the \emph{Carakasaṃhitā},\footnote{\Ca{6.13.188}{500}.} is to 
%bring
%the edges of the flesh to be joined close together, and then allow a large black
%ant to bite the join with its mandibles.  The ant's body is twisted off, and the
%head remains in place, clamping the join together.  This technique has been
%described in detail and illustrated by
%\citeauthor{majn-1975}.\footcite[304\,ff]{majn-1975}  Majno described how this
%method is also known from tribal societies in Brazil and the Congo. Most
%interestingly, he cited an entomologist's report of the technique being known in
%southern Bhutan, in the early 1970s.\footcite[307, n.\,298]{majn-1975}  The
%technique was known in the Islamic and European world through the famous and
%much-translated surgical text by the Iberian Arab scholar Albucasis
%(d.\,1013).\footcite[550--51]{spin-albu}  Majno noted that Albucasis knew the
%technique from Suśruta.  Although Majno demonstrated conclusively that the
%technique is practicable, it is interesting that both Suśruta and Albucasis
%referred to the technique as a matter of hearsay.


\subsection{Rhinoplasty}
\label{sec:rhinoplasty}

One of the best-known surgical techniques associated with \SS\ is rhinoplasty, the
repair or rebuilding of a severed nose. The history of this operation has been
discussed by Wujastyk, and a translation of the Sanskrit passage from the vulgate
edition of the \SS\ was given.\footnote{\cite[67--70, 99--100]{wuja-2003}. See
    also \cite[IB, 327--328, note 186]{meul-hist}, for further literature and
    reflections.} This fascinating technique is certainly old in South Asia, having
    been witnessed by travellers from Niccolo Manucci in the seventeenth century
    onwards.\fvolcite{2}[301]{manu-stor} Many witnesses, including the most famous,
    Cruso and Findlay,\footcite[883, 891\,f.]{cowasjee} described an operation that
    differs from \SS\ in that it takes the grafting skin from the forehead, not the
    cheek.  But the nineteenth-century operation witnessed by \citeauthor{thor-bann}
    is especially interesting, since the technique followed  the \SS\ exactly in
    taking flesh from the cheek, rather than the
    forehead.\footcite[352--353]{thor-bann}

As noted by \citeauthor{meul-hist}, none of the extant commentators -- Jejjaṭa,
Gayadāsa, Cakrapāṇi or Ḍalhaṇa -- explained the technique in any detail beyond
short lexical glosses.\footnote{\cite[IB, 328]{meul-hist}. Ḍalhaṇa noted cryptically, on
    \Su{1.16.27--31}{81a}, that a rather different version of the
    text, cast in \emph{śloka} metre, was also known to him from other sources.
    Ḍalhaṇa's variant bears a resemblance to the description of the operation given in
    printed editions of the \emph{Aṣṭāṅgahṛdayasaṃhitā} at \Ah{Utt.18.59--65}{841}.}
    %501 of
    % 1902 ed.
    This suggests that the commentators may not all have known the technique at
    first-hand. %\q{Perhaps, it is worth mentioning (pace. Meul) the comment by
    % Ḍalhaṇa
    %    –discussed in fn 113– which indicates that he knew the grafted skin had
    % to be
    %    connected. And this is not clear in the mūla.} % -- Done DW

\subsection{The skin flap} \label{skinflap} It is worth highlighting here a
point of critical medical importance: the continued attachment of the skin flap.
One of the crucial innovations of the “Hindu Method” of nasal reconstruction, as
observed  and internationally reported in the eighteenth century, was that the
skin flap taken from the face remained partially connected to its original
location.\footcite[See][67--70]{wuja-2003} This ensured the blood flow essential
to keeping the skin alive while it healed in its new location.\footnote{This
    surgical innovation distinguished the “Hindu Method” from sixteenth-century
    European methods associated with Gaspar Tagliacozzo and others \citep[see,
    e.g.,][\emph{passim}]{carp-acco}.}  The Sanskrit of the vulgate is ambiguous on
    this critical point and the wording of the Nepalese version is unclear. However,
    Ḍalhaṇa clarified the meaning of the vulgate here by stating that when reading the
    expression “connected,” one should understand “connected flesh”.\footnote{\SS\
        \Su{1.16.28}{81}.}  He thus indicated that he understood the flesh to be connected
        to the face.\footnote{See p.\,\pageref{well-joined} below.}  Thus, we cannot know
            definitively at present whether the connection of the flap was known to the
            redactors of the Nepalese version, although it seems likely.  It was probably
            known to the redactors of the vulgate, and was certainly known to Ḍalhaṇa in the
            twelfth century.
        
Earlier in the chapter, in the context of ear-piercing and repair, the vulgate has
a passage that is more explicit and conclusive.  After listing the names and
characteristics of different types of earlobe, the vulgate cites some summary
verses from an unknown source.\footnote{\Su{1.16.11--14}{78}.}  The last of these
    verses says,
    \begin{quote}
            If no lobe exists, an expert may create an ear lobe by scarifying and
then using living flesh still attached to the cheek from which it has
been sliced.\footnote{\Su{1.16.14}{78}: \dev{gaṇḍādutpāṭya māṃsena
    sānubandhena jīvatā | karṇapālīmāpālestu kuryānnirlikhya śāstravit/} 
    Cf.\ the translation of the whole passage by \citet[94]{wuja-2003}.}
    \end{quote}
The commentator Ḍalhaṇa was even more explicit in his gloss on this passage:
\begin{quote}
    “Living” [flesh] means “together with blood”.\footnote{\Su{1.16.14}{78}:
    \dev{jīveti śoṇitasahitenetyarthaḥ/}}
\end{quote}
Thus, Ḍalhaṇa's comment gives us unequivocal evidence for the concept of a living
skin flap in the twelfth century, and it is almost certain that this is also what
the text of the \SS\ intended by the word “living.”  The one remaining historical
problem is that these specific verses, 1.16.11--14, are not present in the
Nepalese version of the text.  This suggests that they were part of a different
tradition of practice with a verse literature that was integrated into the vulgate
text of the \SS\ at a time after the Nepalese version as recorded in 878 \CE, but
at the latest by the time of Ḍalhaṇa in the twelfth century.

If we can assume that the descriptions of ear-surgery and rhinoplasty were part of
a single professional tradition of surgical method, then the above passage, in the
context of ear-lobe repair, strongly supports the idea that rhinoplastic surgery
too was conducted with attention to keeping a living skin flap.

By the late first millennium, had the rhinoplastic technique moved from the
professional competence of scholar-physicians (\emph{vaidya}s) to that of
barber-surgeons (\emph{Ambaṣṭha}s and others)?  Or perhaps the influence was in
the other direction, and a technique known to practitioners elsewhere in South
Asia in the first millennium was integrated into the text of \SS. The rhinoplastic
description consists of only five verses and they are written in the Upendravajrā
metre, which is different from the rest of the chapter.  The description's
appearance at the very end of the chapter, its terseness, its ornate metre, and
the paucity of the commentators' treatment could all be taken as pointing in this
direction.

%
%The famous `Indian rhinoplasty' operation is often cited as
%evidence that \emph{The Compendium}'s surgery was widely known in India 
%even up
%to comparatively modern times. This operation took place in March
%1793 in Poona and was ultimately to change the course of plastic
%surgery in Europe and the world. A Maratha named Cowasjee, who had
%been a bullock-cart driver with the English army in the war of
%1792, was captured by the forces of Tipu Sultan, and had his nose
%and one hand cut off.\footnote{A residual puzzle with this account
%is that `Cowasjee' is a Parsi name, not a Maratha one.} After a
%year without a nose, he and four of his colleagues who had
%suffered the same fate submitted themselves to treatment by a man
%who had a reputation for nose repairs. Unfortunately, we know
%little of this man, except that he was said in one account to be
%of the brick-maker's caste. Thomas Cruso (d.\,1802) and James
%Findlay (d.\,1801),\footnote{\citet{cowasjee} called the second
%surgeon `Trindlay' but this must be an error.
%\citet[37]{carp-acco} had `Findlay', and both surgeons are listed by 
%\citet[409, 411]{craw-roll}.} senior British surgeons in the
%Bombay Presidency, witnessed this operation (or one just like it).
%They appear to have prepared a description of what they saw,
%together with a painting of the patient and diagrams of the skin
%graft procedure.  These details, with diagrams and an engraving
%from the painting, were published at third hand in London in
%1794;\footnote{\citet[883, 891\,f.]{cowasjee}.}
%Fig.\,\ref{fig:cowasjee} shows the illustration that accompanied
%this article.  The key innovation was the grafting of skin from
%the site immediately adjacent to the repair-site, while keeping
%the graft tissue alive and supplied with blood through a
%connective skin bridge. Subsequently, through the publication by
%\citet{carp-acco} describing his successful use of the technique,
%this method of nose-repair gained popularity amongst British and
%European surgeons.
%
%Carpue received personal accounts of other witnesses to this
%operation, and others of the same ilk, which shed more light on
%this episode \citep[appendix II]{carp-acco}.  Carpue's chief
%informant in 1815 was Cowasjee's commanding officer,
%Lieutenant-Colonel Ward.  Ward described the surgeon not as a
%brick-maker, but as an `artist', whose residence was four hundred
%miles distant from Poona. Cowasjee was not the only patient: four
%friends who had suffered the same fate also underwent nose
%reconstruction by the same artist. Most interestingly, the
%understanding in Poona at the time of the operation was that this
%artist-surgeon, who also claimed expertise in repairing torn or
%split lips, was the only one of his kind in India, and that the
%art was hereditary in his family.
%
%Further evidence on this topic is given by the
%seventeenth-century traveller Niccolo Manucci
%(fl.\,1639--ca.\,1709), who described how Shah Jahan's soldiers
%in Kashmir in the 1630s customarily cut off people's noses as a
%form of punishment.\footcite[i.215--6]{manu-stor}.  Even more
%interestingly, Manucci described rhinoplasty operations which
%took place in Bijapur in about 1686:
%\begin{quote}
%    The surgeons belonging to the country cut the skin of the
%    forehead above the eyebrows and made it fall down over the wounds
%    on the nose. Then, giving it a twist so that the live flesh might
%    meet the other live surface, by healing applications they
%    fashioned for them other imperfect noses. There is left above,
%    between the eyebrows, a small hole, caused by the twist given to
%    the skin to bring the two live surfaces together. In a short time
%    the wounds heal up, some obstacle being placed beneath to allow
%    of respiration. I saw many persons with such noses, and they were
%    not so disfigured as they would have been without any nose at
%    all, but they bore between their eyebrows the mark of the
%    incision.\footnote{\cite[ii.301]{manu-stor}.  I am grateful to
%    Mike Miles for having brought this passage
%    to my attention (\citeyear{mile-march1999}).}
%\end{quote}
%This passage provides an important historical precursor to the Poona operation. 
%It also raises interesting questions of its own. What did Manucci mean by
%`surgeons'?  Was he referring to practitioners of the `barber-surgeon' type, or to
%āyurvedic vaidyas? A late nineteenth-century account of the operation suggests
%that the technique was also known amongst Muslim practitioners of the
%barber-surgeon type \citep[352--3]{thor-bann}. Note also that the operative
%details, especially the removal of skin from the cheek rather than the forehead,
%are those of \emph{The Compendium of Suśruta}, rather than that of the Poona
%surgeon.  It remains an important historical problem to discover the time at
%which, in the long period between Suśruta and Manucci, the new mode of 
%performing
%this operation developed.
%
%
%
%The technique used by the Bijapur and Poona surgeons was similar, but not
%identical, to that described in \emph{The Compendium} (see translation,
%p.\,\pageref{su:rhinoplasty}).  \emph{The Compendium}'s version has the skin 
%flap being taken
%from the patient's cheek: Cowasjee's was taken from his forehead, in the same
%manner as that of the Bijapur patients.  The Sanskrit text of Suśruta's
%description is brief, and does not appear to be detailed enough to be followed
%without an oral commentary and practical demonstration, although an 
%experienced
%surgeon might be able to discern the technique even so.  However, no surviving
%manuscript of the text contains any illustration. In fact, there appears to be no
%tradition of anatomical or surgical manuscript illustration in India at all before
%about the eighteenth century. It is hard to see how such techniques could have
%persisted purely textually.
%
%Perhaps the Bijapur and Poona operations were indeed
%extraordinary survivals of a technique from Suśruta's time, but in
%that case it seems to have been transmitted through channels
%outside the learned practice of traditional Indian physicians.
%And it remains an important historical problem to discover just
%when, in the long centuries between Suśruta and Manucci, the new
%mode of performing this operation developed.
%
%
