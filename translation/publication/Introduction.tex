% !TeX root = surgery.tex
%\subsection{Preliminaries}

\subsection{Aim of the Article}

\subsection{Importance of SS.1.16 in the History of Medicine}


Simple forms of surgery have a long history in South Asia. In works datable to at
least 1200 \BC\ we learn how a reed was used as a catheter to cure urine
retention.\footcite[70--71]{zysk-1985} Cauterization too was used to prevent
wounds from bleeding. The \emph{Brāhmaṇa} literature of the early first 
millennium
\BC\ contains more detailed descriptions of animal butchery in the context of
religious sacrifice that involved the enumeration of internal organs and
bones.\footcite{mala-1996}   This exemplifies an early Sanskrit vocabulary for
internal parts of bodies, but it is not the same as medical dissection, whose
methods and purpose is quite different.\footnote{\citet{keit-1908} pointed out
that the enumeration of the bones in the Brāhmaṇas was derived from
correspondences with the numbering of various verse forms, not from anatomical
observation.}  With \emph{The Compendium of Suśruta} (\SS) we find ourselves 
in
the presence of something quite different and more developed, in which the body
was studied specifically for medical and surgical purposes.\footcite{zysk-1986} 
The \emph{Compendium} gives us a historical window onto a school of
professionalised surgical practice which existed almost two millennia ago, and
which in its day was perhaps the most advanced school of surgery in the world.


%Caraka too has brief descriptions of surgical techniques, but Suśruta goes
%into much greater detail, 

\emph{The Compendium of Suśruta} describes how a surgeon should be trained 
and how
various operations should be done.  There are descriptions of ophthalmic couching
(the dislodging of the lens of the eye), perineal lithotomy (cutting for stone in
the bladder), the removal of arrows and splinters, suturing, the examination of
dead human bodies for the study of anatomy, and other
procedures.\footnote{\cites{mukh-1913,desh-2000,nara-2011,wuja-2003,wils-1823}
 and
many other studies.} The author of \emph{The Compendium} claimed that 
surgery is
the most ancient and most efficacious of the eight branches of medical knowledge
(\Su{1.1.15--19}{4}). Discussion with contemporary surgeons has convinced me 
that
many details in the descriptions could only have been written by a practising
surgeon: it is certain that elaborate surgical techniques were a reality in
the author's circle.

I have argued elsewhere that in spite of Suśruta's elaborate descriptions, there
is little historical evidence to show that these practices persisted beyond the
time of the composition of \emph{The Compendium}.\footcite{wuja-indi} A few
references to surgery found in Sanskrit literature between the fourth to the
eighth centuries \AD\ were collected by
\citeauthor{shar-1972}.\footcite[74--8]{shar-1972}  But the stereotypical nature of
most of these references and the paucity of real detail, suggest that the practice
of surgery was rare in this period.

There is some evidence, however, that although surgery ceased to be part of
the professional practice of traditional physicians of the \emph{vaidya}
castes, it migrated to practitioners of the `barber-surgeon' type.  As such,
it was no longer supported by the underpinning of Sanskrit literary
tradition, and so it is harder to find historical data about the
practice. \citet{sirc-raks} discussed some epigraphical evidence for the
heritage and migrations of the `Ambaṣṭha' caste, who appear to have
functioned as barber-surgeons in South India and later migrated to Bengal.
There is also evidence from the eighteenth century of the practice of
smallpox inoculation by traditional 
`\emph{ṭīkādar}s.'\footcite{holw-acco,coul-acco} And some other surgical 
techniques which sound
similar to those described in \emph{The Compendium}, for example for removing 
ulcers, were
observed in the same period.\footcite[79]{babe-scie}

% and the letter of
% 1731 cited by \citet[141\,f., 276]{dhar-indi}.}

While the theoretical aspects of surgery continued to appear in those
medical textbooks that tried to be comprehensive, in practice those who
applied the surgical techniques seem to have been increasingly isolated from
mainstream of āyurvedic practice. It may be that as the caste system grew in
rigidity through the first millennium \AD, taboos concerning physical
contact became almost insurmountable and \emph{vaidya}s seeking to enhance
their status may have resisted therapies that involved intimate physical
contact with the patient, or cutting into the body.  On the other hand,
against this hypothesis it may be argued that the examination of the pulse
and urine gained in popularity, as did massage therapies.

An example of this process may be the famous ophthalmic operation of couching 
for
cataract, which is first described in \emph{The Compendium}.\footcite[Well
outlined by][378--9]{majn-1975} A description of this operation survives in the
ninth-century \emph{Kalyāṇa\-kāraka} composed in eastern India by the Jaina 
author
Ugrāditya.\footcites[67, n.\,76]{meul-1984}[IIA, 151\,ff]{meul-hist}[366--368,
375--378]{shas-kaly}. This procedure, or one very similar to it, also reached
China, through transmission on the routes taken by traders and Buddhist
pilgrims.\footcites[132--48]{unsc-medi}{desh-1999,desh-2000} But by the 
beginning
of the twentieth century it was described by \citet{elli-indi} as long having been
carried out by traditional practitioners of the barber-surgeon type rather than by
physicians trained in the Sanskrit texts.

By the seventeenth century, foreign visitors to India began to remark on how
surgery was virtually non-existent in India. The French traveller
Tavernier\label{tavernier}, for example, reported in 1684 that once when the
King of Golconda had a headache and his physicians prescribed that
blood should be let in four places under his tongue, nobody could be found
to do it, `for the Natives of the Country understand nothing of
Chirurgery'.\footnote{\cite[1.2.103]{tave-trav}; see also
\cite[1.130]{slee-ramb}.}


\subsubsection{Torn ear lobes}

Suśruta's description of the repair of torn ear lobes is again unique for
its time.  \citeauthor{majn-1975} noted that `through the habit of
stretching their earlobes, the Indians became masters in a branch of surgery
that Europe ignored for another two thousand years'.\footcite[291]{majn-1975}  
The different types of
mutilated ear lobe which the \SS\ describes are not always easy to understand
from the Sanskrit: the illustrations supplied in Majno's text help
visualization.\footcites[290--291]{majn-1975}[reproduced with permission 
in][92--93]{wuja-2003}

%One of the subjects unfortunately not covered in the present book is Suśruta's 
%use
%of ants for suturing.\footnote{\Su{4.2.56}{412}.}  The technique, which is also
%described in the \emph{Carakasaṃhitā},\footnote{\Ca{6.13.188}{500}.} is to 
%bring
%the edges of the flesh to be joined close together, and then allow a large black
%ant to bite the join with its mandibles.  The ant's body is twisted off, and the
%head remains in place, clamping the join together.  This technique has been
%described in detail and illustrated by
%\citeauthor{majn-1975}.\footcite[304\,ff]{majn-1975}  Majno described how this
%method is also known from tribal societies in Brazil and the Congo. Most
%interestingly, he cited an entomologist's report of the technique being known in
%southern Bhutan, in the early 1970s.\footcite[307, n.\,298]{majn-1975}  The
%technique was known in the Islamic and European world through the famous and
%much-translated surgical text by the Iberian Arab scholar Albucasis
%(d.\,1013).\footcite[550--51]{spin-albu}  Majno noted that Albucasis knew the
%technique from Suśruta.  Although Majno demonstrated conclusively that the
%technique is practicable, it is interesting that both Suśruta and Albucasis
%referred to the technique as a matter of hearsay.


\subsubsection{Rhinoplasty}
\label{sec:rhinoplasty}

One of the best-known surgical techniques associated with \emph{The 
Compendium of
    Suśruta} is rhinoplasty, the repair or rebuilding of a severed nose. I described
the history of this operation elsewhere and translated the Sanskrit passage from
the vulgate edition of the \SS.\footnote{\cite[67--70, 99--100]{wuja-2003}. See
also \cite[IB, 327--328, note 186]{meul-hist} for further literature and
reflections.}  This fascinating technique is certainly old in South Asia, having
been witnessed by travellers from Marco Polo in the seventeenth century
onwards.\footcite[ii.301]{manu-stor} Many witnesses, including the most famous,
Cruso and Findlay,\footcite[883, 891\,f.]{cowasjee} describe an operation that
differs from \emph{The Compendium} in that it takes the grafting skin from the
forehead, not the cheek.  But the nineteenth-century account of
\citeauthor{thor-bann} is especially interesting, since the technique follows
\emph{The Compendium} exactly in taking flesh from the cheek, not the
forehead.\footcite[352--3]{thor-bann}

As noted by \citeauthor{meul-hist}, none of the known commentators -- Jejjaṭa,
Gayadāsa, Cakrapāṇi or Ḍalhaṇa -- explained the technique in any detail beyond
lexical glosses.\footnote{\cite[IB, 328]{meul-hist}. Ḍalhaṇa also noted that a
rather different version of the text, cast in śloka metre, was also known to him
from other sources (\Su{1.16.27--31}{81a}).  Ḍalhaṇa's variant bears a 
resemblance
to the description of the operation given in printed editions of the
\emph{Aṣṭāṅgahṛdayasaṃhitā} (\Ah{Utt.18.59--65}{841}).} %501 of 1902 ed.
This suggests that the commentators did not in fact know the technique at
first-hand. Perhaps by the late first millennium, the technique had moved into the
professional competence of barber-surgeons?  On the other hand, perhaps the
influence was in the other direction, and a technique known to practitioners
elsewhere in South Asia in the late first millennium was written into the text of
\emph{The Compendium}. The description consists of only five verses and they 
are
written in the Upendravajrā metre, which is different from the rest of the
chapter.  The description's appearance at the very end of the chapter, its
terseness, its ornate metre, and the paucity of the commentators' treatment could
all be taken as pointing in this direction.

%
%The famous `Indian rhinoplasty' operation is often cited as
%evidence that \emph{The Compendium}'s surgery was widely known in India 
%even up
%to comparatively modern times. This operation took place in March
%1793 in Poona and was ultimately to change the course of plastic
%surgery in Europe and the world. A Maratha named Cowasjee, who had
%been a bullock-cart driver with the English army in the war of
%1792, was captured by the forces of Tipu Sultan, and had his nose
%and one hand cut off.\footnote{A residual puzzle with this account
%is that `Cowasjee' is a Parsi name, not a Maratha one.} After a
%year without a nose, he and four of his colleagues who had
%suffered the same fate submitted themselves to treatment by a man
%who had a reputation for nose repairs. Unfortunately, we know
%little of this man, except that he was said in one account to be
%of the brick-maker's caste. Thomas Cruso (d.\,1802) and James
%Findlay (d.\,1801),\footnote{\citet{cowasjee} called the second
%surgeon `Trindlay' but this must be an error.
%\citet[37]{carp-acco} had `Findlay', and both surgeons are listed by 
%\citet[409, 411]{craw-roll}.} senior British surgeons in the
%Bombay Presidency, witnessed this operation (or one just like it).
%They appear to have prepared a description of what they saw,
%together with a painting of the patient and diagrams of the skin
%graft procedure.  These details, with diagrams and an engraving
%from the painting, were published at third hand in London in
%1794;\footnote{\citet[883, 891\,f.]{cowasjee}.}
%Fig.\,\ref{fig:cowasjee} shows the illustration that accompanied
%this article.  The key innovation was the grafting of skin from
%the site immediately adjacent to the repair-site, while keeping
%the graft tissue alive and supplied with blood through a
%connective skin bridge. Subsequently, through the publication by
%\citet{carp-acco} describing his successful use of the technique,
%this method of nose-repair gained popularity amongst British and
%European surgeons.
%
%Carpue received personal accounts of other witnesses to this
%operation, and others of the same ilk, which shed more light on
%this episode \citep[appendix II]{carp-acco}.  Carpue's chief
%informant in 1815 was Cowasjee's commanding officer,
%Lieutenant-Colonel Ward.  Ward described the surgeon not as a
%brick-maker, but as an `artist', whose residence was four hundred
%miles distant from Poona. Cowasjee was not the only patient: four
%friends who had suffered the same fate also underwent nose
%reconstruction by the same artist. Most interestingly, the
%understanding in Poona at the time of the operation was that this
%artist-surgeon, who also claimed expertise in repairing torn or
%split lips, was the only one of his kind in India, and that the
%art was hereditary in his family.
%
%Further evidence on this topic is given by the
%seventeenth-century traveller Niccolo Manucci
%(fl.\,1639--ca.\,1709), who described how Shah Jahan's soldiers
%in Kashmir in the 1630s customarily cut off people's noses as a
%form of punishment.\footcite[i.215--6]{manu-stor}.  Even more
%interestingly, Manucci described rhinoplasty operations which
%took place in Bijapur in about 1686:
%\begin{quote}
%    The surgeons belonging to the country cut the skin of the
%    forehead above the eyebrows and made it fall down over the wounds
%    on the nose. Then, giving it a twist so that the live flesh might
%    meet the other live surface, by healing applications they
%    fashioned for them other imperfect noses. There is left above,
%    between the eyebrows, a small hole, caused by the twist given to
%    the skin to bring the two live surfaces together. In a short time
%    the wounds heal up, some obstacle being placed beneath to allow
%    of respiration. I saw many persons with such noses, and they were
%    not so disfigured as they would have been without any nose at
%    all, but they bore between their eyebrows the mark of the
%    incision.\footnote{\cite[ii.301]{manu-stor}.  I am grateful to
%    Mike Miles for having brought this passage
%    to my attention (\citeyear{mile-march1999}).}
%\end{quote}
%This passage provides an important historical precursor to the Poona operation. 
%It also raises interesting questions of its own. What did Manucci mean by
%`surgeons'?  Was he referring to practitioners of the `barber-surgeon' type, or to
%āyurvedic vaidyas? A late nineteenth-century account of the operation suggests
%that the technique was also known amongst Muslim practitioners of the
%barber-surgeon type \citep[352--3]{thor-bann}. Note also that the operative
%details, especially the removal of skin from the cheek rather than the forehead,
%are those of \emph{The Compendium of Suśruta}, rather than that of the Poona
%surgeon.  It remains an important historical problem to discover the time at
%which, in the long period between Suśruta and Manucci, the new mode of 
%performing
%this operation developed.
%
%
%
%The technique used by the Bijapur and Poona surgeons was similar, but not
%identical, to that described in \emph{The Compendium} (see translation,
%p.\,\pageref{su:rhinoplasty}).  \emph{The Compendium}'s version has the skin 
%flap being taken
%from the patient's cheek: Cowasjee's was taken from his forehead, in the same
%manner as that of the Bijapur patients.  The Sanskrit text of Suśruta's
%description is brief, and does not appear to be detailed enough to be followed
%without an oral commentary and practical demonstration, although an 
%experienced
%surgeon might be able to discern the technique even so.  However, no surviving
%manuscript of the text contains any illustration. In fact, there appears to be no
%tradition of anatomical or surgical manuscript illustration in India at all before
%about the eighteenth century. It is hard to see how such techniques could have
%persisted purely textually.
%
%Perhaps the Bijapur and Poona operations were indeed
%extraordinary survivals of a technique from Suśruta's time, but in
%that case it seems to have been transmitted through channels
%outside the learned practice of traditional Indian physicians.
%And it remains an important historical problem to discover just
%when, in the long centuries between Suśruta and Manucci, the new
%mode of performing this operation developed.
%
%
