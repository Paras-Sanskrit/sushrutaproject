% !TeX root = surgery.tex
\subsection{Preliminaries}

\subsubsection{Aim of the Article}

\subsubsection{Importance of 1.16 in the History of Medicine}


Simple forms of surgery have a long history in South Asia. In works datable to at
least 1200 \BC\ we learn how a reed was used as a catheter to cure urine
retention.\footcite[70--71]{zysk-1985} Cauterization too was used to prevent
wounds from bleeding. The \emph{Brāhmaṇa} literature of the early first 
millennium
\BC\ contains more detailed descriptions of animal butchery in the context of
religious sacrifice that involved the enumeration of internal organs and
bones.\footcite{mala-1996}   This exemplifies an early Sanskrit vocabulary for
internal parts of bodies, but it is not the same as medical dissection, whose
methods and purpose is quite different.\footnote{\citet{keit-1908} pointed out
that the enumeration of the bones in the Brāhmaṇas was derived from
correspondences with the numbering of various verse forms, not from anatomical
observation.}  With \emph{The Compendium of Suśruta} (\SS) we find ourselves 
in
the presence of something quite different and more developed, in which the body
was studied specifically for medical and surgical purposes.\footcite{zysk-1986} 
The \emph{Compedium} gives us a historical window onto a school of
professionalised surgical practice which existed almost two millennia ago, and
which in its day was perhaps the most advanced school of surgery in the world.


%Caraka too has brief descriptions of surgical techniques, but Suśruta goes
%into much greater detail, 

\emph{The Compendium of 
    Suśruta} described how a surgeon should be trained and
how various operations should be done.  There are descriptions of
ophthalmic couching (the dislodging of the lens of the eye), perineal
lithotomy (cutting for stone in the bladder), the removal of arrows and
splinters, suturing, the examination of dead human bodies for the study of
anatomy, and much 
else.\footnote{\cites{mukh-1913,desh-2000,nara-2011,wuja-2003,wils-1823} and 
many other studies.}  
Suśruta 
claims that surgery is the most ancient
and most efficacious of the eight branches of medical knowledge
(1.1.15--19). Many details in his descriptions could only have been written
by a practising surgeon: it is certain that elaborate surgical techniques
were a reality in Suśruta's circle.

% Ear lobe operations\ldots

I have argued elsewhere that in spite of Suśruta's elaborate
descriptions, there is little historical evidence to show that
these practices persisted beyond the time of the composition of
Suśruta's \emph{Compendium} \citep{wuja-indi}. A few  references
to surgery found in Sanskrit literature between the fourth to the
eighth centuries \AD\ were collected by
\citet[74--8]{shar-indi}.  But the stereotypical nature of most
of these references, and the paucity of real detail, suggests
that the practice of surgery was rare in this period.

There is some evidence, however, that although surgery ceased to be part of
the professional practice of traditional physicians of the \emph{vaidya}
castes, it migrated to practitioners of the `barber-surgeon' type.  As such,
it was no longer supported by the underpinning of Sanskrit literary
tradition, and so it becomes harder to find historical data about the
practice. \citet{sirc-raks} discusses some epigraphical evidence for the
heritage and migrations of the `Ambaṣṭha' caste, who appear to have
functioned as barber-surgeons in South India and later migrated to Bengal.
There is also evidence from the eighteenth century of the practice of
smallpox inoculation by traditional `\emph{ṭīkādar}s'
\citep{holw-acco,coul-acco}. And some other surgical techniques which sound
similar to those described in Suśruta, for example for removing ulcers, were
observed in the same period
\citep[79]{babe-scie}.

% and the letter of
% 1731 cited by \citet[141\,f., 276]{dhar-indi}.}

While the theoretical aspects of surgery continued to appear in those
medical textbooks which tried to be comprehensive, in practice those who
applied the surgical techniques seem to have been increasingly isolated from
mainstream of āyurvedic practice. It may be that as the caste system grew in
rigidity through the first millennium \AD, taboos concerning physical
contact became almost insurmountable and \emph{vaidya}s seeking to enhance
their status may have resisted therapies that involved intimate physical
contact with the patient, or cutting into the body.  On the other hand,
against this hypothesis it may be argued that the examination of the pulse
and urine gained in popularity, as did massage therapies.

An example of this process may be the famous ophthalmic operation
of couching for cataract, which is first described in Suśruta's
\emph{Compendium} \citep[well outlined by][378--9]{majn-heal}. A
description of this operation survives in the ninth-century
\emph{Kalyāṇa\-kāraka} composed in eastern India by the Jaina
author Ugrāditya \citep[67, n.\,76]{Meulenbeld1984}. This
procedure, or one very similar to it, also appears to have reached
China, but probably through transmission by Buddhist pilgrim
monks, rather than trained Indian physicians
\citep[132--48]{unsc-medi}. But by the beginning of the twentieth
century it was described by \citet{elli-indi} as long having been
carried out by traditional practitioners of the barber-surgeon
type rather than by physicians trained in the Sanskrit texts.

By the seventeenth century, foreign visitors to India began to remark on how
surgery was virtually non-existent in India. The French traveller
Tavernier\label{tavernier}, for example, reported in 1684 that once when the
King of Golconda had a headache and his physicians prescribed that
blood should be let in four places under his tongue, nobody could be found
to do it, `for the Natives of the Country understand nothing of
Chirurgery'.\footnote{\citet[1.2.103]{tave-trav}; cf.\ also
\citet[1.130]{slee-ramb}.}

The famous `Indian rhinoplasty' operation is often cited as
evidence that Suśruta's surgery was widely known in India even up
to comparatively modern times. This operation took place in March
1793 in Poona and was ultimately to change the course of plastic
surgery in Europe and the world. A Maratha named Cowasjee, who had
been a bullock-cart driver with the English army in the war of
1792, was captured by the forces of Tipu Sultan, and had his nose
and one hand cut off.\footnote{A residual puzzle with this account
is that `Cowasjee' is a Parsi name, not a Maratha one.} After a
year without a nose, he and four of his colleagues who had
suffered the same fate submitted themselves to treatment by a man
who had a reputation for nose repairs. Unfortunately, we know
little of this man, except that he was said in one account to be
of the brick-maker's caste. Thomas Cruso (d.\,1802) and James
Findlay (d.\,1801),\footnote{\citet{cowasjee} calls the second
surgeon `Trindlay' but this must be an error.
\citet[37]{carp-acco} has `Findlay', and both surgeons appear in
\citet[409, 411]{craw-roll}.} senior British surgeons in the
Bombay Presidency, witnessed this operation (or one just like it).
They appear to have prepared a description of what they saw,
together with a painting of the patient and diagrams of the skin
graft procedure.  These details, with diagrams and an engraving
from the painting, were published at third hand in London in
1794;\footnote{\citet[883, 891\,f.]{cowasjee}.}
Fig.\,\ref{fig:cowasjee} shows the illustration that accompanied
this article.  The key innovation was the grafting of skin from
the site immediately adjacent to the repair-site, while keeping
the graft tissue alive and supplied with blood through a
connective skin bridge. Subsequently, through the publication by
\citet{carp-acco} describing his successful use of the technique,
this method of nose-repair gained popularity amongst British and
European surgeons.

Carpue received personal accounts of other witnesses to this
operation, and others of the same ilk, which shed more light on
this episode \citep[appendix II]{carp-acco}.  Carpue's chief
informant in 1815 was Cowasjee's commanding officer,
Lieutenant-Colonel Ward.  Ward described the surgeon not as a
brick-maker, but as an `artist', whose residence was four hundred
miles distant from Poona. Cowasjee was not the only patient: four
friends who had suffered the same fate also underwent nose
reconstruction by the same artist. Most interestingly, the
understanding in Poona at the time of the operation was that this
artist-surgeon, who also claimed expertise in repairing torn or
split lips, was the only one of his kind in India, and that the
art was hereditary in his family.

Further evidence on this topic is given by the
seventeenth-century traveller Niccolo Manucci
(fl.\,1639--ca.\,1709), who described how Shah Jahan's soldiers
in Kashmir in the 1630s customarily cut off people's noses as a
form of punishment \citep[i.215--6]{manu-stor}.  Even more
interestingly, Manucci described rhinoplasty operations which
took place in Bijapur in about 1686:
\begin{quote}
    The surgeons belonging to the country cut the skin of the
    forehead above the eyebrows and made it fall down over the wounds
    on the nose. Then, giving it a twist so that the live flesh might
    meet the other live surface, by healing applications they
    fashioned for them other imperfect noses. There is left above,
    between the eyebrows, a small hole, caused by the twist given to
    the skin to bring the two live surfaces together. In a short time
    the wounds heal up, some obstacle being placed beneath to allow
    of respiration. I saw many persons with such noses, and they were
    not so disfigured as they would have been without any nose at
    all, but they bore between their eyebrows the mark of the
    incision.\footnote{\citet[ii.301]{manu-stor}.  I am grateful to
    Mike Miles (\citeyear{mile-march1999}) for bringing this passage
    to my attention.}
\end{quote}
This passage provides an important historical precursor to the
Poona operation.  It also raises interesting questions of its own.
What did Manucci mean by `surgeons'?  Was he referring to
practitioners of the `barber-surgeon' type, or to āyurvedic
vaidyas?
%A late nineteenth-century account of the operation
%suggests that the technique was also known amongst Muslim
%practitioners of the barber-surgeon type
%\citep[352--3]{thor-bann}.
%Note also that the operative details, including the
%removal of skin from the forehead rather than the cheek, are
%those of the Poona surgeon, rather than of Suśruta.  It
%remains an important historical problem to discover the time at
%which, in the long period between Suśruta and Manucci, the new
%mode of performing this operation developed.


\begin{sloppypar}
    The technique used by the Bijapur and Poona surgeons was similar,
    but not identical, to that described in Suśruta's
    \emph{Compendium} (see translation,
    p.\,\pageref{su:rhinoplasty}).  Suśruta's version has the skin
    flap being taken from the patient's cheek: Cowasjee's was taken
    from his forehead, in the same manner as that of the Bijapur
    patients.  The Sanskrit text of Suśruta's description is brief,
    and does not appear to be detailed enough to be followed without
    an oral commentary and practical demonstration, although an
    experienced surgeon might be able to discern the technique even
    so.  However, no surviving manuscript of the text contains any
    illustration. In fact, there appears to be no tradition of
    anatomical or surgical manuscript illustration in India at all
    before about the eighteenth century. It is hard to see how such
    techniques could have persisted purely textually.
\end{sloppypar}

Perhaps the Bijapur and Poona operations were indeed
extraordinary survivals of a technique from Suśruta's time, but in
that case it seems to have been transmitted through channels
outside the learned practice of traditional Indian physicians.
And it remains an important historical problem to discover just
when, in the long centuries between Suśruta and Manucci, the new
mode of performing this operation developed.


\paragraph{Torn ear lobes}
Suśruta's description of the repair of torn ear lobes is again unique for
its time.  \citet[291]{majn-heal} notes that `through the habit of
stretching their earlobes, the Indians became masters in a branch of surgery
that Europe ignored for another two thousand years'.  The different types of
mutilated ear lobe which Suśruta describes are not always easy to understand
from the Sanskrit: illustrations from Majno's text are reproduced to help
visualization (pp.\,\pageref{fig:earlobes}, \pageref{fig:gandakarna}).

One of the subjects unfortunately not covered in the present book is
Suśruta's use of ants for suturing (Su.4.2.56).  The technique, which is
also described by Caraka (Ca.6.13.188), is to bring
the edges of the flesh to be joined close together, and then allow a large
black ant to bite the join with its mandibles.  The ant's body is twisted
off, and the head remains in place, clamping the join together.  This
technique has been described in detail and illustrated by
\citet[304\,ff]{majn-heal}.  Majno describes how this method is also
known from tribal societies in Brazil and the Congo. Most interestingly, he
cites an entomologist's report of the technique being known in southern
Bhutan, in the early 1970s \citep[307, n.\,298]{majn-heal}.  The technique
was known in the Islamic and European world through the famous and
much-translated surgical text by the Iberian Arab scholar Albucasis
(d.\,1013)
\citep[550--51]{spin-albu}.  Majno notes that Albucasis knew the technique
from Suśruta.  Although Majno demonstrates conclusively that the technique
is practicable, it is interesting that both Suśruta and Albucasis refer to
the technique as a matter of hearsay.

