% !TeX root = surgery.tex
\subsection{The Nepalese Version}
Andrey's contribution here

\subsection{Cakrapāṇidatta and Ḍalhaṇa's Versions}
The commentaries of Cakrapāṇidatta and Ḍalhaṇa, called the \emph{Bhānumatī} and \emph{Nibandhasaṅgraha} respectively, are based on similar versions of the \SS, both of which are significantly different to the Nepalese version. Ḍalhaṇa was aware of Cakrapāṇidatta's work and reiterated many of his predecessor's remarks, so the commentator's interpretation of the root text is largely consistent. 

Trikamajī Ācārya's edition of the \textit{Sūtrasthāna} of the \emph{Bhānumatī} \citep{acar-1939} duplicates the version of the \SS\ in his edition of the \emph{Nibandhasaṅgraha} \citep{vulgate}, except in a few obvious cases where Cakrapāṇidatta glosses a word or compound that is different to the one glossed by Ḍalhaṇa.\footnote{For example, in SS.1.16.18, Cakrapāṇidatta glosses \emph{rājasarṣapa} whereas Ḍalhaṇa glosses \emph{gaurasarṣapa}, and Ācārya reflects this in the root texts of the \emph{Bhānumatī} \citep[130]{acar-1939} and \emph{Nibandhasaṅgraha} \citep[79]{vulgate}.} The duplication of the root text creates the somewhat misleading impression that both commentators had an almost identical version of the \SS. However, there is evidence in SS.1.16 that this was not the case. For example, Ḍalhaṇa comments on four verses (1.16.11–14, \cite[78]{vulgate}) that Cakrapāṇidatta cites separately in his commentary \citep[128–129]{acar-1939}, introducing each one as 'some people read' (\emph{ke cit paṭhanti}). This clearly indicates that these verses were not in the version of the \SS\ upon which Cakrapāṇidatta was commenting, yet Ācārya includes them in the root text of the \emph{Bhānumatī}.

Also, Cakrapāṇidatta does not acknowledge or comment on some verses in the version of the \SS\ known to Ḍalhaṇa. Although it is possible that a commentator may not have remarked on a verse because its meaning was clear, in some cases the commentarial convention of citing the first words of a new verse or passage provides firmer ground for suspecting the absence of a verse in the root text. For example, the prose passage of SS.1.16.18 in the the \emph{Bhānumatī} \citep[130]{acar-1939}, which is SS.1.16.19 in the \emph{Nibandhasaṅgraha} \citep[79]{vulgate}, is followed by several verses that elaborate on the content of the prose passage, and both commentators introduce these verses and cite the opening words of the first verse before glossing specific terms. However, Cakrapāṇidatta does not introduce, cite or comment on the same verses as Ḍalhaṇa (SS.1.16.20–22ab, \cite[79]{vulgate}), and yet the first of the verses commented on by Ḍalhaṇa appears in the root text of Ācārya's edition of the \emph{Bhānumatī} (SS.1.16.19, \cite[130]{acar-1939}), and the others (SS.1.16.20–21ab) are included in parenthesis. A similar instance of this occurs at \emph{Bhānumatī} SS.1.16.31, where Ācārya includes a verse in parenthesis that was commented on by Ḍalhaṇa (SS.1.16.32, \cite[81]{vulgate}) but not by Cakrapāṇidatta. It appears that the manuscript on which Ācārya's edition of the \emph{Bhānumatī} was based does not include the root text.\footnote{See the section below on Ācārya's 1939 edition for details of the sources Ācārya used for this edition.} Therefore, the inclusion of SS.1.16.19–21ab and 31 in the root text of the \emph{Bhānumatī} is an unsubstantiated hypothesis. 

% Ḍalhaṇa 1.16.11–14
%The version of 1.16.11–14 known to Ḍalhaṇa \citep[78]{vulgate} has four verses (\emph{śloka}) at this point that are not in the Nepalese manuscripts. The additional verses iterate the types of joins required for ear flaps that are missing, elongated, thick, wide, etc. All four verses were probably absent in the version of the \emph{Suśrutasaṃhitā} known to Cakrapāṇidatta. He cites the verses separately in his commentary, the \emph{Bhānumatī} \citep[128–129]{acar-1939}, introducing each one as 'some people read' (\emph{ke cit paṭhanti}). However,  in Trikamajī Ācārya's edition of the \emph{Sūtrasthāna} of the \emph{Bhānumatī}, the root text is largely identical to the one commented on by Ḍalhaṇa (\cite{vulgate}), even in instances like this where Cakrapāṇidatta's commentary indicates that he was reading a different version of the \emph{Suśrutasaṃhitā}

% Ḍalhaṇa 1.16.19–20 
%Cakrapāṇidatta \citep[131]{acar-1939} does not comment on these verses, nor verse 15 of the Nepalese version, and so the version of the \emph{Suśrutasaṃhitā} known to him may not have included them.

% Ḍalhaṇa 1.16.32
 %Cakrapāṇidatta \citep[133]{acar-1939} does not comment on this additional verse, which suggests that either he did not know of it or was not inclined to accept it.
 
% Both commentators were aware of a version of the \SS\ that was similar to the Nepalese version % See blog.

In fact, there is some evidence that the Nepalese version was more similar to Cakrapāṇidatta's version than to Ḍalhaṇa's. For example, 1.16.5 of the Nepalese version begins with the compound \emph{doṣasamudayāt} whereas the version known to Ḍalhaṇa (SS.1.16.6, \cite[77]{vulgate}) inserts two compounds, \emph{kliṣṭajihmāpraśastasūcīvyadhāt} and \emph{gāḍhataravartitvāt}, before this. Cakrapāṇidatta (SS.1.16.5, \cite[126–127]{acar-1939}) begins his comment on this passage by glossing \emph{doṣasamudayāt}, which suggests that he was not aware of any compounds prior to this one. If one looks beyond SS.1.16, there are instances where the Nepalese version (1.1.28) and the root text of Cakrapāṇidatta have the same reading, which Ḍalhaṇa mentions as an alternative read by others. For example, 1.1.28 of the Nepalese version has \emph{tatrāsmiñ chāstre}, which is the reading commented on by Cakrapāṇidatta (\cite[17]{acar-1939}). However, Ḍalhaṇa  (SS.1.1.22, \cite[5]{vulgate}) comments on \emph{asmiñ chāstre} and states that others read \emph{tatrāsmiñ chāstre}. Also, in his commentary on SS.1.1.8.1, Ḍalhaṇa (\cite[5]{vulgate}) notes the variant reading \emph{ṣaṣṭyā vidhānaiḥ}, which is not in his root text but evidently was in Cakrapāṇidatta's  (SS.1.1.6, \cite[11]{acar-1939}). As discussed elsewhere (Birch 2021), the reading of \emph{ṣaṣṭyā vidhānaiḥ} is likely a corruption of \emph{ṣaṣṭyābhidhānaiḥ} in the Nepalese version (1.1.9). 




 
% Ḍalhaṇa was aware of the reading in the Nepalese version because he notes in his commentary on 1.16.6 \citep[77]{vulgate} that some read 'because of the accummulation of humours' rather than 'because of piercing with a painful, crooked and unrecommended needle or because of a wick that is too thick.' 

\subsection{Differences between the Nepalese and Subsequent Versions of SS.1.16}

The structural differences between the Nepalese and subsequent versions has been discussed by \citet[27–44]{kleb-2021b}, which include the frame story,\footnote{On this topic, also see the more recent \citet{birc-2021}.} the name of the first book (\emph{sthāna}), the structuring of the text according to chapter and section colophons, and an additional passage in the \emph{Kalpasthāna}. \citet[44–55]{kleb-2021b} also makes general observations on distinct features of the Nepalese version's content and looks specifically at lists of skin lesions arising from urinary disease and vital energies. And in an effort to demonstrate the possibility of greater coherence in the Nepalese version, \citet[101–104]{hari-2011} has compared its classification of snakes with Ḍalhaṇa's version. 

On the whole, these observations indicate that [...synopsis of general conclusions here, Andrey?...]
% 1.16 is missing yathovāca bhagavān dhanvantariḥ|

The following detailed comparison of 1.16 of the Nepalese version with Ḍalhaṇa's \emph{Nibandhasaṅgraha} unfolded as the chapter was edited. The differences appear to emanate largely from attempts to improve or standardise the language of the Nepalese version, add and redact information, and introduce changes to recipes and treatments. Examples have been provided to demonstrate the general observations, and further examples can be found in the footnotes to the translations. 

\subsubsection{Standardising or Improving the Language}

%Sandhi
% 2 °hastena ṛju  →  °hastena rju 

% Neither here nor there
% 2
% vyadhayet → vidhyete (perhaps, picking up on karnau)
% māse → māsi (shift from māsa to mās. Can't see a reason)
% kṛtamaṅgalaṃ svastivācanan → kṛtamaṅgalasvastivācanan (the latter makes better sense, but could have been original, in my opinion, or an attempt to better integrate a gloss that had become part of the text.)
% abhisāntvayamānaḥ → abhisāntvayan (shift from PPP to PAP) % It seems only the latter is correct in the given context. So, it could be just an error in the NV. Emend or Change translation!!! 

\subsubsection{Augmenting the Text}

% Supplementary compounds and phrases for Adding Information

% 
% 1. karṇavyadhavidhim → karṇavyadhabandhavidhim (discussed in footnote)
% 2
% dhātryaṅke → dhātryaṅke kumāradharāṅke vā (discussed in footnote)
% upaveśyābhisāntvayamānaḥ → upaveśya bālakrīḍanakaiḥ pralobhyābhisāntvayan (discussed in footnote)
% chidre → chidra ādityakarāvabhāsite (discussed in footnote)
% [absent] → śanaiḥ śanaiḥ

% Additional Verses and Passages
% Transposing Verses

% Transposing Passages
% 2. pūrvan dakṣiṇaṃ kumārasya vāmaṅ kanyāyāḥ | pratanuṃ sūcyā bahalam ārayā || → pratanukaṃ sūcyā bahalam ārayā || pūrvaṃ dakṣiṇaṃ kumārasya vāmaṅ kanyāyāḥ ||

%
\subsubsection{Redacting Recipes and Treatments}


