% !TeX root = surgery.tex
\chapter{The Printed Editions}

The careful survey of printed editions of the \SS\ by Meulenbeld lists no fewer than 
44 entries.\footcite[IIB, 311--314]{meul-hist}  These range from the first edition 
by 
Madhusūdana Gupta (\citeyear{gupt-1835}) to editions in the 1970s. The 
number of 
reprints and editions since that time might almost double that number.  
Translations began with Hessler's Latin translation in \citeyear{hess-1855} and 
continue up to the present in scores of publications in many 
languages.\footcites[E.g.,][]{zysk-1984}[IIB, 314--315]{meul-hist}

\section{The vulgate}
\label{vulgate}

The great ayurvedic scholar Yādavaśarman Trivikrama Ācārya produced three
successive editions of the \SS\ with the commentary of Ḍalhaṇa, in 1915, 1931 and
1938.  These editions, especially the last, are generally considered the most
scholarly and reliable editions of the work, and have been constantly reprinted up
to the present day.\footnote{See also the studies of these editions by
    \textcites[\S 1.2]{kleb-2021b}[143--144]{wuja-2013}.}  We refer to the last of
    these editions as “the vulgate.”

The 1915 edition was based on three manuscripts.  The 1931 edition used another
seven manuscripts plus two printed editions.  For his final 1938 edition, Ācārya
used a further three manuscripts.\footnote{The following account is
paraphrased from \citeauthor{vulgate}'s own account of their sources
\citep[22]{vulgate}.}  These sources were described  by Ācārya as follows; we provide 
an overview in Table~\ref{tableofeds}.

\subsection{The sources of the 1915 edition}

\begin{enumerate}
    \item[1] Calcutta, Royal Asiatic Society.  Covers the \emph{sūtra, nidāna, śārīra 
        and 
        kalpa sthāna}s.  
    
    \item [2] Jaipur, Pandit Gaṅgādharabhaṭṭaśarman, lecturer at the Royal 
    Sanskrit University.  Covers the \emph{cikitsāsthāna} and the 
    \emph{uttaratantra}.
    
    \item [3]  Bundi, my great friend the royal physician Paṃ.\ Śrīprasādaśarman  
    Covers the \emph{uttaratantra}.
\end{enumerate}
%
\subsection{The sources of the 1931 edition}

\begin{enumerate}
   \item[1] Vārāṇasī, professor of literature, the great Gaurīnāthapāṭhaka.  With 
    the 
    \emph{Nibandhasaṅgraha}. Covers the \emph{nidānasthāna} and 
    \emph{uttaratantra}.
    
    \item [2]  Ahmedabad.  My friend Sva.\ Vā.\ Vaidya Raṇachoḍalāla 
    Motīlālaśarman.  
    With the \emph{Nibandhasaṅgraha}.  Covers the \emph{śārīrasthāna}.
    
    \item [3] From the personal library of my great friend Sva.\ Vā.\ Vaidya
    Murārajīśarman. Extremely old. No commentary.  Covers the 
    \emph{śārīrasthāna}.
    
    \item [4]  Puṇe, BORI library.  With the \emph{Nibandhasaṅgraha}. Covers the
    \emph{śārīrasthāna}.\footnote{Not one of the three MSS of the
    \emph{śārīrasthāna} described in \cite{shar-vaid}.}
    
    \item [5]  Puṇe, BORI library.  With the \emph{Nibandhasaṅgraha}. Complete.  
    With some damaged folia.
    
    \item [6]  Bombay, Asiatic Society.  Incomplete.\footnote{Possibly 
    \MScite{Mumbai 
    AS B.I.3} or \MScite{Mumbai AS B.D.109} \citep[v.\,1, \# 212 and 
    213]{vela-1930}.  But both these have the \emph{Nibandhasaṅgraha}.  The 
    first 
    covers only the \emph{śārīrasthāna}; the second may be complete, but 
    Velankar calls it 
    only “disorderly.”}
    
    \item [7] Varanasi, the private library of Vaidya Tryambakaśāstrī.  Covers the 
    \emph{cikitsāsthāna}.  The variant readings of this MS were compiled by Prof.\ 
    %    Guruprasādaśāstrī and supplied to Ācārya.
    
    \item [8]  A printed edition together with the commentary 
    \emph{Suśrutasandīpanabhāṣya} by Professor Hārāṇacandra Cakravārtti. 
    Complete work.
    This is the 1910 Calcutta edition numbered “t” by Meulenbeld.\footcites[IB, 
    312]{meul-hist}{bhat-1917}
    
    \item [9] A printed edition of the first 43 chapters of the
    \emph{sūtrasthāna}, printed in Bengali script, with the commentaries
    \emph{Bhānumatī}, \emph{Nibandhasaṅgraha}, edited by Vijayaratnasena and
    Niśikāntasena. This is the 1886 Calcutta edition numbered “g” by 
    Meulenbeld.\footcites[IB,
    311]{meul-hist}{sena-1886}
\end{enumerate}
%

\subsection{The sources of the 1938 edition}
% \coffeestainC{1}{1}{180}{0}{-5 mm}
\begin{enumerate}
    \item [1]  Gwalior, from the library of my great friend Paṃ.\ Rāmeśvaraśāstrin 
    Śukla. 
    Covers the \emph{sūtra, nidāna, śārīra, cikitsā and kalpasthāna}s.
    
    \item[2] Bikaner, from the library of the Royal Palace, supplied by Paṃ.\ 
    Candraśekharaśāstrin. Contains the commentary 
    \emph{Nyāyacandrikāpañjikāvyākhyā} by Gayadāsa.  Covers the 
    \emph{nidānasthāna}.      
    This is almost certainly \MScite{Bikaner Anup 
        4390}.\footnote{See Dominik Wujastyk, “MS Bīkāner AnupLib 4390.” 
    \emph{Pandit}. 
    <\url{http://panditproject.org/entity/108068/manuscript}>.}
    
    \item [3] Kathmandu, located in the private library of the Royal Guru Hemarāja 
    Śarman.  An extremely old palm-leaf manuscript. Readings from this MS were 
    compiled by Paṃ Nityānandaśarman Jośī and sent to Ācārya. Covers from the 
    beginning of the work to the end of the ninth chapter of the 
    \emph{cikitsāsthāna}.  
    
    The siglum for this manuscript in footnotes was \dev{tā} for 
    \dev{tālapatrapustake}. 
\end{enumerate}
\begin{table}
    \centering
      %  \vspace{.5\baselineskip}
        \begin{tabular}{c|ccc|ccccccccc|ccc}
        \toprule
        %        \multicolumn{16}{c}{\emph{Manuscripts (\newmoon) and print 
        %editions 
        %                ($\circ$)}} \\
        \emph{edition}            &\multicolumn{3}{c}{1915}
        &                \multicolumn{9}{c}{1931} 
        &              \multicolumn{3}{c}{1938} \\
        
        \emph{source}         & 1 & 2 & 3 & 1 &2  &3  &4  &5  &6  &7  &8  &9  &1  
        &2 &3 \\
        \midrule
        \emph{sthāna} &&&&&&&&&&&&&&&\\        
        \emph{sū}. &  \newmoon&  &  &
        &  &  &  & \newmoon & ? &  & $\circ$ & 
        $\circ^\dag$ &  
        \newmoon & &\newmoon \\
        
        \emph{ni}. &\newmoon  &  &  &
        \newmoon &  &  &  &  \newmoon&  ?&  & $\circ$ &  &  
        \newmoon&\newmoon & \newmoon\\
        
        \emph{śā}. &  \newmoon&  &  &
        & \newmoon & \newmoon & \newmoon & \newmoon &  ? &  &  
        $\circ$&  &  
        \newmoon& &\newmoon \\
        
        \emph{ci}. &  & \newmoon &  &
        &  &  &  &\newmoon & ? &  \newmoon&$\circ$  &  &
        \newmoon & &\newmoon$^{\dag\dag}$ \\
        
        \emph{ka}.  &\newmoon  &  &  &
        &  &  &  &\newmoon  &  ?&  & $\circ$ &  &  
        \newmoon  & & \\
        
        \emph{utt}.  &  & \newmoon &\newmoon  &
        \newmoon  &  &  &  & \newmoon & ? &  & $\circ$ &  &  
        & & \\
        \bottomrule
    \end{tabular}
    \medskip
    {\small\\
        $^{\dag}$    Covers chapters 1--43 only. \quad
        $^{\dag\dag}$ Covers chapters 1--9 only.
        \par}
    \caption{The sources of Yādavaśarman T. 
        Ācārya's  three editions:\\ manuscript coverage (\newmoon) and print coverage
        ($\circ$). \label{tableofeds}}
\end{table}  
% notes to table 2.
%    \addtocounter{footnote}{-1}
%    \footnotetext{Covers chapters 1--43 only.}
%    \stepcounter{footnote}
%    \footnotetext{Covers chapters 1--9 only.}
%
\subsection{Evaluation}

Estimates show that there are approximately 230 extant manuscript witnesses for
the \emph{Suśrutasaṃhitā}.\footnote{This figure is arrived at by summing the MSS
    mentioned by \tvolcite{39}[373--375]{ncc} and in the \cite{ngmcp}. The real figure
    could be many scores higher.  Cf.\ the overview at
    \cite{wuja-2020}.\label{SSmss}}  Although most of these manuscripts cover only
    parts of the whole work, they amount to approximately twenty times the evidence
    that was used by Ācārya for his vulgate editions.

While the descriptions provided by Ācārya of his source materials seems at
first to be moderately comprehensive, Table~\ref{tableofeds} reveals the
underlying paucity of textual sources for these editions.  At first, it
appears that fifteen manuscripts were consulted.  However, we quickly see that
two of the sources were other people's printed editions, and one of those
covered less than a quarter of the work (no.\,9 of 1931).  That reduces the
manuscript base to 13 witnesses. Ācārya does not appear to have seen two of
the manuscripts at all, having been sent collations prepared for him by others
(7 of 1931 and 3 of 1938).  Thus, Ācārya's final edition was based on the
personal consultation of eleven partial manuscripts.   One of them remains
unidentified (6 of 1931). Only a single manuscript covers the whole of the
\emph{Suśrutasaṃhitā}, no.\,5 of the 1931 edition.  Manuscript 1 of 1938 is
the next most complete, but it omits the \emph{uttaratantra}, which comprises
a third of the work.  Manuscript 1 of the 1915 edition is third in size, but
it still omits both of the longest chapters, and thus offers less than half
the work.  For the rest, the evidence is spotty, with each part of the work
being supported by only between four and eight manuscripts, excluding the
printed editions.

Two sources stand out for their historical importance.  The first is no.\,3 of
1931, which Ācārya calls “extremely old.”  It covered the \emph{śārīrasthāna}
only, and unfortunately we know nothing of the later history of this manuscript.
The second is no.\,3 of 1938, which is one of the important Nepalese manuscripts
being considered in the present project. Ācārya's remarks and references to
Hemarājaśarman's introduction to the \emph{Kāśyapasaṃhitā} allow us to identify
this manuscript as \MScite{Kathmandu NAK
    5-333}.\footnote{\cites[22]{vulgate}[56--57]{hema-1938}. Discussed by \citet[\S
    1.1, 2.3]{kleb-2021b}.  See also \cites[IIB,
    25--41]{meul-hist}[161--169]{wuja-2003}.} The editors of the vulgate,
    \citeauthor{vulgate}, stated that this manuscript covered up to the ninth chapter
    of the \emph{cikitsāsthāna}, but in fact it covers the whole
    work.\footcite[22]{vulgate}  Perhaps the editors only received collations for this
    portion of the manuscript and did not know that it was a witness for the whole
    work.

\section{The 1939 edition}        
\label{1939edition}

In 1939, Yādavaśarman Trivikrama Ācārya and Nandakiśora Śarman co-edited an
edition of the \emph{sūtrasthāna} of the \emph{Suśrutasaṃhitā} that was 
published
by the Swami Laxmi Ram ayurvedic centre in Jaipur, and printed at the famous
Nirṇayasāgara Press in Mumbai (see 
Fig.\,\ref{bhanumati}).\footnote{\cite{acar-1939}.  The description of 
the sources 
below is based on Yādavaśarman T. Ācārya's  remarks in his introduction 
(pp.\,3--4). See also the remarks on this edition by
\citet[7]{kleb-2021a}.  On the Swami Laxmi Ram
centre, see \cite{hofe-2007}} The text was edited on the basis of the following 
sources.

\begin{figure}[p]
    \centering
    \includegraphics[draft=false,height=.9\textheight]{media/Bhanumati-page-11upscaled}
    \caption{A page of the 1939 \emph{Bhānumatī} edition, showing the variant 
        readings in the footnotes.}
    \label{bhanumati}
\end{figure}


\subsection{For the Bhānumatī}

\begin{enumerate}
    \item A printed edition.  Covered the \emph{Bhānumatī} up to chapter Su.sū.40.
    The siglum was \dev{mu} for \dev{mudrita}.\footnote{\cite{sena-1886}.  
    The
    manuscript on which this edition was based is probably in the library of the
    Calcutta Sanskrit College, and described in \cite[v.\,X.1]{sast-1917}, which
    is not available to me.  See also \cite[IB, 495, n.\,57]{meul-hist} for
    mention of this manuscript.  The reference at \cite[217]{rao-sans} to CSCL
    accession number 97 in Bengali script may be this manuscript.}
    
    \item A manuscript in the India Office Library library provided through the
    Bhandarkar Oriental Research Institute in Pune.\footnote{At this time,
    manuscripts from Britain were routinely lent to scholars in India and vice
    versa.} This manuscript covered the \emph{Bhānumatī} up to the end of the
    \emph{sūtrasthāna}.  The siglum was \dev{ha} for
    \dev{hastalikhita}.\footnote{\cite{PP109978}; 
    \MScite{London BL H. T. Colebrooke 908}
    (\href{panditproject.org/entity/109978/manuscript}{PanditProject \#109978},
    consulted on July 03, 2021).}
\end{enumerate}

\subsection{For the Suśrutasaṃhitā}

\begin{enumerate}
    \item A palm leaf manuscript from Hemarājaśarman's personal
    library.\footnote{I.e., \MScite{Kathmandu NAK 5-333}.}  The siglum was
    \dev{tā} for \dev{tāḍapatra}.
    
    \item His own published edition. The siglum was \dev{ḍa} for 
    \dev{ḍalhaṇasaṃmataḥ
        pāṭhaḥ}.\footnote{\cite{vulgate}.  It is noteworthy that Ācārya refers to
    his 1938 edition as representing “the Ḍalhaṇa version.”\label{dalhanaversion}}
    
    \item Hārāṇacandra Cakravarti's published edition with his own
    commentary.\footcite{bhat-1917} The siglum was \dev{hā}.
\end{enumerate}
%
\subsection{Evaluation}

The main innovation of this publication was to present the only surviving part
of the commentary on the \SS\ by the great eleventh-century medical scholar
Cakrapāṇidatta, namely the \emph{Bhānumatī}.\footcite[IA, 374--375 and IB,
495--496]{meul-hist} A secondary purpose was to present the text of the
\emph{sūtrasthāna} as read in \MScite{Kathmandu NAK 5-333}, that had recently
been brought to the editors' attention. In their judgement, the Kathmandu
manuscript presented a text that was closer to what Cakrapāṇidatta had before
him than the text according to Ḍalhaṇa.  In spite of this, the editors largely
reproduced the root text of Ḍalhaṇa's version. This was the first \SS\ edition
in which Ācārya used sigla to identify the sources from which variant readings
were reported, so while it has limitations, it for the first time enables us
to get some idea of origins of his readings at the level of individual words and 
sentences (see Figure~\ref{bhanumati}).

\label{ref:dalhana}Ācārya noted in his introduction that the manuscripts
containing  Ḍalhaṇa's commentary all came together with the root-text of the \SS,
and thus the main \SS\ text reflected the readings chosen by Ḍalhaṇa.  But the
manuscripts of the \emph{Bhānumatī} contained the commentary alone, without the
root-text, and had many explanations based on different readings of the root-text
than those of Ḍalhaṇa.  In many of these cases it was hard to infer what readings
Cakrapāṇidatta had before him. But Ācārya noted that Cakrapāṇidatta had a text
before him that had much in common with the text of the Nepalese
manuscript.\footnote{\cite[3--4]{acar-1939}.  See discussion by
    \citet[7]{kleb-2021a}.}

There is compelling evidence that Cakrapāṇidattas's \emph{Bhānumatī} commentary
once covered the whole text of the \SS.\footcite[IA, 375]{meul-hist}  The loss of
the rest of the work ranks amongst the greatest disasters in Āyurvedic literature.
Remarkably, the whole \emph{Bhānumatī} may still have existed in the early
twentieth century. In 1903, Palmyr Cordier reported being privately informed of a
complete copy of the work in a personal manuscript collection in
Benares.\footcite[332]{cord-1903}


