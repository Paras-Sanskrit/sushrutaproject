% !TeX root = incremental_SS_Translation.tex
\newcommand{\plant}[4]{#1 (\emph{#2})\footnoteA{#3; see #4}}
\let\chemical = \plant
\newcommand{\skt}[2]{#1 (\emph{#2})}
\newcommand{\sskt}[2]{\empty}
%
\newcommand{\diff}[1]{\textcolor{red}{#1}}

\section{Kalpasthāna, adhyāya 2}

\subsection{Introduction}

This section begins with several lists of poisonous plants.  The Sanskrit names
for these plants are mostly not standard or familiar from anywhere in Sanskrit or
ethnobotanical literature.  It remains a historical puzzle why these particular
names are so difficult to interpret.  
However, we 
are not the first to encounter
these difficulties. In the twelfth century, the learned commentator on the text,
Ḍalhaṇa, remarked,
\begin{quote}
In spite of having made the greatest effort, it has been impossible to identify
these plants. In the Himalayan regions, Kirātas and Śabaras are able to identify
them.\footnote{After \SS, \emph{kalpasthāna} 2.5 \citep[564]{vulgate}. From the
view of Sanskrit authors, Kirāṭas and Śabaras were tribal peoples.  The
eleventh-century author Bhikṣu Govinda, however, cast his treatise as a dialogue 
with a
Kirāṭa king called Madana who was a master of the alchemical art \citep[IIA,
620]{meul-hist}.}
\end{quote}
Ḍalhaṇa also recorded variant readings of these poison names from the 
manuscripts
that he consulted of the lost commentary of Gayadāsa (fl.\ c.\ \AD\ 1000).
The identities of these poisons have been in doubt for at least a
thousand years.\footnote{See \cite[80--81]{wuja-2003}.}  Identifications have in
many cases been equally impossible for us today.

One path for exploration in this situation is to attempt to reverse-engineer some 
identifications by considering the known toxic plants of India.\footnote{Valuable 
reference sources on Indian plant toxicology in general include 
\cite[chs.\,10, 11]{pill-2013} and \cite[parts 1.II, 3 and 4]{barc-2008}.}

%\subsection{Manuscript notes}

\subsection{Translation}

\begin{translation}
    
    \item[1]
    And now I shall explain \diff{what should be known} about stationary 
    poisons.\footnote{No reference is made to Dhanvantari 
    \citep[see][]{birc-2021}. “Stationary” here is a term contrasted with “moving,” 
    and signifies plants as opposed to animals and insects.}
  
    \item[3]
    \noindent It is said that there are two kinds of poisons,
    \se{sthāvara}{stationary} and \se{jaṅgama}{mobile}. The former
    dwells in ten sites, the latter in sixteen places.
   
    \item[4]
    Traditionally, the ten are: root, leaf, fruit, flower, bark,
    \se{kṣīra}{milky sap}, \se{sāra}{pith}, \se{niryāsa}{resin}, the
    elements (\emph{dhātu})\sse{dhātu}{element}, and the tuber.

    \item[5]
    
    In that context,\label{poisonousplants}
    \begin{itemize}
        \item
        the eight root-poisons are:\q{Expected \citep{pill-2010}:\\ Croton tiglium, L. 
        = Naepala, 
        Jayapala, kanakaphala, titteriphala (NL \#720);
    Calotropis spp.;\\ Citrullus colocynthus (colocynth);\\ Ricinus communis (castor); }
        \begin{enumerate}
        \item \plant{liquorice}{klītaka}{Glycyrrhiza glabra, L.}{AVS 3.84, NK 
        \#1136},\footnote{Liquorice eaten in excess can be poisonous.}
       
        \item \plant{sweet-scented oleander}{aśvamāraka}{Nerium oleander, 
        L.}{ADPS 223, NK \#1709},\footnote{The roots of sweet-scented oleander 
        are highly toxic, as are most parts of the plant \citep{pill-2019}.}
    
        \item \plant{jequirity}{guñjā}{Abrus precatorius, L.}{AVS 1.10, NK \#6, 
        Potter 168},\footnote{Jequirity does indeed contain a dangerous
toxin called Abrin in its seeds and to a lesser extent in its leaves,
but apparently not in its roots or bulb. Abrin is not harmful if eaten,
but an infusion of the bruised (not boiled) seeds injected or rubbed in
the eyes can be fatal \citep[\# 6]{NK}.  The dose can be quite small.}
        % \item \plant{java galangal}{sugandha}{Alpinia galanga, (L.) Willd.?}{AVS 
        %1.106,
        %   NK \#116},\footnote{No apparent toxicity.}
        % \item \plant{Indian sarsaparilla}{sugandha}{Hemidesmus indicus, (L.) R.
        % Br.?}{AVS 3.141, NK \#1210},\footnote{Non-toxic.}
        
        \item \diff{\plant{aconite}{subhaṅgurā}{\emph{$\rightarrow$ bhaṅgura =
            ativiṣā}? Aconitum ferox, Wall.\ ex Ser.}{NK \#38}},\footnote{The plant is
usually called just \emph{bhaṅgurā} without the prefix \emph{su-} “good.”}
        
%        \item \plant{rauwolfia}{sugandhā $\rightarrow$ sarpagandhā}{Rauvolfia
%            serpentina, (L.) Benth.
%            ex Kurz.?}{NK \#2099, ADPS 439; cf.\ Su.5.5.76--78},

        \item \diff{\emph{karaṭā}},\footnote{This poisonous root cannot at present
be identified.  Similar-sounding candidates include \emph{karkaṭaka},
\emph{karaghāṭa} (emetic nut), and \emph{karahāṭa}, but since this is a
prose passage, there would be no reason to alter the word to fit a metre.
\citet[255]{moni-sans} cite an unknown lexical source that equates
\emph{karaṭa} (mn.) with safflower (\emph{Carthamus tinctorius}, L.), but
this plant does not have a poisonous root.} %
%\plant{?}{karaṭā → karaghāṭa → karahāṭa?}{?}{?}},
%
%        \item \plant{luffa}{gargaraka $\rightarrow$ garāgarī?}{Luffa echinata,
%        Roxb.}{NK
%            \#1517},
%
%
%        \item \plant{emetic nut}{karaghāṭa $\rightarrow$ karahāṭa? $\rightarrow$
%            madana}{Randia dumetorum, Lamk.}{NK \#2091},
%
and ending with \item \plant{leadwort}{vidyutśikhā $\rightarrow$ agni- or
    rakta-śikhā?}{Plumbago zeylanica (or rosea?), L.}{NK \#1966, 
    1967},\footnote{The roots of both rose and white leadwort are very toxic.} 

\item
\diff{\plant{`endless'}{ananta}{?}{?}},\q{Note about Gayī's edition.} and 

\item
\emph{vijayā},\footnote{\citet[61, n.\,3]{meul-sear} argued that our text read a
masculine or neuter noun \emph{vijaya}, which never signifies cannabis.  
However,
unlike the vulgate, the unanimous readings of the Nepalese manuscripts give
feminine \emph{vijayā}.  Nevertheless, even this form only started to signify
\emph{Cannabis sativa} L. after the end of the first millennium
\citep{meul-sear,wuja-cann,mchu-2021a}. The \emph{Sauśrutanighaṇṭu} 
gives a number of synonyms for \emph{vijayā}, almost none 
of
which have any poisonous parts \citep[5.77,
10.143]{suve-2000}.  But one of them, \emph{viṣāṇī} (also
\emph{meṣaśṛṅgī}), is sometimes equated with \emph{Dolichandrone falcata 
(DC.)
Seemann} \citep[518]{adps}, a plant used as an abortifacient and fish poison
\citep[\#862]{nadk-1982}.  This identification is tenuous.}
%
%        \footnote{Large doses of the root-extract of rauwolfia can be fatal. 
%        
%        In large doses luffa is emetic and a drastic purgative. }
        \end{enumerate}
        \end{itemize}
    
        \item
        the leaf-poisons include:
             \begin{itemize}            
        \item \plant{`poison-leaf'}{viṣapatrikā}{unknown}{?},
        \item \diff{\plant{`drum-giver'}{lambaradā}{unknown}{?}},
%        \item \plant{`choice tree'}{varadāru}{unknown}{?},
        \item \plant{thorn apple}{karambha}{Datura metel, L.}{AVS 2.305
            (cf.\ Abhidhāna\-mañjarī), NK \#796\,ff., Potter 292\,f., ADPS 132.},
        and
        \item \plant{`big thorn apple'}{mahākarambha}{Datura metel, L.?}{AVS 
        2.305
            (cf.\ Abhidhāna\-mañjarī), NK \#796\,ff., Potter 292\,f., ADPS 132.};
            \end{itemize}

        \item
        the fruits of items like:
        \plant{jequirity}{guñjā}{}{},
        \plant{rūṣkara}{}{}{},
        \plant{viṣa}{}{}{}, and
        \plant{vedikā}{}{}{},
        are
                    \begin{itemize}
         \item \diff{\plant{kumudavati}{kumadavati}{unknown}{?}},
         
        \item \diff{\plant{reṇuka}{?}{?}{Piper aurantiacum Wall.\ 
        \citep[\#1924]{NK} is 
        not poisonous.}},
%        \plant{`little bamboo'}{veṇukā}{Bambusa bambos, Druce?}{NK 
%        \#307},

\item
\diff{\plant{kurūkaka}{?}{?}{?}},

\item
\diff{\plant{`little bamboo'}{veṇuka}{Bambusa bambos, Druce?}{NK 
           \#307}},\footnote{Not poisonous.},

        \item \plant{thorn apple}{karambha}{Datura metel, L.}{AVS 2.305 (cf.\
            Abhidhāna\-mañjarī), NK \#796\,ff., Potter 292\,f., ADPS 132.}, 
        
        \item 
            \plant{`big
            thorn apple'}{mahākarambha}{Datura metel, L.?}{AVS 2.305 (cf.\
            Abhidhāna\-mañjarī), NK \#796\,ff., Potter 292\,f., ADPS 132.},

\item 
\diff{\plant{`pleaser'}{nandanā}{?}{?}},

\item 
\diff{\plant{`crow'}{kāka}{?}{?}},
        
%        \item \plant{ribbed gourd}{karkoṭaka}{Luffa acutangula, (L.) Roxb.? 
%        (Mormodica
%            cochinchinensis, Spreng.? Cf.\ Luffa tuberosa)}{AVS 3.347 (NK \#1640,
%            1643; NK \#1520)}, 
        
%        \item \plant{black cardamom}{hareṇu}{Amomum 
%            subulatum,
%            Roxb.?}{PVS Caraka 2.734, AVS 1.128, NK \#154}, \item \plant{purple
%            calotropis}{khadyotaka $\rightarrow$ arka?}{Calotropis gigantea, (L.) R.
%            Br.}{ADPS 52, AVS 1.341, NK \#427, Potter 63},
%        \item \plant{carmarī}{carmarī}{unknown}{?}, \item 
%        \plant{heliotrope}{ibhagandhā
%            $\rightarrow$ hastiśuṇḍa?}{Heliotropium indicum, L.}{AVS 3.136, NK
%            \#1203},
%        \item \plant{`snake-killer'}{sarpaghāti}{unknown}{?},
%        \item \plant{`gladdener'}{nandana}{unknown}{?}, and
%        \item \plant{`juice-cooker'}{sārapāka}{unknown}{?};\footnote{Bamboo 
%is 
%        not 
%        toxic.
%        Heliotrope flowers are abortifacient in large doses.}
            \end{itemize}
    
        \item
        the flower-poisons include those of:
              \begin{itemize}
            
        \item \plant{rattan}{vetra}{Calamus rotang, L.}{AVS 1.330, NK \#413},
        \item \plant{wild chinchona}{kādamba}{Anthocephalus cadamba, Miq.}{NK 
        \#204},
        \item \plant{black pepper}{vallīja $\rightarrow$ marica}{Piper
            nigrum, L.?}{NK \#1929; Rā.6.115, Dha.4.85, Dha.2.88},
        \item \plant{thorn apple}{karambha}{Datura metel, L.}{AVS 2.305
            (cf.\ Abhidhāna\-mañjarī), NK \#796\,ff., Potter 292\,f., ADPS 132.},
        and
        \item \plant{big thorn apple}{mahākarambha}{Datura metel, L.?}{AVS 
        2.305
            (cf.\ Abhidhāna\-mañjarī), NK \#796\,ff., Potter 292\,f., ADPS 132.};
            \end{itemize}
        
        \item
        the seven bark, \se{sāra}{pith} and \se{niryāsa}{resin} poisons are:
              \begin{itemize}
            
        \item \plant{`gutboiler'}{antrapācaka}{unknown}{?},
        \item \plant{`blade'}{kartarīya}{unknown}{?},
        \item \plant{wild mustard}{saurīyaka}{Cleome viscosa, L.?
            (cf.\ Rā.4.144)}{AVS 2.116, NK \#615},
        \item \plant{emetic nut}{karaghāṭa $\rightarrow$ karahāṭa? $\rightarrow$
            madana}{Randia dumetorum, Lamk.}{NK \#2091},
        \item \plant{thorn apple}{karambha}{Datura metel, L.}{AVS 2.305
            (cf.\ Abhidhāna\-mañjarī), NK \#796\,ff., Potter 292\,f., ADPS 132.},
        \item \plant{wild asparagus}{nandana $\rightarrow$ 
        bahuputrā?}{Asparagus 
        racemosus,
            Willd.}{ADPS 441, AVS 1.218, NK \#264, IGP 103,
            IMP 4.2499ff., Dymock 482ff.},
        and
        \item \plant{munj grass}{nārācaka}{Saccharum bengalense, Retz.?}{NK
            \#2184};\footnote{The bark of wild asparagus (\emph{Asparagus 
            racemosus}, Willd.)
        is toxic.}
            \end{itemize}
        \item
        the three \se{kṣīra}{milky sap}-poisons are:
              \begin{itemize}
            
        \item \plant{purple calotropis}{kumudaghnī $\rightarrow$ arka?}{Calotropis
            gigantea, (L.) R. Br.}{ADPS 52, AVS 1.341, NK \#427, Potter
            63},\footnote{The name of this poison, \emph{kumuda-ghnī}, means 
            `lotus
        killer'.  In Sanskrit literature, the \emph{kumuda} lotus is associated
        with the moon, since it blossoms by night.  Since the sun causes this lotus
        to close, it is therefore an `enemy' of the lotus.  One of the chief words
        for the sun, \emph{arka}, is also the name of \emph{Calotropis gigantea},
        which indeed has a milky juice which is a violent purgative, poison and
        abortifacient.}
        \item \plant{oleander spurge}{snuhī}{Euphorbia neriifolia, L., 
        \textnormal{or}
            E. antiquorum, L.}{ADPS 448, AVS (2.388), 3.1, NK
            \#988, IGP 457b},
        %   \marginpar{`The milky juice or gum which flows from the branches
        %     [of \emph{E. antiquorum}] is an acrid irritant\ldots. Internally it is a
        %     powerful emetic and a violent purgative, even in very small quantities'.
        %     --- NK \#982}
        and
        \item \plant{`web-milk'}{jālakṣīri}{unknown}{?};
            \end{itemize}
        
        \item
        the two \se{dhātu}{element}-poisons are:
              \begin{itemize}
            
        \item \plant{`foam-stone'}{phenāśma}{unknown}{?}, and
        \item \plant{orpiment}{haritāla}{Arsenii trisulphidum}{NK v.\,2,
            p.\,20\,ff.};\footnote{\citet[38--42]{dutt-1922} conjectured that
        `foam-stone' may be impure white arsenic obtained by roasting orpiment.}
            \end{itemize}
        \item
        the thirteen tuber-poisons are:
        \begin{itemize}
             \item \plant{jequirity}{kālakūṭa}{Abrus
            precatorius, L.? Cf.\ RRS 21.14.}{AVS 1.10, NK \#6, Potter
            168.},\footnote{The much later (perhaps sixteenth century) alchemical
        \emph{Rasa\-ratna\-samuccaya} of pseudo-Vāgbhaṭa (21.14) says that the
        \emph{kāla\-kūṭa} poison, here translated as `jequirity', is similar to
        `\emph{kāka\-cañcu}' or `Crow's Beak', which is indeed a name for the
        plant jequirity or
        \emph{Abrus precatorius}, L., more commonly called \emph{guñjā} (not to
        be confused with \emph{gañjā}). The black seed-pod is described as
        having a `sharp deflexed beak' in botanical descriptions, so the
        Sanskrit name is quite graphic and appropriate. The poisonous scarlet
        seeds of \emph{A. precatorius} can have a distinct black dot or tip,
        which could perhaps be translated `\emph{kāla-kūṭa}', or `Black Tip'.
        
        The \emph{Rāja\-nighaṇṭu\-pariśiṣṭa} (9.35) gives \emph{kālakūṭaka} as a
        synonym for \emph{kāras\-kara}, or \emph{Strychnos nux-vomica}, L., 
        whose
        seeds are notoriously poisonous.}
        \item \plant{wolfsbane}{vatsanābha}{Aconitum napellus, L.}{AVS 1.47,
            NK \#42, Potter 4\,f.},
        \item \plant{Indian mustard}{sarṣapa}{Brassica juncea, Czern. \&
            Coss.}{AVS 1.301, NK \#378},
        \item \plant{leadwort}{pālaka $\rightarrow$  citraka}{Plumbago zeylanica
            (indica? rosea?), L.}{Rā. 6.124, ADPS 119, NK \#1966,
            1967},
        \item \plant{`muddy'}{kardama}{unknown}{?}, the
        \item \plant{`Virāṭa's plant'}{vairāṭaka}{unknown}{?},
        \item \plant{nutgrass}{mustaka}{Cyperus rotundus, L.}{ADPS 316,
            AVS 2.296, NK \#782},
        \item \plant{atis root}{śṛṅgīviṣa}{Aconitum heterophyllum, Wall.
            ex Royle}{AVS 1.42, NK \#39},
        % \item \plant{liquorice}{prapuṇḍarīka $\rightarrow$ 
        %madhuka?}{Glycyrrhiza
        %  glabra, L.}{AVS 3.84, NK \#1136},\footnote{Non-toxic.}
        \item \plant{sacred lotus}{prapuṇḍarīka}{Nelumbo nucifera, Gaertn.}{Dutt 
        110, 
        NK
            \#1698}, \item \plant{radish}{mūlaka}{Raphanus sativus, L.}{NK 
            \#2098},
        \item \plant{`alas, alas'}{hālāhala}{unknown}{Cf. Soḍhalanighantu p.43 
        (sub
            bola) = stomaka = vatsanābha}, \item \plant{`big
            poison'}{mahāviṣa}{unknown}{?}, and \item 
            \plant{galls}{karkaṭa}{Rhus
            succedanea, L.}{NK \#2136}.\footnote{Leadwort root is a powerful poison.
        Nutgrass is tuberous, but non-toxic. Atis has highly toxic tuberous
        roots. Neither sacred lotus nor galls are toxic. The `alas, alas' poison
        (\emph{hālāhala}) is the mythical poison produced from the churning of
        the ocean at the time of creation: it occurs in medical texts such as
        the present one, and commentators identify it with one or other of the
        lethal poisons such as wolfsbane or jequirity.
        \citet[126]{agra-indi} makes the intriguing suggestion
        that the word \emph{hālāhala},
        possibly to be identified with Pāṇini's \emph{hailihila} (P.6.2.38),
        may be of Semitic origin, although his evidence
        seems uncertain (\citet[1506a]{stei-pers} cites Persian \emph{halāhil}
        `deadly (poison)' as a loan from Sanskrit). \cite[iii.585]{mayr-kurz}
        also cites a claim for an Austro-Asiatic origin for the word.}
            \end{itemize}

    Thus, there are fifty-five stationary poisons.
    
    \item[6] There are believed to be four kinds of wolfsbane, two kinds of
\emph{mustaka}, and six kinds of Indian \emph{sarṣapa}.  But the rest are said
to be unique types.
    
    
    
    \subsection{The effects of poisons}
    \item[7--10]
    
People should know that root-poisons cause \se{udveṣṭana}{writhing}, 
\se{pralāpa}{ranting}, and
\se{moha}{delirium}, and  leaf-poisons cause yawning, writhing, and 
\se{śvāsa}{wheezing}.
    
 Fruit-poisons cause swelling of the
   scrotum, a burning feeling and writhing.  Flower-poisons will
    cause vomiting, \se{ādhmāna}{distension} and \se{svāpa}{sleep}.  
    
The consumption of poisons from bark, \se{sāra}{pith} and \se{niryāsa}{resin} 
will
cause foul breath, \se{pāruṣya}{hoarseness}, a headache, and a
discharge of \se{kapha}{phlegm}.\footnote{At \Su{1.2.6 }{11}, Ḍalhaṇa
glosses \se{pāruṣya}{hoarseness} as \emph{vāgrūkṣatā}, “a rough, 
dry voice.”}
    
    % 10
    
     The \se{kṣīra}{milky sap}-poisons make one froth at the mouth,  cause loose
stool, and make the tongue feel heavy.\footnote{At \Su{6.54.10}{773}, Ḍalhaṇa
glosses \se{viḍbheda}{loose stool} as \emph{dravapurīṣatā}, “having liquid
stool.” }  The \se{dhātu}{element}-poisons give one a crushing pain in the
chest, make one faint and cause a burning feeling on the palate.
    
    % 11
    These poisons
    are classified as ones which are generally speaking lethal after a period of time.
    
    \item[11--17]
    
    \subsubsection{Symptoms of tuber poisoning}
    The tuber-poisons, though, are severe.  I shall talk about them in detail.
    
    %12
    
    With
    \plant{jequirity}{kālakūṭa}{Abrus precatorius, L.?
        Cf.\ RRS 21.14.}{AVS 1.10, NK \#6, Potter 168.}, there is numbness
    and very severe trembling.
    %shivering
%
    With
    \plant{wolfsbane}{vatsanābha}{Aconitum napellus, L.}{AVS 1.47,
        NK \#38, Potter 4\,f.}, there is rigidity of the neck, and the faeces,
    and urine become yellow.
    
    %13
    With \se{sārṣapa}{sārṣapa}%
%With \plant{Indian mustard roots}{sārṣapa}{Brassica juncāea, Czern \&
%    Coss.}{AVS 1.301, NK \#378}
,\footnote{\emph{Sārṣapa} would normally mean
“connected with mustard,” and excessive consumption of mustard oil can be 
harmful. However, the \emph{Sauśrutanighaṇṭu} (156) gives
\emph{rakṣoghnā} as a synonym for \emph{sarṣapā}. This can be
\textit{Semecarpus anacardium}, L.f., which has some poisonous parts.} the
\skt{wind becomes defective}{vātavaiguṇya}, there is \se{ānāha}{constipation},
and \se{granthi}{lumps} start to appear. %
With \plant{leadwort}{pālaka $\rightarrow$  citraka}{Plumbago zeylanica
    (indica? rosea?), L.}{Rā. 6.124, ADPS 119, NK \#1966, 1967}, there is weakness
in the neck, and speech gets jumbled.\footnote{The verse in the Nepalese version 
ends with a plural verb that does not agree with the dual of the sentence subject.}
    
    %14
    With the one called
    \plant{`muddy'}{kardama}{unknown}{?},
    there is a \se{praseka}{discharge}, the faeces pour out, and  the eyes
    turn yellow.
    %
The
    \plant{`Virāṭa's plant'}{vairāṭaka}{unknown}{?}
causes pain in the body and illness in the head.
    %
    Paralysis of one's arms and legs and trembling are said to be caused by
    \se{mustaka}{mustaka}.%
%    \plant{nutgrass}{mustaka}{Cyperus rotundus, L.}{ADPS 316, AVS 2.296,
%        NK \#782} %
\footnote{The substitution 
    in \MScite{NAK 5-333} affecting 15cd is caused by an eye-skip to the word 
    \emph{viṣeṇa} in 2.17.  \emph{Mustaka} commonly refers to Cyperus 
    rotundus, L.; the root is used in āyurveda but is 
    not poisonous.  However other dictionaries list \emph{mustaka} amongst 
    serious poisons, for example \emph{Rājanighaṇṭu} (22 v.\,42) and 
    \emph{Rasaratnasamuccaya} 16, v.\,80.  However, its ancient identity is still 
    doubtful.}
    \item[ 15b]
    With \se{mahāviṣa}{great aconite}\q{-> ativiṣa}
%    \plant{atis root}{śṛṅgīviṣa}{Aconitum heterophyllum, Wall.
%        ex Royle}{AVS 1.42, NK \#39}, 
    one's limbs grow weak, there is a burning
    feeling and swelling of the belly.\footnote{The poisonous root 
    \se{mahāviṣa}{great poison} is not clearly identifiable, although \emph{viṣa} 
    is commonly aconite.  Verse 6 above notes that there are several kinds of 
    aconite.}
    \item[ 16a]
    With \se{puṇḍarīka}{puṇḍarīka},
%    \plant{sacred lotus}{puṇḍarīka}{Nelumbo nucifera, Gaertn.}{Dutt 110,
%        NK \#1698},
    one's eyes go red, and one's belly becomes distended.\footnote{The word 
    \emph{puṇḍarīka} very commonly means sacred lotus, Nelumbo nucifera, 
    Gaertn. The entire plant is edible and cannot be the poison intended here.  
    \citet[252]{sing-1972} noted that this poison is unidentified and that it is also 
    listed as a poison in \Cs{ci.23.12}{}.}\q{Look up the ca. reference.}
    \item[ 16b]
    With \se{mūlaka}{mūlaka},
%    \plant{radish}{mūlaka}{Raphanus sativus, L.}{NK \#2098}es,
    one's body is drained of colour and the limbs are paralysed.\footnote{The word 
    \emph{mūlaka} very commonly means the radish, \emph{Raphanus sativus}, 
    L. The root is edible and cannot be the poison intended here.  
    \citet[317]{sing-1972} noted that this poison is unidentified.}
    
    %17
    \item[ 17a]
        
    With \se{Aconite}{hālāhala}, a man turns a \se{dhyāma}{dark colour}, and
gasps.\footnote{Identification of \emph{hālāhala} is  uncertain. It may simply
be a mythical poison, or its specific identity may have been lost over the
centuries. Late \emph{nighaṇṭu}s identify it as \emph{stomaka} =
\emph{vatsanābha}, i.e., \emph{Aconitum napellus}, L. 
(\emph{Soḍhalanighantu}
p.43). Ḍalhaṇa on \Su{5.2.17}{564} interprets our “gasps” as “the man laughs
and grinds his teeth.”  But this gloss is probably displaced and intended to apply 
to verse 2.18.}

% 5.221 Rājanighaṇṭu

\item[ 17b] With \plant{atis root}{śṛṅgīviṣa}{Aconitum
    heterophyllum, Wall.\ ex Royle}{AVS 1.42, NK \#39}, one gets violent
\se{granthi}{knots} and stabbing pains in the 
heart.\footnote{\citet[407]{sing-1972} noted that \emph{vatsanābha} and 
\emph{śṛṅgīviṣa} are two different varieties of poisonous Aconites that are 
difficult to distinguish.}
    
    %18
    \item[ 18a]
    With
    \se{monkey}{markaṭa}, one leaps up, laughs, and 
    bites.\footnote{\citet[299]{sing-1972} said of \emph{markaṭa}, “an 
    unidentified vegetable poison.”  Cf.\ \cite[v.36]{suve-2000} for synonyms that 
    lead to the non-toxic jujube tree.}
    
    %{galls}{karkaṭa}{Rhus succedanea, L.}{NK \#2136}
    
    \item[ 18b-19a]
%    Experts said that the thirteen cited highly potent tuber-poisons should be 
%known to have possessed ten features:
%    %()Experts said that one should know that these thirteen cited highly potent 
%%tuber-poisons have ten features:
    %Experts said that these thirteen highly potent tuber-poisons which are 
    %mentioned here consist of ten features.)
    
    Experts have said that one should know that the thirteen highly potent 
    tuber-poisons, which are mentioned here, have ten \se{guṇa}{qualities}.
    
    \item[ 19b--20a]
    
    The ten are:
    \begin{itemize}
        \item    \se{rūkṣa}{dry}, 
        \item hot, 
        \item sharp, 
        \item \se{sūkṣma}{rarified},
        \item     fast-acting, 
        \item \se{vyavāyin}{pervasive}, 
        \item \se{vikāsin}{expansive}, 
        \item \se{viśada}{limpid},
        \item     light, and 
        \item indigestible.    
    \end{itemize}
    %20b
    \item[ 20b]
    Because of dryness, it may cause inflammation of the wind; because of heat
    it inflames the choler and blood. 
    %21
    Because of the sharpness it unhinges the
    mind, and it cuts through the connections with the \skt{sensitive
        points}{marman}.  Because it is rarified it can infiltrate and distort
    the parts of the body.\footnote{We read the active \emph{vikaroti} with 
    Ḍalhaṇa against the 
    transmitted passive \emph{vikriyeta}, since it must be the parts of the body 
    that are distorted, not the poison.}    
    

\item[22]
Because it is fast-acting it kills quickly, and because of its pervasiveness
it affects one's \skt{whole physical constitution}{prakṛti}.\footnote{Ḍalhaṇa
on \Su{5.2.22}{565} explained this as “\se{akhiladehavyāptirūpam}{takes the
form of pervading the whole body}.”}  Because of its expansiveness it enters
into the \se{doṣa}{humour}s, \se{dhātu}{bodily constiuents}s, and even the
impurities\sskt{impurity}{mala}.  Because it is limpid it overflows, and
because it is light it is difficult to treat.  Because it is indigestible it
is hard to eliminate.  Therefore, it causes suffering for a long time.
    
    \item[ 24]
    Any poison that is instantly lethal, whether it be
    stationary, mobile, or artificial, will be known to 
    have all ten of these qualities.
    
    
  
    
    \subsection{Slow-acting poison}
    \item[25cd--26]  
    \begin{verse}
        A poison that is old or destroyed by
        anti-toxic medicines, or else dried up by blazing fire, wind, or sunshine, or
        which has just lost its qualities by itself,\footnote{Ḍalhaṇa specified that this 
        refers to the ten 
        qualities that are mentioned above (\Su{5.2.26}{565}).} becomes a 
        \skt{slow-acting poison}{dūṣīviṣa}.\footnote{Ḍalhaṇa cited this verse at 
        \Su{1.46.83}{222} while explaining \emph{dūṣīviṣa}.}
                Because it has lost its potency it is
        no longer perceived.  Because it is surrounded by \se{kapha}{phlegm} it 
        has an aftermath that lasts for a very long time.
        
        \item[27] If he is suffering from this, the colour of his stools changes,
he gets sourness and a bad taste with great thirst. Stammering and close
to death, wandering about, he may feel faint, giddy, and
aroused.\footnote{Similar symptoms of slow-acting poison are described at
\Su{2.7.11--13}{296} in the context of  \se{duṣyodara}{contamination
dropsy}.  This this may explain why the vulgate inserted reference to this
disease at this point.}
        
        
        
%        Also, he has
%        the symptoms of \skt{contaminated
%            dropsy}{duṣyodara}.
%        \footnote{\label{dusyodara}`Contaminated dropsy'
%        (\emph{duṣyodara} or \emph{dūṣyudara}) is described elsewhere as a
%        condition which arises when women of ill-character mix nail clippings,
%        hair, urine, faeces, or menstrual blood with a man's food, in order to
%        gain power over him (2.7.11--13).}



        \item[28]
        If it lodges in his \se{āmāśaya}{stomach}, he becomes sick because of wind 
        and phlegm; if it lodges in his \se{pakvāśaya}{intestines}, he becomes sick 
        because of  wind and 
        choler.  A man's hair and limbs fall away and he looks like a
        bird whose wings have been chopped off.
        \item[29a--c]
        If it lodges in one of the body tissues such as 
        \se{rasa}{chyle}, it causes the diseases arising
        from the body tissues, that have been said to be wrong.\footnote{The 
        expression \emph{ayathāyathoktān} “stated to be unsuitable” is hard to 
        understand here, but is clearly transmitted in the Nepalese version.}
        and it rapidly becomes inflamed on days that are nasty
        because of cold and wind.
        
        \item[29d--31] Listen to its initial \se{liṅga}{symptoms}: it causes
heaviness due to sleep, yawning, \se{viśleṣa}{disjunction} and
\se{harṣa}{horripilation} and a \se{aṅgamarda}{bruising of the
    limbs}.\footnote{Ḍalhaṇa \Su{5.2.30ab}{565} glossed “disjunction” as the
loss of function of the joints in regard to movement.} Next, it causes
\se{annamada}{intoxication from food} and indigestion, \se{arocaka}{loss
    of appetite}, the condition of having a \se{koṭha}{skin disease} with
\se{maṇḍala}{round blotches},\footnote{The last ailment could perhaps be
ringworm.} % 5.2.31
\diff{\se{kṣaya}{dwindling away} of flesh}, swelling of the feet, hands, and
face, \diff{the fever called \textit{pralepaka}}, vomiting and
diarrhoea.\footnote{The \emph{pralepaka} fever was described by Ḍalhaṇa,
at \Su{6.39.52}{675}, as an accumulation of phlegm in the joints.  Its
symptoms are described in 6.39.54} The slow-acting poison might cause
\diff{wheezing, thirst and fever, and it might also cause distension of the
abdomen.}
        
        %Perhaps his colour may drain away and he may faint or have \se{viṣamajvara}{irregular fever}.  It may cause heightened,
        %powerful thirst.
        
        \item[32]
 
            These various disorders are of many different types: one poison may 
            produce
            madness, while another one may cause \se{ānāha}{constipation}, and 
            yet
            another may ruin the semen. One may cause \diff{emaciation}, while 
            another
            \se{kuṣṭha}{pallid skin disease}.
 
    \end{verse}

    
    \item[33]  
Something is “corrupted” by repetitively keeping to bad locations, times,
  foods, and sleeping in the daytime.  Or, traditionally, “corrupting poison” 
  (\se{dūṣī-viṣa}{slow-acting poison}) is so called because
    it may corrupt (\emph{dūṣayet}) the \se{dhātu}{body tissue}s.  
    
    
    
    
    
    \item[34-]
    \subsubsection{The stages of toxic shock}

    In the first shock of having taken a stationary poison, a person's tongue becomes dark brown and stiff, he grows faint, and panics.
    
    
    
    % FROM HERE Harṣal 35-38
    \item[35]
    In the second, he trembles, feels exhausted, has a burning feeling, as well as a
    sore throat.  When the poison reaches the \se{āmāśaya}{stomach}, it causes
    pain in the \se{hṛd}{chest}.
    
    
    
    \item[36]
    In the third,his palate goes dry, he gets violent \se{śūla}{pain} in the 
    \se{āmāśaya}{stomach}, and his eyes become weak, swollen and yellow.

    \item[37]
    In the fourth shock, it causes the intestines and stomach to
    \se{sāda}{be exhausted}, he gets hiccups, a cough,  a rumbling in the
    \se{antra}{gut}, and his head becomes heavy too.
    
     \item[38]
    In the fifth he dribbles \se{kapha}{phlegm}, goes a bad colour,
    his \diff{\se{parśvabheda}{ribs crack}},  all his humours are irritated, and he
    also has a pain in his \se{pakvādhāna}{intestines}.
   
   
    \item[39a]
    In the sixth, he loses consciousness and he completely loses
    control of his bowels.
    
    \item[39b]
    In the seventh, there are breaks in his shoulders, back and loins, and he  
stops breathing.\footnote{%
%In \Su{1.15.24}{72}, Ḍalhaṇa glossed 
%\emph{kriyā-sannirodha} 
%as “cessation of the activities of the body, speech and mind” 
%(\emph{kriyāṇāṃ kāyavāṅmānasīnāṃ sannirodhaḥ}), while 
Here at \Su{5.2.24}{566} Ḍalhaṇa glossed \emph{sannirodha} as
“complete cessation, i.e., of breath” (\emph{sannirodhaḥ 
samyaṅnirodhaḥ, ucchvāsasya iti śeṣaḥ}).
The manuscripts all read \emph{skanda} where \emph{skandha} must be 
intended; this confusion is known from Buddhist Hybrid Sanskrit 
\citep[608]{edge-1953}.}
    
    % next  40-44
    
      % from here down it's DW's old translation of the vulgate text.
    \subsubsection{Remedies for the stages of slow poisoning}
  
    \item[40] In the first shock of the poison, the physician should make the man,
who has vomited and been sprinkled with cold water, drink an
\se{agada}{antidote} mixed with with honey and ghee.
    
    \item[41a] In the second, he should make the man who has vomited and been
purged drink as before;
    
    \item[41b]
    on the third, drink an antidote and a beneficial
    \se{nasya}{nasal medicine} as well as an \se{añjana}{eye salve}.
    
    
    \item[42a] In the fourth, the physician should make him drink an antidote that
is salt with a little oil.\footnote{At \Su{6.52.30}{769} Ḍalhaṇa noted that
\emph{sindhu} can be interpreted as \se{saindhava}{salt}.}
    
    
    % got to here. 
    
    \item[42b]
    In the fifth, he should be prescribed the antidote together with a
    \se{kvātha}{decoction} of honey and
    \plant{liquorice}{madhuka}{Glycyrrhiza glabra, L.}{AVS 3.84, NK \#1136}.
    \item[43]
    In the sixth, the cure is the same as for diarrhoea.
    %
    And in the seventh, he should have medicated powder blown up his nose, and
    after having a `\se{kākapada}{crow's foot}' cut made on his head, he
    should have a piece of bloody meat put on
    it.\footnote{\label{su:kakapada}Suśruta explains the term \emph{avapīḍa}
    `medicated nasal powder' as the procedure either of administering
    \se{avapīḍa}{nasal drops}, or blowing medicated powder into the nose
    (4.40.44--46): it is particularly recommended for unconscious or incapable
    patients.  The `crow's-foot' procedure is also recommended later in the
    `Section on Procedures' (5.5.24a) in cases of snake-bite. It is also
    described by Caraka (see p.\,\pageref{sa:kakapada} below).}
    
    \item[44]
    In the intervals between each shock, assuming that the above actions have
    been performed, one should give the patient cold porridge together with ghee
    and honey, to take away the poison.
    
    % Dominik - 
    
    \item[45--46]
    \begin{sloppypar}
        Both kinds of poison are destroyed by a porridge prepared with the
        \se{niṣkvātha}{stewed juice} of the following:
        \plant{luffa}{koṣātakya}{Luffa cylindrica, (L.) M. J. Roem. \textnormal{or}
            L. acutangula, (L.) Roxb.}{ADPS 252, NK \#1514 etc.},
        \plant{migraine tree}{agnimantha}{Premna corymbosa, Rottl.}{IMP 1927, 
        ADPS 21,
            NK \#2025, AVS 4.348; GJM 523: = P. integrifolia/serratifolia, L.},
        \plant{velvet-leaf}{pāṭhā}{Cissampelos pariera, L.}{ADPS 366,
            NK \#592, GJM 573, IMP 1.95; cf. AVS 2.277},
        \plant{`sun-creeper'}{sūryavallī $\rightarrow$ jīvantī?}{Holostemma
            ada-kodien, Schultes}{ADPS 195, AVS 3.167, NK \#1242, IMP 3.1619},
        \plant{heart-leaved moonseed}{amṛtā}{Tinospora cordifolia,
            (Willd.) Hook.f. \& Thoms.?}{ADPS 38, NK \#2472 \& 624,
            Dastur \#229},
        \plant{myrobalan}{abhayā}{Terminalia chebula, Retz.}{ADPS 172, NK 
        \#2451,
            Potter 214}s,
        \plant{siris}{śirīṣa}{Albizia lebbeck, Benth.}{AVS 1.81, NK \#91},
        \plant{white siris}{kiṇihī}{Albizia procera, (Roxb.) Benth.}{GVDB 98,
            NK \#93},
        \plant{selu plum}{śelu}{Cordia myxa, L. non
            Forssk.}{GJM 529 (2),
            IGP 291b, cf.\ IMP 3.1677f; cf. AVS 2.180 (C. dichotoma, Forst.f.),
            NK \#672 (C. latifolia, Roxb.)},
        \plant{white clitoria}{giryāhvā}{Clitoria ternatea, L.}{AVS 2.129, NK
            \#621},
        the two kinds of
        \plant{turmeric}{rajanī}{Curcuma longa, L.}{ADPS 169, AVS 2.259,
            NK \#750},
        the two
        \plant{hogweed}{punarnavā}{Boerhaavia diffusa, L.}{ADPS 387,
            AVS 1.281, NK \#363}s (red and white),
        \plant{black cardamom}{hareṇu}{Amomum subulatum, Roxb.?}{PVS
            Caraka 2.734, AVS 1.128, NK \#154},
        the \se{trikaṭu}{three pungent spices} % \marginpar{kaṭu=piquant?}
        (\plant{dried ginger}{śuṇṭhī}{Zingiber officinale, Roscoe.}{ADPS 50, NK
            \#2658, AVS 5.435, IGP 1232},
        \plant{long pepper}{pippalī}{Piper longum, L.}{ADPS 374, NK
            \#1928}, and \plant{black pepper}{marica}{Piper nigrum, L.}{ADPS
            294, NK \#1929}), %
        %the two \plant{Indian sarsaparilla}{sārive}{Hemidesmus indicus,
        %(L.) R.
        %  Br. \textnormal{and} Cryptolepis buchanani, Roemer \&
        %  Schultes}{ADPS 434, AVS 3.141, NK \#1210}s,
        the two \se{sārive}{Indian sarsaparillas}
        (\plant{country sarsaparilla}{anantā}{Hemidesmus indicus, (L.) R. 
        Br.}{ADPS 434,
            AVS 3.141--5, NK \#1210}
        and
        \plant{black creeper}{pālindī}{Ichnocarpus frutescens, (L.)
            R.Br. \textnormal{or} Cryptolepis buchanani, Roemer \&
            Schultes}{AVS 3.141, 3.145, 3.203, NK \#1283, \#1210, ADPS
            434})
        %
        and
        \plant{country mallow}{balā}{Sida cordifolia, L.}{ADPS 71, NK \#2297}.
    \end{sloppypar}
    
    \item[ 47--49]
    \subsection{The `invincible' ghee}
    \label{ajeya}
    There is a famous ghee called \se{ajeya}{`Invincible'}. It rapidly
    destroys all poisons and `always conquers'. It is made with a
    \se{kalka}{mash} of the following plants:
    \plant{liquorice}{madhuka}{Glycyrrhiza glabra, L.}{AVS 3.84, NK \#1136},
    \plant{Indian rosebay}{tagara}{Tabernaemontana divaricata (L.) R.Br.\ ex
        Roem.\ \& Schultes.}{GJM 557, AVS 5.232},
    %\plant{Indian valerian}{tagara}{Valeriana wallichii, DC.}{NK \#2558, cf.\
    %  Potter 311},
    \plant{costus}{kuṣṭha}{Saussurea costus, Clarke}{NK \#2239},
    \plant{deodar}{bhadradāru}{Cedrus deodara,
        (Roxb.ex D.Don) G. Don}{AVS 41, NK \#516},
    \plant{black cardamom}{hareṇu}{Amomum subulatum, Roxb.?}{PVS
        Caraka 2.734, AVS 1.128, NK \#154},
    \plant{Alexandrian laurel}{punnāga}{Calophyllum inophyllum,
        L.}{AVS 1.338, NK \#425},
    \plant{cherry}{elavāluka}{Prunus cerasus, L.?}{BVDB 58, NK \#2037},
    \plant{cobra's saffron}{nāgapuṣpa}{Mesua ferrea, L.}{NK \#1595},
    \plant{water-lily}{utpala}{Nymphaea stellata, Willd.}{GJM 528,
        IGP 790; Dutt 110, NK \#1726},
    \plant{white clitoria}{sitā $\rightarrow$ śvetā?}{Clitoria
        ternatea, L.}{AVS 2.129, NK \#621},
    \plant{embelia}{viḍaṅga}{Embelia ribes, Burm. f.}{ADPS 507,
        AVS 2.368, NK \#929, Potter 113},
    \plant{sandalwood}{candana}{Santalum album, L.}{ADPS 111, NK \#2217},
    \plant{cassia cinnamon}{patra}{Cinnamomum tamala,
        (Buch.-Ham.) Nees}{AVS 2.84, NK \#},
    \plant{`going-to-my-darling'}{priyaṅgu}{Callicarpa macrophylla,
        Vahl.}{AVS 1.334, NK \#420},
    \plant{rosha grass}{dhyāmaka}{Cymbopogon martinii (Roxb.) Wats}{AVS 
    2.285,
        NK \#177},
    the two turmerics
    (ordinary
    \plant{turmeric}{rajanī}{Curcuma longa, L.}{ADPS 169, AVS 2.259,
        NK \#750}
    and
    \plant{Indian barberry}{dāruharidrā}{Berberis aristata, DC.}{Dymock 1.65, NK
        \#685, GJM 562, IGP 141}),
    the two \se{bṛhatī}{Indian nightshade}s
    (\plant{poison berry}{bṛhatī}{Solanum violaceum, Ortega}{ADPS 100,
        NK \#2329, AVS 5.151}
    and
    \plant{yellow-berried nightshade}{kṣudrā}{Solanum virginianum, L.}{ADPS
        100, NK \#2329, AVS 5.164}),
    %the two \se{sārive}{Indian sarsaparilla}s
    %  (\plant{country sarsaparilla}{anantā}{Hemidesmus indicus, (L.) R. 
    %Br.}{ADPS 434,
    %    AVS 3.141--5, NK \#1210}
    %  and
    %  \plant{black creeper}{pālindī}{Ichnocarpus frutescens, (L.)
    %    R.Br.}{AVS 3.203, 3.145, NK \#1283, ADPS 434}),
    the two \se{sārive}{Indian sarsaparillas}
    (\plant{country sarsaparilla}{anantā}{Hemidesmus indicus, (L.) R. Br.}{ADPS 
    434,
        AVS 3.141--5, NK \#1210}
    and
    \plant{black creeper}{pālindī}{Ichnocarpus frutescens, (L.)
        R.Br. \textnormal{or} Cryptolepis buchanani, Roemer \&
        Schultes}{AVS 3.141, 3.145, 3.203, NK \#1283, \#1210, ADPS
        434}),
    %
    \plant{beggarweed}{sthirā $\rightarrow$ śālaparṇī}{Desmodium
        gangeticum (L.) DC}{Dymock 1.428, GJM 602, NK \#1192;
        ADPS 382, 414 and AVS 2.319, 4.366 are confusing},
    and
    \plant{`spotted-leaf'}{sahā $\rightarrow$ pṛśniparṇī}{Uraria
        lagopoides, DC}{GJM 577, Dymock 1.426, IMP 1.750ff., NK \#2542;
        ADPS 382, AVS 2.319 4.366 are confusing}.
    
    
    % Jason - 
    
    \item[ 50--52]
    \subsection{Curing the `slow-acting' poison}
    
    \begin{sloppypar}
        Someone suffering from `\se{dūṣīviṣa}{slow-acting poison}' should be well
        sweated, and purged both top and bottom.  Then he should in all cases be
        made to drink the following antidote which removes `slow-acting
        poison':
    \end{sloppypar}
    
    Take
    \plant{long pepper}{pippalī}{Piper longum, L.}{ADPS 374, NK \#1928},
    \plant{rosha grass}{dhyāmaka}{Cymbopogon martinii (Roxb.) Wats}{AVS 
    2.285,
        NK \#177},
    \plant{spikenard}{māṃsī}{Nardostachys grandiflora, DC.}{NK \#1691},
    \plant{lodh tree}{śāvara $\rightarrow$ lodhra}{Symplocos racemosa,
        Roxb.}{ADPS 279, NK \#2420},
    \plant{nutgrass}{paripelava $\rightarrow$ plava $\rightarrow$ 
    mustā?}{Cyperus
        rotundus, L.}{ADPS 316, AVS 2.296, NK \#782},
    \chemical{soda crystals}{suvarcikā $\rightarrow$ suvarjikā}{Sodium
        carbonate}{NK 2, p.\,101},
    \plant{cardamom}{sūkṣmailā}{Elettaria cardamomum, Maton}{AVS 2.360, NK 
    \#924,
        Potter 66},
    \plant{`scented pavonia'}{toya $\rightarrow$ bālaka}{Pavonia odorata,
        Willd.}{ADPS 498, NK \#1822},
    and
    \se{kanakagairika}{`gold-chalk' ochre}.
    %
    This antitoxin, taken with honey, eliminates `slow-acting poison'. It is
    called `\se{dūṣīviṣāri}{slow-acting poison antidote}', and there is no
    situation where it is not recommended.
    
    % Deepro - 
    
    \item[ 53--54]
    If there are any \se{upadrava}{side-effect}s, such as fever, a burning
    feeling, hiccups, \se{ānāha}{constipation}, depletion of the semen,
    distension, diarrhoea, fainting, illness in the heart,
    \se{jaṭhara}{bellyache}, madness, trembling, or others, then one
    should treat each one in its own terms, as well as using the anti-toxic
    medicines.
    
    \item[ 55]
    `Slow-acting poison' is \se{sādhya}{curable} if caught immediately; it is
    \se{yāpya}{treatable} if it is of a year's standing; but it cannot be cured
    in someone who has unhealthy habits or who is \se{kṣīṇa}{weak}.
    

    \begin{center}
        Thus ends the second chapter, called `on the knowledge of stationary 
        poisons',
        in the  Procedures Section of Suśruta's \emph{Compendium}.
    \end{center}
    \end{translation}


\endinput 
% PLANTS OF GARDEN OR WOODS
% WITH POISONOUS ROOTS AND STEMS
% Arisaema triphyllum
% Colchicum autumnale
% Convallaria majalis
% Dicentra spp.
% Gloriosa superba
% Hyacinthus spp.
% Iris spp.
% Narcissus spp.
% Ornithogalum umbellatum
% Phytolacca americana
% Podophyllum peltatum
% Jack-in-the-pulpit
% Autumn Crocus
% Lily-of-the-Valley
% Bleeding-heart and
% Dutchman’s Breeches
% Glory-lily
% Hyacinth
% Iris, Flags
% Narcissus, Daffodil
% Star-of-Bethlehem
% Pokeweed
% May-apple, Mandrake
%--  DONALD WYMAN
%http://arnoldia.arboretum.harvard.edu/pdf/articles/1966-26--a-few-poisonous-plants.pdf
